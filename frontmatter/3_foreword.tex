\chapter{Foreword}\label{chap-foreword}
\bigskip
 
Thousands of years ago, in the middle of chaos, uncertainty, and fear on an enormous battlefield, supreme knowledge was given to humanity. We know it now as the \textit{Bhagavad Gītā}. 

The \textit{Gītā} is a narrative between God and a conflicted warrior named Arjuna. The Supreme Lord, in the form of Kṛṣṇa, plays the humble role of a charioteer who masterfully instructs Arjuna about the profound mysteries of life. 

The truth is, the world we are living in today is the same battlefield Arjuna found himself on. The only difference is that the battle each one of us faces is not on the outside, but on the inside. 

In the face of these trials, we need a source of greater wisdom to help us overcome the devastating impact on our thinking, emotions, and reactions. Fortunately for us, Kṛṣṇa’s instructions have stood the test of time and provide the knowledge to help us triumph over the obstacles we face. 

Paramahamsa Vishwananda’s commentary on Kṛṣṇa’s teachings opens up and applies the wisdom of the \textit{Bhagavad Gītā} to our lives today. Ultimately, it helps us understand that we have a relationship with God, and that the goal of life is to become fully aware of that divine connection. 

If we can integrate the teachings of the \textit{Bhagavad Gītā} and Paramahamsa Vishwananda’s commentary in this way, we will begin a uniquely personal journey that takes us from the mind to the heart. And, despite the challenges along the way, it is the guidance of the \textit{guru} that can help us to walk it with trust and courage. 

--SC