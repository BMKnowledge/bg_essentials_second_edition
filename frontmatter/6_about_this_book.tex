\chapter{About This Bhagavad Gītā}\label{chap-about-this-bhagavad-gita}

It is rare when a book has the potential to become a lifelong spiritual companion. \textit{Bhagavad Gītā Essentials} is designed to be just that: an essential part of your life. Small enough to carry with you wherever you go, yet profound enough to carry you all the way to God. It is succinct enough to read in a matter of hours, yet deep enough to contemplate for decades to come. 

\textit{Bhagavad Gītā Essentials} evolved out of two live discourses given by Paramahamsa Vishwananda: an 18-day course in 2014 and an 8-day course in 2016. The essence of these commentaries has been distilled to reveal the deeper meaning of this timeless scripture. All 701 Sanskrit verses of the \textit{Bhagavad Gītā} are presented herein along with translations that are infused with the beauty of \textit{bhakti}: love and devotion for God. 

In addition, this \textit{Bhagavad Gītā} has been specially designed to help you get the most out of its verses and commentary:
\begin{itemize}[itemsep=1pt, topsep=6pt]
    \item Each chapter begins with highlights of the knowledge it contains and introduces key points to help you navigate your own journey.

    \item Within each chapter, the verses are grouped together under titles which help you stay connected to its themes.

    \item Finally, definitions of key Sanskrit terms have also been included at the end of each chapter to help you better understand the philosophy behind the verses and commentaries.
\end{itemize}

\section{How to Savor this Bhagavad Gītā}\label{sec-how-to-savor-this-bhagavad-gita}

The \textit{Bhagavad Gītā} offers a most precious treasure: the timeless discourse of Lord Kṛṣṇa which Paramahamsa Vishwananda’s commentary unravels and delivers straight to the heart. 

What you do with it depends on the value you give it. Receiving the Gītā with this in mind allows you to draw on its wisdom in the most powerful way. There are many ways you can savor this scripture: 

\begin{itemize}[itemsep=1pt, topsep=6pt]

\item Reading it cover to cover gives you an overview of the entire scripture. 

\item Reading only the verses from beginning to end allows you to stand next to Arjuna and take direct instruction from the Lord Himself. 

\item Concentrating on the introductory text and commentaries of each chapter helps you discover the secrets hidden within the verses. 

\item Simply letting the Bhagavad Gītā fall open to any page allows the Divine to reveal something to you in the moment. Take it as a gift and contemplate on what you find. 

\end{itemize}

Whether you read a verse a day, a chapter a week, the verses alone, only the commentary, or you just open the Bhagavad Gītā to see what will be revealed to you in the moment, you’ll surely want to keep it close at hand and make it a part of your daily life.

\section{Special Note from Paramahamsa Vishwananda}\label{sec-special-note-from-paramahamsa-vishwananda}

\enquote{The \textit{Bhagavad Gītā} is not a novel. It's not just a book which you read whenever you have time. You have to soak your mind with it. Not only reading it one time.}

\enquote{You have to dive into it, because each line of the \textit{Bhagavad Gītā}, each phrase which Kṛṣṇa has uttered, has a deeper meaning in your life. It’s not outside of your life. He didn’t say something alien to you. Actually, what He has spoken 5000 years ago is still relevant to now. If you see what Arjuna went through, everybody goes through the same thing. That’s why you have to dive into it. Read it one time, two times, three times, hundreds of times. Become a living and walking \textit{Bhagavad Gītā}.}