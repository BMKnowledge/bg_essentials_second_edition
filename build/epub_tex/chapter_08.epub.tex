\chapter{Akṣara-brahma-yoga}\label{chap-aksara-brahma-yoga}
\chaptersubtitle{The Eternal Abode}
%%%%%%%%%%%%%%%%%%%%%%%%%%%%%%%%%%%%%%%%%%%%%%%%%%%%%%%%%

\noindent Kṛṣṇa begins by clarifying terms He has used at the end of the previous chapter, and in response to Arjuna’s question, explains how He can be known at the time of death. It is our last thought before leaving the body that decides our destination. But this thought is a product of our state of consciousness and therefore Kṛṣṇa urges Arjuna to incessantly fix his mind upon Him. Following on from chapter six, Kṛṣṇa further describes the technique on how to achieve this. By infusing devotion (\textit{bhakti}) into the practice of yoga given by Kṛṣṇa thus far, one can control the senses and life breath to reach the Supreme.

This chapter then outlines the cosmology of the universe. Kṛṣṇa states that each life in this world is ultimately suffering since nothing we experience is permanent. The material plane is made up of many realms, the highest of which is where Brahmā\footnote{The deity born from the Supreme Lord responsible for creating within the physical universe.} resides. But even there one is subject to constant birth and death. Each of \textit{Brahmā’s} Days and Nights lasts thousands of ages (\textit{yugas}), and every time we are helplessly brought out into existence and withdrawn into a dormant state. But beyond his abode there is another realm that is beyond this cosmic cycle and does not decay. Through pure devotion one can reach this highest Abode which is none other than Kṛṣṇa. Hence, Kṛṣṇa's Abode is not the place where Kṛṣṇa resides, but is Kṛṣṇa Himself. This realization is considered to be the highest realization one can attain in life, and is also referred to as God-realization.

Those who know the \textit{Vedas} understand that there are two times to leave the body: one is auspicious and leading to freedom from rebirth, the other brings one back to bondage. But the devotee who is fixed on Kṛṣṇa is not bewildered by these timings. While those who follow the rituals and practice of scripture gain a certain limited reward, the one who practices \textit{bhakti-yoga}\footnote{The path of loving devotion to God.} attains Kṛṣṇa Himself.

\section{Verses 1-4: Kṛṣṇa Defines Different Terms}\label{sec-verses-1-4-krsna-defines-different-terms}


\par\noindent\textbf{8.1}\par
\begin{verse}
\textit{arjuna uvāca} \\
\textit{kiṁ tad brahma kim adhyātmaṁ kiṁ karma puruṣottama} \\
\textit{adhibhūtaṁ ca kiṁ proktam adhidaivaṁ kim ucyate}

\noindent Arjuna said: What is that Brahman? What is \textit{adhyātma}? What is meant by \textit{karma}? What is meant by \textit{adhibhūta}? And what is said to be \textit{adhidaiva}, O best of men?
\end{verse}
\par



\par\noindent\textbf{8.2}\par
\begin{verse}
\textit{adhiyajñaḥ kathaṁ ko ’tra dehe ’smin madhusūdana} \\
\textit{prayāṇa-kāle ca kathaṁ jñeyo ’si niyatātmabhiḥ}

\noindent Who is the \textit{adhiyajña} and how does He exist here in this body, O Madhusūdana? And how are You to be known at the time of death by the self-controlled?
\end{verse}
\par



\par\noindent\textbf{8.3}\par
\begin{verse}
\textit{śrī-bhagavān uvāca} \\
\textit{akṣaraṁ brahma paramaṁ svabhāvo ’dhyātmam ucyate} \\
\textit{bhūta-bhāvodbhava-karo visargaḥ karma-saṁjñitaḥ}

\noindent Bhagavān Kṛṣṇa said: Brahman is the Supreme Imperishable, \textit{adhyātma} is said to be that which possesses individual existence, and the cosmic emanation which gives rise to material beings is known as \textit{karma}.
\end{verse}
\par



\par\noindent\textbf{8.4}\par
\begin{verse}
\textit{adhibhūtaṁ kṣaro bhāvaḥ puruṣaś cādhidaivatam} \\
\textit{adhiyajño ’ham evātra dehe deha-bhṛtāṁ vara}

\noindent O best of embodied beings, \textit{adhibhūta} refers to the perishable material existence, the \textit{adhidaivata} is the cosmic governing deity, and it is I who am the \textit{adhiyajña} present in the body.
\end{verse}
\par




\enquote{Lord Kṛṣṇa has already spoken about all these terms, but Arjuna is still trying to understand and his mind is getting confused. He says, \enquote{You are the Lord of everything. You surpass everything. It would be simpler if You would just tell me, ‘I am the Lord!’ I get confused with all these terms.}}

\enquote{Arjuna didn’t really have a confused mind, but a mind that wanted to understand, a scientific, rational mind. But it was good that his mind was in such a state, otherwise he would not have kept Lord Kṛṣṇa talking, giving the \textit{Gītā}, to him and to humanity.}

\enquote{This indestructible Self, which is above all the material nature, is the higher nature itself. It is this divinity, \textit{Para-Brahman}, Kṛṣṇa Himself, who is the source and the core of the individual soul.}

\enquote{When this higher nature of God manifests and takes form in this lower reality, through His \textit{prakṛti}, through His will, it takes the aspects of the body, the senses, the mind, and the intellect, but it stays completely out of it. The \textit{ātmā} presides over the body, the mind, the senses, and the intellect.}

\enquote{When something is limited by matter or by existence and parts of it are taken away, little by little, it reaches a point when every part has been taken away and it’s finished! But no matter how many parts of the Lord come out of the Lord, He is still full. He doesn’t get diminished! He can manifest Himself many times in this world, yet there’s no limit to Him.}

\enquote{Lord Kṛṣṇa declares that the supreme reality, the \textit{jīvātmā},\footnote{Term interchangeable with \enquote{\textit{jīva},} which refers to the embodied Self.} and the lower material creation, are three different aspects of the Lord. Even if due to ignorance, the lower material nature affects the body, the mind and the intellect, it doesn’t touch the individual soul. Due to ignorance, the people who dwell in the worldly, materialistic reality, completely focused on the outer world, completely focused on the negative qualities, can’t perceive the soul, the \textit{ātmā}, inside themselves.}

\enquote{On the other hand, people who are on the spiritual path, who are dedicated to their \textit{sādhana}, who are calm and stable, can perceive and feel the \textit{ātmā}. Through the knowledge of the Self, through meditating on the \textit{Śrīmad Bhagavad Gītā}, and putting it into practice, one takes the path of \textit{bhakti}. One becomes aware of the \textit{ātmā}. To become spiritual means that you have the awareness that you are not only this body, you are not only this mind: you realize that you are the individual soul, and in that state, you praise the Lord and love the Lord. And then you ascend to such a degree that you perceive the Lord continuously within yourself and have the great joy, the bliss, to serve Him.}

\enquote{Kṛṣṇa also says that beyond the \textit{ātmā} is \enquote{the supreme, indestructible Self (\textit{akṣara}).} The ignorant one can’t perceive the Self, but the Self is ever-present. Sometimes It reveals itself through a certain feeling, through a certain awareness. In the same way, the Supreme soul reveals Himself to the \textit{bhakta} as the Supersoul, the highest of all.}

\enquote{When the mind unites with the heart, something else comes out. When the mind is purified to become intellect, that intellect enters the heart. What awakens is the consciousness, the consciousness of the soul\ldots When the consciousness is purified, it reveals the soul. But when that soul is purified, it reveals the Supersoul, the Supreme soul.}

\enquote{To those who are fully absorbed into the divine bliss, the Lord reveals Himself. He reveals there is no difference between Him and the individual soul. It is the same quality, the only difference is quantity. The Light is the same, but the intensity of that Light differs. When you have a light, you can look at it. But the Light of the Lord you can’t look at. The spark taken from the fire you can look at, you can probably put it out with your hand also. But if you have a huge fire and you put your hand in it, you will get burned. Like this, you have different degrees of burning and you have different degrees of light.}

\section{Verses 5-16: Fixing the Mind on Kṛṣṇa at Death}\label{sec-verses-5-16-fixing-the-mind-on-krsna-at-death}


\par\noindent\textbf{8.5}\par
\begin{verse}
\textit{anta-kāle ca mām eva smaran muktvā kalevaram} \\
\textit{yaḥ prayāti sa mad-bhāvaṁ yāti nāsty atra saṁśayaḥ}

\noindent And the one who, at the time of death remembers Me alone, reaches My state of being after giving up the body. In this regard, there is no doubt.
\end{verse}
\par



\par\noindent\textbf{8.6}\par
\begin{verse}
\textit{yaṁ yaṁ vāpi smaran bhāvaṁ tyajaty ante kalevaram} \\
\textit{taṁ tam evaiti kaunteya sadā tad-bhāva-bhāvitaḥ}

\noindent O son of Kuntī, whatever state of being one remembers at the time of death while leaving the body, that very state one attains thereafter, having always been absorbed in it.
\end{verse}
\par



\par\noindent\textbf{8.7}\par
\begin{verse}
\textit{tasmāt sarveṣu kāleṣu mām anusmara yudhya ca} \\
\textit{mayy arpita-mano-buddhir mām evaiṣyasy asaṁśayaḥ}

\noindent Therefore, always remember Me and fight. With your mind and intellect fixed in Me, without doubt you will certainly attain Me.
\end{verse}
\par



\par\noindent\textbf{8.8}\par
\begin{verse}
\textit{abhyāsa-yoga-yuktena cetasā nānya-gāminā} \\
\textit{paramaṁ puruṣaṁ divyaṁ yāti pārthānucintayan}

\noindent O Pārtha! Constantly being aware of the divine Supreme Lord by engaging in repeated practice of \textit{yoga} with a mind that doesn't wander anywhere else, one attains Him.
\end{verse}
\par



\par\noindent\textbf{8.9–10}\par
\begin{verse}
\textit{kaviṁ purāṇam anuśāsitāram aṇor aṇīyāṁsam anusmared yaḥ} \\
\textit{sarvasya dhātāram acintya-rūpam āditya-varṇaṁ tamasaḥ parastāt} \\
\textit{prayāṇa-kāle manasācalena bhaktyā yukto yoga-balena caiva} \\
\textit{bhruvor madhye prāṇam āveśya samyak sa taṁ paraṁ puruṣam upaiti divyam}

\noindent One who is absorbed in devotion and at the time of death meditates with an unwavering mind on the most ancient One, the all-knowing supreme controller whose form is inconceivable, the One who is more subtle than an atom, the maintainer of all, who is beyond darkness and brilliant like the sun, while firmly keeping the \textit{prāṇa} between the eyebrows by the power of \textit{yoga}, such a person attains the Divine Supreme Being.
\end{verse}
\par



\par\noindent\textbf{8.11}\par
\begin{verse}
\textit{yad akṣaraṁ veda-vido vadanti viśanti yad yatayo vīta-rāgāḥ} \\
\textit{yad icchanto brahmacaryaṁ caranti tat te padaṁ saṅgraheṇa pravakṣye}

\noindent I shall now tell you about that goal which the knowers of the \textit{Vedas} call the Imperishable, which ascetics who are free from desire attain, and desiring which, they practice celibacy.
\end{verse}
\par



\par\noindent\textbf{8.12–13}\par
\begin{verse}
\textit{sarva-dvārāṇi saṁyamya mano hṛdi nirudhya ca} \\
\textit{mūrdhny ādhāyātmanaḥ prāṇam āsthito yoga-dhāraṇām} \\
\textit{oṁ ity ekākṣaraṁ brahma vyāharan mām anusmaran} \\
\textit{yaḥ prayāti tyajan dehaṁ sa yāti paramāṁ gatim}

\noindent Having controlled all the gates of the body, withdrawing the mind into the heart and placing the \textit{prāṇa} within the head, fixed in \textit{yogic} concentration, chanting the sacred syllable \textit{oṁ} which embodies the Absolute and thinking of Me intensely, one who abandons the body and leaves the world in this way, reaches the supreme goal.
\end{verse}
\par



\par\noindent\textbf{8.14}\par
\begin{verse}
\textit{ananya-cetāḥ satataṁ yo māṁ smarati nityaśaḥ} \\
\textit{tasyāhaṁ sulabhaḥ pārtha nitya-yuktasya yoginaḥ}

\noindent O Pārtha, for the ever-united \textit{yogī} who is always single-minded and continuously remembers Me, I am easily attainable.
\end{verse}
\par



\par\noindent\textbf{8.15}\par
\begin{verse}
\textit{mām upetya punar janma duḥkhālayam aśāśvatam} \\
\textit{nāpnuvanti mahātmānaḥ saṁsiddhiṁ paramāṁ gatāḥ}

\noindent Upon attaining Me, the great ones do not again take a transient birth, which is the abode of suffering, since they have reached the highest perfection.
\end{verse}
\par



\par\noindent\textbf{8.16}\par
\begin{verse}
\textit{ā-brahma-bhuvanāl lokāḥ punar āvartino ’rjuna} \\
\textit{mām upetya tu kaunteya punar janma na vidyate}

\noindent O Arjuna, all realms up to the world of Brahmā are subject to rebirth. But after attaining Me, there is no further rebirth, O son of Kuntī.
\end{verse}
\par


\enquote{Among all the human beings, how many people die with an attitude of leaving the body to attain the Lord? Only very few. After passing through 8.4 million births, and manifesting through the entire chain of creation, the individual soul attains a human body. It’s a very rare opportunity to attain the level of a human being who can then strive to attain the Lord. One has to grasp such an opportunity.}

\enquote{Bhagavān Kṛṣṇa says to Arjuna, \enquote{Fight! But keep your mind focused on Me.} This means that one must do one’s duty in life. Wherever you are, whatever you are doing in this world, God has placed you there for a reason. So accept it and do your action properly! However, your mind should be fully focused on Him and not only on the outside world.}

\enquote{It’s very important to control how you breathe. Through control, through \textit{prāṇāyāma}, one learns how to be centered inwardly. People who live a hectic life in the outside world also breathe in a hectic way. They are not just working hard, but they are draining themselves severely\ldots [they] very quickly have heart attacks and die young---the life force is finished.}

\enquote{Since time immemorial, the \textit{yogīs} have meditated on the \textit{oṁ} vibration, the primordial cosmic sound, the sound that contains everything and that is beyond the creation, preservation, and destruction of the universe. Lord Kṛṣṇa Himself says, \enquote{I am \textit{oṁ}.} When the \textit{bhaktas} are fully absorbed in the divine sound \enquote{\textit{oṁ}} they become fully immersed in it and go deeper and deeper. At that moment, everything becomes still, the body is still, the mind is still, the consciousness is still: only the cosmic sound, \textit{oṁ}, is vibrating in the third eye. One loses oneself, one loses consciousness of the body, and one perceives the Lord. At that point, it’s not only a vibration: the Supreme Lord Nārāyaṇa Himself takes a certain aspect. He manifests Himself, reveals Himself in one’s consciousness. The one who has attained that state, \enquote{the highest status,} becomes a true \textit{bhakta}, who has realized the Absolute.}

\enquote{Once one has attained that \enquote{highest status,} even if one dwells in the outside world, engaging oneself in many activities, one’s mind, one’s consciousness, is never deviated. The Divine Himself is crystallized into one’s mind and then one is free. Whereas the minds of people who have not realized that state, are always busy running towards the outside world. The outside reality is crystallized into their minds and they can never be free. Therefore, \textit{japa-kriyā}, the continuous repetition of the Divine Names, is very important\ldots This leads one to God-realization; not just to Self-realization, but to God-realization. Then one is free from the cycle of birth and death. One perceives the supreme reality, not only in oneself, but everywhere.}

\enquote{He reveals Himself to the one who aims for Self-realization, and to the one who aims to reach even beyond Self-realization, who aims for God consciousness. In Self-realization, the Self is revealed to the intellect and the mind, but higher than Self-realization is God-realization. Once the Lord has revealed Himself within the core of the Self, one is fully absorbed in His Love and attains Him, the Supreme Puruṣa.}

\enquote{When you have this Love for God, He limits Himself and takes an aspect, a form, because as you long for Him, He also longs for you. This is how God manifests in a certain form, making it easy for you to come to Him.}

\enquote{It’s very difficult for those who dwell in the outside world, who have no faith or love, no knowledge of the glory of God, to attain Him. Whereas it becomes easy for a \textit{bhakta}. When people reach the state of entering the heart, what they perceive through that Love miraculously changes their lives. Due to their readiness, they are brought to the spiritual path, they are brought to the spiritual master, so that they can drink the divine nectar, grow, and perfect themselves. As they have a longing for God, God also has a similar longing for them; the Lord gathers all the people who have this same yearning and brings them together. Then they have the privilege of being in touch with saints and God-realized souls.}

\section{Verses 17-28: Breaking Free of Brahmā's Creation}\label{sec-verses-17-28-breaking-free-of-brahma-s-creation}

\Verse[8.17]
{sahasra-yuga-paryantam ahar yad brahmaṇo viduḥ \\
rātriṁ yuga-sahasrāntāṁ te ’ho-rātra-vido janāḥ}
{Those who understand that a Day of Brahmā lasts for a thousand \textit{mahā-yugas}\footnote{Defined as an age or epoch in Hindu cosmology. There are four distinct yugas: \textit{Satya-yuga}, \textit{Tretā-yuga}, \textit{Dvāpara-yuga}, and \textit{Kali-yuga}, each period being shorter, darker, and less righteous than the preceding. Mankind is currently in the age of \textit{Kali-yuga}. In this verse, Kṛṣṇa states that both a Day and a Night of Brahmā lasts a \enquote{thousand \textit{Mahā-yugas}} which is actually a thousand cycles of the four \textit{yugas} mentioned, a period that lasts trillions of years.} and a Night of Brahmā lasts for another thousand \textit{mahā-yugas}, are the knowers of a cosmic Day and Night.}


\par\noindent\textbf{8.18}\par
\begin{verse}
\textit{avyaktād vyaktayaḥ sarvāḥ prabhavanty ahar-āgame} \\
\textit{rātry-āgame pralīyante tatraivāvyakta-saṁjñake}

\noindent At the beginning of a Day of Brahmā all beings come forth from the unmanifest, and at the dawn of Brahmā's Night, they are dissolved back into the state known as the unmanifest.
\end{verse}
\par



\par\noindent\textbf{8.19}\par
\begin{verse}
\textit{bhūta-grāmaḥ sa evāyaṁ bhūtvā bhūtvā pralīyate} \\
\textit{rātry-āgame ’vaśaḥ pārtha prabhavaty ahar-āgame}

\noindent Having come into manifestation again and again, the same multitude of beings is helplessly withdrawn at the coming of the Night. At the dawn of the next Day, yet again they manifest, O Pārtha.
\end{verse}
\par



\par\noindent\textbf{8.20}\par
\begin{verse}
\textit{paras tasmāt tu bhāvo ’nyo ’vyakto ’vyaktāt sanātanaḥ} \\
\textit{yaḥ sa sarveṣu bhūteṣu naśyatsu na vinaśyati}

\noindent However, there is an eternal unmanifest existence which is different from and superior to this unmanifest state of \textit{prakṛti}; it is not destroyed even when all material beings are annihilated.
\end{verse}
\par



\par\noindent\textbf{8.21}\par
\begin{verse}
\textit{avyakto ’kṣara ity uktas tam āhuḥ paramāṁ gatim} \\
\textit{yaṁ prāpya na nivartante tad dhāma paramaṁ mama}

\noindent This unmanifest realm, also called \enquote{The Imperishable,} is said to be the highest goal; it is My Supreme Abode, having attained which one does not return.
\end{verse}
\par



\par\noindent\textbf{8.22}\par
\begin{verse}
\textit{puruṣaḥ sa paraḥ pārtha bhaktyā labhyas tv ananyayā} \\
\textit{yasyāntaḥ-sthāni bhūtāni yena sarvam idaṁ tatam}

\noindent This Supreme Being within whom all beings exist and who pervades this entire universe, is however only attained by exclusive devotion, O Pārtha.
\end{verse}
\par



\par\noindent\textbf{8.23}\par
\begin{verse}
\textit{yatra kāle tv anāvṛttim āvṛttiṁ caiva yoginaḥ} \\
\textit{prayātā yānti taṁ kālaṁ vakṣyāmi bharatarṣabha}

\noindent O best of the Bharatas, I will now describe to you the specific moments of leaving this world, during which \textit{yogīs} either return to this world or don't.
\end{verse}
\par



\par\noindent\textbf{8.24}\par
\begin{verse}
\textit{agnir jyotir ahaḥ śuklaḥ ṣaṇ-māsā uttarāyaṇam} \\
\textit{tatra prayātā gacchanti brahma brahma-vido janāḥ}

\noindent Those who have realized Brahman attain the Supreme, while leaving this world during the influence of Agni, the light, the day, the bright fortnight, or during the six months of the northern course of the sun.
\end{verse}
\par


\Verse[8.25]
{dhūmo rātris tathā kṛṣṇaḥ ṣaṇ-māsā dakṣiṇāyanam \\
tatra cāndramasaṁ jyotir yogī prāpya nivartate}
{But the \textit{yogī} who departs during the smoke, the night, the dark fortnight, or the six months of the southern course of the sun, reaches the light of the moon and then returns.\footnote{Verses 24-25 make reference to the \textit{Upaniṣads} which describe different ways of leaving the body. The auspicious six months during the light are known as Uttarāyaṇa, and the inauspicious months are known as Dakṣiṇāyana. Both are recognized in the Hindu calendar today.}}


\par\noindent\textbf{8.26}\par
\begin{verse}
\textit{śukla-kṛṣṇe gatī hy ete jagataḥ śāśvate mate} \\
\textit{ekayā yāty anāvṛttim anyayāvartate punaḥ}

\noindent These two paths of the world, the light and the dark, are considered to be eternal. By following one of these paths, one does not come back, but by following the other, one returns.
\end{verse}
\par



\par\noindent\textbf{8.27}\par
\begin{verse}
\textit{naite sṛtī pārtha jānan yogī muhyati kaścana} \\
\textit{tasmāt sarveṣu kāleṣu yoga-yukto bhavārjuna}

\noindent Pārtha, the \textit{yogī} who knows these two paths is never bewildered. Therefore, at all times be engaged in \textit{yoga}, O Arjuna.
\end{verse}
\par



\par\noindent\textbf{8.28}\par
\begin{verse}
\textit{vedeṣu yajñeṣu tapaḥsu caiva dāneṣu yat puṇya-phalaṁ pradiṣṭam} \\
\textit{atyeti tat sarvam idaṁ viditvā yogī paraṁ sthānam upaiti cādyam}

\noindent Whatever merits are gathered from the study of the \textit{Vedas}, the performance of sacrifices, or the practice of austerities and charity, the \textit{yogī} who knows this teaching of Mine crosses beyond all of it and reaches the original, Supreme Abode.
\end{verse}
\par


\enquote{During the Day and Night of Brahmā, everything is created and everything is destroyed. This also represents the ignorance in man: even if one has this deep, inner wisdom anchored inside, when one has to make a choice between right and wrong, the mind doesn’t see clearly enough to make the right decision. Even if one has the feeling, \enquote{If I do that, I will suffer. God is guiding me on the way, I know that I have to realize my Self,} when one acts, one always does something to make oneself suffer. So one falls again and again into the darkness.}

\enquote{It is said that 100 years in the life of Brahmā is equivalent to one breath of Mahā-Viṣṇu.\footnote{A cosmic manifestation of the Supreme Lord Nārāyaṇa in which He classically holds the conch, discus, mace, and lotus in each of His four hands.} So while Mahā-Viṣṇu is inhaling one time, Brahmā disappears. This means that in just one breath of Mahā-Viṣṇu, the entire collection of countless universes are annihilated and enter into the body of Mahā-Viṣṇu. When Mahā-Viṣṇu exhales, creation happens again. And that takes 311 trillion years!}

\enquote{How easy it is to attain the Lord! You don’t have to go through all these billions of years. You don’t even have to go through one inhaling and exhaling of Mahā-Viṣṇu. You don’t have to go through all this, you can just be a \textit{bhakta} and enjoy His Love. Whether He breathes or not, you don’t have to bother about it.}

\enquote{The ones who are constantly immersed in His Love, the Divine Love within their own Self, realize the essence of God. They realize that God is not just the One they pray to, outside themselves, but He is the essence of their \textit{bhakti}, the essence of what they want to achieve. They realize that He is actually inside them. Those \textit{bhaktas} who have developed such a devotion, who are absorbed in their worship, in their \textit{sādhana}, in their chanting, will quickly reach the state of God-realization.}

\enquote{In such a state of \textit{bhakti}, of deep, supreme devotion, one must long for the Lord like a ship which is out of the water longs for water, or like a man who is drowning longs for air. Kṛṣṇa says here that He gives Himself only to a \textit{bhakta} who has such a longing for realization, who has such a longing for Him. So if you want to reach God-realization, you have to be very dedicated, you can’t be superficial, you can’t be insincere, or just a part-time devotee on the spiritual path. To attain the Lord, you have to do everything for the sake of developing that supreme devotion.}

\enquote{Death doesn’t have any effect on the ones who are completely surrendered to Him [Kṛṣṇa]. Not even \textit{Yamarāja}, the lord of death, has power over the \textit{bhaktas} of the Lord. They leave this world when He wills them to leave and they come back again when He wills them to come back. A \textit{yogī}, a \textit{bhakta} is fully absorbed into God consciousness, so he will not stay here forever.}

\enquote{The \textit{bhaktas} who are surrendered, who have not fully realized themselves before departing this world, go to higher \textit{lokas} to perfect themselves and then they come back again. And when they come back again, they come for the benefit of others. Kṛṣṇa says, \enquote{Because they are fully surrendered, I Myself send them down.} The saints, who are born again on Earth for the welfare of society, come to remind others about the goal of life.}
