\chapter{Śraddhā-traya-vibhāga-yoga}\label{chap-sraddha-traya-vibhaga-yoga}
\chaptersubtitle{The Three Divisions of Faith}
%%%%%%%%%%%%%%%%%%%%%%%%%%%%%%%%%%%%%%%%%%%%%%%%%%%%%%%%%

\noindent In the previous chapter, Kṛṣṇa warns Arjuna about abandoning the rules of scripture, for it is through scripture that one can understand one’s duty. Prompted by Kṛṣṇa's statements on the authority of scripture, Arjuna asks about those who act with faith but ignore scriptural rules. Kṛṣṇa replies that our faith is not necessarily reliable since it is dictated by the three \textit{guṇas}. This answer initiates a new line of discourse where different aspects of life are analyzed through the lens of \textit{sattva}, \textit{rajas} and \textit{tamas}.

Kṛṣṇa begins by discussing the different types of food. Then He explains the various ways of performing ritual (\textit{yajña}). He describes the austerity (\textit{tapas}) of the body, speech and mind and how these also fall under the influence of the \textit{guṇas}. Then finally, the different ways of giving charity (\textit{dāna}) are outlined.

Kṛṣṇa stresses that these actions (\textit{yajña}, \textit{tapas}, \textit{dāna}) when done under the influence of \textit{sattva-guṇa}, are performed with detachment, free from any desire.

In the closing few verses, the Lord speaks about the sacred words \textit{oṁ tat sat}. This \textit{mantra} has been chanted during the \textit{Vedic} sacrifice since ancient times. Each syllable and its relevance to ritual is explained. But simply following rules is not enough. Kṛṣṇa emphasizes once again the importance of sincere faith when carrying out any action such as ritual, charity or austerity. Without it, one cannot obtain any benefit in this world or the next.

\section{Verses 1-6: The Faith and the Guṇas}\label{sec-verses-1-6-the-faith-and-the-gunas}


\par\noindent\textbf{17.1}\par
\begin{verse}
\textit{arjuna uvāca} \\
\textit{ye śāstra-vidhim utsṛjya yajante śraddhayānvitāḥ} \\
\textit{teṣāṁ niṣṭhā tu kā kṛṣṇa sattvam āho rajas tamaḥ}

\noindent Arjuna said: What is the standing of those who disregard the rules of scripture yet worship with faith, O Kṛṣṇa? Is it in \textit{sattva}, \textit{rajas}, or \textit{tamas}?
\end{verse}
\par



\par\noindent\textbf{17.2}\par
\begin{verse}
\textit{śrī-bhagavān uvāca} \\
\textit{tri-vidhā bhavati śraddhā dehināṁ sā svabhāva-jā} \\
\textit{sāttvikī rājasī caiva tāmasī ceti tāṁ śṛṇu}

\noindent Lord Kṛṣṇa said: The faith of embodied beings, arising from their inherent nature, is of three types, namely \textit{sattvic}, \textit{rajasic}, and \textit{tamasic}. Hear My description of them.
\end{verse}
\par



\par\noindent\textbf{17.3}\par
\begin{verse}
\textit{sattvānurūpā sarvasya śraddhā bhavati bhārata} \\
\textit{śraddhā-mayo ’yaṁ puruṣo yo yac-chraddhaḥ sa eva saḥ}

\noindent O descendant of Bharata, everyone's faith is in accordance with their being. A person is made of faith---whatever their faith, that is what they become.
\end{verse}
\par



\par\noindent\textbf{17.4}\par
\begin{verse}
\textit{yajante sāttvikā devān yakṣa-rakṣāṁsi rājasāḥ} \\
\textit{pretān bhūta-gaṇāṁś cānye yajante tāmasā janāḥ}

\noindent Those of \textit{sattvic} nature worship the \textit{devas}. Those of \textit{rajasic} nature worship Yakṣas and Rākṣasas, while others, those persons with a \textit{tamasic} nature, worship ghosts and spirits.
\end{verse}
\par



\par\noindent\textbf{17.5–6}\par
\begin{verse}
\textit{aśāstra-vihitaṁ ghoraṁ tapyante ye tapo janāḥ} \\
\textit{dambhāhaṅkāra-saṁyuktāḥ kāma-rāga-balānvitāḥ} \\
\textit{karṣayantaḥ śarīra-sthaṁ bhūta-grāmam acetasaḥ} \\
\textit{māṁ caivāntaḥ śarīra-sthaṁ tān viddhy āsura-niścayān}

\noindent Those who practice severe austerities not ordained by scripture, filled with hypocrisy and egoism, driven by desire and attachment, and lacking wisdom---they torture the elements within their body as well as Me, who dwells within the body. Know such people have a demonic resolve.
\end{verse}
\par


\enquote{Like Bhagavān has explained, there are two kinds of people: people who have divine qualities and people who have demonic qualities. Now, Bhagavān continues explaining how one achieves those qualities. Here you will see that it’s very much about how our faith and how the food that we eat affects ourselves and our minds. The surroundings, the companionship, and with whom you associate is very important. In this chapter Bhagavān explains the three types of faith. People who have \textit{sattvic} qualities, people who have \textit{rajasic} qualities and people who have \textit{tamasic} qualities.}

\enquote{\textit{Bhaktas} with \textit{sattvic} qualities turn their minds towards godly things, towards the Divine, towards enlightenment, towards awakening. They worship the Lord with faith, with love and dedication. All they do is for the Lord, nothing is for their self-interest or gain.}

\enquote{They don’t build their friendship, relationship or prayer expecting something in return. They will not even ask God, \enquote{Lord, give me this,} or, \enquote{Give me that, Lord, I love You, reveal Yourself to me.} As if God doesn’t know when one is ready. They will not even say, \enquote{God, give me Your Love.} Those devotees who are surrendered to God, pray only for the sake and pleasure of the Lord Himself.}

\enquote{Those who are \textit{rajasic} and who pray to the demigods, pray because they are greedy. Due to their pride and arrogance, due to always wanting more and more, they worship and praise people, always befriending people for their own selfish motives. They are always trying to gain something from people.}

\enquote{The ones with \textit{tamasic} qualities pray to the ghosts. They like to abuse the power of nature. They utilize the force of nature to do their own dirty job through black magic, through voodoo and working with spirits of lower forces\ldots When they die, they also become like the ones they have been worshiping. Because their mind is fully absorbed into that, they have conditioned themselves and programmed themselves so that they are in darkness continuously due to their aspect of always wanting to harm others. So, automatically that becomes their nature. And when they die, they also become like the ones they have been worshiping. And if they come on Earth, they become wicked people.}

\section{Verses 7-10: Types of Food and the Guṇas}\label{sec-verses-7-10-types-of-food-and-the-gunas}


\par\noindent\textbf{17.7}\par
\begin{verse}
\textit{āhāras tv api sarvasya tri-vidho bhavati priyaḥ} \\
\textit{yajñas tapas tathā dānaṁ teṣāṁ bhedam imaṁ śṛṇu}

\noindent The food that is dear to all is also of three types, as are sacrifice, austerity, and charity. Listen to the distinction between them.
\end{verse}
\par



\par\noindent\textbf{17.8}\par
\begin{verse}
\textit{āyuḥ-sattva-balārogya-sukha-prīti-vivardhanāḥ} \\
\textit{rasyāḥ snigdhāḥ sthirā hṛdyā āhārāḥ sāttvika-priyāḥ}

\noindent Foods that increase longevity, alertness, strength, health, happiness, and satisfaction, that are tasty, smooth, crisp, and pleasant, are dear to those of \textit{sattvic} temperament.
\end{verse}
\par



\par\noindent\textbf{17.9}\par
\begin{verse}
\textit{kaṭv-amla-lavaṇāty-uṣṇa-tīkṣṇa-rūkṣa-vidāhinaḥ} \\
\textit{āhārā rājasasyeṣṭā duḥkha-śokāmaya-pradāḥ}

\noindent Foods that are bitter, sour, very salty, very hot, pungent, dry, and burning, and that produce pain, sorrow, and disease, are liked by those of \textit{rajasic} temperament.
\end{verse}
\par



\par\noindent\textbf{17.10}\par
\begin{verse}
\textit{yāta-yāmaṁ gata-rasaṁ pūti paryuṣitaṁ ca yat} \\
\textit{ucchiṣṭam api cāmedhyaṁ bhojanaṁ tāmasa-priyam}

\noindent Food that is stale, tasteless, foul, putrid, leftover, and impure is dear to those of \textit{tamasic} temperament.
\end{verse}
\par


\enquote{The effect of the \textit{guṇas} is also present in the food that we take. So, what we eat is very important, because the food that we eat acts on the brain. And you see that in life itself. People who are aggressive and short-tempered have so many \textit{tamasic} qualities. The meat eaters are always nervous and aggressive because, although you can boil the meat, can you transform the energy in it? You can’t. You can cook it, but the energy doesn’t change while cooking. The sadness of the animal, the \textit{karma} of that animal from previous lives, and also what they did to be incarnated as an animal is also put into that.}

\enquote{Bhagavān gives these explanations to Arjuna, even though Arjuna has not mentioned food. He hasn’t asked which kinds of food we should eat. Bhagavān explains that, for faith, it is also important what one puts into one’s body. Besides worshiping the Lord, having the faith in the practices that one is doing, and the \textit{sādhana} that one is doing, food is also very important. The \textit{sādhana} also depends on the food. If you sit in your \textit{sādhana} and you have heavy things inside of you, it pulls you down. You can’t sit meditating, you feel drowsy.\enquote{}}

\enquote{There is a saying that says, \enquote{By looking at the food people eat, you can tell how they are.} Just by the food. Because people eat certain kinds of food according to their tendencies. Taste is the tendency, but in reality what is behind that taste will also determine the type of food people eat. So, somebody who has \textit{sattvic} qualities will be eating vegetarian food. That helps one’s mind to be more calm and peaceful, and so it also helps one on the spiritual path.}

\enquote{\textit{Sattvic} qualities act on the intellect, they make the intellect pure, transparent, accurate. It gives one the mental stability so that one doesn’t fall into depression. When the mind is stable, you are strong in your will. So, it gives you both inner strength and outer strength. Inner strength is the will. You don’t get unbalanced. However, you also become stronger in the body, because \textit{sattvic} food builds up the immune system in the body.}

\section{Verses 11-13: Types of Sacrifice and the Guṇas}\label{sec-verses-11-13-types-of-sacrifice-and-the-gunas}


\par\noindent\textbf{17.11}\par
\begin{verse}
\textit{aphalākāṅkṣibhir yajño vidhi-diṣṭo ya ijyate} \\
\textit{yaṣṭavyam eveti manaḥ samādhāya sa sāttvikaḥ}

\noindent Sacrifice performed as prescribed by scripture by those without any desire for reward, who have mental resolve that it is to be performed out of duty alone, is \textit{sattvic} in nature.
\end{verse}
\par




\par\noindent\textbf{17.12}\par
\begin{verse}
\textit{abhisandhāya tu phalaṁ dambhārtham api caiva yat} \\
\textit{ijyate bharata-śreṣṭha taṁ yajñaṁ viddhi rājasam}

\noindent But know that sacrifice performed with an aim for the reward and with the aim of deceiving is \textit{rajasic} in nature, O best of the Bharatas.
\end{verse}
\par



\par\noindent\textbf{17.13}\par
\begin{verse}
\textit{vidhi-hīnam asṛṣṭānnaṁ mantra-hīnam adakṣiṇam} \\
\textit{śraddhā-virahitaṁ yajñaṁ tāmasaṁ paricakṣate}

\noindent Sacrifice not based on scriptural authority, without the offering of food, without recitation of \textit{mantras}, lacking faith, and devoid of gifts to \textit{brāhmaṇas}, is considered to be \textit{tamasic} in nature.
\end{verse}
\par


\enquote{People who have \textit{sattvic} qualities, the devotees, they give from the heart. Whereas, somebody who has \textit{tamasic} qualities, whatever they give has this arrogance inside of it: \enquote{I have done this, I have done that. I have given this.} These are hypocrites; they are full of ego.}

\enquote{There are different kind of sacrifices. Charity is even considered a sacrifice like the \textit{yajña}, the fire ceremony. When you do a \textit{yajña}, it is a sacrifice. When you do a prayer, it is a sacrifice. When you chant the Name of the Lord, it is a sacrifice. Here Bhagavān says that whatever kind of sacrifice is offered by someone \enquote{\ldots without desire for personal rewards, which is executed according to the right principle of the \textit{śāstras}, and with concentrated mind, that is \textit{sattvic}.}}

\enquote{The one who knows the true knowledge of the Self and what one wants to achieve---the Lord Himself---surrenders; one wants to achieve this Love for God. Because, like I said, in the \textit{śāstras} it’s only about that: how to attain the Love of God. Nothing else! The ones who have this deep longing inside of them, whatever they do in life they will do only to please the Lord.}

\enquote{Of course, you have to work. You have to do your duty. Bhagavān has not said that you have to run away from your duty. But with what mindset are you doing it? This is very important. With what aim are you doing it? This is also very important. So here He also defines the mindset of what you do. If it is \textit{sattvic}, then everything you do will bring you closer and closer to your inner Self. You will be closer and closer to the supreme reality. Enlightenment! Then, you are free.}

\enquote{When one is doing something for personal gain like, \enquote{O God, please give me a car!} \enquote{O, God, please give me a husband!} \enquote{God, please give me a wife!} \enquote{God, give me wealth!} \enquote{O God, give me a house!} \enquote{O God, give me fame, give me prestige, give me victory, give me health, give me bliss, give me joy,} it’s only about \enquote{give me, give me, give me, give me.} So, Bhagavān says that this falls into the \textit{rajasic} qualities which are always connected to personal gain and motive. So, one is never free.}

\enquote{So, it’s also very important how we pray and what we are praying for. Praying is not to ask and ask and ask, because this is the mind. I remember that once I told someone, \enquote{Pray to God!} Then the person said, \enquote{But, I don’t need to ask Him anything.} No. God doesn’t want anything from you, and also truly inside of yourself you don’t want anything from Him except Himself. And for that, even if you ask Him or not, He will give Himself to you the moment you long for Him. Longing is not about asking. It’s a feeling which awakens inside, which draws Him. It’s a magnet. It is small, but it grows. The more there is Love and longing for Him, the more that magnet grows and becomes big. And wherever He is, He is attracted to you and comes to you. Imagine that this magnet is so big that even attracts the Lord who is in Heaven, and pulls Him towards you.}

\enquote{In the \textit{Vedas} there are certain \textit{mantras}: the hymns which bring one closer to the Lord. But, there are other rituals which are not in the Vedas, in the \textit{śāstras} and the holy scriptures, and they are used for personal gain. Bhagavān says that these rituals don’t reach the Lord. They reach only the Yakṣas, the demons and entities which fulfill only the lower wishes of people.}

\enquote{Bhagavān reminds us that one should balance what one receives. But these people do everything only due to expectations. When they do their sacrifices, they don’t even bother to feed anybody or to give anything to anyone. They don’t even offer \textit{prasāda}.}

\enquote{Here Bhagavān reminds you that when you receive, you should always give something in return, a \textit{dakṣiṇā}. This is very important\ldots People without faith don’t give \textit{dakṣiṇā}. They don’t have any gratitude. They do everything out of expectation; they lack reverence due to their arrogance, self-conceit, delusion, hypocrisy, egoism. They lack understanding. They lack knowledge. Bhagavān says that is \textit{tamasic}.}

\section{Verses 14-19: Types of Austerity and the Guṇas}\label{sec-verses-14-19-types-of-austerity-and-the-gunas}


\par\noindent\textbf{17.14}\par
\begin{verse}
\textit{deva-dvija-guru-prājña-pūjanaṁ śaucam ārjavam} \\
\textit{brahmacaryam ahiṁsā ca śārīraṁ tapa ucyate}

\noindent The worship of the \textit{devas}, \textit{brāhmaṇas}, teachers, and enlightened beings, as well as purity, honesty, celibacy, and nonviolence---this is said to be bodily austerity.
\end{verse}
\par



\par\noindent\textbf{17.15}\par
\begin{verse}
\textit{anudvega-karaṁ vākyaṁ satyaṁ priya-hitaṁ ca yat} \\
\textit{svādhyāyābhyasanaṁ caiva vāṅ-mayaṁ tapa ucyate}

\noindent Speech that does not cause distress to others, is truthful, pleasant, and beneficial, as well as the recitation of the \textit{Vedas}---this is said to be verbal austerity.
\end{verse}
\par



\par\noindent\textbf{17.16}\par
\begin{verse}
\textit{manaḥ-prasādaḥ saumyatvaṁ maunam ātma-vinigrahaḥ} \\
\textit{bhāva-saṁśuddhir ity etat tapo mānasam ucyate}

\noindent Calmness of mind, benevolence, silence, self-restraint, and purity of being---this is said to be mental austerity.
\end{verse}
\par



\par\noindent\textbf{17.17}\par
\begin{verse}
\textit{śraddhayā parayā taptaṁ tapas tat tri-vidhaṁ naraiḥ} \\
\textit{aphalākāṅkṣibhir yuktaiḥ sāttvikaṁ paricakṣate}

\noindent When practiced with supreme faith by those who are disciplined and don't desire any reward, this threefold austerity is considered \textit{sattvic} in nature.
\end{verse}
\par



\par\noindent\textbf{17.18}\par
\begin{verse}
\textit{satkāra-māna-pūjārthaṁ tapo dambhena caiva yat} \\
\textit{kriyate tad iha proktaṁ rājasaṁ calam adhruvam}

\noindent Austerity that is performed with hypocrisy, for the sake of honor, praise, and good reputation alone, is said to be \textit{rajasic}, unsteady, and impermanent in this world.
\end{verse}
\par



\par\noindent\textbf{17.19}\par
\begin{verse}
\textit{mūḍha-grāheṇātmano yat pīḍayā kriyate tapaḥ} \\
\textit{parasyotsādanārthaṁ vā tat tāmasam udāhṛtam}

\noindent Austerity practiced with a deluded notion, for the sake of torturing oneself or inflicting harm on others, is considered \textit{tamasic}.
\end{verse}
\par


\enquote{One has to be aware of the tongue, and control what one says, for not causing any harm. It is said that it’s easy if you beat somebody up; the wound will get healed and they will forget about it. But, the wound which you inflict to others caused by the tongue is more painful, because that wound hurts the heart directly, and when it hurts the heart, it hurts Bhagavān Himself.}

\enquote{Here Bhagavān talks about the discipline of the mind where all propensities, evil tendencies and evil qualities have been eliminated; all the prejudice, anger, greed, infatuation, lust, arrogance, jealousy and all other kinds of negative things have been removed from the mind. You don’t have any grudge towards anybody, you don’t have any enmity, envy or intolerance. Then one’s mind attains a calm, gentle state of silence and self-control.}

\enquote{Purifying everything that comes out of the mind happens through \textit{sādhana}. When you sit for meditation, entering the deep silence within yourself, it also reflects on the outside reality.}

\enquote{When one does something in a calm state and with a calm mind, it bears positiveness. But, if somebody does something in a restless way, in an agitated way, it always goes wrong. You see it in your life itself. When you take a decision very quickly, without even considering it, you always take the wrong decision. Even if the right decision was so clear in front of you, you blind yourself. The mind becomes blind so that you don’t even perceive. You can’t make the distinction between what is right and what is wrong. Then you always tend to take the wrong decision and afterwards you say, \enquote{Oh I knew it! I should have listened to the first intuitive feeling that arose.} But that’s how you have to learn to listen to your intuition. When the mind is calm and gentle, when the mind is silenced through meditation, when you are absorbed in that, you learn to control it and then you also learn to take the right decision. So, the mind is tamed. It’s not running very quickly.}

\enquote{People who are always looking on the outside, who are doing all these big, big shows---it’s only for a short-time glory, a short-time name, short-time fame. You see that in the world nowadays. One year you have a new actor or new actress who is famous. How long? They will be famous for three years and after that you don’t even know where they have gone. They have come with a big \enquote{Ta-Da!} They have gone and you never hear about them again. And then one day you say, \enquote{Is she still alive? Have you ever heard about her again?} No, you don’t know where she is. You don’t even know when she comes and when she goes.}

\enquote{These are \textit{rajasic} qualities and \textit{rajasic} qualities don’t last for long. The moment the beauty is gone, the fame, the name, then everything will disappear with it. Here, Bhagavān says, \enquote{Keep away from this kind of association also! Keep away from this kind of honor! Keep away from this kind of glory.} Because, if you invite all these qualities inside of you, you will have only short-term happiness. And a \textit{bhakta} doesn’t long for short-term happiness. A \textit{bhakta} longs for long-term happiness. And that long-term happiness is given to you only by God.}

\section{Verses 20-22: Types of Charity and the Guṇas}\label{sec-verses-20-22-types-of-charity-and-the-gunas}


\par\noindent\textbf{17.20}\par
\begin{verse}
\textit{dātavyam iti yad dānaṁ dīyate ’nupakāriṇe} \\
\textit{deśe kāle ca pātre ca tad dānaṁ sāttvikaṁ smṛtam}

\noindent Charity that is given for the sake of giving, to someone from whom no benefit is expected, and according to the proper place, time, and recipient, is said to be \textit{sattvic} in nature.
\end{verse}
\par



\par\noindent\textbf{17.21}\par
\begin{verse}
\textit{yat tu pratyupakārārthaṁ phalam uddiśya vā punaḥ} \\
\textit{dīyate ca parikliṣṭaṁ tad dānaṁ rājasaṁ smṛtam}

\noindent But charity that is given for the sake of receiving in return, or in hope of some other result, or done grudgingly, is considered \textit{rajasic}.
\end{verse}
\par



\par\noindent\textbf{17.22}\par
\begin{verse}
\textit{adeśa-kāle yad dānam apātrebhyaś ca dīyate} \\
\textit{asat-kṛtam avajñātaṁ tat tāmasam udāhṛtam}

\noindent Charity that is given at the wrong place and time, to unworthy recipients, with contempt and without respect, is considered \textit{tamasic}.
\end{verse}
\par


\enquote{\enquote{Charity that is done from a sense of duty,} here Bhagavān refers to the ones who do charity, like helping people, with the true sense of serving, which is not just for a show, but truly from the heart. It is when you give and forget about it, not even expecting anything, any gratification nor even a \enquote{thank you.} This is the sense of duty. People with \textit{sattvic} qualities don’t even expect the blessing of the Lord.}

\enquote{Whoever needs help and goes to a person who is \textit{sattvic} is welcomed at any time. And here Bhagavān says that such a person who has the openness to welcome anybody at any time, any place and doesn’t expect anything in return, is fixed in the \textit{sattvic} qualities. They are always overwhelmed with gratitude themselves. They don’t expect anything from anybody else outside because they feel that they are full!}

\enquote{The mind should be free when you serve. When you are doing service, when you are doing \textit{sevā}, you have to have a mind which is calm and enjoys the \textit{sevā}. Let that joy of service arise through you. When you have that joy of service inside of you, that arises from the \textit{sattvic} quality. So, serve without expecting.}

\enquote{The Supreme Lord is watching everything. His eyes are everywhere. And all that you are doing, if you do it with love, will reach Him.}

\enquote{If somebody is begging on the road\ldots if somebody is asking, \enquote{Oh, I am very hungry, give me food} and you see that they are sincere and they are really longing for food, give. But if you see that the people will indulge themselves, they are begging for drugs and you knowingly give alms, you are participating in harming them. So, their \textit{tamasic} quality will also reflect upon you. That’s why it is very important to be watchful.}

\enquote{You don’t need to judge anyone, but see how you can make that person’s life better. Otherwise, instead of making their life better, what is happening? You are making their life worse and that will not profit them or you.}

\enquote{There are all these big, big charities who have big, big names; in reality even if your intention is right, by giving gifts to these places, you are participating in all their wrongdoing.}

\section{Verses 23-27: The Sacred Mantra OṀ TAT SAT}\label{sec-verses-23-27-the-sacred-mantra-om-tat-sat}

\Verse[17.23]
{oṁ tat sad iti nirdeśo brahmaṇas tri-vidhaḥ smṛtaḥ \\
brāhmaṇās tena vedāś ca yajñāś ca vihitāḥ purā}
{\enquote{\textit{Oṁ} \textit{tat} \textit{sat}}---this is considered the threefold indication of Brahman by which the \textit{brāhmaṇas}, the \textit{Vedas}, and \textit{yajñas} were established in former times.}


\par\noindent\textbf{17.24}\par
\begin{verse}
\textit{tasmād oṁ ity udāhṛtya yajña-dāna-tapaḥ-kriyāḥ} \\
\textit{pravartante vidhānoktāḥ satataṁ brahma-vādinām}

\noindent Therefore, the acts of worship, charity, and austerity prescribed by scripture are always begun with the chanting of \textit{oṁ} by the followers of the \textit{Vedas}.
\end{verse}
\par


\Verse[17.25]
{tad ity anabhisandhāya phalaṁ yajña-tapaḥ-kriyāḥ \\
dāna-kriyāś ca vividhāḥ kriyante mokṣa-kāṅkṣibhiḥ}
{Without expecting the fruit of their action, the seekers of liberation chant \enquote{\textit{tat}} when performing various acts of sacrifice, austerity, and charity.}

\Verse[17.26]
{sad-bhāve sādhu-bhāve ca sad ity etat prayujyate \\
praśaste karmaṇi tathā sac-chabdaḥ pārtha yujyate}
{The term \enquote{\textit{sat}} is used in the sense of \enquote{existence} and \enquote{virtue.} It is also used for an auspicious action, O Pārtha.}

\Verse[17.27]
{yajñe tapasi dāne ca sthitiḥ sad iti cocyate \\
karma caiva tad-arthīyaṁ sad ity evābhidhīyate}
{Perseverance in sacrifice, austerity, and charity is called \enquote{\textit{sat}}; and any action performed for the sake of these is also termed \enquote{\textit{sat}.}}

\enquote{The three syllables \textit{oṁ}, \textit{tat}, \textit{sat} are Bhagavān Himself. It is the Lord, together with this entire creation, consisting of the act of sacrifice and performance of the sacrifice.}

\enquote{If ever you do your prayer, your \textit{mantra}, yet you are not sure about it, at the end of the prayer, think of the Lord, think of Bhagavān Kṛṣṇa and just say, \enquote{\textit{oṁ}, \textit{tat}, \textit{sat}.} By doing that, if you made a mistake by pronouncing any \textit{mantra}, a mistake in performing any \textit{mantra} with whatever attitude or with whatever aim that arose during that time, you will be free from the consequences, because you have offered it to Him.}


\textbf{OṀ:} \enquote{The pronunciation of \textit{oṁ} is Bhagavān Himself, the manifestation of the cosmic sound, the word incarnated, the awakening. When people attain realization, they start to vibrate the cosmic sound of \textit{oṁ}. When you do your Atma Kriya Yoga, when you are inhaling, you are always concentrating on it, infusing yourself with it, reminding yourself continuously of \textit{oṁ}, not outside of you, but inwardly. You are purifying each channel, each chamber, each energy point inside your body and you rise.}

\enquote{Everything comes from \textit{oṁ} and everything will go back into \textit{oṁ}. There is nothing else. That’s why the \textit{brāhmaṇas}, the great sages start all the \textit{mantras} with \textit{oṁ}. When compiling all the \textit{mantras}, \textit{Veda-vyāsa} put \textit{oṁ} at the beginning of all \textit{mantras}. He said, \enquote{The benefit of \textit{oṁ} is that it’s a reminder of God.} That’s why when we start with \textit{oṁ}, it reminds us that it is \textit{sattvic} in itself. It is directed towards the Divine. It’s not in vain. It’s not for self-gratification, but it’s for pleasing the Lord. It is Him.}

\textbf{TAT:} \enquote{The self-discipline helps one to free oneself from all forms of expectations. \textit{Tat} stands for that: the one who is doing all actions in a state of disinterest, but full of \textit{bhāva} for the Lord, full of Love for the Lord, full of surrender to the Lord. There’s not a moment, wherever they are, whatever they are doing, when they are not in constant remembrance that their life is for the Lord Himself. That’s what a \textit{bhakta} should attain when they say, \enquote{\textit{hari oṁ tat sat}.} When you say, \enquote{\textit{tat},} it’s desireless, free and the mind is focused on the Lord Himself.}

\textbf{SAT:} \enquote{Here \textit{sat} stands for that freedom, that everything is the Lord and that one is not the doer. One perceives that God is doing all. Here Bhagavān says that using the word \textit{sat} means that one does all in the state of divine awareness, full of Love, disinterested about the fruit of their action, not attached to any fruit of the action, lives freely in a state of service to the Lord, perceiving the Lord all the time, loving Him in all His creation.}

\enquote{Here Bhagavān says that \textit{sat}, in this quality, purifies the heart of \textit{bhaktas} and makes them realize that He is seated inside of them. This realization that God is the doer of all, and whatever we do is for His sake, knowing deeply that it is only Him that is doing all, that’s \textit{sat}. That’s the Truth. Here it is not \textit{satya} (truth), it is \textit{sat} referring to the Almighty, everything.}
\section{Verse 28: Action Without Faith is Useless}\label{sec-verse-28-action-without-faith-is-useless}

\Verse[17.28]
{aśraddhayā hutaṁ dattaṁ tapas taptaṁ kṛtaṁ ca yat \\
asad ity ucyate pārtha na ca tat pretya no iha}
{O Pārtha, whatever is offered as sacrifice, given in charity, practiced as austerity, and whatever else is carried out---if it is devoid of faith, it is called \enquote{\textit{asat}.} It is worthless in this world and the next.}

\enquote{If one is performing any action without faith, it’s of no use doing it. Because that action which is performed without faith it is a forbidden act. This action will ruin oneself, this action brings sinful qualities to one and the fruit of it is also very destructive. People doing such actions without faith, doing all the service, all the charities, performing all the prayers, practicing all the \textit{sādhana} in a hypocrite way, here Bhagavān says that it’s called \textit{asat}. As this \textit{asat} is just a show, people have to bear the fruit, people have to reap the painful consequences of their acts.}

\enquote{Whatever one does, one has to bear the fruit. If you have planted a seed which is not good, it will grow, but don’t expect the tree to bear sweet fruits, even if you water it with sweet water. It will bear fruits according to the seed that you planted. If you plant a good seed, it shall grow and multiply a thousand-fold. When one does a sinful act, or act with a selfish motive, even if it appears very nice on the outside, one will bear the consequence of that. It all depends on how much faith people have. If they have faith in what they are doing, it’s good, they will also bear that kind of things in their life. If they have faith in the Lord and completely surrender to the Lord, their faith will also grow and bring them towards the Lord, and their faith will also bring the Lord to them.}

