\chapter{Arjuna-viṣāda-yoga}\label{chap-arjuna-visada-yoga}
\chaptersubtitle{The Lamentation of Arjuna}
%%%%%%%%%%%%%%%%%%%%%%%%%%%%%%%%%%%%%%%%%%%%%%%%%%%%%%%%%

\noindent The \textit{Gītā} begins with a vivid description of the scene on the Kurukṣetra battlefield. Key warriors on both sides are named. The blowing of conches and the beating of drums signal the start of this terrible war. It is at this point that the focus shifts to Arjuna and Kṛṣṇa. Eager to assess the enemy ranks, Arjuna asks Kṛṣṇa to take him to the middle of the battlefield so he can take a closer look at those he is about to fight.

Among the opposing army, he sees relatives, friends, and revered elders. The sight of those who were once so dear to him causes Arjuna to lose his resolve. He can no longer see the point in engaging in this battle which will inevitably destroy his family. Confused about his duty and overwhelmed with compassion for his enemies, he begins to pour out his heart to Kṛṣṇa. Surely, he argues, this war cannot be based on righteousness. He repeatedly makes the point that killing one’s own family for the sake of a kingdom will only produce dire consequences for the future. After making his case, the chapter ends with Arjuna casting aside his bow in grief.

\section{Verse 1: Dhṛtarāṣṭra's Inquiry}\label{sec-verse-1-dhrtarastra-s-inquiry}


\par\noindent\textbf{1.3}\par
\begin{verse}
\textit{paśyaitāṁ pāṇḍu-putrāṇām ācārya mahatīṁ camūm}\\
\textit{vyūḍhāṁ drupada-putreṇa tava śiṣyeṇa dhīmatā}

\noindent O teacher, behold this mighty army of the Pāṇḍavas, arrayed by the son of Drupada, your intelligent disciple.
\end{verse}
\par

\enquote{This verse starts with the word \enquote{\textit{dharma-kṣetra}.} \enquote{\textit{Dharma}} means righteous, \enquote{\textit{kṣetra}} means the field---so this is the field of righteousness.}

\enquote{One of the meanings of this war is life, where the \enquote{good} side fights with the \enquote{not good} side. This war is not outside, it is also happening inside the human body. Your physical body is the \textit{dharma-kṣetra}. You have incarnated to do your \textit{dharma} (duty) in this field.}

\enquote{Life is also a \textit{dharma-kṣetra}. You have come to fulfill your divine purpose. When you are in tune with your true Self, you realize your purpose in life\ldots and that’s what the word \enquote{\textit{dharma-kṣetra}} is reminding you of. Do your \textit{dharma}! Awaken! This \textit{dharma} can be done with the greatest gift which God has given: this field, this body. And when you start doing your \textit{dharma}, you’ll get good merit! But, if you run away from your \textit{dharma}, then you turn towards the dark side.}

\enquote{This blind king, Dhṛtarāṣṭra, represents the mind---the mind which is blind and wants to always stay blind. The mind is hanging on to the outside so much that it has power only when it is focused on something exterior: on the material, on relationships, on gaining this or that. This is the nature of the mind. The mind is blind.}

\enquote{Both families were from the Kuru dynasty. But the king refused to recognize the Pāṇḍavas. The mind doesn’t recognize the good qualities which are present in oneself. The mind can only look towards the senses, looking always towards the outside. The Self, and the positive qualities which are present inside, are not comprehended by it.}
