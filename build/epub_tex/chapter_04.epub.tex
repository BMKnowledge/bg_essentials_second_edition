\chapter{Jñāna-vibhāga-yoga}\label{chap-jnana-vibhaga-yoga}
\chaptersubtitle{The Path of Knowledge}
%%%%%%%%%%%%%%%%%%%%%%%%%%%%%%%%%%%%%%%%%%%%%%%%%%%%%%%%%

\noindent Chapter four begins with Kṛṣṇa telling Arjuna how He originally taught this ancient knowledge of \textit{karma-yoga} to Vivasvān, the Sun-god. Due to the passage of time it was lost and now it is being revived once again. It is at this point that Kṛṣṇa declares emphatically His identity as the \textit{avatāra}, the Lord who has descended to protect \textit{dharma}\footnote{Everything in this world is part of a natural cosmological order. As a result, each individual has a set purpose in line with this order. \textit{Dharma} is the fulfillment of this purpose.} and vanquish evil.

The discourse then shifts back to a discussion on \textit{karma-yoga} and how we can understand the true nature of action. The overall point is that action which produces results (\textit{karma}) is not connected with one’s external activity but rather the state of consciousness we are in. Our thoughts, feelings, and desires also produce \textit{karmic} results. If we are free of desire and detached no matter how much activity we are engaged in, there are no \textit{karmic} consequences to our actions. However, even if we do our best to abstain from any external activity, so long as we have desires and attachment, we will always be creating further results. In other words, the level of renunciation and the amount of \textit{karma} produced cannot be simply gauged from one’s outside behavior.

Following on from this, Kṛṣṇa discusses the ritual fire sacrifice or \textit{yajña}.\footnote{Referring to the \textit{Vedic} fire sacrifice. Various ritual offerings are made in an effort to appease the different gods and gain results. Throughout the \textit{Gītā}, however, Kṛṣṇa introduces \textit{yajña} as being more about the internal sacrifice of spiritual practice rather than just a fire ritual.} Traditionally this \textit{yajña} is seen as a way of satisfying the gods and fulfilling material desires. But Kṛṣṇa explains that this \textit{yajña} should not be understood as just an external act, but an internal process of restraint and dedication. Controlling the senses, breath, and taking up vows are all different forms of \textit{yajña}. Rather than focusing on the external ritual for material gain, \textit{karma-yoga} is all about the inner \textit{yajña} which eventually leads to knowledge of the Self (\textit{Brahma-jñāna}). Once this goal is achieved, the realization of who we truly are destroys the possibility of incurring any additional \textit{karma}.

\newpage
\section{Verses 1-6: The Revival of Ancient Wisdom}\label{sec-verses-1-6-the-revival-of-ancient-wisdom}

\par\noindent\textbf{4.1}\par
\begin{verse}
\textit{śrī-bhagavān uvāca} \\
\textit{imaṁ vivasvate yogaṁ proktavān aham avyayam} \\
\textit{vivasvān manave prāha manur ikṣvākave ’bravīt}

\noindent Bhagavān Kṛṣṇa said: I taught this imperishable \textit{yoga} to Vivasvān;\footnote{Also referred to as the Sun-god, who was given divine knowledge by Kṛṣṇa at the beginning of the age.} Vivasvān taught it to Manu; Manu declared it to Ikṣvāku.
\end{verse}
\par



\par\noindent\textbf{4.2}\par
\begin{verse}
\textit{evaṁ paramparā prāptam imaṁ rājarṣayo viduḥ} \\
\textit{sa kāleneha mahatā yogo naṣṭaḥ parantapa}

\noindent Having received it in disciplic succession, the royal sages thus knew this knowledge. But due to the passage of great time, Arjuna, this \textit{yoga} was lost to the world, O conqueror of enemies.
\end{verse}
\par



\par\noindent\textbf{4.3}\par
\begin{verse}
\textit{sa evāyaṁ mayā te ’dya yogaḥ proktaḥ purātanaḥ} \\
\textit{bhakto ’si me sakhā ceti rahasyaṁ hy etad uttamam}

\noindent The same ancient \textit{yoga} has today been taught to you by Me, since you are My devotee and friend; it is indeed the highest mystery.
\end{verse}
\par



\par\noindent\textbf{4.4}\par
\begin{verse}
\textit{arjuna uvāca} \\
\textit{aparaṁ bhavato janma paraṁ janma vivasvataḥ} \\
\textit{katham etad vijānīyāṁ tvam ādau proktavān iti}

\noindent Arjuna said: Your birth was recent, and the birth of Vivasvān was long ago. How shall I understand that You taught this knowledge in the beginning?
\end{verse}
\par



\par\noindent\textbf{4.5}\par
\begin{verse}
\textit{śrī-bhagavān uvāca} \\
\textit{bahūni me vyatītāni janmāni tava cārjuna} \\
\textit{tāny ahaṁ veda sarvāṇi na tvaṁ vettha parantapa}

\noindent Bhagavān Kṛṣṇa said: Many births of Mine have passed, O Arjuna, and so have many of yours. I know them all, but you do not know them, O vanquisher of enemies.
\end{verse}
\par



\par\noindent\textbf{4.6}\par
\begin{verse}
\textit{ajo ’pi sann avyayātmā bhūtānām īśvaro ’pi san} \\
\textit{prakṛtiṁ svām adhiṣṭhāya sambhavāmy ātma-māyayā}

\noindent Although I am unborn and unchanging, and although I am the Lord of all beings, controlling My own nature, I manifest in this world by My own potency.
\end{verse}
\par


\enquote{Here Kṛṣṇa is revealing His eternal relationship. He says, \enquote{This ancient \textit{yoga} that I am revealing to you, I am giving to you out of the Love which I have for you. This is the same \textit{yoga} which I have given to the Sun-god billions of years ago, and I am now giving it to you. Nothing has changed. Even if you have forgotten it, due to passing from generation to generation and it has disappeared in time, here it is on this battlefield, for you are My devotee and you are My friend.}}

\enquote{When you are in tune with the Divine, the Divine reveals to you your true \textit{dharma} of life: the reason you have incarnated on Earth. Then all the illusion, all the other \textit{dharma} which you have thought about, disappears! Only then, you have true knowledge of how to serve the Lord. Only then, can the Lord reveal Himself to you. Therefore, Arjuna should not entertain any doubt about the Truth of the Lord when He says, \enquote{I gave this knowledge to the Sun-god. I existed before everything was created or manifested.}}

\enquote{This secret was only given to people by their \textit{gurus} when they were ready. If you give this \textit{Brahma-jñāna} (knowledge of the Self) to people who are not ready to grasp it with their hearts, they will not know what to do with it. If their minds are too focused on the outside world, it will not make any sense to them; it will just be meaningless.}

\enquote{Kṛṣṇa sees this quality of a \textit{bhakta} inside of Arjuna, this longing! That’s what the \textit{guru}, the Teacher, the master looks for in the devotees, in the disciples. He checks to see if the \enquote{soil} is ready, if all the wild grass has been removed and if the disciples are ready to do their \textit{dharma} or not. If people are in a state of confusion, if they are given such knowledge and such blessing, they will fall into deeper confusion. But if one is wholeheartedly surrendered to the \textit{guru}, one doesn’t need to worry about anything. Everything will be taken care of.}

\section{Verses 7-8: Kṛṣṇa Declares Himself as the Avatāra of the Age}\label{sec-verses-7-8-krsna-declares-himself-as-the-avatara-of-the-age}


\par\noindent\textbf{4.7}\par
\begin{verse}
\textit{yadā yadā hi dharmasya glānir bhavati bhārata} \\
\textit{abhyutthānam adharmasya tadātmānaṁ sṛjāmy aham}

\noindent O Arjuna, whenever there is a decline in \textit{dharma}, and an increase of \textit{adharma}, I manifest Myself.
\end{verse}
\par



\par\noindent\textbf{4.8}\par
\begin{verse}
\textit{paritrāṇāya sādhūnāṁ vināśāya ca duṣkṛtām} \\
\textit{dharma-saṁsthāpanārthāya sambhavāmi yuge yuge}

\noindent I manifest Myself age after age for the protection of the good, for the destruction of the wicked, and for the sake of establishing \textit{dharma}.
\end{verse}
\par


\enquote{Here Bhagavān Kṛṣṇa is saying He comes forth from age to age whenever righteousness is covered by unrighteousness, whenever evil starts to predominate in the world and when it is threatening and destroying cultures and the world. For the good and benefit of this world, Bhagavān manifests Himself and He manifests by His own willpower.}

\enquote{People can’t understand the Ultimate, God Himself, the Lord of all creation, the One who is ever-pure, who is awakened and ever-free, who is playing every role in His creation. It is too difficult for the mind to understand. Even if one tries to understand it, it’s far beyond the mind. He (Kṛṣṇa) says, \enquote{Even if you become a great \textit{yogī}, even if you are centered in the Self, you will realize only one step of realization. My true nature, the Self of the Self, can’t even be realized by \textit{yogīs}.} That’s why He says, \enquote{I am veiled. I came into birth by My \textit{māyā}: when I will it, I manifest. I’m ever-free to take any form, at any time. I am not bound by time or by space. I am not bound by anything, not even by the scriptures.}}

\enquote{God is ever-free, He can do whatever He wants. Some people would ask, \enquote{How can this be possible? This is just a human being, how can He be God?} Arjuna is asking the same question: \enquote{How can He be God?} Here you see the doubting human mind. But God is beyond that. Even if He appears to be very human, He is not human. The saints and sages can only have a glimpse of that reality. What He shows to the sages is only a little aspect of Himself. This raises them to a very high degree of spirituality, but not to the Ultimate Himself.}

\enquote{Kṛṣṇa says, \enquote{I manifest Myself from time to time, taking various forms, appearing in different places. Sometimes I remain unknown to the world, incognito. Yet whatever I do, I do it for the sake of uplifting mankind.}}

\enquote{Humans are blinded by one \textit{māyā} which we call \enquote{illusion.} You are blinded by this \textit{māyā} until you are ready. When you make yourself ready through \textit{sādhana} (spiritual practice), the veil of \textit{māyā} is removed. Through your \textit{sādhana}, when you’re surrendered to your \textit{sādhana}, you are moving away from external desires and the objects of senses which hold you back, drag you down, and root you here to this world.}

\enquote{By doing your \textit{dharma} and by doing your \textit{sādhana} with the aim of attaining the Lord, the Lord allows this first veil of \textit{māyā} to be removed from your eyes. Then you see differently, and you perceive the glory of the Lord everywhere. However, you are not perceiving Himself everywhere because by His own will, He veils Himself by another \textit{māyā}, \textit{yoga-māyā},\footnote{The specific potency of God that conceals His true identity.} so people won’t recognize Him and realize who He is. This is due to God’s humility. He says, \enquote{I veil Myself with another \textit{māyā}, \textit{yoga-māyā}, so that even if you do your \textit{sādhana}, even if you try your best to see who I am, until I will it, you will never see Me. You can try as much as you want, you can be the greatest \textit{yogī}, but you will not know Me until I will it.} Here He reveals that this \textit{yoga-māyā} is different from the \textit{māyā} that covers mankind.}

\enquote{Even being the Lord of the whole universe, He humbles Himself for the sake of humanity, so that He can uplift humanity. This is wonderful. He willingly chooses to veil Himself. He chooses to put a limitation on Himself in order to incarnate, but He is not bound by any limitation. He appears to be limited, but He is not.}

\section{Verses 9-15: How Surrender Leads to Freedom}\label{sec-verses-9-15-how-surrender-leads-to-freedom}

\par\noindent\textbf{4.9}\par
\begin{verse}
\textit{janma karma ca me divyam evaṁ yo vetti tattvataḥ} \\
\textit{tyaktvā dehaṁ punar janma naiti mām eti so ’rjuna}

\noindent Whoever knows the essence of My divine birth and deeds, O Arjuna, is not reborn again. After leaving the body, they come to Me.
\end{verse}
\par



\par\noindent\textbf{4.10}\par
\begin{verse}
\textit{vīta-rāga-bhaya-krodhā man-mayā mām upāśritāḥ} \\
\textit{bahavo jñāna-tapasā pūtā mad-bhāvam āgatāḥ}

\noindent Freed from attachment, fear, and anger, absorbed in Me, having taken refuge in Me, and purified by the austerity of knowledge, many have attained My nature.
\end{verse}
\par



\par\noindent\textbf{4.11}\par
\begin{verse}
\textit{ye yathā māṁ prapadyante tāṁs tathaiva bhajāmy aham} \\
\textit{mama vartmānuvartante manuṣyāḥ pārtha sarvaśaḥ}

\noindent In whatever way people approach Me, in the same way I reciprocate. Everybody follows My path in all respects, O Arjuna.
\end{verse}
\par



\par\noindent\textbf{4.12}\par
\begin{verse}
\textit{kāṅkṣantaḥ karmaṇāṁ siddhiṁ yajanta iha devatāḥ} \\
\textit{kṣipraṁ hi mānuṣe loke siddhir bhavati karma-jā}

\noindent Those desiring success of their actions, sacrifice to the \textit{devas}; for in this human world, results born of actions manifest quickly.
\end{verse}
\par


\Verse[4.13]
{cātur-varṇyaṁ mayā sṛṣṭaṁ guṇa-karma-vibhāgaśaḥ \\
tasya kartāram api māṁ viddhy akartāram avyayam}
{The system of the four \textit{varṇas}\footnote{The different social classes within \textit{Vedic} culture, namely the \textit{brāhmaṇas} (sages and priests), \textit{kṣatriyas} (warriors and royal leaders), \textit{vaiśyas} (merchants responsible for commerce), and \textit{śūdras} (responsible for general labor). In order to ensure a prosperous society, these different groups were allocated different responsibilities for the welfare of everyone at large.} was created by Me, based on the division of predominant qualities and corresponding actions. Despite being their creator, know Me to be a non-doer and changeless.}


\par\noindent\textbf{4.14}\par
\begin{verse}
\textit{na māṁ karmāṇi limpanti na me karma-phale spṛhā} \\
\textit{iti māṁ yo ’bhijānāti karmabhir na sa badhyate}

\noindent Actions do not taint Me, since I do not have a desire for their fruits. One who knows Me in this way is not bound by actions.
\end{verse}
\par



\par\noindent\textbf{4.15}\par
\begin{verse}
\textit{evaṁ jñātvā kṛtaṁ karma pūrvair api mumukṣubhiḥ} \\
\textit{kuru karmaiva tasmāt tvaṁ pūrvaiḥ pūrva-taraṁ kṛtam}

\noindent Understanding this, even the ancient seekers of liberation performed action; therefore, you should also perform your duty just as people in the past have done.
\end{verse}
\par


\enquote{Those who do their work with an attitude of surrender, knowing that only God is the doer, will attain Him. Whoever surrenders sees that whatever Lord Kṛṣṇa does is for the good of this world. Every activity of His is for the upliftment of humanity. He has also given the \textit{Gītā} for this. Whoever surrenders to the Word of the \textit{Gītā} and practices its teaching wholeheartedly, with love in their hearts, is free, and attains Him.}

\enquote{But it’s not just about saying, \enquote{I love God!} You also have to serve Him. You have to have this deep knowledge through meditation that He is the core of the Self in each person. You have to train the mind to see Him in everybody and in everything. Therefore, one is not waiting to die to attain Him, to realize Him. No. Now, in this moment, do everything to see Him, do everything to feel His Love. Whether you attain Him now or not, don’t be concerned about it. Let Him give you whatever is needed. Learn to accept it. It’s not only after you leave the body that you will attain God. No. First you attain Him now in every moment of your life.}


\enquote{True happiness can be given only by the Lord Himself. Only when you long for Him, He reciprocates the feeling that you have inside of you, not just one time, but one hundred times more\ldots You think that you are thinking of Him. No. He is thinking of you. You think that you are going to Him; He is running towards you. You think of Him, you long for Him, but in His heart, He is pining for you.}

\section{Verses 16-23: Defining Action and Inaction}\label{sec-verses-16-23-defining-action-and-inaction}

\par\noindent\textbf{4.16}\par
\begin{verse}
\textit{kiṁ karma kim akarmeti kavayo ’py atra mohitāḥ} \\
\textit{tat te karma pravakṣyāmi yaj jñātvā mokṣyase ’śubhāt}

\noindent What is action? What is inaction? Even the wise are confused in this regard. I shall now explain to you the nature of action, having known which you will be freed from misfortune.
\end{verse}
\par



\par\noindent\textbf{4.17}\par
\begin{verse}
\textit{karmaṇo hy api boddhavyaṁ boddhavyaṁ ca vikarmaṇaḥ} \\
\textit{akarmaṇaś ca boddhavyaṁ gahanā karmaṇo gatiḥ}

\noindent Verily one must understand what is action, what is forbidden action, as well as what is inaction. The nature of action is indeed hard to understand.
\end{verse}
\par



\par\noindent\textbf{4.18}\par
\begin{verse}
\textit{karmaṇy akarma yaḥ paśyed akarmaṇi ca karma yaḥ} \\
\textit{sa buddhimān manuṣyeṣu sa yuktaḥ kṛtsna-karma-kṛt}

\noindent One who sees inaction in action and also action in inaction is wise among people. Such a person is fit for liberation and performs all duties.
\end{verse}
\par



\par\noindent\textbf{4.19}\par
\begin{verse}
\textit{yasya sarve samārambhāḥ kāma-saṅkalpa-varjitāḥ} \\
\textit{jñānāgni-dagdha-karmāṇaṁ tam āhuḥ paṇḍitaṁ budhāḥ}

\noindent One whose every action is free from the egoistic desire to enjoy and whose actions are burnt up by the fire of knowledge, is declared a sage by the wise.
\end{verse}
\par



\par\noindent\textbf{4.20}\par
\begin{verse}
\textit{tyaktvā karma-phalāsaṅgaṁ nitya-tṛpto nirāśrayaḥ} \\
\textit{karmaṇy abhipravṛtto ’pi naiva kiñcit karoti saḥ}

\noindent Having renounced attachment to the fruits of action, without reliance on anything worldly, ever-content in the Self, such a sage truly does not act at all despite being engaged in activity.
\end{verse}
\par



\par\noindent\textbf{4.21}\par
\begin{verse}
\textit{nirāśīr yata-cittātmā tyakta-sarva-parigrahaḥ} \\
\textit{śārīraṁ kevalaṁ karma kurvan nāpnoti kilbiṣam}

\noindent Free from desire, possessing a controlled mind, having relinquished all sense of ownership, and acting only to maintain the body, they do not incur sin.
\end{verse}
\par



\par\noindent\textbf{4.22}\par
\begin{verse}
\textit{yadṛcchā-lābha-santuṣṭo dvandvātīto vimatsaraḥ} \\
\textit{samaḥ siddhāv asiddhau ca kṛtvāpi na nibadhyate}

\noindent Content with what comes of its own accord, having crossed beyond the pairs of opposites, being free from envy and equal-minded in success and failure, that sage is not bound despite acting.
\end{verse}
\par



\par\noindent\textbf{4.23}\par
\begin{verse}
\textit{gata-saṅgasya muktasya jñānāvasthita-cetasaḥ} \\
\textit{yajñāyācarataḥ karma samagraṁ pravilīyate}

\noindent For the liberated person free from attachments, whose mind is established in wisdom, and acts only for sacrifice, \textit{karma} is entirely dissolved.
\end{verse}
\par


\enquote{Thinking involves performance of action; the mind wandering around is still action. You can’t stop that. But you can transcend these worldly activities. You can see that everything is predominated by the Lord.}

\enquote{If you perform \enquote{inaction} without the direction and supervision of a master, you will not achieve anything. You can sit for meditation, you can control your speech, you can control the body, but this is not true calmness, because there is a motive behind it. Even if the mind, speech and body are controlled, if the motive is egoistic, if there is an expectation, it will not lead you anywhere.}

\enquote{The realized ones, the enlightened souls have the true knowledge: they know which action to perform, and which motive to use, so it will lead them to complete renunciation, and not only superficially. So that’s why those who are seeking to liberate themselves should surrender to a \textit{guru}, who will lead them out of this game of action, inaction, and prohibited action. Prohibited action means action done with deceit, hypocrisy, violence, greed, pride. On the other hand, when people do work with devotion, they are humble. When one is surrendered to the \textit{guru}, automatically one is humble; one is ready to accept what the \textit{guru} says, which means one has to remove one’s own will and ways of seeing and embrace what the \textit{guru} is giving.}

\enquote{Of course, the \textit{guru} has to have certain qualities, otherwise he can’t help you progress. In the world nowadays, there are so many \textit{gurus}, so many teachers. There are many \textit{gurus} who teach \textit{yoga}. But if these \textit{gurus} don’t teach you to attain the perfection of the soul, and if they just focus on the perfection of the body or the mind, it’s not enough. That’s why I said that it’s very important to know the \textit{guru} who can help you advance and attain the ultimate reality. There are many learned people in the world who know all the scriptures. Their head is full of book knowledge, but this doesn’t give them liberation, it doesn’t free them. These people fail to determine what their true nature is. That’s why it is necessary to approach a true \textit{guru} and surrender to Him.}

\enquote{Nowadays, there are many schools of \textit{yoga}; each one claims that their \textit{yoga} is better than the others. But when one is realized, one sees that all ways are right, because God puts people where they have to be. So if there is true realization, there is no judgment. If one has divine knowledge, one should not have any judgment towards anyone; one should not have any enmity or aggressiveness towards anyone.}

\enquote{If people are doing their \textit{sādhana}, either for their own benefit or to gain something from the world, this can be considered prohibited action and they won’t gain anything from it. Whereas a true \textit{guru} teaches one to love God beyond everything and to surrender to God. Then one is free.}

\enquote{The wise people perform all their duties in a renounced way. They renounce the fruit of the action and the thinking, \enquote{I am doing this.} If you think that you are doing it, you will expect the result. But if your mind is not on \enquote{I, I, I,} if the mind is upon Him, thinking, \enquote{He is doing it through me,} no matter what result will come, it will be the fruit of that action which the Lord has performed through you.}

\enquote{When the Lord performs a certain action through you, there is always perfection in it, and you will always have peace of mind. You will be free from \textit{karma}, because you are not creating the \textit{karma}. As the Lord is free from \textit{karma}, so shall you be.}

\newpage
\section{Verses 24-33: The Different Types of Yajña}\label{sec-verses-24-33-the-different-types-of-yajna}


\par\noindent\textbf{4.24}\par
\begin{verse}
\textit{brahmārpaṇaṁ brahma havir brahmāgnau brahmaṇā hutam} \\
\textit{brahmaiva tena gantavyaṁ brahma-karma-samādhinā}

\noindent Brahman\footnote{In this context, the term is used to describe the supreme reality, Śrī Hari.} is the instrument, Brahman is the oblation which is offered into the fire which is also Brahman. Brahman alone is to be reached by one who is absorbed in Brahman-centered action.
\end{verse}
\par



\par\noindent\textbf{4.25}\par
\begin{verse}
\textit{daivam evāpare yajñaṁ yoginaḥ paryupāsate} \\
\textit{brahmāgnāv apare yajñaṁ yajñenaivopajuhvati}

\noindent Some \textit{yogīs} exclusively offer sacrifice to the \textit{devas},\footnote{The limited celestial beings who occupy different heavenly realms within the material universe. Although not perfected beings, they are generally associated with positivity.} while others sacrifice offerings into the fire of Brahman by the spirit of sacrifice alone.
\end{verse}
\par



\par\noindent\textbf{4.26}\par
\begin{verse}
\textit{śrotrādīnīndriyāṇy anye saṁyamāgniṣu juhvati} \\
\textit{śabdādīn viṣayān anya indriyāgniṣu juhvati}

\noindent Some offer senses such as hearing and so on into the fires of restraint. Others offer objects of perception such as sound and other sense objects into the fires of the senses.
\end{verse}
\par



\par\noindent\textbf{4.27}\par
\begin{verse}
\textit{sarvāṇīndriya-karmāṇi prāṇa-karmāṇi cāpare} \\
\textit{ātma-saṁyama-yogāgnau juhvati jñāna-dīpite}

\noindent Others offer the actions of all the senses and of the life airs into the fire of \textit{yoga} that is based on self-control and illumined by true knowledge.
\end{verse}
\par



\par\noindent\textbf{4.28}\par
\begin{verse}
\textit{dravya-yajñās tapo-yajñā yoga-yajñās tathāpare} \\
\textit{svādhyāya-jñāna-yajñāś ca yatayaḥ saṁśita-vratāḥ}

\noindent Some offer material objects as sacrifice, others make offerings through austerity, likewise others offer their practice of \textit{yoga}, while some ascetics observe strict vows and offer their scriptural study and knowledge as sacrifice.
\end{verse}
\par



\par\noindent\textbf{4.29}\par
\begin{verse}
\textit{apāne juhvati prāṇaṁ prāṇe ’pānaṁ tathāpare} \\
\textit{prāṇāpāna-gatī ruddhvā prāṇāyāma-parāyaṇāḥ}

\noindent Still there are others devoted to \textit{prāṇāyāma}, who while regulating the movement of \textit{prāṇā} and \textit{apāna}, offer \textit{prāṇā} into \textit{apāna}, and likewise \textit{apāna} into \textit{prāṇā}.
\end{verse}
\par



\par\noindent\textbf{4.30}\par
\begin{verse}
\textit{apare niyatāhārāḥ prāṇān prāṇeṣu juhvati} \\
\textit{sarve ’py ete yajña-vido yajña-kṣapita-kalmaṣāḥ}

\noindent Others who restrain their eating offer their life forces into the vital airs. All these \textit{yogīs} whose sins are destroyed by sacrifice know the meaning of sacrifice.
\end{verse}
\par



\par\noindent\textbf{4.31}\par
\begin{verse}
\textit{yajña-śiṣṭāmṛta-bhujo yānti brahma sanātanam} \\
\textit{nāyaṁ loko ’sty ayajñasya kuto ’nyaḥ kuru-sattama}

\noindent Enjoying the nectar of remnants of sacrifice, they attain the eternal Brahman. O best of Kurus, even this world is not for one who does not sacrifice; what to speak of the world beyond?
\end{verse}
\par



\par\noindent\textbf{4.32}\par
\begin{verse}
\textit{evaṁ bahu-vidhā yajñā vitatā brahmaṇo mukhe} \\
\textit{karma-jān viddhi tān sarvān evaṁ jñātvā vimokṣyase}

\noindent In this way, many kinds of \textit{yajñas} are described in the \textit{Vedas}. Know all of them to be based on action. Knowing this, you shall be liberated.
\end{verse}
\par



\par\noindent\textbf{4.33}\par
\begin{verse}
\textit{śreyān dravya-mayād yajñāj jñāna-yajñaḥ parantapa} \\
\textit{sarvaṁ karmākhilaṁ pārtha jñāne parisamāpyate}

\noindent O Parantapa, the sacrifice of wisdom is superior to ritual sacrifice of external materials, since all action, without exception, culminates in wisdom, Pārtha.
\end{verse}
\par


\enquote{There is a saying that goes, \enquote{God is always with people, whether people want it or not. But people are not always with God.} However, someone who is realized has already attained the state of realizing that everything, wherever one goes, whatever one does, is only Him. Whatever one is doing, it is Him doing it: every step, every thought, everything. If one doesn’t self-judge oneself, one will see that it is only the Lord Himself taking every step, doing every action.}

\enquote{If the aim is to attain God, it doesn’t matter where you are or what you are doing, however, it’s very important how you are doing your duty, how you are doing the offering, with what aim, and with what attitude you are doing it. If it is with pride, it doesn’t reach Him.}

\enquote{There is not only one way, but the aim is one! The aim is only God! And if everybody understands this aim, one will be free; it doesn’t matter where one is. That’s why at the beginning, Lord Kṛṣṇa Himself says to Arjuna that each one has been placed in the right place to do their duty. And if one does one’s duty properly, with the attitude of surrendering to the Divine, the Divine will free that person. That’s why He says, \enquote{Arjuna accept your duty! And be free.}}

\enquote{\textit{Yajña} doesn’t only mean when you sit and pour \textit{ghī} in the fire. Whatever you do is a sacrifice upon the altar of the Lord. It’s a continuous offering.}

\enquote{That’s what Bhagavān is asking for right now: to sacrifice these material things and to sacrifice the way you see things. Transform. Know that everything is the Divine only. Find the Divine inside of you. Find the Divine in every action you are doing. Release yourself. Let the Self take over. Let God take over your life. Let yourself be consumed by the Love of God. That’s what realization is. When your heart is consumed with the Love of God, you will have that awareness. And that’s what Atma Kriya Yoga\footnote{Given by Paramahamsa Sri Swami Vishwananda for humanity to realize the teachings of the \textit{Bhagavad Gītā}. It brings constant awareness of the Divine in every action and consists of a series of techniques which include meditation, \textit{prāṇāyāma}, \textit{āsanas}, \textit{mudras}, \textit{japa}, and OM Chanting.} does inside: it elevates you to that realization.}

\section{Verse 34: The Importance of the Guru}\label{sec-verse-34-the-importance-of-the-guru}

\par\noindent\textbf{4.34}\par
\begin{verse}
\textit{tad viddhi praṇipātena paripraśnena sevayā} \\
\textit{upadekṣyanti te jñānaṁ jñāninas tattva-darśinaḥ}

\noindent Learn such wisdom by submission, extensive inquiry, and service from a spiritual master. Such realized beings will impart it to you, for they directly perceive the Truth.
\end{verse}
\par


\enquote{Here Bhagavān Kṛṣṇa Himself is saying that if you just take a book and read it, it will not lead you anywhere. Here He is giving the importance of the master.}

\enquote{Surrender to the feet of the master. Serve the master with your heart filled with love and humility. Earn that grace from the master. Make yourself ready, make yourself willing, make yourself worthy. Let the master guide you. The master has already reached oneness with the Divine.}

\enquote{Often people say, \enquote{No, you don’t need a master in life, you don’t need a \textit{guru}, you can do it by yourself.} They are all stupid. Walk in a jungle without a guide, and you will be lost.}

\enquote{Even if something is very near, just like realization IS there, one needs a guide. This is the importance of a master.}

\enquote{The \textit{guru} wants the disciples to reach a certain level. They know that they have a certain mission, a certain goal, and a certain \textit{dharma} to fulfill and to achieve; if they are not doing it, the master has to be strict. This strictness doesn’t mean that the master doesn’t love the disciple. He does. It’s like when an abscess grows on the leg of a child. First the parents use the medicine to cure the child, but if it’s not getting cured, the parents will not hesitate to cut the limb of the child to save the child’s life. If it’s even necessary to cut off the leg of the child, they will do it. How difficult this would be! But the parents care for their children. In the same way, the \textit{guru} is sometimes strict so that all the \enquote{evil-doers} will be destroyed, so that you can grow and rise spiritually.}

\enquote{When the mind is upset, one doesn’t\ldots see that the \textit{guru} is doing everything for the benefit of the disciple’s progress---even if He has to sacrifice Himself, as did Christ. The \textit{guru} is always sacrificing Himself for the sake of His disciples and devotees.}

\section{Verses 35-42: The Fruit of Karma-yoga Is Realized Knowledge}\label{sec-verses-35-42-the-fruit-of-karma-yoga-is-realized-knowledge}

\par\noindent\textbf{4.35}\par
\begin{verse}
\textit{yaj jñātvā na punar moham evaṁ yāsyasi pāṇḍava} \\
\textit{yena bhūtāny aśeṣāṇi drakṣyasy ātmany atho mayi}

\noindent Having realized it, you will not again go into delusion, O Pāṇḍava. By that wisdom you will see the entire reality, first in the \textit{ātmā} and then in Me.
\end{verse}
\par



\par\noindent\textbf{4.36}\par
\begin{verse}
\textit{api ced asi pāpebhyaḥ sarvebhyaḥ pāpa-kṛt-tamaḥ} \\
\textit{sarvaṁ jñāna-plavenaiva vṛjinaṁ santariṣyasi}

\noindent Even if you are the most sinful of all sinners, you can cross over all distress by the boat of wisdom.
\end{verse}
\par



\par\noindent\textbf{4.37}\par
\begin{verse}
\textit{yathaidhāṁsi samiddho ’gnir bhasma-sāt kurute ’rjuna} \\
\textit{jñānāgniḥ sarva-karmāṇi bhasma-sāt kurute tathā}

\noindent Arjuna, just as a blazing fire turns wood to ashes, so does the fire of knowledge turn all \textit{karma} to ashes.
\end{verse}
\par



\par\noindent\textbf{4.38}\par
\begin{verse}
\textit{na hi jñānena sadṛśaṁ pavitram iha vidyate} \\
\textit{tat svayaṁ yoga-saṁsiddhaḥ kālenātmani vindati}

\noindent In this world nothing is as purifying as knowledge. In due time, someone perfected in \textit{karma-yoga} naturally discovers it within.
\end{verse}
\par



\par\noindent\textbf{4.39}\par
\begin{verse}
\textit{śraddhāvāḻ labhate jñānaṁ tat-paraḥ saṁyatendriyaḥ} \\
\textit{jñānaṁ labdhvā parāṁ śāntim acireṇādhigacchati}

\noindent One who has faith, is dedicated, and has controlled the senses attains knowledge of the \textit{ātmā}. Having attained it, one quickly realizes supreme peace.
\end{verse}
\par



\par\noindent\textbf{4.40}\par
\begin{verse}
\textit{ajñaś cāśraddadhānaś ca saṁśayātmā vinaśyati} \\
\textit{nāyaṁ loko ’sti na paro na sukhaṁ saṁśayātmanaḥ}

\noindent But one who is ignorant, faithless, and full of doubt is utterly lost. Neither this world, nor the next, nor any happiness exist for a doubtful person.
\end{verse}
\par



\par\noindent\textbf{4.41}\par
\begin{verse}
\textit{yoga-sannyasta-karmāṇaṁ jñāna-sañchinna-saṁśayam} \\
\textit{ātmavantaṁ na karmāṇi nibadhnanti dhanañjaya}

\noindent O Dhanañjaya, actions don't bind the one who has renounced selfish actions by \textit{karma-yoga}, whose doubts have been dispelled by knowledge, and who has realized the \textit{ātmā}.
\end{verse}
\par



\par\noindent\textbf{4.42}\par
\begin{verse}
\textit{tasmād ajñāna-sambhūtaṁ hṛt-sthaṁ jñānāsinātmanaḥ} \\
\textit{chittvainaṁ saṁśayaṁ yogam ātiṣṭhottiṣṭha bhārata}

\noindent Therefore, cut this doubt present in your heart, which has arisen from ignorance of the \textit{ātmā}, with the sword of knowledge. Arise and perform \textit{karma-yoga}, O descendant of Bharata.
\end{verse}
\par


\enquote{The Truth about the soul is deeply mysterious because it cannot be known by the mind. It is only a few who have a controlled mind, who are centered in the Lord, who are fully dedicated to the spiritual path, who receive the grace of realization.}

\enquote{Many people practice meditation, but only a few really attain realization. And who are these few? The ones who are completely surrendered.}

\enquote{You have a human body, and in this human body, God has given you the power to find Him! This means that He has great hope for you! He has great hope that you can do it! He is waiting for you. If you let yourself be weak, He can’t help you. But if you make yourself strong, He will carry you.}

\enquote{If you are not ready, the true knowledge doesn’t come to you. If you are not ready, the \textit{guru} doesn’t reveal Himself to you. But when you are ready, you are guided to where you have to be. The soul sends an invisible \enquote{email} to Mother Nature and as the \textit{guru} receives that \enquote{email,} He says, \enquote{Ah, this one is ready!}}

\enquote{When one has faith, one is ready to evolve, ready to change and ready to transform. Faith makes you conquer and control the mind and senses.}

\enquote{It is through faith that He brings you on your spiritual path, to get better and to grow\ldots It is through faith that you can have respect for your Teacher. It is through faith that you can accept what the Teacher is giving you, and in that faith, humility and peace of mind awaken.}

\enquote{Many claim they know better. They come and then they leave, due to that pride. They don’t want to let go of their pride. They don’t want to let go of their ego. They don’t want to change.}

\enquote{By surrendering to the feet of the master, all the errors of lives are removed. Once you have this right knowledge about the \textit{ātmā}, the Truth will be revealed. That supremacy of the Lord will be revealed, not outside, but first inside of yourself.}
