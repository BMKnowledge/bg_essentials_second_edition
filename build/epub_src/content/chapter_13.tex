\chapter{Kṣetra-kṣetrajña-vibhāga-yoga}\label{chap-ksetra-ksetrajna-vibhaga-yoga}
\chaptersubtitle{The Distinction of Matter and Spirit}
%%%%%%%%%%%%%%%%%%%%%%%%%%%%%%%%%%%%%%%%%%%%%%%%%%%%%%%%%

\noindent At this point, Kṛṣṇa makes another dramatic shift in His teaching. These final six chapters provide supplemental teaching, which connects and expands upon the main discourse already given in the \textit{Gītā}. Arjuna is now given the grounding knowledge needed to walk a spiritual path. Chapter thirteen is a further development on \textit{jñāna-yoga}, and how one can realize oneself as the \textit{ātmā}.

The chapter begins with the Lord elaborating on the relationship between the Field (\textit{kṣetra}) and the knower of the field (\textit{kṣetrajña}). This \textit{kṣetra} is the body made up of material matter. The \textit{kṣetrajña} is of two types, the inner Self and the Supersoul. The latter is Kṛṣṇa Himself who exists alongside the inner Self as the Witness and Maintainer of life. The Lord then describes the various elements and the different modifications which make up and affect the Field.

Next, Kṛṣṇa explains the state of one who has real knowledge. Qualities such as equanimity, detachment, and freedom from ego are all symptoms of one who is truly realized.

Having established the different components of the Field, who the knowers of the field are, and the nature of one who has knowledge, Kṛṣṇa then describes what we should strive to know: the unfathomable, all-pervasive Lord.

He further points out that both material nature and the Self are beginningless, and as long as the Self is entangled with it, it undergoes suffering and enjoyment. But the Paramātmā (Supersoul) is also present within. Seekers perceive it in various different ways. Those who have gained this perception behold the Supersoul everywhere and in all living beings. They also recognize that all action is performed within the material world and the inner Self is completely untouched. The one who is able to grasp this truth of the \textit{kṣetra} and \textit{kṣetrajña} finally becomes liberated.

\section{Verses 1-7: The Field and Its Knower}\label{sec-verses-1-7-the-field-and-its-knower}



\begin{html}
<div class="verse" data-ref="13.1">
  <span class="verse-ref">13.1</span>
  <div class="sa">
    arjuna uvāca<br/>
    prakṛtiṁ puruṣaṁ caiva kṣetraṁ kṣetrajñam eva ca<br/>
    etad veditum icchāmi jñānaṁ jñeyaṁ ca keśava
  </div>
  <div class="en">
    \hspace*{1em}Arjuna said: \\ O Keśava, I wish to know about \textit{puruṣa} and about \textit{prakṛti}, about the \enquote{field} and its knower, about knowledge and the object of knowledge.
  </div>
</div>
\end{html}



\begin{html}
<div class="verse" data-ref="13.2">
  <span class="verse-ref">13.2</span>
  <div class="sa">
    śrī-bhagavān uvāca<br/>
    idaṁ śarīraṁ kaunteya kṣetram ity abhidhīyate<br/>
    etad yo vetti taṁ prāhuḥ kṣetrajña iti tad-vidaḥ
  </div>
  <div class="en">
    \hspace*{1em}Bhagavān Kṛṣṇa said: \\ This body, O Kaunteya, is called the \enquote{field.} One who has knowledge of it is called \enquote{knower of the field} by the learned.
  </div>
</div>
\end{html}



\begin{html}
<div class="verse" data-ref="13.3">
  <span class="verse-ref">13.3</span>
  <div class="sa">
    kṣetrajñaṁ cāpi māṁ viddhi sarva-kṣetreṣu bhārata<br/>
    kṣetra-kṣetrajñayor jñānaṁ yat taj jñānaṁ mataṁ mama
  </div>
  <div class="en">
    O descendant of Bharata, understand that I am the knower of the field in all fields and also know that, in My view, true knowledge is the knowledge regarding the field and its knower.
  </div>
</div>
\end{html}



\begin{html}
<div class="verse" data-ref="13.4">
  <span class="verse-ref">13.4</span>
  <div class="sa">
    tat kṣetraṁ yac ca yādṛk ca yad-vikāri yataś ca yat<br/>
    sa ca yo yat-prabhāvaś ca tat samāsena me śṛṇu
  </div>
  <div class="en">
    Hear briefly from Me about the field, what its nature is, what modification it undergoes, from where each modification arises, as well as about the knower of the field and His power.
  </div>
</div>
\end{html}


\Verse[13.5]
{ṛṣibhir bahudhā gītaṁ chandobhir vividhaiḥ pṛthak \\
brahma-sūtra-padaiś caiva hetumadbhir viniścitaiḥ}
{This knowledge has been sung about differently by various sages in multiple ways through hymns and also through conclusive arguments of the \textit{Brahma-sūtra}.\footnote{A philosophical scripture that summarizes the spiritual ideas of the \textit{Upaniṣads}.}}

\Verse[13.6–7]
{mahā-bhūtāny ahaṅkāro buddhir avyaktam eva ca \\
indriyāṇi daśaikaṁ ca pañca cendriya-gocarāḥ \\
icchā dveṣaḥ sukhaṁ duḥkhaṁ saṅghātaś cetanā dhṛtiḥ \\
etat kṣetraṁ samāsena sa-vikāram udāhṛtam}
{The five principal elements, the ego, the intellect, the primordial material substance,\footnote{The unmanifest state of \textit{prakṛti} not yet differentiated into the various forms of material nature.} the ten senses,\footnote{The \textit{jñānendriyas}, which are the five modes of perception (sight, hearing, touch, taste, smell), and the \textit{karmendriyas}, which are the five organs of action (hands, feet, voice, anus, and genitals).} the mind, the five objects of the senses,\footnote{Form/color, sound, touch sensations, flavor, and aroma} desire, aversion, pleasure and pain, the body itself, consciousness and one’s resolve---in brief this is described as the field along with its modifications.}

\enquote{Here Bhagavān is saying, \enquote{\ldotsMy dear Arjuna, I know you have been confused with My cosmic form. You have heard Me talking about so many things. I know you are confused, but at the same time you want to know. I will reveal to you in a very simple way, in an easy way, the truth about the purpose of having a human body, the greatness of a human body, and why you have received this human body. I will reveal to you the true identity of the Supersoul, which is unlimited, birthless, eternal, whereas this body and everything which is created is the opposite. The body itself goes through constant changes. At all times you are getting old and at all times this body is moving towards its end.}}

\enquote{The body is like a field. When a farmer sows seeds with time the seeds start to grow. In this body, the seeds you have sown in your past lives bear fruit in this life. This body is the field where the seeds of \textit{karma} you harvest reap the fruit of your actions in their due time.}

\enquote{\enquote{\textit{Kṣetra}} means also perishable, so everything that is linked to \textit{kṣetra} is everything that is matter, tangible, everything that decays, impermanent, perishable.}

\enquote{The knower of the field is the \enquote{gardener,} the Observer, the \textit{ātmā}, the One who observes life from deep within. When you sit in meditation, who perceives everything? There is an Observer observing how the mind itself is working and observing the body. There is somebody else inside of you. You are not that mind, and you are not the body. The \textit{ātmā}, the Great Observer of life is \textit{kṣetrajña}. This is the Great Observer that observes the game of life, observes the game where the mind is diverted to the object of senses, observes the senses and so on. But yet it is completely separate from everything it observes; it is not a part of it.}

\enquote{The ones who realize this great mystery rise above everything; they rise above gender, and above nature itself. This \textit{ātmā} is not bound by transient limitation; it is not bound by anything that one perceives. It is genderless, changeless, without modification, eternal. When one realizes it, one enters the consciousness of this \textit{ātmā}, and the essence of the Divine starts manifesting.}

\enquote{Here Bhagavān Kṛṣṇa says to Arjuna, \enquote{Understand Me, I am the Great Observer of life, I am this Para-Puruṣa. I am this Paramātmā, the great soul, the Supersoul.}}


\begin{html}
<div class="verse" data-ref="13.8–12">
  <span class="verse-ref">13.8–12</span>
  <div class="sa">
    amānitvam adambhitvam ahiṁsā kṣāntir ārjavam<br/>
    ācāryopāsanaṁ śaucaṁ sthairyam ātma-vinigrahaḥ<br/>
    indriyārtheṣu vairāgyam anahaṅkāra eva ca<br/>
    janma-mṛtyu-jarā-vyādhi-duḥkha-doṣānudarśanam<br/>
    asaktir anabhiṣvaṅgaḥ putra-dāra-gṛhādiṣu<br/>
    nityaṁ ca sama-cittatvam iṣṭāniṣṭopapattiṣu<br/>
    mayi cānanya-yogena bhaktir avyabhicāriṇī<br/>
    vivikta-deśa-sevitvam aratir jana-saṁsadi<br/>
    adhyātma-jñāna-nityatvaṁ tattva-jñānārtha-darśanam<br/>
    etaj jñānam iti proktam ajñānaṁ yad ato ’nyathā
  </div>
  <div class="en">
    Freedom from pride, sincerity, nonviolence, patience, integrity, service to the \textit{guru}, purity, steadfastness, and self-restraint; dispassion for sense objects, absence of egotism, perceiving the deficiency of birth, death, old age, and suffering; non-attachment, lack of attachment toward son, spouse, home, and other such things; constant equanimity in desirable or undesirable circumstances, undeviating devotion with exclusive absorption in Me; a preference for solitary places, aversion from mundane society; being ever established in knowledge of the Self and contemplation of the goal of spiritual knowledge---all of this is declared to be knowledge; whatever is contrary to that is ignorance.
  </div>
</div>
\end{html}


\enquote{When one becomes spiritual, one enters the heart, and one perceives the divinity inside. The body becomes a temple, where you come face to face with God. Here, inside you, runs the Gaṅgā, the Yamunā, and the Sarasvatī rivers\ldots your body is this place of solitude; it is this mountain, this wonderful nature that we look at---like when you look outside the \textit{āśrama} here, you see the beautiful valley. When a \textit{bhakta} perceives this beauty inside themselves and enters this calm state within the body, they perceive these three rivers, Gaṅgā, Yamunā, and Sarasvatī, flowing inside them. Even with all the noise outside, they’re drawn towards the sound of the water, they are drawn to the cosmic sound inside them, and they enter this deepness residing inside. One achieves an inner peace, an inner tranquility, a holy atmosphere, and if one enters this state, it has a deep impact on the body. And such is this impact, that it is said that, at that moment, there is such a beam that, in a state of complete focus and \textit{śānti}, peace, there is also a cessation of disease. In the state of complete calmness, all germs, all illnesses, are removed from the body. For some saints, there comes a point that the body itself stops aging. For example, at the time of the war He was partaking in, Bhagavān Kṛṣṇa Himself was a great-grandfather, but yet He appeared only 25 years old. He was not bound by age. He was going through the \textit{līlā}, and when the \textit{līlā} was complete, when the body had reached a certain age, He kept it from changing. So the aging process stopped.}

\enquote{With the true knowledge, one knows that this body goes through changes and that one is the \textit{ātmā}. The \textit{ātmā} is eternal and it constantly longs for the Lord. The \textit{ātmā} constantly wants to attain Him, to be with Him, to return home until one enters a deep state of \textit{samādhi}. Then, in this state one enters the universal and eternal presence of the Lord. One attains the inner vision, the \textit{darśana} of the Lord within oneself\ldots In this state all pride, all arrogance, all hypocrisy, everything disappears. And this is the state that the \textit{bhakta} achieves, just by loving the Lord, just by serving Him.}

\section{Verses 13-19: The Highest Object of Knowledge}\label{sec-verses-13-19-the-highest-object-of-knowledge}


\begin{html}
<div class="verse" data-ref="13.13">
  <span class="verse-ref">13.13</span>
  <div class="sa">
    jñeyaṁ yat tat pravakṣyāmi yaj jñātvāmṛtam aśnute<br/>
    anādimat paraṁ brahma na sat tan nāsad ucyate
  </div>
  <div class="en">
    I shall explain that which is to be known, realizing which one attains immortality---the beginningless Parabrahman; it is said to be neither manifest nor unmanifest.
  </div>
</div>
\end{html}



\begin{html}
<div class="verse" data-ref="13.14">
  <span class="verse-ref">13.14</span>
  <div class="sa">
    sarvataḥ pāṇi-pādaṁ tat sarvato ’kṣi-śiro-mukham<br/>
    sarvataḥ śrutimal loke sarvam āvṛtya tiṣṭhati
  </div>
  <div class="en">
    His hands and feet are everywhere; His eyes, heads, and faces are everywhere; His ears are on all sides; and He exists, pervading all things.
  </div>
</div>
\end{html}



\begin{html}
<div class="verse" data-ref="13.15">
  <span class="verse-ref">13.15</span>
  <div class="sa">
    sarvendriya-guṇābhāsaṁ sarvendriya-vivarjitam<br/>
    asaktaṁ sarva-bhṛc caiva nirguṇaṁ guṇa-bhoktṛ ca
  </div>
  <div class="en">
    He illumines the senses and sense objects and yet is devoid of senses; He is unattached and yet the maintainer of all; He is beyond the \textit{guṇas} and yet the experiencer of the \textit{guṇas}.
  </div>
</div>
\end{html}



\begin{html}
<div class="verse" data-ref="13.16">
  <span class="verse-ref">13.16</span>
  <div class="sa">
    bahir antaś ca bhūtānām acaraṁ caram eva ca<br/>
    sūkṣmatvāt tad avijñeyaṁ dūra-sthaṁ cāntike ca tat
  </div>
  <div class="en">
    He is inside and outside all beings, moving yet unmoving. Being so subtle, that Supreme Being is incomprehensible: far away, and yet so near.
  </div>
</div>
\end{html}



\begin{html}
<div class="verse" data-ref="13.17">
  <span class="verse-ref">13.17</span>
  <div class="sa">
    avibhaktaṁ ca bhūteṣu vibhaktam iva ca sthitam<br/>
    bhūta-bhartṛ ca taj jñeyaṁ grasiṣṇu prabhaviṣṇu ca
  </div>
  <div class="en">
    That Supreme Being is undivided and yet exists as if divided among beings. He is the sustainer of all beings, their destroyer, and creator---He is that which is to be known.
  </div>
</div>
\end{html}



\begin{html}
<div class="verse" data-ref="13.18">
  <span class="verse-ref">13.18</span>
  <div class="sa">
    jyotiṣām api taj jyotis tamasaḥ param ucyate<br/>
    jñānaṁ jñeyaṁ jñāna-gamyaṁ hṛdi sarvasya viṣṭhitam
  </div>
  <div class="en">
    That Supreme Being is the light of lights, beyond all darkness. He is knowledge, the object of knowledge, the goal of knowledge, and is situated in the hearts of all.
  </div>
</div>
\end{html}



\begin{html}
<div class="verse" data-ref="13.19">
  <span class="verse-ref">13.19</span>
  <div class="sa">
    iti kṣetraṁ tathā jñānaṁ jñeyaṁ coktaṁ samāsataḥ<br/>
    mad-bhakta etad vijñāya mad-bhāvāyopapadyate
  </div>
  <div class="en">
    In this way, I have briefly explained the field, knowledge, and the object of knowledge. Upon realizing this, My devotee attains My nature.
  </div>
</div>
\end{html}


\enquote{Kṛṣṇa is indicating that even if everything comes out from Him, He is still superior. God (as the Supersoul) is still superior to the soul itself. This is equivalent to the example I gave of the mighty ocean and the droplet. They are the same in their qualities, but in quantity the ocean is far superior to a droplet. The droplet can’t claim to be the ocean! The ocean is mighty. Bhagavān says here, \enquote{This individual soul is just a spark of Myself. I am superior to this; I am the whole mighty ocean itself.}}

\enquote{The minds of men think, \enquote{I am praying to God. When God has time, He will listen to me,} or \enquote{When He is not busy, then He will connect and fulfill the prayer.} No, through His innumerable ears, He hears all praises. When you pray, when you ask, He can hear everything, because He is seated deep inside everyone.}

\enquote{The duty of man is to realize God, to realize the Divine within themselves and to attain Him. He is the Light of all; He is the knowledge itself, and the object of knowledge. So whatever you do in life is only to serve Him, to attain Him. There is not a single moment in your life that He is not present. This is why I said, \enquote{Do your \textit{dharma}!} When you do your \textit{dharma} properly, you embrace this knowledge; you’re attuned to it. When you learn to accept life, and whatever God gives you in life, wholeheartedly, this is the object of knowledge that God has given you to attain perfection. Without this, you can’t attain perfection. If you fight what He has given you as an object to attain Him, you will not be able to do your \textit{dharma} in life. Then it is far away. Even one who appears the nearest and dearest to you will appear far and very difficult to attain. But when you embrace what He gives you at every moment, you will see how easy it is. He is seated in the hearts of men. That is beautiful!}

\enquote{Even though the Lord is present everywhere, in everything---everything is His manifestation---yet the heart of man is a different reality; it is more superior than any other manifestation of Him. Bhagavān is saying that the heart is very special. Your heart is not just a heart. For example, the sun is present everywhere. But when the sun’s rays reflect on a mirror, the light is different: it is greatly intensified and it is stronger. Even if the rays of the sun are equally present everywhere, when there is a cloud, the sun pierces through the cloud. However, when the sun’s rays hit a mirror or a magnifying glass, the light gets intensified. Normally the sun would not burn anything. But if it reaches a certain angle on a magnifying glass and becomes intensified, then it burns. It is the same with the hearts of men. The Lord is present everywhere, but in the heart of human beings His Light shines more brightly. The heart of man is the seat of the Lord; if He is seated in the heart, He becomes greatly intensified. The heart of man is the manifestation of God. It is there that one can feel Him, meet Him, speak with Him. This heart, which is inside the body, is unique; this is what makes human beings unique.}

\section{Verses 20-24: The Self Within Material Nature}\label{sec-verses-20-24-the-self-within-material-nature}


\begin{html}
<div class="verse" data-ref="13.20">
  <span class="verse-ref">13.20</span>
  <div class="sa">
    prakṛtiṁ puruṣaṁ caiva viddhy anādī ubhāv api<br/>
    vikārāṁś ca guṇāṁś caiva viddhi prakṛti-sambhavān
  </div>
  <div class="en">
    Know that both \textit{prakṛti} and \textit{puruṣa} are beginningless, and that the \textit{guṇas} and all transformations originate from \textit{prakṛti}.
  </div>
</div>
\end{html}



\begin{html}
<div class="verse" data-ref="13.21">
  <span class="verse-ref">13.21</span>
  <div class="sa">
    kārya-kāraṇa-kartṛtve hetuḥ prakṛtir ucyate<br/>
    puruṣaḥ sukha-duḥkhānāṁ bhoktṛtve hetur ucyate
  </div>
  <div class="en">
    Agency in relation to cause and effect is caused by material nature, while the sense of experiencing joy and sorrow is caused by the conscious Self.
  </div>
</div>
\end{html}



\begin{html}
<div class="verse" data-ref="13.22">
  <span class="verse-ref">13.22</span>
  <div class="sa">
    puruṣaḥ prakṛti-stho hi bhuṅkte prakṛti-jān guṇān<br/>
    kāraṇaṁ guṇa-saṅgo ’sya sad-asad-yoni-janmasu
  </div>
  <div class="en">
    Situated in \textit{prakṛti}, the individual Self experiences the \textit{guṇas} arising from \textit{prakṛti}. For the Self, attachment to these \textit{guṇas} is the cause of birth in superior or inferior species.
  </div>
</div>
\end{html}



\begin{html}
<div class="verse" data-ref="13.23">
  <span class="verse-ref">13.23</span>
  <div class="sa">
    upadraṣṭānumantā ca bhartā bhoktā maheśvaraḥ<br/>
    paramātmeti cāpy ukto dehe ’smin puruṣaḥ paraḥ
  </div>
  <div class="en">
    Within this body, there also exists the supreme Puruṣa, said to be the witness, the One who grants permission, the supporter, the enjoyer, the great Lord, the Paramātmā.
  </div>
</div>
\end{html}



\begin{html}
<div class="verse" data-ref="13.24">
  <span class="verse-ref">13.24</span>
  <div class="sa">
    ya evaṁ vetti puruṣaṁ prakṛtiṁ ca guṇaiḥ saha<br/>
    sarvathā vartamāno ’pi na sa bhūyo ’bhijāyate
  </div>
  <div class="en">
    One who knows \textit{puruṣa} and \textit{prakṛti} along with its \textit{guṇas} in this way, is not born again, whatever situation one may be in.
  </div>
</div>
\end{html}


\enquote{People say this world is an illusion, that nothing is existent. This is not true. How can there be no existence when everything here is manifested? With which power does it manifest? With the \textit{māyā} of the Lord Himself, so it is not an illusion. Even the air has molecules. And these molecules are a manifestation; there’s a molecule of the Divine Himself in each of them. So, why is the world considered an illusion? It is an illusion because you don’t see the divinity. You don’t perceive divinity through the mind, in the way you see things. However, behind this \enquote{illusion} is the Lord’s energy, which pervades everything. This is why it is not an illusion---this \enquote{illusion} is filled with the Lord Himself. A realized soul perceives that everything around them is the potency and the manifestation of the Lord. Not just merely an illusion. What would be the use of devotion? What would be the use of going on a spiritual path? It would become useless. But people always say one should see this as an illusion. It is true that it is an illusion, but behind the illusion is the Lord who pervades that illusion. And this is what a \textit{bhakta} who loves the Lord and sees the Lord everywhere will see: that in each atom it is only Him.}

\enquote{Due to the \textit{guṇas}, you are born in a certain category. \textit{Bhaktas} are born in high categories. When you are born, the soul has only one aim: to attain the Lord. No matter how the world is, how it rotates, how strong the illusion of the world might be, how strong the surroundings themselves can be, a \textit{bhakta} is born as a \textit{bhakta}. They carry on from their previous life performing their devotional practices. As I said earlier, a saint is born a saint due to his past life. There is evolution, but one attains the state of sainthood in this life due to \textit{bhakti}, and this is through grace. If one is attached to the \textit{guṇas}, due to these \textit{guṇas}, one takes a certain aspect in life and manifests it here. And \textit{prakṛti} makes this manifestation possible so that one can cleanse and evolve towards God-realization. Even those who are far away from Light, or carry very heavy \textit{karma} with them, or are born in an evil womb, are constantly advancing towards God-realization.}

\enquote{Sometimes you are born due to a certain \textit{karma}, but yet, for you to become a devotee, it’s not a coincidence. The Lord placed you where you have to finish certain \textit{karma} with other people. Once you come on the spiritual path, once this love and longing of realizing yourself awakens, everything changes. Somebody asked me last time, \enquote{Swamiji, why have you waited so long to call me? Why now? I’m old, I can’t really serve you. You should have called me earlier.} If I had called you earlier, you would not even want to know who I am. Because what makes you a devotee, what called you on the spiritual path, was not yet awakened inside of you. Once that call awakens inside of you, that’s when you come on the spiritual path. So, it is always at the right time.}

\enquote{The moment one attains realization, one doesn’t have anything to do with \textit{prakṛti}, which means that, at that moment, all creation of \textit{karma} ceases. Due to this divorce from \textit{prakṛti}, such a soul is not bound by ignorance. If one is not bound by the ignorance of separateness, such a realized soul perceives the Supreme Lord everywhere, in everything. \textit{Prakṛti} doesn’t have a grip on this person so there is no creation of \textit{karma}. When you do your Atma Kriya Yoga, it makes you dwell in another consciousness, or awareness.}

\enquote{The \textit{ātmā} goes through endless births until the \textit{ātmā} realizes itself. But just one realization can erase all the \textit{karmas} of many lives. This is what Atma Kriya Yoga does! When one practices Atma Kriya Yoga, one cleanses \textit{karmas} that have been created through many, many lives, as well as the \textit{karma} that is being created now---it comes up and Atma Kriya Yoga makes it disappear. Finished! It helps one to free oneself fully from the cycle of birth and death.}

\section{Verses 25-35: Different Ways to Perceive the Supersoul}\label{sec-verses-25-35-different-ways-to-perceive-the-supersoul}


\begin{html}
<div class="verse" data-ref="13.25">
  <span class="verse-ref">13.25</span>
  <div class="sa">
    dhyānenātmani paśyanti kecid ātmānam ātmanā<br/>
    anye sāṅkhyena yogena karma-yogena cāpare
  </div>
  <div class="en">
    Some perceive the \textit{ātmā} by the mind within themselves through meditation, others through the path of analytical knowledge, and still others by \textit{karma-yoga}.
  </div>
</div>
\end{html}



\begin{html}
<div class="verse" data-ref="13.26">
  <span class="verse-ref">13.26</span>
  <div class="sa">
    anye tv evam ajānantaḥ śrutvānyebhya upāsate<br/>
    te ’pi cātitaranty eva mṛtyuṁ śruti-parāyaṇāḥ
  </div>
  <div class="en">
    Others, however, who are unaware of these paths, worship based on what they hear from others. Dedicated to what has been heard, they too pass beyond death.
  </div>
</div>
\end{html}



\begin{html}
<div class="verse" data-ref="13.27">
  <span class="verse-ref">13.27</span>
  <div class="sa">
    yāvat sañjāyate kiñcit sattvaṁ sthāvara-jaṅgamam<br/>
    kṣetra-kṣetrajña-saṁyogāt tad viddhi bharatarṣabha
  </div>
  <div class="en">
    O best of the descendants of Bharata, whatever comes into existence, whether immovable or movable, know it to arise from the union of the field and the knower of the field.
  </div>
</div>
\end{html}



\begin{html}
<div class="verse" data-ref="13.28">
  <span class="verse-ref">13.28</span>
  <div class="sa">
    samaṁ sarveṣu bhūteṣu tiṣṭhantaṁ parameśvaram<br/>
    vinaśyatsv avinaśyantaṁ yaḥ paśyati sa paśyati
  </div>
  <div class="en">
    Whoever perceives the Supreme Lord residing equally in all beings, as the imperishable among the perishable, truly sees.
  </div>
</div>
\end{html}



\begin{html}
<div class="verse" data-ref="13.29">
  <span class="verse-ref">13.29</span>
  <div class="sa">
    samaṁ paśyan hi sarvatra samavasthitam īśvaram<br/>
    na hinasty ātmanātmānaṁ tato yāti parāṁ gatim
  </div>
  <div class="en">
    Equally seeing the Lord residing everywhere, one does not harm the Self by the mind and therefore attains the supreme goal.
  </div>
</div>
\end{html}



\begin{html}
<div class="verse" data-ref="13.30">
  <span class="verse-ref">13.30</span>
  <div class="sa">
    prakṛtyaiva ca karmāṇi kriyamāṇāni sarvaśaḥ<br/>
    yaḥ paśyati tathātmānam akartāraṁ sa paśyati
  </div>
  <div class="en">
    One who sees that all actions are performed solely by \textit{prakṛti} and that the \textit{ātmā} is a non-doer, truly sees.
  </div>
</div>
\end{html}



\begin{html}
<div class="verse" data-ref="13.31">
  <span class="verse-ref">13.31</span>
  <div class="sa">
    yadā bhūta-pṛthag-bhāvam eka-stham anupaśyati<br/>
    tata eva ca vistāraṁ brahma sampadyate tadā
  </div>
  <div class="en">
    When one perceives that the existence of distinct beings is rooted in the One and emanates from it, one attains Brahman.
  </div>
</div>
\end{html}



\begin{html}
<div class="verse" data-ref="13.32">
  <span class="verse-ref">13.32</span>
  <div class="sa">
    anāditvān nirguṇatvāt paramātmāyam avyayaḥ<br/>
    śarīra-stho ’pi kaunteya na karoti na lipyate
  </div>
  <div class="en">
    O Kaunteya, due to being beginningless and beyond the \textit{guṇas}, the unchanging Paramātmā neither acts nor is He stained by action, despite dwelling within the body,
  </div>
</div>
\end{html}



\begin{html}
<div class="verse" data-ref="13.33">
  <span class="verse-ref">13.33</span>
  <div class="sa">
    yathā sarva-gataṁ saukṣmyād ākāśaṁ nopalipyate<br/>
    sarvatrāvasthito dehe tathātmā nopalipyate
  </div>
  <div class="en">
    Just as the all-pervading space remains unaffected due to its subtlety, the Self present everywhere in the body is not tainted by the body.
  </div>
</div>
\end{html}



\begin{html}
<div class="verse" data-ref="13.34">
  <span class="verse-ref">13.34</span>
  <div class="sa">
    yathā prakāśayaty ekaḥ kṛtsnaṁ lokam imaṁ raviḥ<br/>
    kṣetraṁ kṣetrī tathā kṛtsnaṁ prakāśayati bhārata
  </div>
  <div class="en">
    Just as the sun alone illumines this whole world, so too does the knower of the field illumine the whole field, O descendant of Bharata.
  </div>
</div>
\end{html}



\begin{html}
<div class="verse" data-ref="13.35">
  <span class="verse-ref">13.35</span>
  <div class="sa">
    kṣetra-kṣetrajñayor evam antaraṁ jñāna-cakṣuṣā<br/>
    bhūta-prakṛti-mokṣaṁ ca ye vidur yānti te param
  </div>
  <div class="en">
    Those who perceive the distinction between the field and the knower of the field in this way through the eye of wisdom, as well as how living beings can be liberated from material nature, attain the Supreme.
  </div>
</div>
\end{html}


\enquote{Those who don’t have time to sit and do \textit{japa}, their \textit{sādhana}, their \textit{yoga}, their meditation, yet they hear the glories of the Lord, hear of His \textit{līlā}, hear about the greatness of the \textit{bhaktas} of the Lord---with faith and concentration, not superficially but with the absorption of a \enquote{sponge}---will also be free. See how merciful He is?}

\enquote{The heart of the \textit{bhakta} is like the sun shining on a clear mirror, reflecting the light onto others. The Lord is seated equally inside all hearts, yet not everyone perceives this. However, the realized soul, the one who has attained God-realization, perceives the same Lord everywhere, even in the one who is not realized. This is why the saintly evolved beings, who have attained God-realization, don’t judge anybody.}

\enquote{Those who have this knowledge of the Self, those who have this knowledge to move towards God consciousness, are free; they will achieve this state. And those who do not have this knowledge or lack this equal vision are constantly in judgment; they constantly criticize each other. They are very friendly to somebody who does good towards them, and they reject those and are unfriendly to those who are mean to them. Such people have to undergo this repetition of birth and death. Bhagavān is saying to everyone, \enquote{Don’t! Break through! See this unity! Awaken this universal Love! Love how I love you! And do everything in this attitude as a service to the Lord, remembering God at all times.} This will free oneself. Especially when you take a spiritual path, one has to love.}

\enquote{When one starts saying, \enquote{I did this! I did that!,} this big \enquote{I} arises, and at that moment there is creation of \textit{karma}; the mind becomes very active, alert and very possessive.}

\enquote{The ones who have realized themselves are detached from \textit{prakṛti}. You walk, but it’s not you who is walking. You have the awareness that it is the body that is walking. When you die, it’s not you who dies. It is the \textit{ātmā} that leaves the body. You don’t die. In every act and every move done with such a realization, you are always in connection with the supreme reality. You are always swimming in the ocean of that bliss. Everything that you do will be infused with that bliss, infused with that knowledge of the Self, infused with that Love. Here Bhagavān is saying, \enquote{Enter into that supreme reality within yourself. Meet Me there. Know that everything else is just an act}\ldots like the light, the bulb is just \textit{prakṛti}, the light is the reality. You can feel the light---you can feel the electricity even without the light. You touch, you put your little finger in the socket, you will feel it. You don’t need to see if the light is on or not.}

\enquote{So, when that truth shines, the seers start to see. That’s why when people are around a Self-realized person, their \textit{karma} starts to activate very quickly, because that \enquote{observer} that is inside of them also wants to reach that same level. That’s what purification is. Association with those who love God will purify others. This is the duty of the \textit{bhaktas} of the Lord: to let the Light of the Lord shine through them to this world.}