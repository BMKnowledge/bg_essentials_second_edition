\chapter{Puruṣottama-yoga}\label{chap-purusottama-yoga}
\chaptersubtitle{Knowing the Supreme Person}
%%%%%%%%%%%%%%%%%%%%%%%%%%%%%%%%%%%%%%%%%%%%%%%%%%%%%%%%%

\noindent The last two chapters have primarily explained the nature of the material world in relation to the \textit{ātmā}. Now Kṛṣṇa will focus on the Supreme Lord who is beyond both matter and the soul.

The opening verses compare bondage to the material world to being entangled in an upside-down banyan tree. With the weapon of detachment, one has to cut through the endless branches, roots, and leaves, and only then can we reach the Lord’s Abode. Kṛṣṇa states that the soul is an eternal fragment of Himself that enters creation. It moves from body to body, carrying with it the five senses and the mind. Only those with true knowledge are able to perceive it within.

Having described the soul’s position, Kṛṣṇa explains how He as the Lord sustains creation in various ways. He is the energy of the Moon, the fire that digests food. He is the one who resides in the hearts of all beings and is the knower of the Vedas.

The individual soul has two states: one that is in bondage (\textit{kṣara}) and one that has risen above the material world (\textit{akṣara}), considered to be a state of Self-realization; but Kṛṣṇa is beyond both. He is known as the Paramātmā, the Lord who pervades creation and sustains it. The \textit{Vedas} also celebrate Him as Puruṣottama, the highest Personality. Those who understand Kṛṣṇa as such, knowing all there is to know, have reached a state of God-realization. They worship Him with their entire being.

\section{Verses 1-4: The Tree of Life}\label{sec-verses-1-4-the-tree-of-life}

\Verse[15.1]
{\hspace*{1em}śrī-bhagavān uvāca \\
ūrdhva-mūlam adhaḥ-śākham aśvatthaṁ prāhur avyayam \\
chandāṁsi yasya parṇāni yas taṁ veda sa veda-vit}
{\hspace*{1em}Bhagavān Kṛṣṇa said: \\
They speak of an imperishable Aśvattha tree having roots above and branches below, whose leaves are the \textit{Vedic} hymns---one who knows that tree knows the \textit{Vedas}.}

\Verse[15.2]
{adhaś cordhvaṁ prasṛtās tasya śākhā guṇa-pravṛddhā viṣaya-pravālāḥ \\
adhaś ca mūlāny anusantatāni karmānubandhīni manuṣya-loke}
{Nourished by the \textit{guṇas}, its branches extend both above and below, with sense objects as shoots. The roots, which bind to action, extend downward into the human world.}

\Verse[15.3–4]
{na rūpam asyeha tathopalabhyate nānto na cādir na ca sampratiṣṭhā \\
aśvattham enaṁ suvirūḍha-mūlam asaṅga-śastreṇa dṛḍhena chittvā \\
tataḥ padaṁ tat parimārgitavyaṁ yasmin gatā na nivartanti bhūyaḥ \\
tam eva cādyaṁ puruṣaṁ prapadye yataḥ pravṛttiḥ prasṛtā purāṇī}
{Its form as described cannot be perceived in this world---neither its beginning, end, nor foundation. After cutting down this deep-rooted Aśvattha tree with the firm weapon of detachment, thereafter that abode should be sought, from which, once attained, one does not return, declaring, \enquote{I surrender to the primordial Lord, from whom this ancient cosmic manifestation has emanated.}}

\enquote{Here Bhagavān says that as long as this tree of creation which is firmly rooted, very strong and very powerful---is not destroyed, if it is not cut down with the sword of detachment, dispassion, one will never be free. One will never seek the highest goal; one stays blind. Until one develops detachment, one will always remain blind. Bhagavān is saying, \enquote{Awake!}}

\enquote{If we dwell only in the outside world, it is useless; it doesn’t bring anything. Someone who doesn’t have any knowledge of spirituality, who is not longing for the Divine, is rooted completely in the world outside, and they become like the trees that have a beginning and an end. They identify themselves with the body! They identify themselves with the \textit{guṇas}! They don’t think that what one goes through is only short-term happiness, and everlasting suffering. You are happy for a maximum of an hour, and then, it’s over! Bhagavān said, \enquote{Find eternal happiness!} And this eternal happiness is not outside everywhere, in the things outside, it is inside you. And when you are fully attached to the Divine, even this one hour can become very long; you enjoy the Supreme Lord at all times; you are not attached to outside things only, but are fully focused, fully rooted in divine consciousness; then the outside doesn’t matter.}

\enquote{Through dispassion one advances towards God-realization. These qualities that awaken, that one attains, through devotion and surrender, cause one not to return into this world. Who else can give this supreme blissful state to you? Can another human being give this to you? No, they can’t!}

\enquote{The human being varies. You see this in your life: when you’re freshly in love, love is sublime; you’re so blind that you don’t see anything else. But after a month, when you start to see again, it’s not the same. At the beginning it’s beautiful! But after a month, two months, three months, a year, you only need one little thing to happen to make you go crazy. And with that little thing, all the love you have disappears.}

\enquote{Bhagavān doesn’t tell you that you should not live your life on the outer world. You can live your life, with your husband, wife, children, and so on, but be in a state of God consciousness. Be aware of God in your life. Know where you stand and what your goal is---this is very important! If being in this world you don’t know the goal you want to reach, it’s impossible to be happy. You will be like these ignorant people who go through life like animals. They don’t even know there is a life inside them. And they don’t know there is life after death! But one day they will come back again and realize it.}


\section{Verses 5-6: Reaching the Highest Realm}\label{sec-verses-5-6-reaching-the-highest-realm}

\Verse[15.5]
{nirmāna-mohā jita-saṅga-doṣā adhyātma-nityā vinivṛtta-kāmāḥ \\
dvandvair vimuktāḥ sukha-duḥkha-saṁjñair gacchanty amūḍhāḥ padam avyayaṁ tat}
{The undeluded ones who are free from pride and delusion, who have conquered the defect of attachment and are constantly devoted to the spiritual path, who have rejected selfish desires and are liberated from the dualities of pleasure and pain, attain that imperishable abode.}

\Verse[15.6]
{na tad bhāsayate sūryo na śaśāṅko na pāvakaḥ \\
yad gatvā na nivartante tad dhāma paramaṁ mama}
{That realm is not illumined by the sun or the moon, nor by fire. Having reached that supreme abode of Mine, one does not return again.}

\enquote{To be free from this \textit{aparā-prakṛti}, this lower nature, one has to detach oneself and be free from egoism. One has to be free from attachment, free from all kinds of desires, free from \enquote{the duality of joy and grief,} the pairs of opposites: agreeable or disagreeable; honor or dishonor; praise or blame; joy or sorrow; these qualities have many names. Those who transcend these are not enslaved by this attraction and aversion, and so they are free. They attain Enlightenment.}

\enquote{When a \textit{sādhu}, a saint, touches someone, they are transformed. The saint becomes like the philosopher’s stone, which by touching metal, transforms it into gold. People become like this when they are free. They radiate light; wherever they go, they disperse darkness.}

\section{Verses 7-15: The Soul Is an Eternal Fragment of Kṛṣṇa}\label{sec-verses-7-15-the-soul-is-an-eternal-fragment-of-krsna}

\Verse[15.7]
{mamaivāṁśo jīva-loke jīva-bhūtaḥ sanātanaḥ \\
manaḥ-ṣaṣṭhānīndriyāṇi prakṛti-sthāni karṣati}
{An eternal particle of Myself becomes embodied as the \textit{jīva} in the world of mortal beings. It takes on the six senses, including the mind, which are situated in \textit{prakṛti}.}

\Verse[15.8]
{śarīraṁ yad avāpnoti yac cāpy utkrāmatīśvaraḥ \\
gṛhītvaitāni saṁyāti vāyur gandhān ivāśayāt}
{When the \textit{ātmā} departs from one body and acquires another, it leaves, carrying along these six senses, just as the wind carries fragrances from a source.}

\Verse[15.9]
{śrotraṁ cakṣuḥ sparśanaṁ ca rasanaṁ ghrāṇam eva ca \\
adhiṣṭhāya manaś cāyaṁ viṣayān upasevate}
{Using the ear, the eye, the sense of touch, the tongue, the nose, and the mind, the \textit{jīva} experiences the sense objects.}

\Verse[15.10]
{utkrāmantaṁ sthitaṁ vāpi bhuñjānaṁ vā guṇānvitam \\
vimūḍhā nānupaśyanti paśyanti jñāna-cakṣuṣaḥ}
{The deluded ones cannot perceive the Self associated with the \textit{guṇas} as it remains in the body, experiences sense objects, or departs from the body. Only those who possess the eye of knowledge can perceive it.}

\Verse[15.11]
{yatanto yoginaś cainaṁ paśyanty ātmany avasthitam \\
yatanto ’py akṛtātmāno nainaṁ paśyanty acetasaḥ}
{The striving \textit{yogīs} can perceive this \textit{ātmā} residing within themselves, but those who have not perfected their mind and lack awareness cannot behold it, despite striving.}

\Verse[15.12]
{yad āditya-gataṁ tejo jagad bhāsayate ’khilam \\
yac candramasi yac cāgnau tat tejo viddhi māmakam}
{Know that the light of the sun, which illumines the entire universe, as well as the brilliance in the moon and in fire, are Mine alone.}

\Verse[15.13]
{gām āviśya ca bhūtāni dhārayāmy aham ojasā \\
puṣṇāmi cauṣadhīḥ sarvāḥ somo bhūtvā rasātmakaḥ}
{Pervading the Earth, I sustain all beings with My energy. Becoming the Moon, full of nectar, I nourish all herbs.}

\Verse[15.14]
{ahaṁ vaiśvānaro bhūtvā prāṇināṁ deham āśritaḥ \\
prāṇāpāna-samāyuktaḥ pacāmy annaṁ catur-vidham}
{As the digestive fire dwelling in the body of living beings, and joining with the \textit{prāṇa}\footnote{Vital life force connected with the breath.} and \textit{apāna}, I digest the four kinds of food.}

\Verse[15.15]
{sarvasya cāhaṁ hṛdi sanniviṣṭo mattaḥ smṛtir jñānam apohanaṁ ca \\
vedaiś ca sarvair aham eva vedyo vedānta-kṛd veda-vid eva cāham}
{I dwell in the hearts of all. From Me arise knowledge, memory, and forgetfulness. I alone am to be known by all the \textit{Vedas}. I am the creator of Vedānta and the sole knower of the \textit{Vedas}.}

\enquote{Kṛṣṇa is saying that it is Him who gives the energy to act. Every action is only Him. The Lord seated in the heart pervades everything and observes the great activities of the body, mind and intellect. Without Him giving that energy, nothing is possible. You are thinking, \enquote{I am doing this, I am doing that,} just pleasing that petty self, looking for acknowledgment, and forgetting that if God hadn’t given you that energy to act, you would not be able to act\ldots It is by His will that every action is happening, every movement is happening. Everything works through His energy, each part of the body.}

\enquote{When you are so attached to your pride and ego, you are drained of that energy. It doesn’t mean that the energy is not there. It is there, but you are focusing that energy in the wrong place, and pride and ego consume lots of energy. It takes lots of energy to think about yourself.}

\enquote{The realized ones perceive only the Divine. Everywhere they look, everything they hear, everything they do, is an engagement of realization. They perceive the Lord within themselves and in all creation. For those who have true knowledge, who have wisdom, who meditate, contemplate, and do their \textit{sādhana}, the Lord is ever-present; there is not a single moment that He is not present.}

\enquote{But the ignorant are not drawn towards spirituality; they are far away. For those whose mind is impure, unsteady, always flickering---a mind that is constantly moving, jumping from one thing to the other---it’s very difficult to purify themselves. For those with a mind filled with impurities of many kinds, it is very difficult to detach, but not impossible.}

\enquote{They perceive that this happiness reflecting from you is not artificial happiness! It’s not superficial! It emanates from deep within you and they want it! They perceive it, it reacts on their mind, it reacts on them, and they see there is something different in you. But due to their ignorance, pride, and fear, they judge. However, this is a good sign. Just by this, you know you have had an effect on them. And to have an effect, whether it appears positive or not, when you beam this light and happiness, it is really always positive.}

\enquote{The ultimate aim of all the knowledge, the rituals, the worship, and the spiritual practices is to attach one to the Lord Himself, to realize the Lord, to realize God, to attain the qualification of God-realization. Bhagavān says here that those who attain God-realization fulfill all that is written in the \textit{Vedas}---not a part of what is written, but all. The ones who are completely surrendered to Him and realize Him---the Lord inside their own Self---go above the Vedas.}

\enquote{People can read the Vedas, they can chant many \textit{mantras}, but if this has not led them to God-realization, to seeing the Lord, it’s useless.}

\section{Verses 16-20: Kṛṣṇa is the Puruṣottama Beyond the Self}\label{sec-verses-16-20-krsna-is-the-purusottama-beyond-the-self}


\Verse[15.16]
{dvāv imau puruṣau loke kṣaraś cākṣara eva ca \\
kṣaraḥ sarvāṇi bhūtāni kūṭa-stho ’kṣara ucyate}
{There are two \textit{puruṣas} in this world: the perishable and the imperishable. All material beings are perishable, while the immutable Self is imperishable.}

\Verse[15.17]
{uttamaḥ puruṣas tv anyaḥ paramātmety udāhṛtaḥ \\
yo loka-trayam āviśya bibharty avyaya īśvaraḥ}
{But different from these is the Supreme Puruṣa, described as the Paramātmā, the unchanging Lord who pervades and maintains the three worlds.}

\Verse[15.18]
{yasmāt kṣaram atīto ’ham akṣarād api cottamaḥ \\
ato ’smi loke vede ca prathitaḥ puruṣottamaḥ}
{Because I am beyond the perishable and also higher than the imperishable, I am known as the Supreme Puruṣa in the world and in the \textit{Vedas}.}

\Verse[15.19]
{yo mām evam asammūḍho jānāti puruṣottamam \\
sa sarva-vid bhajati māṁ sarva-bhāvena bhārata}
{One who is not deluded and thus understands Me as the Supreme \textit{Puruṣa} knows all and worships Me with the entirety of their being, O descendant of Bharata.}

\Verse[15.20]
{iti guhyatamaṁ śāstram idam uktaṁ mayānagha \\
etad buddhvā buddhimān syāt kṛta-kṛtyaś ca bhārata}
{O sinless one, in this way, the most secret teaching has been revealed by Me. By understanding it, one becomes truly wise and accomplished in all respects.}

\enquote{Whoever rises above the threefold universe, the three \textit{guṇas}, perceives that everywhere only the Supersoul pervades. This is the all-pervading God, who has entered the whole universe, who has manifested the whole universe, who upholds the whole universe. It is Him who is all-supporting; this cosmic energy enables both the mutable and the immutable to function, to act, to manifest.}

\enquote{He who is unaffected, unchanged, who controls and rules over all, is the Almighty God. He is Kṛṣṇa Himself. Bhagavān says that when one rises above this, one perceives only Him within the core of oneself. And when one perceives Him within the core of oneself, one becomes immortalized. One sees the Truth and then one can really enjoy life---this you can call \enquote{enjoying life}! When you have risen above this, then you can enjoy life. There’s nothing to worry about! Others who have said that they are enjoying life are full of misery and pain. They are happy only for a short time, and then---misery. But the realized soul, one who perceives the Lord everywhere, is ever free! They truly enjoy life because they are not bound by any \textit{karma}. Wherever they go, whatever they do, is in a state of complete freedom.}

\enquote{The \textit{guṇas} and the mode of \textit{prakṛti}, are perishable, by nature. Everything that is manifested has an end. Whatever appears through the elements has an end. Bhagavān says, \enquote{I transcend the perishable. I transcend \textit{prakṛti}, I transcend material nature itself.} But here He doesn’t say just this! He says, \enquote{I am higher than the imperishable\ldots I am above the soul. I am the Supersoul, the soul of the souls.} By saying that He is superior, by claiming superiority over the imperishable, over the soul itself, it means that He is different\ldots He says the imperishable soul, the individual soul itself, is part of Him. It is also eternal, immortal, beyond the perishable.}

\enquote{The soul is the controlled, God is the Controller, and the body is the vehicle. The Lord is the worshiped, the praised one, and the soul, with the true knowledge, becomes the worshiper.}

\enquote{The \textit{bhakta} who recognizes the Lord to be the Almighty, universal, all-powerful, Supreme and Beloved of all, automatically develops faith in the Divine, in the virtues, in the glory, in the Truth, in the essence, in the greatness of God. Through \textit{bhakti}, his mind is absorbed in the Lord. The ears hear only the glory of the Lord, the chanting of the Name of the Lord; the speech is only focused on singing the praises of the Lord, and the eyes see only the \enquote{good} the Lord in everything. Such are the minds of the \textit{bhaktas}: they serve Him, they long for Him, they recognize that everything around is from Him. Such \textit{bhaktas} are absorbed in so much Love and attachment to the Supreme Lord, they will perform all their duties, carrying out the daily routine of their work, but absorbed in the Love for God. Everything they touch, everything they do, everything they hear, everyone who is in front of them---everything becomes an act of worship.}

\enquote{Here Bhagavān says that whoever has surrendered perceives Him everywhere, and He is the one that takes full control of the \textit{bhakta}. And through different \textit{līlās}, the different pastimes, He acts specifically for each individual; each one has their own personal relationship with Him. He manifests in a certain aspect for that person, and this is how He makes love grow---love for God, and love for all! Because one who loves God truly, perceives the Lord everywhere.}

