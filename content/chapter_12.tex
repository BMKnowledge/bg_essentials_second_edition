\chapter{Chapter 12: Bhakti-yoga}\label{chap-bhakti-yoga}

\paragraph*{\MakeUppercase{The Yoga of Love}}
%%%%%%%%%%%%%%%%%%%%%%%%%%%%%%%%%%%%%%%%%%%%%%%%%%%%%%%%%

\noindent After the great revelation of the cosmic form, Arjuna poses a question to Kṛṣṇa: Which is the better way to attain Him, as a devotee who is constantly engaged in worship of His personal form, or as one who contemplates upon the formless aspect of the Divine?

Kṛṣṇa's response is clear, the way of devotion is higher. The path of contemplation on the formless aspect of the Divine is filled with difficulty and suffering, whereas those devotees who are constantly absorbed in the Lord are delivered through grace.

Kṛṣṇa then outlines a step-by-step guide as to how one can ceaselessly fix one’s mind upon Him. In the beginning, one should develop a mood of detachment through \textit{karma-yoga} described in the earlier chapters. Then this internal renunciation should be suffused with \textit{bhakti}, dedicating all actions to Kṛṣṇa. Finally, one will be in a position to perform a meditation practice which can steady the mind to remain exclusively on the Lord.

In the closing verses we hear about the various qualities of the true devotees. They are beyond the ways of the world, untouched by joy and sorrow, detached from pain and pleasure. At the end of each statement, Kṛṣṇa repeatedly states that such devotees are dear to Him.

\section{Verses 1-5: Which is Superior, Devotion or Knowledge?}\label{sec-verses-1-5-which-is-superior-devotion-or-knowledge}

\Verse[12.1]
{\hspace*{1em}arjuna uvāca \\
evaṁ satata-yuktā ye bhaktās tvāṁ paryupāsate \\
ye cāpy akṣaram avyaktaṁ teṣāṁ ke yoga-vittamāḥ}
{\hspace*{1em}Arjuna said: \\
Among the devotees who worship You, constantly absorbed in the way described, and those who contemplate on the imperishable unmanifest---who has the highest knowledge of \textit{yoga}?}

\Verse[12.2]
{\hspace*{1em}śrī-bhagavān uvāca \\
mayy āveśya mano ye māṁ nitya-yuktā upāsate \\
śraddhayā parayopetās te me yuktatamā matāḥ}
{\hspace*{1em}Bhagavān Kṛṣṇa said: \\
Those whose mind is absorbed in Me, are ever united with Me and worship Me with supreme faith---I consider them to be most perfected in \textit{yoga}.}

\Verse[12.3–4]
{ye tv akṣaram anirdeśyam avyaktaṁ paryupāsate \\
sarvatra-gam acintyaṁ ca kūṭa-stham acalaṁ dhruvam \\
sanniyamyendriya-grāmaṁ sarvatra sama-buddhayaḥ \\
te prāpnuvanti mām eva sarva-bhūta-hite ratāḥ}
{But those who have subdued all their senses, who remain equal-minded at all times, who are intent on the welfare of all beings, and worship the imperishable, indefinable, unmanifest, all-pervading, inconceivable, unchangeable, immovable, eternal Supreme---they too reach Me alone.}

\Verse[12.5]
{kleśo ’dhikataras teṣām avyaktāsakta-cetasām \\
avyaktā hi gatir duḥkhaṁ dehavadbhir avāpyate}
{Greater is the hardship of those whose minds are attached to the unmanifest. That state is attained with difficulty for those who are embodied.}

\enquote{Here Arjuna is asking, which one is better? Those devotees who have love and longing and want to have this connection with the Supreme Lord, or those who are meditating upon the formless God?}

\enquote{Kṛṣṇa is saying, \enquote{Those who surrender to Me, who serve Me, who pray to Me, who long for Me: that is the easiest way.}}

\enquote{You have a body, a mind, and a brain that acts. All the pictures one sees are just images of things. And every day you build up 100,000 new images and input them into \enquote{this computer} inside your head. So, how can you picture the emptiness? That’s why certain words don’t have an image, like \enquote{no} or \enquote{emptiness.} They are blank. The brain goes into meltdown. But yet, when this mind that is in meltdown perceives the cosmic Lord, it is in awe; it is wonderstruck and shocked. And then the mind will say, \enquote{I always thought You were attribute-less; I always thought You were formless. I always thought You were just energy, flying from left to right.} No, no, my dear.}

\enquote{Beyond this energy is someone who controls all this energy. There’s a generator that creates this energy. Like the electricity that is everywhere: you touch it there, you will get electricity, you touch it here, you’ll get electricity, but the generator that is making this energy is a manifestation. A generator can’t be gliding in the air, flying somewhere. Even in an electric eel, the electricity is created inside, it goes through different organs, and then it arises as electricity outside. And this is what one perceives outside as electricity; just a manifestation, just a feeling. This is the same. Who can perceive this? Not everybody, but only the one whose mind is calm and who is surrendered. This is the state of a true \textit{bhakta}.}

\enquote{You may take the path of \textit{tapasya}, letting go of the world completely, doing certain hard austerities: torture! God is not a masochist! He says that it is possible to attain the state of the unmanifest, the void, but it takes rigorous, strong effort. And one goes through lots of suffering and pain. It’s a big roller coaster, yet one does get there, if one survives!}

\enquote{A \textit{bhakta} who worships God with attributes does go through this, but more easily. By worshiping the Lord, the Lord becomes easily relatable and very easy to adapt to. Then the Lord becomes the mother who carries the child, who cares for the child. This is why in other verses Bhagavān praises \textit{bhakti}.}

\newpage
\section{Verses 6-7: The Lord Rescues His Devotees}\label{sec-verses-6-7-the-lord-rescues-his-devotees}

\Verse[12.6–7]
{ye tu sarvāṇi karmāṇi mayi sannyasya mat-parāḥ \\
ananyenaiva yogena māṁ dhyāyanta upāsate \\
teṣām ahaṁ samuddhartā mṛtyu-saṁsāra-sāgarāt \\
bhavāmi na cirāt pārtha mayy āveśita-cetasām}
{I swiftly deliver those who surrender all actions to Me, who have Me as their supreme goal, who meditate on Me, worship Me with exclusive devotion, and whose minds are absorbed in Me, from the ocean of birth and death, O Pārtha.}

\enquote{Bhagavān Himself is saying that He will deliver those whose minds are still and focused on the different forms of the Lord, surrendered to Him; those who perform all their actions, their duties, but being free from the result don’t have any expectation; who don’t acknowledge the action as their own and who surrender everything. At night when they go to bed they say, \enquote{Kṛṣṇa, it’s only You.} They say, \enquote{\textit{kṛṣṇārpaṇam astu},} offering everything to Him. The Lord says, \enquote{To them I give Myself. These who are single-minded in their devotion, who are constantly fixed on Me, I deliver them from the cycle of birth and death.} They are the ones who have this exclusive Love inside them and who have cultivated this Love. They recognize that the Lord is everywhere. Through this Love they perceive in each person they meet, they meet the Lord Himself. With such Love they become selfless, free from egoism. There’s no trace of selfishness inside of them. There’s no pride, because they are not thinking of \enquote{I, I, I.} They become true \textit{yogīs}, unshakable in their faith. Through this Love that they have constantly, they become like a lover. They are constantly in Love; the mind constantly dwells on Him; \enquote{who fix on Me all their consciousness.} Their mind is fully absorbed only on Him. It’s like you are continuously in Love. Wherever you go, whatever you do, you think of your beloved.}

\enquote{This was the case of the \textit{gopīs}, the ladies from Vṛndāvana. Their minds, when they were doing their duties---attending to their husbands and children---were always on their beloved Kṛṣṇa. And it was for Kṛṣṇa's sake that they did everything. In their husbands, they didn’t see their husbands; they saw Kṛṣṇa. In their children, they didn’t see their children; they saw only Kṛṣṇa. All are His manifestation, the manifestation of the Lord Himself. And one who has devotion---who listens to His glory, meditates upon His glory, His virtues, does \textit{bhajan}, \textit{kīrtan}, chants His Name, and does \textit{japa}---the Lord frees. So, it’s easy! And this is the key. Love frees oneself. True Love frees oneself.}

\section{Verses 8-12: The Steps to Perfect Devotion}\label{sec-verses-8-12-the-steps-to-perfect-devotion}

\Verse[12.8]
{mayy eva mana ādhatsva mayi buddhiṁ niveśaya \\
nivasiṣyasi mayy eva ata ūrdhvaṁ na saṁśayaḥ}
{Place your mind on Me alone and let all your intellect be fixed in Me; hereafter you shall dwell in Me alone, without doubt.}

\Verse[12.9]
{atha cittaṁ samādhātuṁ na śaknoṣi mayi sthiram \\
abhyāsa-yogena tato mām icchāptuṁ dhanañjaya}
{But if you are unable to fix your mind on Me in a stable manner, then strive to attain Me, O Dhanañjaya, by the constant practice of \textit{yoga}.}

\Verse[12.10]
{abhyāse ’py asamartho ’si mat-karma-paramo bhava \\
mad-artham api karmāṇi kurvan siddhim avāpsyasi}
{But if you are incapable of practice, be devoted to My works. Also by performing actions for My sake, you shall attain perfection.}

\Verse[12.11]
{athaitad apy aśakto ’si kartuṁ mad-yogam āśritaḥ \\
sarva-karma-phala-tyāgaṁ tataḥ kuru yatātmavān}
{But if you are incapable of doing even that, then resort to My \textit{yoga}, give up the fruit of all action, and act with a controlled mind.}

\Verse[12.12]
{śreyo hi jñānam abhyāsāj jñānād dhyānaṁ viśiṣyate \\
dhyānāt karma-phala-tyāgas tyāgāc chāntir anantaram}
{Knowledge is better than practice. Meditation is better than knowledge, and renouncing the fruits of action is better than meditation because from such renunciation, peace immediately follows.}

\enquote{Bhagavān is making it easier and easier and easier for the devotee.}

\enquote{You have time to \enquote{lazy yourself.} You have time to do everything which keeps you attached to the world. But to give your time for God? People find many excuses for not doing it\ldots Stop finding excuses that you don’t have time, but train your mind to think that whatever you are doing, you are doing it for God.}

\enquote{Here He is giving the assurance: \enquote{Those who are constantly thinking of Me, those who do everything with Me in their minds, whose meditation is focused upon Me, those with single-minded devotion,\ldots those who are absorbed in Me in their daily work---for them, I make everything easy.}}

\enquote{Bhagavān is saying that for the ones who recognize God alone as their supreme refuge and constantly bear Him in mind through their speech and controlling of the body; their every action becomes a sacrifice, a charity. All actions become penance. So to the one who can’t control sitting in meditation for five minutes, Bhagavān says, \enquote{Go and do your work! Do charity! Do your daily chores with great love and surrender.} Bhagavān is saying that this work is also a form of meditation and if you do it with love, if you enjoy what you are doing, it will become an adoration to the Lord, to Bhagavān Himself. It will become a prayer! This is where the expression \enquote{work is worship} comes from. For those who can’t sit down to meditate, those that don’t have time, due to many other obligations, or for those who are very hyperactive, He says, do your duty properly with a mind of surrender, a mind which is constantly focused on Him. Know that with such faith, whatever you do is a sacrifice, a charity, is penance to Him. All this is transformed into worship. Bhagavān says, \enquote{Even to this kind of a \textit{bhakta}, I will also reveal Myself.}}

\enquote{The ones who are not attached to the fruit of their action perceive Him; the Lord will reveal Himself. He will reveal Himself by His own mercy and Love! But the \textit{bhaktas} who are on the way to the Lord should remove all feelings of doership and not be attached to their action. One should not even consider oneself the instrument of God. Praise the Lord in all His glory. Because, you see, if you start seeing yourself as the instrument, there is a separation. And when there is separation, pride will arise.}


\section{Verses 13-20: The Qualities of a Devotee Who is Dear to Kṛṣṇa}\label{sec-verses-13-20-the-qualities-of-a-devotee-who-is-dear-to-krsna}

\Verse[12.13–14]
{adveṣṭā sarva-bhūtānāṁ maitraḥ karuṇa eva ca \\
nirmamo nirahaṅkāraḥ sama-duḥkha-sukhaḥ kṣamī \\
santuṣṭaḥ satataṁ yogī yatātmā dṛḍha-niścayaḥ \\
mayy arpita-mano-buddhir yo mad-bhaktaḥ sa me priyaḥ}
{My devotee who is without hatred toward any living being, who is friendly and compassionate, who is free from the sense of possession and doership, who faces pain and pleasure with equanimity, who is forgiving, contented, who always engages in \textit{yoga}, is self-controlled, and firm in resolve, and whose mind and intellect are dedicated to Me, is dear to Me.}

\Verse[12.15]
{yasmān nodvijate loko lokān nodvijate ca yaḥ \\
harṣāmarṣa-bhayodvegair mukto yaḥ sa ca me priyaḥ}
{One who does not disturb the world and is not disturbed by the world, who is free from excitement, impatience, fear, and passion---such a person is dear to Me.}

\Verse[12.16]
{anapekṣaḥ śucir dakṣa udāsīno gata-vyathaḥ \\
sarvārambha-parityāgī yo mad-bhaktaḥ sa me priyaḥ}
{My \textit{bhakta} who is free from expectations, who is pure, skilled, impartial, free from agitation, and who has given up all selfish pursuits, is dear to Me.}

\Verse[12.17]
{yo na hṛṣyati na dveṣṭi na śocati na kāṅkṣati \\
śubhāśubha-parityāgī bhaktimān yaḥ sa me priyaḥ}
{One who is full of devotion, who neither rejoices nor hates, who neither laments nor desires, and who renounces both the pleasant and unpleasant, is dear to Me.}

\Verse[12.18–19]
{samaḥ śatrau ca mitre ca tathā mānāpamānayoḥ \\
śītoṣṇa-sukha-duḥkheṣu samaḥ saṅga-vivarjitaḥ \\
tulya-nindā-stutir maunī santuṣṭo yena kenacit \\
aniketaḥ sthira-matir bhaktimān me priyo naraḥ}
{One who is equal to both enemy and friend and likewise to honor and dishonor, who is indifferent to cold and heat, pleasure and pain, who is free from all attachments, unaffected by both praise and blame, silent, and content with anything, who is without home, steady in mind, and full of devotion---such a person is dear to Me.}

\Verse[12.20]
{ye tu dharmāmṛtam idaṁ yathoktaṁ paryupāsate \\
śraddadhānā mat-paramā bhaktās te ’tīva me priyāḥ}
{But those devotees who regard Me as the Supreme and devotedly follow this nectar of \textit{dharma} as taught by Me with full faith, are exceedingly dear to Me.}

\enquote{These last seven verses of this chapter are where Bhagavān Kṛṣṇa reveals the qualities of a \textit{bhakta} who is completely surrendered to Him, and who has Him as the supreme aim, the supreme goal. Whatever one does is only for Him. These seven verses are the basis of religion for the upliftment of humanity. The fulfillment of human existence lies in the heart of a \textit{bhakta}. Whosoever follows this ideal---these seven verses that we have just talked about---makes one the ideal \textit{bhakta}, the true \textit{bhakta}. And this \textit{bhakta} attains God-realization. The ones who have these qualities attain God-realization and they become immortalized. Due to this realization they will always have the mark of the Lord on them.}

\enquote{The \textit{bhaktas} don’t have any egoism. The \textit{bhaktas} are an ocean of mercy and forgiveness. The \textit{bhaktas} always stay in deep connection with the Lord. Nothing can deviate the \textit{bhaktas} who are surrendered to their spiritual path. Those who can be deviated are not \textit{bhaktas}. They will be deviated; they will \enquote{fly} somewhere else. They are \enquote{shoppers,} who like going from one shop to another. They would say, \enquote{I go to this shop to get some sweets.} Once they have finished eating the sweets, they will go to another shop to buy other things. They are always \enquote{shopping.} Or, you can also call it \enquote{hopping}: \textit{guru} hopping, spiritual hopping or any kind of hopping. They are always \enquote{hop on,} \enquote{hop off.} They hop, hop, hop; it is hopeless! A true \textit{bhakta} doesn’t have this \enquote{hopping-ness,} this \enquote{hopeless} state. A true \textit{bhakta} has stability and is anchored! Rooted!}

\enquote{When one practices spirituality, one has to have a firm faith and firm steadiness. It’s of no use \enquote{hopping} around, from here to here. You should make one single jump from your head to your heart. I always say that from the head to the heart is only 40 centimeters---a very short way. That’s only one hop! A leap of faith, which makes one stronger.}

\enquote{If you see someone in front of you with an affliction, automatically you get afflicted also. You feel the pain. But does it change the world if you feel this pain? No, the world doesn’t change. If you want to help the world, you have to be strong. If you let the light inside of you dim itself, you can’t shine. But by being self-centered, being absorbed in the will of God inside you, this becomes your strength. And by being strong, you shine this Light into the world. This is how peace happens! This is how things change in this world. If you want peace, yet you say, \enquote{O my goodness, it’s terrible!} What image is in your mind? Which is the more powerful image in your mind? Is it the peaceful image or the terrible image? It’s always like this! Because this is the mind of the world. This is how people function: they hang on negativity. However, the outside world should not bother you. This doesn’t mean that you don’t care. You do care, but you should be focused! Because you have Love inside you, you care, but you don’t become weak. And in this way, you can bring forth change.}

\enquote{The \textit{bhakta} is open to many possibilities. This is why He says, \enquote{Don’t be attached to the fruit of your action, because I create the possibilities. If A here doesn’t work, I will create B. If B doesn’t work, I’ll make C, but I’ll lead you to perfection. There’s always a plan B, and there’s always a plan C. And if C doesn’t work, I’ll create D! If D doesn’t work, I’ll create E! But if you hang on to one outcome, then you are sitting on A, and A will lead you only to A.}}

\enquote{One who is surrendered on their spiritual path, and one who has the chance to meet the \textit{satguru}, knows that the \textit{satguru} looks after their welfare as a mother looks after her baby. One tries one’s best, though; it doesn’t mean that you can sit down and say, \enquote{Okay, I won’t do anything. Guruji will look after it.} No, you have to put in your maximum effort, but be detached from it. You have to try your best. And due to this, when \textit{guru}/God sees that you are making an effort, that you have taken one step---and it is very important that the \textit{guru} sees this in the devotee---then he says, \enquote{Ahh, this one is ready! We can help advance him.} This is in reality what happens on the spiritual path.}

\enquote{The ones whose aim is only God are \enquote{the perfect devotees.} In this verse He says that this single-pointedness is the mark of a truly surrendered devotee. The devotees who have such focus dwell eternally with the Lord. For the saints who have achieved unity with God, God dwells constantly with them. There’s no difference between the saints and God. When a saint touches something, it’s purified. The touch of a saint is the same as from the Lord; the Lord touches through the saints.}

\enquote{Bhagavān Kṛṣṇa praises His devotees, praises those who are surrendered to Him, showing how dear He is surrendered to them. You think that you are longing for Him? No, He is longing more for you. You think that you love Him? Your love is like a grain of dust, but His Love is like an ocean. You think that you are walking towards Him? He is running towards you. That’s the greatness and the humility of the Lord Himself.}