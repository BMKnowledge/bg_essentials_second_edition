\chapter{Rāja-vidyā-guhya-yoga}\label{chap-raja-vidya-guhya-yoga}
\chaptersubtitle{The Secret of Devotion}


%%%%%%%%%%%%%%%%%%%%%%%%%%%%%%%%%%%%%%%%%%%%%%%%%%%%%%%%%

\noindent Taking up many of the themes in chapter seven, Kṛṣṇa now explores the way of \textit{bhakti} in more depth. This, He states, is the highest knowledge and the greatest secret. Once again, we learn how the Lord supports and controls the material world, but at the same time is both transcendent and detached from the process. Those great souls who understand this truth engage in acts of devotional worship.

Kṛṣṇa then reiterates His immanence in the world: He Himself is the various aspects of the \textit{yajña}, the sound of \textit{oṁ}, the \textit{Vedas} and the very goal of life. Those who fail to perceive this, carry out the \textit{Vedic} ritual and attempt to enjoy heavenly delights. But once their merit (\textit{puṇya}) is exhausted, they are forced back into this world. They cannot see that it is Kṛṣṇa who is ultimately the enjoyer of all sacrifices and He alone provides prosperity and security. Depending on where we focus our minds (family, spirits, gods or Kṛṣṇa) our destination after death will be decided.

The closing verses describe the power of \textit{bhakti}. Kṛṣṇa states that the simplest of offerings done with love will be accepted by Him, and so He urges Arjuna to make everything in life an offering to Him. Although Kṛṣṇa is equal to all, the devotees are the ones who dwell in the heart of Kṛṣṇa, and He in theirs. He declares that even if one is the worst sinner, \textit{bhakti} to the Lord will transform that person onto the righteous path. Finally, Arjuna is told to fix his mind upon Kṛṣṇa, to bow down and worship Him. If this is done, he is guaranteed to come to the Lord.

\section{Verses 1-3: Knowledge of Bhakti is the Highest}\label{sec-verses-1-3-knowledge-of-bhakti-is-the-highest}

\Verse[9.1]
{\hspace*{1em}śrī-bhagavān uvāca \\
idaṁ tu te guhyatamaṁ pravakṣyāmy anasūyave \\
jñānaṁ vijñāna-sahitaṁ yaj jñātvā mokṣyase ’śubhāt}
{\hspace*{1em}Bhagavān Kṛṣṇa said: \\
I shall reveal to you who are without envy, the most secret knowledge, along with the process of its realization, knowing which you shall be freed from the miseries of life.}

\Verse[9.2]
{rāja-vidyā rāja-guhyaṁ pavitram idam uttamam \\
pratyakṣāvagamaṁ dharmyaṁ susukhaṁ kartum avyayam}
{This is the king of wisdom, the king of mysteries, and the supreme means of purification. It is perceptible through direct experience, based on \textit{dharma}, easy to practice, and imperishable.}

\Verse[9.3]
{aśraddadhānāḥ puruṣā dharmasyāsya parantapa \\
aprāpya māṁ nivartante mṛtyu-saṁsāra-vartmani}
{Not attaining Me, those who have no faith in this path return to this cycle of birth and death, O vanquisher of enemies.}

\enquote{People who are attached to earthly existence, who are attached to the objects of the senses, whose senses and mind revolve around vices, due to their ignorance, are not worthy to receive the Word of God. In the Bible, Christ says that we should not cast pearls before pigs. The pigs will trample over them, not knowing their value. Kṛṣṇa says that this deep knowledge is to be given only to the \textit{bhaktas} and should not be given to people who are not ready, because they will not know what to do with it. They will not even understand it.}

\enquote{By chanting the Lord’s Name and letting your mind be focused on Him, you will get rid of all the vices and evil ways. It will cleanse you of all your past \textit{karmas} and release you from endless suffering. When you are rid of all endless suffering, the \textit{ātmā} reveals itself.}

\enquote{Even when the \textit{ātmā} reveals itself, that’s not everything. You have to attain the supreme bliss. You have to attain the supreme peace. You have to attain God-realization. God-realization here means when the Divine reveals Himself. When you realize your Self, you have paved the way towards the divine qualities. This is where all the divine qualities start reflecting through you. You become calm, you become radiant, you become very peaceful, you become non-judgmental. Only then can you do service to society. Only then can you go around and serve freely. The right hand gives and the left hand doesn’t know about it. So, you become automatically a great aid for this world. divine virtue starts flowing through you and the way you perceive the world changes.}

\enquote{Kṛṣṇa says, \enquote{What I am asking of you is not difficult: to surrender your mind to My Feet, to think of Me, place your mind upon My image, My form; nothing can destroy that. Once you build a connection to Me, once you build a relationship to Me, once you attain that Love, there is no going back and it is imperishable because nothing can destroy it.} That knowledge of the Self is a secret that has been kept always, and it was given only through Initiation and only when one was ready. But in reality, it’s very simple and very easy. Even if it sounds very difficult, it is the easiest thing to do.}

\enquote{The soul incarnates in 8.4 million species before attaining a human body. Only through the grace of God, through spiritual advancement, can a soul attain a human body. The ones who surrender to the Divine within themselves, realize how rare it is to get this human existence. When they realize how important this human existence is, they hold on tight to their spiritual path, their meditation, and their Atma Kriya Yoga practice. If they lack faith in the teachings of the Lord, if they are reading or listening to the \textit{Gītā} without faith, they will find it difficult to attain this Love, to surrender and to realize God. Those who find it difficult to realize God, first within the core of themselves, and then everywhere, come back again and again to advance towards unity with the Supreme Divine Being.}

\enquote{Without faith, it is impossible to center the mind. Without faith, it is impossible to love. Without faith, it is not easy to chant. But with faith, it becomes easy.}

\section{Verses 4-10: Kṛṣṇa's Relationship to the World}\label{sec-verses-4-10-krsna-s-relationship-to-the-world}

\Verse[9.4]
{mayā tatam idaṁ sarvaṁ jagad avyakta-mūrtinā \\
mat-sthāni sarva-bhūtāni na cāhaṁ teṣv avasthitaḥ}
{This entire universe is pervaded by Me in My unmanifest form. All beings are in Me, but I am not in them.}

\Verse[9.5]
{na ca mat-sthāni bhūtāni paśya me yogam aiśvaram \\
bhūta-bhṛn na ca bhūta-stho mamātmā bhūta-bhāvanaḥ}
{And yet these beings do not truly abide in Me. Behold My divine mystery: although I uphold all beings, and I Myself am the source of their existence, I am not in them.}

\Verse[9.6]
{yathākāśa-sthito nityaṁ vāyuḥ sarvatra-go mahān \\
tathā sarvāṇi bhūtāni mat-sthānīty upadhāraya}
{Understand that just as the mighty wind moves everywhere, but always remains in space, all beings rest in Me.}

\Verse[9.7]
{sarva-bhūtāni kaunteya prakṛtiṁ yānti māmikām \\
kalpa-kṣaye punas tāni kalpādau visṛjāmy aham}
{All beings are withdrawn into My \textit{prakṛti}\footnote{Substance from which all material creation manifests. It forms part of Kṛṣṇa's nature and is therefore under His control.} at the end of a \textit{kalpa}, and at the beginning of a new \textit{kalpa}, I send them forth again, O son of Kuntī.}

\Verse[9.8]
{prakṛtiṁ svām avaṣṭabhya visṛjāmi punaḥ punaḥ \\
bhūta-grāmam imaṁ kṛtsnam avaśaṁ prakṛter vaśāt}
{Using My own \textit{prakṛti}, I repeatedly bring forth this entire multitude of beings who remain powerless under the influence of this very \textit{prakṛti}.}

\Verse[9.9]
{na ca māṁ tāni karmāṇi nibadhnanti dhanañjaya \\
udāsīnavad āsīnam asaktaṁ teṣu karmasu}
{And yet these actions do not bind Me, Dhanañjaya, since I am detached from them, remaining always indifferent.}

\Verse[9.10]
{mayādhyakṣeṇa prakṛtiḥ sūyate sa-carācaram \\
hetunānena kaunteya jagad viparivartate}
{Under My direction this \textit{prakṛti} brings forth all the moving and non-moving beings. By this reason alone the world operates as it does, O Kaunteya.}

\enquote{He says, \enquote{All existences are situated in Me, but I am not in them.} The material, the tangible things exist inside of Kṛṣṇa, but Kṛṣṇa as the Supreme Lord is beyond this material existence and He stays always unconnected. He remains intact, even at the time of \textit{pralaya}\footnote{The periodic dissolution of the material world, in which creation is withdrawn into the body of God.} when the whole creation is dissolved.}

\enquote{Imagine a cloudy sky. The sky is completely covered by the clouds, but does this affect the sky? No, it doesn’t. When the clouds are dispersed, you see that the sky is still there and that the sky was always there. It’s not touched by the clouds at all. In the same way, God is ever-existing. He doesn’t cease to exist, even if the whole world disappears, because He existed even before creation.}

\enquote{So, He is never touched by anything, even if the world disappears. Even right now, while we are talking, there are universes appearing and disappearing. But the glory of the Lord shines everywhere. There is no place where He is not shining, however, nothing material touches Him.}

\enquote{The ego, the pride, the big \enquote{I, I, I} are just like clouds in the sky, and when they disappear, only the sky stays. In the same way, the \textit{bhakta} realizes that beyond this \enquote{I, I, I,} there is only One that ever-exists---God. One attains the state where the world stops existing and what exists is only God.}

\enquote{In His \textit{nirguṇa} aspect, God is like the ether: uniform, formless, actionless, infinite, unattached and immutable, immune to all changes. Like the wind, like air, all beings spring from God, abide in God and naturally come back into God.}

\enquote{However, God also has a \textit{saguṇa} aspect. That’s why He can conceive the whole creation. Without this, how would He be? He would be complete emptiness. First one has to see that all is present in His cosmic form and that is the Supreme Being. However, He is also the formless, actionless, infinite, unattached. He is beyond everything, yet He pervades everything and infuses energy into everything. When we refer to a small hole saying that it is empty, in reality it’s not empty. Air pervades it. There is air in the hole. In the same way, you can take an empty bottle and say that the bottle is empty, yet it’s not fully empty, because there is air inside.}

\enquote{When one rises in awareness, one perceives that every atom, every particle, is pervaded by Bhagavān in His formless state. However, all this drama of countless universes, countless worlds, which are appearing and disappearing, are in His cosmic form, in His \textit{saguṇa} form: beyond that there is nothing else. And when one reaches that state of realization, and perceives that He is the One who gives the power to the atom, one will see that divine reality; and that’s God consciousness. The one who wants to see only the atom, will perceive only the atom and not the power which is behind the atom, giving it the energy to act and manifest.}

\enquote{At the end of the dissolution period, another creation manifests. The whole world dissolves into Him and, afterwards, a new world arises from Him. The dissolution period also enables the \textit{ātmā}, the individual soul, to move to a higher level of realization.}

\enquote{\textit{Pralaya} is very long in human years. In this reality, \textit{pralaya} takes trillions of human years, but in the celestial reality, \textit{pralayas} can happen every day. After \textit{pralaya}, the soul comes back to this reality, to this world. As I said before, to advance to God consciousness, one has to be born in this world here, in this reality. That’s why all the \textit{avatāras} come to remind people how important it is to realize God in one’s life! And this chance is not given in every life, due to the surroundings, the environment, the association, whether it’s low or high association. If it’s a high association, it’s fine. If it is a low association, finish! Nevertheless, whether one takes hundreds of millions of years or hundreds of millions of incarnations, in the end, one will return to where one has come from: to Him. It is Him only.}

\enquote{If people have not reached this state of realization in their lives, due to attachment to the \textit{guṇas} and \textit{karma} created, they are born again and again. They are under the control of \textit{māyā}: they are so attached to the body and to the outside world that it becomes very difficult for them to let go. They don’t perceive the cosmic nature of the Lord within their own individual nature. They become blind. They can’t perceive that it is the Supreme Lord Himself who is the core of the individual. Here I’m not referring to the individual Self, or to the ego self, but to the Supreme Self, Nārāyaṇa, who is seated deep in the core of one’s heart, in the core of one’s Self.}

\enquote{Bhagavān Kṛṣṇa says that He is the Controller, the Supervisor of this creation; He is the Supervisor behind \textit{prakṛti}. All existences and activities of \textit{prakṛti} derive from Him; but it is His \textit{prakṛti} which carries on all such activities as the creation of the universe.}

\enquote{Here you can take the example of a farmer. As the supervisor, a farmer plants the seeds in the earth, and then the earth carries on. Even if the farmer has put the seeds in the earth, the earth brings forth a variety of plants according to the kinds of seeds which the farmer planted. Like that, God, Nārāyaṇa, implants the seeds of life in \textit{prakṛti}, and as the supervisor, He then supervises everything. So, \textit{prakṛti} evolves. He plants the seeds of creation inside Her, then \textit{prakṛti} makes the entire creation evolve and all the beings, and species come into existence, into manifestation. Mother Nature carries on doing what She has to do. And the farmer is not attached. The farmer doesn’t go around saying, \enquote{Grow, grow, grow!} No. He puts the seeds in the earth, but then Mother Nature takes Her course by Herself. The Lord says that by His will He brought forth the creation and then everything was carried on by Nature. It’s the same thing when a child is born. You may want to have a child, but if the Supreme Lord Himself doesn’t will it, it will not happen. One can say, \enquote{There is a tube which is blocked on this side} or \enquote{There is no strength in the man,} and so on. Science can say many things, but when the Divine wills something, it can happen and then even science gets completely baffled\ldots He can bend His own nature and make things happen that are beyond the limitations of the mind of man to understand.}

\newpage

\section{Verses 11-15: Recognizing and Worshiping the Lord}\label{sec-verses-11-15-recognizing-and-worshiping-the-lord}

\Verse[9.11]
{avajānanti māṁ mūḍhā mānuṣīṁ tanum āśritam \\
paraṁ bhāvam ajānanto mama bhūta-maheśvaram}
{Unaware of My supreme nature as the great controller of all beings, fools disregard Me when appearing in a human form.}

\Verse[9.12]
{moghāśā mogha-karmāṇo mogha-jñānā vicetasaḥ \\
rākṣasīm āsurīṁ caiva prakṛtiṁ mohinīṁ śritāḥ}
{The aspirations, actions, and knowledge of such ignorant people are in vain. They have adopted a malicious, demonic, and truly deluded nature.}

\Verse[9.13]
{mahātmānas tu māṁ pārtha daivīṁ prakṛtim āśritāḥ \\
bhajanty ananya-manaso jñātvā bhūtādim avyayam}
{But the great souls, O Pārtha, have assumed My divine nature. They adore Me single-mindedly, knowing Me to be the imperishable source of all beings.}

\Verse[9.14]
{satataṁ kīrtayanto māṁ yatantaś ca dṛḍha-vratāḥ \\
namasyantaś ca māṁ bhaktyā nitya-yuktā upāsate}
{Constantly glorifying Me, striving for Me, being fixed in firm vows, and bowing down to Me with devotion, they worship Me and are constantly united with Me.}

\Verse[9.15]
{jñāna-yajñena cāpy anye yajanto mām upāsate \\
ekatvena pṛthaktvena bahudhā viśvato-mukham}
{Yet others make offerings by the sacrifice of knowledge and worship Me with a view of oneness, others with a view of duality, and yet others worship Me by various means as having faces everywhere.}

\enquote{Wise indeed are those who recognize who is in front of them\ldots Those who are detached, those who are free, those whose minds are focused, will see the Divine. But those who are attached and blind, even if the Supreme Lord Himself is in front of them, like the Kauravas, they will not see who He is.}

\enquote{Out of compassion for all embodied souls, in order to place them all under His protection, with a view to establish righteousness, to redeem His devotees, His people, the Supreme Lord manifests Himself. He takes various forms and performs various \textit{līlās}\footnote{When the Lord takes on limitation to uplift and interact with His devotees.} in this world to remind mankind of His divinity, to remind mankind of the importance of surrendering to God: \enquote{Not my will be done, Thy will be done!}}

\enquote{Kṛṣṇa reveals to Arjuna that He is the Supreme Lord who has manifested Himself in various Incarnations, like Matsya, Kūrma, Varāha, Nārasiṁha, Vāmana, Paraśurāma, Rāma, and so on. And it is Him now, the same Lord, the same fullness: it is the Supreme Lord fully present in Vāsudeva Kṛṣṇa.}

\enquote{Bhagavān says that the great souls adore Him with a single-pointed mind. Even if they have realized God, even if they see Him inside themselves and know that everything is Him, they never stop adoring Him. These realized souls have exclusive Love for God. Their minds don’t feel attracted towards anything other than God. They perceive God within themselves and everywhere, but they don’t stop praising and singing the glory of God. These realized souls are never separated from God. As \textit{bhaktas} have this unbearable yearning for Him, the Lord Himself also longs for the \textit{bhaktas} in the same way. As the \textit{bhaktas} perceive the Lord inside themselves, the Lord also perceives the \textit{bhaktas} inside Him. This is the relationship, the connection, that a realized soul has with the Lord. Slowly, slowly the Lord is revealing to Arjuna the essence of \textit{bhakti}.}

\enquote{The ones who hear the glory of the Lord with a steadfast mind, the ones whose hearts choke and have tears roll down from their eyes hearing of the divinity of the Lord, are fit for God-realization. Those devotees who have such longing for the Lord are fit for God-realization, because they love God for the sake of loving Him. They don’t love God just to make a bargain with Him. They love Him because they love Him. That’s \enquote{Just Love}! The Lord praises these \textit{bhaktas} and He Himself rejoices in the form of a \textit{bhakta}.}

\enquote{There may be terrible calamities, the wind can be blowing left and right, a big tsunami can happen, but it doesn’t in the slightest affect the true \textit{bhakta}. They are not shaken in their resolution or determination. When the \textit{bhaktas} have this firm resolve, have taken a path and surrendered, nothing which happens in the outside world can shake them. It doesn’t matter how much they are criticized or what others say. Nothing can move them. Such \textit{bhaktas} attain the supreme reality. They are fit for God-realization. But the flickering ones, God discharges them for their next lives.}

\section{Verses 16-19: The Immanence of Kṛṣṇa in the World}\label{sec-verses-16-19-the-immanence-of-krsna-in-the-world}

\Verse[9.16]
{ahaṁ kratur ahaṁ yajñaḥ svadhāham aham auṣadham \\
mantro ’ham aham evājyam aham agnir ahaṁ hutam}
{I am the ritual; I am the sacrifice; I am the offerings made to the ancestors; I am the herbs; I am the \textit{mantra}; I alone am the clarified butter; I am the fire; I am the sacrificial offering.}

\Verse[9.17]
{pitāham asya jagato mātā dhātā pitāmahaḥ \\
vedyaṁ pavitram oṁ-kāra ṛk sāma yajur eva ca}
{I am the father, mother, sustainer, and grandfather of this universe. I am the object of knowledge, the purifier, the syllable \textit{oṁ}, as well as the \textit{Ṛg Veda}, \textit{Sāma Veda}, and \textit{Yajur Veda}.}

\Verse[9.18]
{gatir bhartā prabhuḥ sākṣī nivāsaḥ śaraṇaṁ suhṛt \\
prabhavaḥ pralayaḥ sthānaṁ nidhānaṁ bījam avyayam}
{I am the goal, the supporter, the Lord, the witness, the abode, the refuge, and the friend of all. I am creation, destruction, and sustenance. I am the reservoir of everything and the eternal seed.}

\Verse[9.19]
{tapāmy aham ahaṁ varṣaṁ nigṛhṇāmy utsṛjāmi ca \\
amṛtaṁ caiva mṛtyuś ca sad asac cāham arjuna}
{O Arjuna, I am the one who emanates heat; I send forth and hold back the rain; I am immortality as well as death; I am the eternal and the perishable.}


\enquote{Lord Kṛṣṇa is explaining to Arjuna about His omnipresence; about how He is in everything. He is gradually guiding Arjuna to perceive that supreme reality. People may say, \enquote{Okay, God is everywhere.} However, this doesn’t mean that they really perceive that God is everywhere.}

\enquote{The Lord is taking Arjuna beyond that point to reveal the Truth to him through \textit{bhakti}. When \textit{bhakti} awakens, the \textit{jñāna} is transformed: he not only perceives the formless in everything, but he realizes that it’s only the Lord who pervades everything, and that everything is just His emanation.}

\enquote{He says, \enquote{I am the sacred \textit{oṁ}. I am the \textit{praṇava}. I am this life force that sustains everything.} Kṛṣṇa is not only talking about the sound itself. Sound is vibration and everything in this world is caused by vibration. The whole universe, everything, is created due to that vibration. He says, \enquote{I am the cause of that vibration. I am that \textit{oṁkāra}. I am this sound that vibrates in the whole universe and brings everything into being. I am that vibration itself, and whoever through \textit{sādhana} realizes and hears this cosmic sound, attains God-realization. Through this sound I purify everything}\ldots When one is doing \textit{OM Chanting}, the \textit{oṁ} vibrates, it purifies.}

\enquote{Everything, all beings, emanate from Him alone and He is the \enquote{eternal resting place,} so everything goes back to Him. Everything comes out of God and everything goes back to God, and everything is happening inside of Him through His \textit{yoga-māyā}.}

\enquote{He is the ultimate Truth. He is the One who takes people from the darkness of ignorance and brings them to the Light. He takes people from untruth to Truth. He is the One who leads each person to realize the \textit{ātmā}, releasing them from the cycle of birth and death.}

\enquote{Rituals may change, Names may change, everything around may change, but the Lord, the essence that is behind everything is only Him. And those who have the eyes to see, who have the eyes of truth and who have the eyes of wisdom, they will perceive the same Lord that they worship everywhere\ldots Everything becomes an offering, everything is a prayer, because the glories of the Lord are endless and have no limitation. His mercy doesn’t have any limitation. His wisdom doesn’t have any limitation. His creation is endless.}

\section{Verses 20-25: The Fate of Those Who Worship the Lesser Gods}\label{sec-verses-20-25-the-fate-of-those-who-worship-the-lesser-gods}

\Verse[9.20]
{trai-vidyā māṁ soma-pāḥ pūta-pāpā \\
yajñair iṣṭvā svar-gatiṁ prārthayante \\
te puṇyam āsādya surendra-lokam \\
aśnanti divyān divi deva-bhogān}
{Those who follow the three \textit{Vedas}, purified from sin by drinking the \textit{soma} juice,\footnote{A powerful plant extract that formed a central part of the \textit{Vedic} ritual sacrifice.} pray to attain heaven\footnote{In the context of this verse, heaven is not the Supreme Abode of the Lord, but is considered to be the material realm of the demigods where one can temporarily enjoy celestial pleasures.} and worship Me with sacrifices. After reaching the realm of Indra,\footnote{The leader of the \textit{devas} and the ruler of the heavens.} they enjoy celestial pleasures of the \textit{devas} in heaven.}

\Verse[9.21]
{te taṁ bhuktvā svarga-lokaṁ viśālaṁ \\
kṣīṇe puṇye martya-lokaṁ viśanti \\
evaṁ trayī-dharmam anuprapannā \\
gatāgataṁ kāma-kāmā labhante}
{Having enjoyed the vast realm of heaven, they return to the world of mortals once their merit is exhausted. In this way, those who follow the \textit{Vedic} rituals while desiring material enjoyment attain repeated birth.}

\Verse[9.22]
{ananyāś cintayanto māṁ ye janāḥ paryupāsate \\
teṣāṁ nityābhiyuktānāṁ yoga-kṣemaṁ vahāmy aham}
{But for those who are single-mindedly focused on Me and worship Me, for them who are absorbed in Me and ever in union with Me, I personally take responsibility of their prosperity and welfare.}

\Verse[9.23]
{ye ’py anya-devatā-bhaktā yajante śraddhayānvitāḥ \\
te ’pi mām eva kaunteya yajanty avidhi-pūrvakam}
{O Kaunteya, even those \textit{bhaktas} of other deities who worship with faith, in fact worship Me alone, however, not in a proper manner.}

\Verse[9.24]
{ahaṁ hi sarva-yajñānāṁ bhoktā ca prabhur eva ca \\
na tu mām abhijānanti tattvenātaś cyavanti te}
{Indeed, I alone am the enjoyer and the Lord of all sacrifices. But such people do not know Me in truth; hence they fall down.}

\Verse[9.25]
{yānti deva-vratā devān pitṝn yānti pitṛ-vratāḥ \\
bhūtāni yānti bhūtejyā yānti mad-yājino ’pi mām}
{The worshipers of the \textit{devas} go to the \textit{devas}. The worshipers of the ancestors go to the ancestors. The worshipers of the spirits reach the spirits. But My \textit{bhaktas} surely come to Me.}

\enquote{Here Bhagavān Kṛṣṇa says that people who don’t realize the Truth about the Self and God, but who surrender to the demigods, attain the fruit of transitory pleasures in the celestial \textit{lokas} of the demigods, according to what they did in their lives. However, their good \textit{puṇya} is transitory. It is not permanent. Due to that \textit{puṇya}, they go for instance to \textit{Indraloka} and get entertained a little bit by \textit{Indra} and company. But later, when their \textit{puṇya} is finished, when \enquote{their bank account is empty,} \textit{Indra} will say, \enquote{My dear, now you have used up all your money. You don’t have anything else to bet with, so you have to leave this casino.} Then they have to come back and be born again here on Earth to work towards realization. If they fail to realize God, if they don’t have the chance to surrender to the Feet of the \textit{guru} so that they can be freed in this life, they will return many, many, many, many more times, and keep repeating the same things. Only the ones who surrender attain God-realization. The ones who surrender to the feet of the master are fortunate. The master will carry them across the ocean of \textit{saṁsāra},\footnote{\textit{Karmic} cycle of repeated birth and death.} lead them out of this delusion of birth and death, and make them realized.}

\enquote{People often work with the lower frequencies of nature as they don’t have the knowledge of the higher Self. They start doing lots of shamanic rituals without knowing the true knowledge of the Self and this leads them to an inferior state. Then they are never free. Then they are stuck in that reality. They take birth as those animals, because their minds are so focused on them.}

\enquote{God has placed men on Earth to advance towards Him and not to please the demigods or the entities of nature that have a low vibration. There was a time when all of you incarnated as animals, but that reality was finished a long time ago. When you grow up, you let go of your childhood, so you don’t keep acting like a small kid, do you?}

\enquote{Due to the faith that those ones have in the demigods or the entities of nature, the merciful Lord Kṛṣṇa says, \enquote{Know that they are worshiping Me alone. I am the sole object of all worship.} Even if they end up going to the \textit{lokas} of those demigods or incarnate into a lower state, God will give them opportunities to rise higher, towards God-realization. To have the grace of being on the spiritual path in order to attain the supreme reality is very rare.}

\section{Verse 26: Kṛṣṇa Only Wants the Love of the Devotee}\label{sec-verse-26-krsna-only-wants-the-love-of-the-devotee}

\Verse[9.26]
{patraṁ puṣpaṁ phalaṁ toyaṁ yo me bhaktyā prayacchati \\
tad ahaṁ bhakty-upahṛtam aśnāmi prayatātmanaḥ}
{One who offers to Me a leaf, a flower, a fruit, or some water with devotion, I accept such an offering given with devotion by one whose mind is pure.}

\enquote{This is one of the most beautiful verses in the \textit{Gītā} which shows the humility of the Lord.}

\enquote{How beautiful this is! In Hindu prayers we always offer flowers, leaves, fruits, and water. These things are easily obtained by everybody. The Lord is not asking for big things. Actually, He doesn’t ask for anything! It’s not about His expectation! He is referring to \textit{bhakti}: \enquote{Whatever people are offering to Me, whatever desire, whatever they have inside their hearts}; the most simple things; the offering of Love itself, the sacrifice of Love itself, is what the Lord takes and that is what He accepts. It’s not about the flower. The flower is here and then finish, soon it has disappeared! The water, and the fruits are offered in front of Him, and then after that, we take the offerings as \textit{prasāda}. But, in reality, what the Lord accepts is the Love that is in the heart of the devotee. It’s not through knowledge that one conquers the Lord, but through simplicity, through the simplest things done with devotion. If one offers a leaf, a flower, a few drops of water with gratitude, with a mind focused on the Divine, on the Lord Himself, He accepts that offering of Love.}

\enquote{That intensity of faith and worship, that intensity of Love and surrender that you have, comes from the individual soul. God can’t make you love Him. You love Him as an act of surrender and offering. You choose to love Him. He can’t make that for you. That’s why He longs for that Love. That’s why when one is fully absorbed in devotion, that Love becomes so intensified in oneself that it brings you to the Lord. This intensity of your worship is something that emerges from deep within you, out. This inclination, this power, this truthfulness which emerges from your soul, that’s what Bhagavān longs for. He doesn’t need anything. He has everything. The whole universe, the entire creation is His.}

\enquote{There are so many people who go to temples and do so many offerings, and so many people who go around doing so much service, but they are full of pride. These offerings\ldots don’t even reach His shadow, because that intention is devoid of Love, is devoid of surrender.}

\enquote{What Bhagavān accepts from the heart of the devotee is this Love and surrender. You think that you are surrendering, but actually He is. By the action of surrendering to Him, He runs and surrenders to you.}

\section{Verses 27-33: The Power of Bhakti}\label{sec-verses-27-33-the-power-of-bhakti}

\Verse[9.27]
{yat karoṣi yad aśnāsi yaj juhoṣi dadāsi yat \\
yat tapasyasi kaunteya tat kuruṣva mad-arpaṇam}
{Whatever you do, whatever you eat, whatever you offer in sacrifice, whatever you give, whatever austerity you practice, O Arjuna, do it as an offering to Me.}

\Verse[9.28]
{śubhāśubha-phalair evaṁ mokṣyase karma-bandhanaiḥ \\
sannyāsa-yoga-yuktātmā vimukto mām upaiṣyasi}
{With a mind absorbed in the practice of offering all actions to Me, you will become free from the bondage of \textit{karma} in the form of good and bad results. Once liberated in this way, you will attain Me.}

\Verse[9.29]
{samo ’haṁ sarva-bhūteṣu na me dveṣyo ’sti na priyaḥ \\
ye bhajanti tu māṁ bhaktyā mayi te teṣu cāpy aham}
{I am equal to all beings; there is no one hated by Me nor dear to Me; but those who worship Me with devotion are in Me and I in them.}

\Verse[9.30]
{api cet sudurācāro bhajate mām ananya-bhāk \\
sādhur eva sa mantavyaḥ samyag vyavasito hi saḥ}
{Even if a person of exceedingly wicked conduct turns to Me with exclusive devotion, they must be considered virtuous, for such a person has truly resolved rightly.}

\Verse[9.31]
{kṣipraṁ bhavati dharmātmā śaśvac-chāntiṁ nigacchati \\
kaunteya pratijānīhi na me bhaktaḥ praṇaśyati}
{Quickly such a person becomes righteous and attains everlasting peace. O Kaunteya, let it be declared that My devotee never perishes.}

\Verse[9.32]
{māṁ hi pārtha vyapāśritya ye ’pi syuḥ pāpa-yonayaḥ \\
striyo vaiśyās tathā śūdrās te ’pi yānti parāṁ gatim}
{O Pārtha, having taken refuge in Me, also women, \textit{vaiśyas}, \textit{śūdras}, or even those born from sinful wombs will attain the supreme destination.}

\Verse[9.33]
{kiṁ punar brāhmaṇāḥ puṇyā bhaktā rājarṣayas tathā \\
anityam asukhaṁ lokam imaṁ prāpya bhajasva mām}
{Then what to speak of \textit{brāhmaṇas}, virtuous \textit{bhaktas}, and royal sages. Therefore, having come into this temporary world devoid of happiness, you should adore Me.}

\enquote{Kṛṣṇa says that whatever you are doing in life, breathing, doing your \textit{sādhana} (spiritual practice), eating, drinking, sitting, or listening, make it an act of offering. Make it an act of surrendering to the Lord. Make it a prayer to the Lord and offer the prayer to Him. Offer all the merit of whatever you do to Him. You will attain Him and He, who is not attainable by any \textit{yogī}, will give Himself to you.}

\enquote{When all good actions are offered and dedicated to the Lord, automatically the result is God-realization. But if one does a practice to gain a certain result, this will create \textit{karma} which will lead one to be born again. Like that, one will be thrown into the bondage and attachment to this world. One will have to come back again. And when one comes back again, bear in mind that you won’t know how that life will be. In this life, one may have attained the grace to be on the spiritual path and to associate with people who have a similar aim and goal, so that makes it easy. But one can’t know about the next life. Here Bhagavān says, \enquote{Do everything as an act of surrender to Me, ‘with your soul in union with the Divine through renunciation.’} Don’t be attached to anything. Your mind should only be focusing on the Lord within yourself. The heart of a devotee is not stained by anything, apart from Divine Love. Through that Love, \enquote{you will become free and attain Me.}}

\enquote{At night, before sleeping, close your eyes, think of Nārāyaṇa Kṛṣṇa and say, \enquote{Lord, I offer You my whole day. I place whatever I have done, ‘good’ or ‘not good,’ upon Your altar. I surrender every action to you.} The Lord Himself will grant you what is needed and will release you from the grip of \textit{māyā}.}

\enquote{God doesn’t hate anybody. God loves everyone, but hate and love is the choice of men, and they don’t go together. Either you love or you don’t. Here Bhagavān is equal in all of that. He is like the sun that doesn’t look on whom it is shining. He is like the rain which falls on everybody equally, on the bad and the good. He is the tree which gives shelter to everybody.}

\enquote{There are so many saints who were sinners, but through the grace of the Lord, through the association of the right people, they were transformed and changed. Bhagavān says that no matter how big the sins are, if the sinners surrender to Him with love and devotion, if they focus their minds upon Him and make Him their sole refuge, they will become saints. St. Paul is an example, \textit{Vālmīki} is another example, Mary of Egypt is another example, and like them there are so many others.}

\enquote{When one tries to surrender to Him with devotion, all one’s evilness, all one’s negativity will be overcome. He says, \enquote{Don’t hang onto this negativity. Don’t focus on it. Your aim is to focus on God, not on the negativity.} It’s of no use beating yourself with a whip saying, \enquote{I am a sinner, I am a sinner, I am a sinner.} No! Perceive it, but cross over it, go beyond it. Your mind should focus on God, not on your sin. If you are concentrating on your sin, then you become sinful. Then you accumulate more sin. But if you concentrate and focus on the Lord Himself, then you are free.}



\section{Verse 34: Devotion Takes Us to the Goal}\label{sec-verse-34-devotion-takes-us-to-the-goal}



\Verse[9.34]
{man-manā bhava mad-bhakto mad-yājī māṁ namaskuru \\
mām evaiṣyasi yuktvaivam ātmānaṁ mat-parāyaṇaḥ}
{Focus your mind on Me, be devoted to Me, offer worship to Me, and bow down to Me. Engaging your mind in this way and holding Me as the supreme goal, you will certainly attain Me.}

\enquote{Here He says to Arjuna, \enquote{Focus your mind on Me, be devoted to Me, offer worship to Me.} Can you see the power that is behind this? Kṛṣṇa is not saying this superficially. No! Here is God Himself talking to Arjuna with all power! There is no pride in it. He says, \enquote{Here I am, the Supreme Lord Himself. I’m God Himself. I’m the God of which speak the Vedas. It is I who am here talking to you and I am telling you how to surrender to Me! Focus your mind on Me. Be devoted to Me. Awake. Let your heart dwell on Me. Offer worship to Me. Serve Me in your worship, in your charity, in your daily chores. Do it as a form of worship. Bow down to Me. Be humble. Engaging your mind in this manner and regarding Me as the supreme goal, you will come to Me.} If the mind and the heart are focused on the Lord, and everything one does is done with this attitude of surrender to the Lord, there is no doubt that one will reach Him.}
