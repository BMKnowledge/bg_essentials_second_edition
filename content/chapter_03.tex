\chapter{Karma-yoga}\label{chap-karma-yoga}
\noindent\textbf{CHAPTER 3}\par
\paragraph*{\MakeUppercase{Understanding the Nature of Action}}
%%%%%%%%%%%%%%%%%%%%%%%%%%%%%%%%%%%%%%%%%%%%%%%%%%%%%%%%%

\noindent Despite the explanation of \textit{karma-yoga}, Arjuna remains confused: If one has to renounce action, then why is he being urged to fight? Fundamentally Arjuna does not see the difference between \textit{karma} and \textit{karma-yoga}. \textit{Karma} is any action done with the body or mind that produces further consequences. \textit{Karma-yoga} is about performing action in a detached state, thereby being free of any such consequences.

Kṛṣṇa replies to Arjuna’s question by saying that realization of the Self can be achieved through two methods: by setting aside all duties and contemplating on the \textit{ātmā} (\textit{jñāna-yoga}), or by carrying out one’s duties in a detached way (\textit{karma-yoga}). But, out of the two, \textit{karma-yoga} is the easiest. \textit{Jñāna-yoga}, in the context of the \textit{Gītā}, ultimately requires renunciation of external action, and action as a whole is incredibly difficult to renounce. The very nature of life causes us to act, we cannot escape it. Therefore, we should carry out our duty without any attachment to the results.

By following the example of Kṛṣṇa and other great personalities, action should be performed selflessly for the welfare of the world. Eventually the true \textit{yogī} sees that, despite acting, oneself is not the doer. It is simply the various forms of the material world acting upon each other. Kṛṣṇa concludes chapter three by stating that the root cause of deviating from the path of \textit{yoga} is desire. It clouds our vision and prevents us from realizing true knowledge. As a result, we should strive to conquer this enemy through our spiritual practices.

\section{Verses 1-2: Arjuna's Question}\label{sec-verses-1-2-arjuna-s-question}
\Verse[3.1]
{\hspace*{1em}arjuna uvāca \\
jyāyasī cet karmaṇas te matā buddhir janārdana \\
tat kiṁ karmaṇi ghore māṁ niyojayasi keśava}
{\hspace*{1em}Arjuna said: \\
O Janārdana, if you consider intelligence to be superior to action, then why do you compel me to commit this terrible deed, O Keśava?}

\Verse[3.2]
{vyāmiśreṇeva vākyena buddhiṁ mohayasīva me \\
tad ekaṁ vada niścitya yena śreyo ’ham āpnuyām}
{You are somewhat confusing my intellect with statements that appear to contradict each other; therefore please tell me clearly the one path by which I may attain the highest good.}


\enquote{In deep attention, Arjuna has been listening to Kṛṣṇa, but inside of himself, he is not fully cleansed. That doubt is still there, that mind is still active. He wants to understand in depth the knowledge that the Lord wants to give him\ldots He is asking, \enquote{How can I do certain duties, have certain responsibilities, certain relationships, but not attach myself to them?}}

\enquote{Kṛṣṇa is talking to Arjuna, but in reality He is talking to everyone. Because He is not talking to a person, He is talking to the soul. People may differ in beliefs, they can differ in faith, but knowledge, true knowledge, is universal.}

\enquote{Even if Bhagavān Kṛṣṇa had talked about \textit{jñāna-yoga} and had given Arjuna this great knowledge, they were not on the same wavelength. Their vibration is different. Bhagavān Kṛṣṇa is the Supreme Lord Himself. So, when He is talking, He is in His fullness. Whereas, Arjuna is listening, but he is still in a lower level. He has to rise to the same level so that he can understand.}

\enquote{Here, Kṛṣṇa is not just talking about the knowledge that you read in a book or that you hear from others. He is also talking about the inner knowledge: how to attain God consciousness.}

\enquote{Arjuna surrenders to the Lord. He is humbly submitting himself to the Lord and is willing to change himself\ldots He is still hanging onto the mind, but with attentiveness and with the willingness to change. Arjuna is removing his own will, little by little. He is renouncing his own will so that his inner will, which is in accordance with the \textit{guru}, which is in accordance with Śrī Kṛṣṇa Himself, will free him.}

\enquote{This chapter is about how one can work in daily life, accept what God has given, do one’s \textit{karma-yoga} outside, yet be separate from it; how to be focused and be free from that action. This chapter will allow you to realize that all the work that you have to do here can help you participate in your realization.}


\section{Verses 3-16: The Inescapable Nature of Action}\label{sec-verses-3-16-the-inescapable-nature-of-action}
\Verse[3.3]
{\hspace*{1em}śrī-bhagavān uvāca \\
loke ’smin dvi-vidhā niṣṭhā purā proktā mayānagha \\
jñāna-yogena sāṅkhyānāṁ karma-yogena yoginām}
{\hspace*{1em}Bhagavān Kṛṣṇa said: \\
O sinless one, in this world there are two kinds of paths leading to the realization of the \textit{ātmā} that were mentioned earlier. It is attained by \textit{jñāna-yoga}\footnote{The path of external renunciation, where outside activity is given up in pursuit of inner inquiry.} for the followers of Sāṅkhya, and by \textit{karma-yoga} for the \textit{yogīs}.}

\Verse[3.4]
{na karmaṇām anārambhān naiṣkarmyaṁ puruṣo ’śnute \\
na ca sannyasanād eva siddhiṁ samadhigacchati}
{A person does not attain freedom from \textit{karma} by simply abstaining from duties, and no one attains perfection by renunciation alone.}

\Verse[3.5]
{na hi kaścit kṣaṇam api jātu tiṣṭhaty akarma-kṛt \\
kāryate hy avaśaḥ karma sarvaḥ prakṛti-jair guṇaiḥ}
{No one can ever remain still for even a moment without performing action. Everyone is helplessly forced to act by the \textit{guṇas} born of \textit{prakṛti}.}

\Verse[3.6]
{karmendriyāṇi saṁyamya ya āste manasā smaran \\
indriyārthān vimūḍhātmā mithyācāraḥ sa ucyate}
{One who mentally contemplates on sense objects while restraining the external senses is considered to be a hypocrite whose mind is deluded.}

\Verse[3.7]
{yas tv indriyāṇi manasā niyamyārabhate ’rjuna \\
karmendriyaiḥ karma-yogam asaktaḥ sa viśiṣyate}
{But, Arjuna, one who subdues the senses by the mind, is free from attachment, and engages in \textit{karma-yoga} with the organs of action, excels.}

\Verse[3.8]
{niyataṁ kuru karma tvaṁ karma jyāyo hy akarmaṇaḥ \\
śarīra-yātrāpi ca te na prasidhyed akarmaṇaḥ}
{Perform your prescribed duty, since action is superior to the avoidance of action. Further, not even the sustenance of your body would be possible by inaction.}

\Verse[3.9]
{yajñārthāt karmaṇo ’nyatra loko ’yaṁ karma-bandhanaḥ \\
tad-arthaṁ karma kaunteya mukta-saṅgaḥ samācara}
{This world is bound by all actions other than those performed in the spirit of sacrifice. Therefore, Kaunteya, perform action for the sake of sacrifice, while being free from attachment.}

\Verse[3.10]
{saha-yajñāḥ prajāḥ sṛṣṭvā purovāca prajāpatiḥ \\
anena prasaviṣyadhvam eṣa vo ’stv iṣṭa-kāma-dhuk}
{In the beginning, after creating humans along with \textit{yajña}, Brahmā said: \enquote{By this shall you prosper; let it be that which grants your desired wishes.}}

\Verse[3.11]
{devān bhāvayatānena te devā bhāvayantu vaḥ \\
parasparaṁ bhāvayantaḥ śreyaḥ param avāpsyatha}
{\enquote{By this, may you nurture the \textit{devas}, and may the \textit{devas} in return nurture you. In this way, nurturing one another, you will obtain the highest welfare.}}

\Verse[3.12]
{iṣṭān bhogān hi vo devā dāsyante yajña-bhāvitāḥ \\
tair dattān apradāyaibhyo yo bhuṅkte stena eva saḥ}
{The \textit{devas}, pleased by sacrifice, will surely bestow on you the enjoyments you desire. One who enjoys the gifts of the \textit{devas} without offering them anything in return is verily a thief.}

\Verse[3.13]
{yajña-śiṣṭāśinaḥ santo mucyante sarva-kilbiṣaiḥ \\
bhuñjate te tv aghaṁ pāpā ye pacanty ātma-kāraṇāt}
{The righteous who eat the remnants of sacrifices are freed from all sins. But the sinful ones who cook only for their own gratification eat only sin.}

\Verse[3.14]
{annād bhavanti bhūtāni parjanyād anna-sambhavaḥ \\
yajñād bhavati parjanyo yajñaḥ karma-samudbhavaḥ}
{All beings are sustained by food and food is produced from rain; rain comes from sacrifice; and sacrifice arises from action.}

\Verse[3.15]
{karma brahmodbhavaṁ viddhi brahmākṣara-samudbhavam \\
tasmāt sarva-gataṁ brahma nityaṁ yajñe pratiṣṭhitam}
{Know that ritual action comes from the \textit{Vedas} and the \textit{Vedas} originate from the imperishable Lord; thus the all-pervading Brahman is ever established in sacrifice.}

\Verse[3.16]
{evaṁ pravartitaṁ cakraṁ nānuvartayatīha yaḥ \\
aghāyur indriyārāmo moghaṁ pārtha sa jīvati}
{O Arjuna, one who does not follow this cycle thus set in motion here in this world, leads an impure life and revels in the senses. Such a person lives in vain.}

\enquote{The moment you are born in this world, you are born into the world of action, a world which is in constant movement. There is not a single moment when you are not working. Even when you think you are sitting and relaxing, you are still working. You are breathing, you are thinking, you are working.}

\enquote{One may say, \enquote{Oh, I will not do this work, because there will be no profit in it.} But here the mind is still hanging onto the work. Kṛṣṇa says that if \textit{karma-yoga} is done with such an attitude, there is no freedom. He says, \enquote{No one ever attains perfection by mere renunciation of works.} If one is fake with oneself, if one is not in truth towards oneself, freely and happily renouncing work, one will not be free. One will not do the work outside, but it will still play very much in the mind, saying, \enquote{I have renounced this, I am a great \textit{yogī}. I do my meditation properly. My back is straight.} But the mind always continues to play. This is not really doing meditation! And this is not really doing the job properly, focused on the Divine. It is not the outside that matters, it is God that matters. It is to find God in everything. Put Him first! If He is first, it is easy. But if He is not first, it is very difficult to let go, even if you pretend to let go! The ones who are only pretending are hypocrites!}

\enquote{You can’t stop the senses. You can’t stop the natural tendency of the senses, but you can control it. Even if you try to stop it, how would you do your \textit{karma-yoga}? How would you do your action outwardly? It would never happen, because any form of action is through the senses. However, if the senses are under control, if the mind is under control, then one excels and rises above the senses.}

\enquote{If one reaches the state of a true \textit{bhakta}, where all the activities of the body, mind and senses are completely surrendered, one sees that one is not the doer. One attains the Self; one becomes the Great Observer. One sees that one is the witness of all these activities. Instead of giving up the performance of their duties or stage of life, instead of running away, the devotees who practice the path of knowledge must renounce the sense of doership, the sense of possession, and the attachment to that desire. Lord Kṛṣṇa says, \enquote{One will not reach perfection by renouncing the act, but by transcending it, by transforming it into service to the Lord in whatever one does, and by being separate from self-gratification, self-glory.} Be humble. Remove all this egoism inside of you. Remove all this pride inside of you and be free.}

\enquote{Do everything and offer everything upon the altar of the Lord. Surrender and dedicate all actions to the Lord, and then the action or the \textit{karma} ceases to be a burden because it’s being directed towards Him. So that’s why Bhagavān has, first, given Arjuna the wisdom of the Self. First, He has told him what is the Self, what is the body, what is the mind. Now He is saying, \enquote{Utilize everything.} How will you utilize it? Do your actions without complaining. Accept what God has given you, where God has put you; whatever job He has given you, accept it. Because where He has put you is where you have to be. Maybe you don’t like it. Maybe you think that you could have done much better, but now you are here. What you could have had is in the past. You are not there. Learn to accept it right now without any complaints. Accept the result even if you don’t agree with it, because in the mind, whenever you do something, you have a certain expectation. But you don’t know about the result. Whatever result comes, learn to accept it. If you do that, this attitude will pave the way for \textit{karma-yoga}, the \textit{yoga} of action.}

\section{Verses 17-26: The Responsibility of the Wise}\label{sec-verses-17-26-the-responsibility-of-the-wise}
\Verse[3.17]
{yas tv ātma-ratir eva syād ātma-tṛptaś ca mānavaḥ \\
ātmany eva ca santuṣṭas tasya kāryaṁ na vidyate}
{But the person who delights in the \textit{ātmā} alone, who is satisfied with the \textit{ātmā} and rejoices only in the \textit{ātmā}, duties do not exist.}

\Verse[3.18]
{naiva tasya kṛtenārtho nākṛteneha kaścana \\
na cāsya sarva-bhūteṣu kaścid artha-vyapāśrayaḥ}
{For such a person there is nothing to be gained by acting or renouncing action in this world, nor are they dependent on any other being for any purpose.}

\Verse[3.19]
{tasmād asaktaḥ satataṁ kāryaṁ karma samācara \\
asakto hy ācaran karma param āpnoti pūruṣaḥ}
{Therefore, always perform the work that ought to be done without attachment. One who performs action without attachment certainly attains the Supreme.}

\Verse[3.20]
{karmaṇaiva hi saṁsiddhim āsthitā janakādayaḥ \\
loka-saṅgraham evāpi sampaśyan kartum arhasi}
{Verily, by selfless action alone did Janaka\footnote{A great king who achieved realization through the practice of \textit{karma-yoga}. An ideal example of one who acts in the world but is detached from his actions.} and others reach perfection. Indeed, also considering the welfare of the world, you should act.}

\Verse[3.21]
{yad yad ācarati śreṣṭhas tat tad evetaro janaḥ \\
sa yat pramāṇaṁ kurute lokas tad anuvartate}
{Whatever an eminent person does, other people also do; whatever standard they set, the world follows.}

\Verse[3.22]
{na me pārthāsti kartavyaṁ triṣu lokeṣu kiñcana \\
nānavāptam avāptavyaṁ varta eva ca karmaṇi}
{For Me, O Pārtha, there is nothing unattained which needs to be gained in the three worlds. Although there is no duty for Me, I certainly engage in action.}

\Verse[3.23]
{yadi hy ahaṁ na varteyaṁ jātu karmaṇy atandritaḥ \\
mama vartmānuvartante manuṣyāḥ pārtha sarvaśaḥ}
{For if, at any time, I were not to tirelessly engage Myself in action, O Pārtha, the world would follow My example.}

\Verse[3.24]
{utsīdeyur ime lokā na kuryāṁ karma ced aham \\
saṅkarasya ca kartā syām upahanyām imāḥ prajāḥ}
{If I did not perform action these worlds would perish; I would be the author of confusion and would cause the destruction of these people.}

\Verse[3.25]
{saktāḥ karmaṇy avidvāṁso yathā kurvanti bhārata \\
kuryād vidvāṁs tathāsaktaś cikīrṣur loka-saṅgraham}
{O Bhārata, just as the ignorant act with attachment to their work, so should the wise act without any attachment, being intent on the welfare of the world.}

\Verse[3.26]
{na buddhi-bhedaṁ janayed ajñānāṁ karma-saṅginām \\
joṣayet sarva-karmāṇi vidvān yuktaḥ samācaran}
{A wise person should not disturb the minds of the ignorant who are attached to work. Engaged, one should inspire others while carefully performing all duties.}

\enquote{The realized one doesn’t look for any personal gratification. There is no obligation for him to do something or not do something! In the scriptures there are many rules about what one has to do, and what one has not to do. The \textit{Vedas} are full of these rules. But here, Lord Kṛṣṇa says that the one who is completely surrendered and realizes that it is only God, Lord Nārāyaṇa Himself, who is doing everything, is not bound by what he has to do or not do. He is not bound by his activities. He is not controlled by the senses. Nor is he controlled by the scriptures. The scriptures don’t compel him to do or not to do anything. He doesn’t have anything to abandon nor can he fail by not doing something. Why? Because he is guided. All his actions are guided by the Lord Himself. He is not bound by his own mind. Here Kṛṣṇa says to Arjuna, \enquote{This is another level now! This is a level where only God exists, where there is no ‘I.’}}

\enquote{This doesn’t mean that God-realized souls don’t act in the world. They do! But they are not bound by their actions. They help to maintain the balance in the world, but they have no attachment to what they do or to what they don’t do. They go with their intuition. It’s not them---they are guided. Everything happens, everything is taken care of, automatically. Such a state is the highest good for humanity. Not only for them, but also for the people around. They are centered in the Supreme. In this state, there is no creation of \textit{karma}. Such a person renounces all!}


\enquote{Great heroes and great saints reached perfection during their lifetimes. You don’t need ten lifetimes to do that. You only need one lifetime. You just need complete surrender. The willingness to surrender has to be there!}

\enquote{Here Bhagavān Himself is reminding: \enquote{Look at Me, dear Arjuna. I am free from these three worlds. \textit{Māyā} doesn’t have any power over Me. Illusion is far away. I am free from that and I don’t need anything, because I have created everything. Everything comes from Me. I am in everything. But I’m doing My duty, not running away from My duty. I have taken this form for a purpose.}}

\enquote{The Lord is the ideal person. He never neglects His duty, because He knows that if He does, others will take Him as an example\ldots Thus, this will bring great imbalance to the world.}

\enquote{\enquote{Set an example, My dear Arjuna! People will follow that example and people will also be strong. Be strong in your faith! Be strong in your path so that you can also help others to find their way in life.}}

\enquote{If God has given you a certain duty to do in life, if He has given you a family to look after, if your duty is the family, do your duty properly! If you are a mother, be a good mother! If you are a father, be a good father; if you are a husband or a wife, do your duty accordingly.}

\enquote{When you do your duty properly in the world, it can be a blessing to others. Your children watch you. Your children learn first from you. When they look at you, they imprint your example into their lives. And they will pass on this imprint to their children later on. If you are depressed, if you are weak, that’s what you will give to your children, to humanity, to the world.}

\enquote{Kṛṣṇa says, \enquote{There is salvation for all!} Even if there is a very, egoistic person, there is salvation for him. There is also salvation for people who are into the mind. Nobody is completely lost. But He says, \enquote{Go slowly! Let them first see the changes inside of you. Be the example. They will change. Be patient! They will learn to be patient. Be loving! They will learn to love. Be strong!}}

\newpage
\section{Verses 27-35: Transcending the Guṇas and Karma}\label{sec-verses-27-35-transcending-the-gunas-and-karma}
\Verse[3.27]
{prakṛteḥ kriyamāṇāni guṇaiḥ karmāṇi sarvaśaḥ \\
ahaṅkāra-vimūḍhātmā kartāham iti manyate}
{Actions are in every way performed by the \textit{guṇas} of \textit{prakṛti}. Yet, one whose mind is deluded by the sense of individuality, thinks, \enquote{I am the doer.}}

\Verse[3.28]
{tattva-vit tu mahā-bāho guṇa-karma-vibhāgayoḥ \\
guṇā guṇeṣu vartanta iti matvā na sajjate}
{But one who knows the truth about the distinction of the Self from the \textit{guṇas} and actions, O mighty armed one, realizes that the senses act under the influence of the \textit{guṇas} and thus remains unattached.}

\Verse[3.29]
{prakṛter guṇa-sammūḍhāḥ sajjante guṇa-karmasu \\
tān akṛtsna-vido mandān kṛtsna-vin na vicālayet}
{Those who are deluded by the \textit{guṇas} of \textit{prakṛti} are attached to the actions produced by them. However, one of perfect wisdom should not unsettle the ignorant who have incomplete knowledge.}

\Verse[3.30]
{mayi sarvāṇi karmāṇi sannyasyādhyātma-cetasā \\
nirāśīr nirmamo bhūtvā yudhyasva vigata-jvaraḥ}
{Dedicating all your actions to Me, with a mind centered in the Self, free from desire and possessiveness and having cast aside sorrow, engage in battle.}

\Verse[3.31]
{ye me matam idaṁ nityam anutiṣṭhanti mānavāḥ \\
śraddhāvanto ’nasūyanto mucyante te ’pi karmabhiḥ}
{Those people who have faith, who are free from envy and always practice this teaching of Mine, are also released from \textit{karma}.}

\Verse[3.32]
{ye tv etad abhyasūyanto nānutiṣṭhanti me matam \\
sarva-jñāna-vimūḍhāṁs tān viddhi naṣṭān acetasaḥ}
{But those who disregard My teaching and those who do not practice it, know them to be lost, devoid of reasoning, and deluded with respect to their entire knowledge.}

\Verse[3.33]
{sadṛśaṁ ceṣṭate svasyāḥ prakṛter jñānavān api \\
prakṛtiṁ yānti bhūtāni nigrahaḥ kiṁ kariṣyati}
{Even an enlightened person acts in accordance with his own nature. All beings follow their nature, so what will restraint do?}

\Verse[3.34]
{indriyasyendriyasyārthe rāga-dveṣau vyavasthitau \\
tayor na vaśam āgacchet tau hy asya paripanthinau}
{For each of the senses, attraction and aversion are based on their respective objects. One should not allow oneself to come under their sway, as they are obstacles for a seeker.}

\Verse[3.35]
{śreyān sva-dharmo viguṇaḥ para-dharmāt svanuṣṭhitāt \\
sva-dharme nidhanaṁ śreyaḥ para-dharmo bhayāvahaḥ}
{It is better to carry out one’s own duty imperfectly, than to do another's duty perfectly. Even death in one’s own duty is better, since the duty of another is filled with danger.}

\enquote{Lord Kṛṣṇa says that you just need to have faith and trust in Him and try to follow what He says. He is not imposing. He just says, \enquote{Try it!} He is talking to whoever believes in His word. Whoever trusts and tries to put into practice in their life what He is teaching, whoever tries to awaken His Love inside themselves and love as He loves the world, they will be released from the bondage, from the effect of the three \textit{guṇas}. They will be free from the cycle of birth and death. He is giving His assurance to everybody, to the people of all nationalities, creeds, colors, and castes. He is giving His assurance saying, \enquote{If you try to follow what I say, you will be free! You’ll attain supreme bliss! If you surrender all your actions to Me, it doesn’t matter what you do, you’ll be transformed. And I assure you that if this is done, if you are practicing your \textit{sādhana} or if you are not practicing your \textit{sādhana}, you’ll attain salvation!}}

\enquote{Kṛṣṇa emphasizes that the evil which is described in the scriptures comes from attraction and aversion. On the other hand, normal deeds, inspired by faith and devotion, make one strong, make one renounce all evil deeds, and make one free. Letting go of the vices which are cherished in the heart, one should devote oneself to God, devote oneself to devotional practices, by keeping the mind focused on the Divine. If one does such an action, it will only bear good deeds. Even if someone is in the world outside and can’t let go of the world, but the mind, from time to time, runs to the Divine---for even five minutes a day---just by doing that, one will be free!}

\enquote{Very often people come and say, \enquote{Swamiji, what is my \textit{dharma}?} I say, \enquote{Well, look, what do you do?} \enquote{I work in an office.} I say, \enquote{Okay, that’s your \textit{dharma}. Accept it!} But they don’t want to hear that. Do you know what they want to hear? \enquote{Your \textit{dharma} is to be spiritual.} But they don’t know that spirituality is also in their duty of daily life, in what God has given them to do!}

\enquote{If you don’t accept the outside, what God has given, you will never be able to handle a greater duty when it is given to you. God will not even give you a greater duty to do! That’s why He has placed each one in the world differently. Each one has his own right place and if one sees this perfection of the Lord, one will do one’s \textit{dharma} at that place, at that time.}

\enquote{For some, it is meant to live completely for God! But for some, no. They are still on the way. They are still in preparation; they are still in kindergarten. Very often, you will see this in children: when they look at grown-ups, they also want to be a grown-up. They don’t know what it is to be an adult because they see it only from the outside and say, \enquote{Oh, it’s a party! When will we grow up?} They are so eager to grow up. But then one doesn’t let time carry oneself. One doesn’t learn to trust in the will of the Divine, that God has put everyone where they have to be.}

\enquote{Each one is right where one has to be. Learn to accept this! If you start to fight it, of course, your mind will say, \enquote{Oh, how wonderful it would be if I moved there.} But if you do not go to the right place, this will create a complete imbalance inside of you. And when this confusion starts happening inside of you, you go backwards. Stay wherever you are and accept the will of God, accept that He has put you there, and let yourself flow with it, until He clearly calls you to go elsewhere. If it is meant to be, He will call you, and when you develop this trust, then you are free!}

\enquote{Kṛṣṇa says to Arjuna, \enquote{Be centered in your duty! Even if it is without any merit, don’t try to run away. This would be the reaction of a cowardly person. Better is death in your own duty!} When God calls you to your true \textit{dharma} to do what you have come to this world to do, if you die in that, it is your salvation!}

\section{Verses 36-43: Desire as the Great Enemy of the Wise}\label{sec-verses-36-43-desire-as-the-great-enemy-of-the-wise}
\Verse[3.36]
{\hspace*{1em}arjuna uvāca \\
atha kena prayukto ’yaṁ pāpaṁ carati pūruṣaḥ \\
anicchann api vārṣṇeya balād iva niyojitaḥ}
{\hspace*{1em}Arjuna said: \\
Then, O Kṛṣṇa, what is it that impels one to act wrongly, even against one’s own will, as if compelled by some force?}

\Verse[3.37]
{\hspace*{1em}śrī-bhagavān uvāca \\
kāma eṣa krodha eṣa rajo-guṇa-samudbhavaḥ \\
mahāśano mahā-pāpmā viddhy enam iha vairiṇam}
{\hspace*{1em}Bhagavān Kṛṣṇa said: \\
It is desire born of \textit{rajo-guṇa}\footnote{The \textit{guṇa} relating to the quality of passion and activity.} and its resulting anger. Know it to be the all-devouring, exceedingly wicked enemy in this world.} 

\Verse[3.38]
{dhūmenāvriyate vahnir yathādarśo malena ca \\
yatholbenāvṛto garbhas tathā tenedam āvṛtam}
{Just as a fire is covered by smoke, a mirror by dust, and an embryo by the amnion, so is this world enveloped by desire.}

\Verse[3.39]
{āvṛtaṁ jñānam etena jñānino nitya-vairiṇā \\
kāma-rūpeṇa kaunteya duṣpūreṇānalena ca}
{O Kaunteya, knowledge is veiled by the eternal enemy of the wise in the form of desire, which is as insatiable as fire.}

\Verse[3.40]
{indriyāṇi mano buddhir asyādhiṣṭhānam ucyate \\
etair vimohayaty eṣa jñānam āvṛtya dehinam}
{The senses, the mind, and the intellect are said to be desire’s abode. Through these, desire deludes the \textit{jīva} by concealing its wisdom.}

\Verse[3.41]
{tasmāt tvam indriyāṇy ādau niyamya bharatarṣabha \\
pāpmānaṁ prajahi hy enaṁ jñāna-vijñāna-nāśanam}
{Therefore, O best of the Bharatas, by first controlling the senses, defeat this source of sin that destroys knowledge and direct realization.}

\Verse[3.42]
{indriyāṇi parāṇy āhur indriyebhyaḥ paraṁ manaḥ \\
manasas tu parā buddhir yo buddheḥ paratas tu saḥ}
{The senses are superior to matter, superior to the senses is the mind, higher than even the mind is the intellect, and even greater than the intellect is the \textit{ātmā}.}

\Verse[3.43]
{evaṁ buddheḥ paraṁ buddhvā saṁstabhyātmānam ātmanā \\
jahi śatruṁ mahā-bāho kāma-rūpaṁ durāsadam}
{O mighty-armed Arjuna, knowing that which is greater than the intellect and steadying your mind by the intellect, slay this enemy in the form of desire, which is difficult to overcome.}

\enquote{Desire is everywhere, in many kinds and in many forms. All the actions of people are due to these desires. Desires are like dust which accumulates on the surface of a mirror, and then the mirror can no longer reflect any object. Even so, the mind gets soiled by impurities in the form of sin. It is no longer able to reveal the true nature of things or its own true duty. That’s why someone with an impure and sinful mind cannot correctly judge things. They don’t know what is \enquote{good} and what is \enquote{bad.} In this deep ignorance, the light of Truth is covered. Knowledge gets covered by ignorance\ldots It is desire alone which makes the mind restless. It is through desire that one becomes evil and does sinful acts.}

\enquote{There are two forms of desire. The desire of the mind drags one completely to the outside. The other desire, the longing of the heart, is the gateway to Heaven. When you train yourself to transform the desires of the mind through meditation, and perform your duty, and accept the will of God, then these desires, which can’t be killed, are transformed from \textit{rajasic} to \textit{sattvic} in nature. Don’t go into the game of desires. Let’s say you feel anger: in place of acting on that anger, in place of putting fuel into that fire, try to control yourself. The anger will leave you and transform itself.}

\enquote{When anger arises in the heart of man, it deprives him of his power of discrimination. He is unable to think, so he will not heed the consequences of whatever he does in a fit of anger. He doesn’t realize what he is doing. When the anger grows, one’s memory starts to get confused, and when the brain starts to get confused, one loses all control. One forgets about relationships; one forgets about everyone around. One also forgets what one has to do, or not to do. In this state, one is unable to plan; one doesn’t have any determination. One loses all reasoning. When one doesn’t control the senses, one is reduced to this state. And when a man is reduced to this state, it becomes easy for him to abandon the path of duty, the path of \textit{dharma}. It becomes easy for that person to lose himself.}

\enquote{If someone doesn’t take advantage of this human life to realize and attain God-realization, attain the grace of God, or receive His Vision, this life is wasted\ldots In this lifetime itself, you can attain God. This is mercy. God loves you! God cares for you! God incarnates! God traces your life and makes everything possible for you to realize Him! But He can’t force you to do it. He shows you the way to come to Him, through His Love. He says, \enquote{I am waiting for you! Please come, realize, awaken! First do your effort, control yourself! I will help you.}}

\enquote{Lord Kṛṣṇa says, \enquote{First control the senses: they are easier to control, because they are outside. Control too much sleep. Control what you are eating. Control what you are doing in your actions. Control the mind. Don’t let the mind dwell on the attraction of worldly perishable things, which will bring you sorrow.} Use all the senses, transforming them by serving God, by hearing His glory, by chanting and by concentrating on the Divine Names, on the different Names of God. Use the hands, the touch, to serve God through the action. Like that, you don’t go into the game of worldly enjoyments. Keep yourself busy with the right things. If you start running after these desires, if you entertain them in the mind, it will destroy knowledge and discrimination.}

\enquote{Desire destroys the greater knowledge inside, the knowledge of the Divine. It is through this knowledge that the Divine acts inside people. If this knowledge is veiled, then life will be very difficult, because one will not be guided by knowledge and discrimination. One will always feel self-pity and sadness.}

\enquote{Kṛṣṇa says to Arjuna, \enquote{Be knowledgeable about the Self! Have the power to discriminate! Because, if you have knowledge and the power to discriminate, desire has no power.} You can uproot desire itself and transform it.}

\enquote{In \textit{karma-yoga}, if you do everything in this world with an attitude of surrender, you will attain Him. Kṛṣṇa says to Arjuna, \enquote{Without controlling the senses, it will not be possible. If you give in, you will bring about your own damnation. But with control, you will free yourself. It is in your hands!} God will help you when you start helping yourself. Once you take your first step, He will give you courage; He will give you support; He will bring you to where you have to be.}

