\chapter{Mokṣa-sannyāsa-yoga}\label{chap-moksa-sannyasa-yoga}
\noindent\textbf{CHAPTER 18}\par
\paragraph*{\MakeUppercase{The Path to Liberation}}
%%%%%%%%%%%%%%%%%%%%%%%%%%%%%%%%%%%%%%%%%%%%%%%%%%%%%%%%%

This is the final chapter of the \textit{Bhagavad Gītā} and it is the longest. It begins with Arjuna seeking clarification between two terms \enquote{\textit{sannyāsa}} and \enquote{\textit{tyāga}.} Kṛṣṇa defines \enquote{\textit{sannyāsa}} as the complete giving up of all external duties, while \enquote{\textit{tyāga}} on the other hand is the internal detachment from the fruits of one’s actions. He insists that acts such as ritual, austerity, and charity must always be performed and so naturally \textit{tyāga} is judged as superior to \textit{sannyāsa}.

Carrying on the theme of the previous chapter, Kṛṣṇa delves into a number of aspects in relation to the \textit{guṇas}. The different kinds of renunciation, the different types of knowledge, the nature of action, the performer of action, the different levels of intelligence, the types of determination, and the kinds of happiness one experiences; all of these are discussed in relation to \textit{sattva}, \textit{rajas} and \textit{tamas}.

We find Kṛṣṇa once again returning to \textit{karma-yoga} which dominated the opening chapters. But this time, Kṛṣṇa connects \textit{karma-yoga} with \textit{bhakti}. Previously in the earlier chapters, \textit{karma-yoga} was advocated because it changed the motive behind action, from selfish gain to the detached performance of social duty. Now however, Kṛṣṇa states that action should be performed as a selfless act of worship.

He also reiterates how the perfection of this path brings one to the highest realization. By carrying out our duties while depending entirely on the Lord, we gain full mastery over the mind. Established within, beyond the ways of material existence, we attain God-realization, the highest state of devotion, where we receive the knowledge of who Kṛṣṇa truly is.

Drawing His teaching to a close, Kṛṣṇa emphasizes the path of devotion (\textit{bhakti}), stating it is the easiest and best way to realize true wisdom. Kṛṣṇa advises Arjuna to dedicate all actions to Him and not give into his pride. Even if he refuses to fight, his nature is that of a warrior and so he will be compelled to fight. With that in mind, the Lord advises him to think carefully over everything that has been said and decide for himself what is best. In one last grand declaration, Kṛṣṇa tells Arjuna:

\begin{customquote}\fontsize{10}{11}\selectfont
\textit{Fix your mind on Me, be devoted to Me, worship Me, bow down before Me, and you shall certainly come to Me. I promise you this for certain, since you are dear to Me. Abandon all \textit{dharmas} and surrender to Me alone. I will liberate you from all wrongdoings; do not worry.}

\textit{- Bhagavad Gītā, 18.65-66}
\end{customquote}

After listening intently for eighteen chapters, Arjuna finally tells Kṛṣṇa that his doubts have vanished. His confusion has gone and now he is filled with the determination to carry out the Lord’s instruction.

The concluding verses return to Sañjaya as the narrator. Thrilled at the divine words of Kṛṣṇa, and recalling with ecstasy the cosmic form of the Lord, he tells the blind King Dhṛtarāṣṭra:

\begin{customquote}\fontsize{10}{11}\selectfont
\textit{It is my firm conviction that wherever there is Kṛṣṇa, the Lord of \textit{yoga}, and wherever there is Pārtha the bearer of the bow, there will be prosperity, victory, success, and morality.}

\textit{- Bhagavad Gītā, 18.78}
\end{customquote}



\section{Verses 1-6: Sannyāsa or Tyāga, Which Is Better?}\label{sec-verses-1-6-sannyasa-or-tyaga-which-is-better}

\Verse[18.1]
{\hspace*{1em}arjuna uvāca \\
sannyāsasya mahā-bāho tattvam icchāmi veditum \\
tyāgasya ca hṛṣīkeśa pṛthak keśi-niṣūdana}
{\hspace*{1em}Arjuna said: \\
O mighty-armed Kṛṣṇa, O Lord of the senses and slayer of Keśi! I wish to know distinctly the essence of \textit{sannyāsa} and \textit{tyāga}.}

\Verse[18.2]
{\hspace*{1em}śrī-bhagavān uvāca \\
kāmyānāṁ karmaṇāṁ nyāsaṁ sannyāsaṁ kavayo viduḥ \\
sarva-karma-phala-tyāgaṁ prāhus tyāgaṁ vicakṣaṇāḥ}
{\hspace*{1em}Bhagavān Kṛṣṇa said: \\
The wise understand \textit{sannyāsa} as the relinquishment of actions driven by desire and declare \textit{tyāga} to be the renunciation of the fruits of all action.}

\Verse[18.3]
{tyājyaṁ doṣa-vad ity eke karma prāhur manīṣiṇaḥ \\
yajña-dāna-tapaḥ-karma na tyājyam iti cāpare}
{Certain learned people declare that all action is inherently flawed and should thus be given up, while others maintain that acts of sacrifice, charity, and austerity should not be given up.}

\Verse[18.4]
{niścayaṁ śṛṇu me tatra tyāge bharata-sattama \\
tyāgo hi puruṣa-vyāghra tri-vidhaḥ samprakīrtitaḥ}
{O best of the Bharatas, listen to My definite conclusion regarding \textit{tyāga}, which is said to be of three kinds, O best among men.}

\Verse[18.5]
{yajña-dāna-tapaḥ-karma na tyājyaṁ kāryam eva tat \\
yajño dānaṁ tapaś caiva pāvanāni manīṣiṇām}
{The acts of sacrifice, charity, and austerity should not be given up but should be performed, as they purify even the wise.}

\Verse[18.6]
{etāny api tu karmāṇi saṅgaṁ tyaktvā phalāni ca \\
kartavyānīti me pārtha niścitaṁ matam uttamam}
{But, O Pārtha, even these three practices should be performed while giving up attachment and the desire for reward. That is My decisive and conclusive opinion.}

\enquote{Bhagavān is saying that these are the two kinds of renunciates: the \enquote{\textit{sannyāsī}} and the \enquote{\textit{tyāgī}}; the ones who have let go. But if one takes the path of \enquote{\textit{sannyāsa},} and yet has not let go of the outside from deep within the heart itself, then one has not let go of anything.}

\enquote{Whereas the true sages, whose minds are fully absorbed into the Lord, even if they renounce the world outside, for them there is no difference between \enquote{\textit{sannyāsa}} and \enquote{\textit{tyāga}.} This is why I said that for an enlightened one, it doesn’t matter where they are, what they do. They are free, they are ever-free.}

\enquote{A \enquote{\textit{tyāgī}} is somebody who works in the outside world, yet is not attached to the fruit of their action. Whatever they do, their mind is so focused on the Lord, they are so absorbed in the \textit{sāttvika-bhāva} inside of them, that they don’t get drawn into any external objectives or aims in life, apart from serving the Lord. They do their \textit{karma-yoga}, with their determination fixed only on serving. They have \textit{bhakti}, devotion.}

\enquote{You don’t need to let go of everything and run into a cave, but you have to do your duty. The wise seekers of God-realization, the ones who seek the Lord truly, do their duty in a very joyous way. They do their duty with much Love, with a spirit of disinterest, non-attached to the fruit of the action. They always do it for the benefit of society, for the benefit of others. They do it with an attitude of detachment, without creating any bondage. They do their prayers, their sacrifices, their charity, their penance, but with a pure heart.}

\enquote{Bhagavān says that one must perform one’s duty in this way, because it purifies oneself, it purifies the wise. And one has to get purified, because without being purified, one can’t advance on the spiritual path. By doing one’s duty in such a way, as Bhagavān said, one cleanses the \textit{karma} of the past; one purifies not only this life, but past lives also.}

\enquote{Bhagavān says to people---to all of you who live in the outside world, \enquote{Do your duty full-heartedly!} Do your duty as a husband, a wife, a child, in your job. Whatever you do, do it with full love. And do it with an attitude of surrendering to the Lord. Make your work your prayer, your worship. Make your work your meditation, your \textit{sādhana}, by doing it properly. Bhagavān says that when you have such an attitude, when you have such enthusiasm, it helps to purify you. Then the result doesn’t matter; you don’t have to worry about the result, because if you surrender to the Lord, He will take care of everything.}

\section{Verses 7-12: Different Kinds of Renunciation}\label{sec-verses-7-12-different-kinds-of-renunciation}

\Verse[18.7]
{niyatasya tu sannyāsaḥ karmaṇo nopapadyate \\
mohāt tasya parityāgas tāmasaḥ parikīrtitaḥ}
{Renunciation of prescribed action is never appropriate, however. Renouncing such action out of delusion is said to be \textit{tamasic} in nature.}

\Verse[18.8]
{duḥkham ity eva yat karma kāya-kleśa-bhayāt tyajet \\
sa kṛtvā rājasaṁ tyāgaṁ naiva tyāga-phalaṁ labhet}
{If one gives up any duty because it is difficult, out of fear of bodily hardship, that person, performing \textit{rajasic} renunciation, does not obtain the fruit of renunciation.}

\Verse[18.9]
{kāryam ity eva yat karma niyataṁ kriyate ’rjuna \\
saṅgaṁ tyaktvā phalaṁ caiva sa tyāgaḥ sāttviko mataḥ}
{O Arjuna, when whichever prescribed action is performed as a matter of duty, giving up attachment to it as well as its fruit, such renunciation is considered to be \textit{sattvic}.}

\Verse[18.10]
{na dveṣṭy akuśalaṁ karma kuśale nānuṣajjate \\
tyāgī sattva-samāviṣṭo medhāvī chinna-saṁśayaḥ}
{The wise \textit{tyāgī}, established in \textit{sattva}, is free from doubts and neither resents unpleasant work nor becomes attached to pleasant work.}

\Verse[18.11]
{na hi deha-bhṛtā śakyaṁ tyaktuṁ karmāṇy aśeṣataḥ \\
yas tu karma-phala-tyāgī sa tyāgīty abhidhīyate}
{For embodied beings, it is impossible to give up action entirely. But one who renounces the rewards of action is a true renunciate.}

\Verse[18.12]
{aniṣṭam iṣṭaṁ miśraṁ ca tri-vidhaṁ karmaṇaḥ phalam \\
bhavaty atyāgināṁ pretya na tu sannyāsināṁ kvacit}
{Undesirable, desirable, and mixed---these are the three types of results from action. After death, these results affect those who have not renounced, but never those who have renounced.}

\enquote{Due to this ignorance, one does not perceive that every duty that is allocated to one is very important; it contributes to the advancement of one towards liberation itself. Because every duty that is given is given as a test, to see whether you are ready for something greater or not. If you are not ready, why should God give you realization, why should God give you liberation? If you can’t handle what He has given you in your daily life and you try to run away from it, forget about liberation.}

\enquote{Even a hermit sitting in a forest or a cave is doing a lot of work. Maybe one doesn’t see this from the outside, but they are actually doing a great deal: they pray, they do their duty.}

\enquote{These kind of hermits commit themselves to a certain responsibility in life. They give themselves fully to that duty, because they know that it’s the only thing that connects them to God; and they have great joy in serving the Lord---they always put God first. They don’t bother about themselves; they don’t think, \enquote{This will pollute me. I won’t touch this. I will not do this job, because this job is not my kind of job. I only want a proper job that will be profitable for realization only.} No, whatever God gives, to the one who is surrendered, one shall accept it! Whatever duty has been given, one should take it with great gratitude and do it! That’s a life towards God! That’s a life towards God-realization!}

\enquote{The ones who are fully-realized, the \textit{bhaktas}, do not judge the action that has been given to them. Whether they like it or not, they find the same happiness in doing it, because it has been given by God. They accept it---even if it is \enquote{unpleasant}---even if they have to clean toilets, they will find great joy in taking on that duty, because they do it for God. It’s not for themselves but for God. Even if the duty that is given to them is very easy, very nice, very beautiful, their attitude is the same.}

\enquote{If they have a certain duty which is not according to what they want, they don’t feel, \enquote{Oh no, I can’t do this. This has to be done only by certain other people---it’s their duty, not my duty.} They don’t say, \enquote{This is not my duty, I won’t do this.} No. Whatever is asked, if it has been given as a duty to perform, one should find great joy in it, whether one wants to do it or not. And that’s what Bhagavān says is the sign of a true renunciate; someone with a truly \textit{sattvic} mind.}

\section{Verses 13-19: The Five Kinds of Action}\label{sec-verses-13-19-the-five-kinds-of-action}

\Verse[18.13]
{pañcaitāni mahā-bāho kāraṇāni nibodha me \\
sāṅkhye kṛtānte proktāni siddhaye sarva-karmaṇām}
{O strong-armed Arjuna, learn from Me about the five factors, declared in the doctrine of \textit{Sāṅkhya}, that ensure the accomplishment of all action.}

\Verse[18.14]
{adhiṣṭhānaṁ tathā kartā karaṇaṁ ca pṛthag-vidham \\
vividhāś ca pṛthak ceṣṭā daivaṁ caivātra pañcamam}
{The five factors of action are: the seat of action (the body), the actual performer, the various sense organs, the different kinds of effort, and the Lord Himself, the fifth in this regard.}

\Verse[18.15]
{śarīra-vāṅ-manobhir yat karma prārabhate naraḥ \\
nyāyyaṁ vā viparītaṁ vā pañcaite tasya hetavaḥ}
{Whatever work a person undertakes with body, speech, and mind, whether right or wrong, these five are its cause.}

\Verse[18.16]
{tatraivaṁ sati kartāram ātmānaṁ kevalaṁ tu yaḥ \\
paśyaty akṛta-buddhitvān na sa paśyati durmatiḥ}
{This being the case, the fool who sees only himself as the doer, does not see at all, due to underdeveloped intelligence.}

\Verse[18.17]
{yasya nāhaṅkṛto bhāvo buddhir yasya na lipyate \\
hatvāpi sa imāḻ lokān na hanti na nibadhyate}
{One who is free from the notion of ego and whose understanding is not tainted by action, does not kill and is not bound, even though slaying all these people.}

\Verse[18.18]
{jñānaṁ jñeyaṁ parijñātā tri-vidhā karma-codanā \\
karaṇaṁ karma karteti tri-vidhaḥ karma-saṅgrahaḥ}
{Knowledge, the object of knowledge, and the knower are the three stimulants of action. The instrument, the act itself, and the performer are the three components of action.}

\Verse[18.19]
{jñānaṁ karma ca kartā ca tridhaiva guṇa-bhedataḥ \\
procyate guṇa-saṅkhyāne yathāvac chṛṇu tāny api}
{Knowledge, action, and the performer of action are threefold, according to the division of \textit{guṇas}. Now hear about them properly as described in Saṅkhyā.}

\enquote{Bhagavān says that whatever one does, whether it’s good or evil; whatever sacrifices, charity, austerities, study; whatever work\ldots; is due to these five causes [the body, the performer, the sense organs, the different kinds of effort, and the Lord Himself]. Whatever binds one, according to what one has to do, what kind of category one falls into, what stage of life one is in, whatever temperament, whatever manners, whatever character one has, is due to these five causes. Whatever duty has been given, wherever you are, whatever religion you follow, whatever morality you have, whatever action you perform, is due to these five causes.}

\enquote{The ones who are ignorant don’t see that these qualities of the body are just the game of \textit{prakṛti}. They don’t see that all these qualities are not eternal. Life will carry on again, in another body and so on. They don’t realize that the \textit{ātmā} that is inside of them is eternal. Due to their false identification with the body; due to their claiming that they are doing this, doing that; dancing in this game of good and evil, and evil and good, and so on, they will remain attached to \textit{karma}, and \textit{karma} attached to them. And due to this \textit{karma}, they can’t be free; they don’t see this. Whereas those who have the eyes of true knowledge, will see with the eyes of the heart. And when they see with the eyes of the heart, they perceive the supreme reality of the Self. They perceive that human beings are created to be free, not to be attached to themselves. And automatically they will change, they will become humble! So when one is surrendered to the spiritual path, this is what happens---humbleness awakens.}

\section{Verses 20-22: Knowledge and the Guṇas}\label{sec-verses-20-22-knowledge-and-the-gunas}

\Verse[18.20]
{sarva-bhūteṣu yenaikaṁ bhāvam avyayam īkṣate \\
avibhaktaṁ vibhakteṣu taj jñānaṁ viddhi sāttvikam}
{That knowledge by which one sees one changeless reality in all beings, undivided in different bodies---know it to be \textit{sattvic} in nature.}

\Verse[18.21]
{pṛthaktvena tu yaj jñānaṁ nānā-bhāvān pṛthag-vidhān \\
vetti sarveṣu bhūteṣu taj jñānaṁ viddhi rājasam}
{The knowledge which perceives a multitude of various different beings in all creatures due to their distinct nature---know that knowledge to be \textit{rajasic}.}

\Verse[18.22]
{yat tu kṛtsnavad ekasmin kārye saktam ahaitukam \\
atattvārthavad alpaṁ ca tat tāmasam udāḥrtam}
{But the knowledge that is attached to a single object as if it were everything, which is devoid of logic, oblivious to truth, and insignificant---that knowledge is declared to be \textit{tamasic}.}

\enquote{In this verse, Bhagavān says that those realized souls who have true knowledge of the Self, perceive the Lord everywhere\ldots Even though the bodies are separated, there are diversities, there are dualities; the one with \textit{sattvic} knowledge, with true knowledge, perceives that \textit{vasudhaiva-kuṭumbakaṁ}---there’s only one family. They perceive the Lord inside themselves and in everything; not only in human beings, but also in animals, in plants and in the entire creation.}

\section{Verses 23-25: Action and the Guṇas}\label{sec-verses-23-25-action-and-the-gunas}

\Verse[18.23]
{niyataṁ saṅga-rahitam arāga-dveṣataḥ kṛtam \\
aphala-prepsunā karma yat tat sāttvikam ucyate}
{Prescribed work performed without attachment, without desire or hatred, by one who does not seek any reward, is said to be \textit{sattvic}.}

\Verse[18.24]
{yat tu kāmepsunā karma sāhaṅkāreṇa vā punaḥ \\
kriyate bahulāyāsaṁ tad rājasam udāhṛtam}
{Action performed with a longing to fulfill desire, with egotism, or excessive endeavor, is said to be \textit{rajasic}.}

\Verse[18.25]
{anubandhaṁ kṣayaṁ hiṁsām anapekṣya ca pauruṣam \\
mohād ārabhyate karma yat tat tāmasam ucyate}
{Action that is done out of delusion, irrespective of consequences, loss, injury, and one's own ability is said to be \textit{tamasic}.}

\enquote{Here Bhagavān is again and again reminding Arjuna that one has to do one’s duty without any attachment, without any expectation, even without the mind thinking that one is doing something good to please the Lord.}

\enquote{If people are fully absorbed in the Divine, in God consciousness, it doesn’t matter what the work is, whatever they do---in the eyes of the outside it can be a very terrible work or terrible things---Bhagavān says they are the dearest ones to the Lord. They have this \textit{sattvic} quality.}

\enquote{It doesn’t matter where they come from; what background they have; what predominant qualities they had---whether before they had \textit{rajasic}, \textit{tamasic} or even \textit{sattvic} qualities---they rise above these qualities. A \textit{bhakta} who is fully surrendered rises above even \textit{sattvic} qualities.}

\enquote{That’s why it is said that when you are spiritual, you have to learn to accept what God has given you---whatever your job is, whatever duty has been given to you. A Self-realized person doesn’t question, because who knows best? The Lord knows best what is good for you. A Self-realized person has trust in the Supreme that the duty granted to them is right for them, whatever it is. So here Bhagavān is saying that this is a person with \textit{sattvic} qualities. They don’t have any selfish motives, no negativity is cherished in their heart. Everything gets purified.}

\section{Verses 26-28: The Performer of Action and the Guṇas}\label{sec-verses-26-28-the-performer-of-action-and-the-gunas}

\Verse[18.26]
{mukta-saṅgo ’nahaṁ-vādī dhṛty-utsāha-samanvitaḥ \\
siddhy-asiddhyor nirvikāraḥ kartā sāttvika ucyate}
{A performer of action who is free from attachment and self-absorption, who is endowed with determination and enthusiasm, and is undisturbed in success and failure, is said to be \textit{sattvic}.}

\Verse[18.27]
{rāgī karma-phala-prepsur lubdho hiṁsātmako ’śuciḥ \\
harṣa-śokānvitaḥ kartā rājasaḥ parikīrtitaḥ}
{A performer of action who longs for the fruits of his actions, who is attached, greedy, violent, impure, and affected by joy and sorrow, is said to be \textit{rajasic}.}

\Verse[18.28]
{ayuktaḥ prākṛtaḥ stabdhaḥ śaṭho naiṣkṛtiko ’lasaḥ \\
viṣādī dīrgha-sūtrī ca kartā tāmasa ucyate}
{A performer of action who is undisciplined, vulgar, stubborn, vicious, dishonest, lazy, morose, and a procrastinator, is said to be \textit{tamasic}.}

\enquote{Those who are non-attached, who are free, who don’t lose balance even when they face all kinds of problems, worries, or tests, will take everything as a sacred obligation; they are surrendered to it. They have a mind which is calm and steady; a mind which is still; which is not engaged in any action that harms others or in seeking reward. Bhagavān says that these people have virtue, they have firmness, they have zeal. They love their duty and will not fail to do it, however hard it is. Whatever has been given to them, they will happily do it; they will do it with great joy, with great enthusiasm. So they don’t worry about whether they succeed in what they do, or if they fail. Bhagavān says that these are the qualities of a \textit{sattvic} person, somebody who is endowed with \textit{sattvic} qualities and virtues.}

\enquote{These people, Bhagavān says, receive the Light of God. This Light is hidden, deep inside the heart, hidden from the eyes of everyone. But for the true seeker of God, who loves God and serves Him, for them, the hidden Light is made manifest. Here Bhagavān says that for people with these qualities, nothing is impossible. Because it’s not about them, it’s about thinking that everything revolves around God. For a \textit{bhakta}, the mind of the \textit{bhakta} focuses on Kṛṣṇa all the time. The \textit{bhakta} perceives Kṛṣṇa in whatever they do. They perceive the same in everybody: the Supreme Nārāyaṇa, sitting in everyone, whoever they meet. Because they have perceived the supreme Light within themselves, they have perceived Nārāyaṇa within themselves.}

\newpage

\section{Verses 29-32: Intelligence and the Guṇas}\label{sec-verses-29-32-intelligence-and-the-gunas}

\Verse[18.29]
{buddher bhedaṁ dhṛteś caiva guṇatas tri-vidhaṁ śṛṇu \\
procyamānam aśeṣeṇa pṛthaktvena dhanañjaya}
{O Dhanañjaya, listen to the threefold distinction of intelligence and determination according to the \textit{guṇas}, as I describe them entirely one-by-one.}

\Verse[18.30]
{pravṛttiṁ ca nivṛttiṁ ca kāryākārye bhayābhaye \\
bandhaṁ mokṣaṁ ca yā vetti buddhiḥ sā pārtha sāttvikī}
{That intelligence which understands bondage and liberation, prescribed action and renunciation of action, what ought to be done and what ought not to be done, what should be feared and what should not, is \textit{sattvic}, O Pārtha.}

\Verse[18.31]
{yayā dharmam adharmaṁ ca kāryaṁ cākāryam eva ca \\
ayathāvat prajānāti buddhiḥ sā pārtha rājasī}
{The intelligence by which one does not correctly understand what \textit{dharma} is and what \textit{adharma} is,  what ought to be done and what ought not to be done, is \textit{rajasic}, O Pārtha.}

\Verse[18.32]
{adharmaṁ dharmam iti yā manyate tamasāvṛtā \\
sarvārthān viparītāṁś ca buddhiḥ sā pārtha tāmasī}
{The intelligence which is covered in darkness, that regards \textit{adharma} as \textit{dharma} and all things as their opposite, is considered to be \textit{tamasic}, O Pārtha.}

\enquote{People who discriminate are always in peace. They just know what is wrong and what is right, and this decisiveness inside of them enlightens themselves. They have this sense of discrimination which will reveal to them what they have to do and what not to do. They know what to do to not get bound into this external reality and they know how to free themselves. Such an intellect of a devotee who is surrendered to the Lord has no fear. They don’t have any cowardice. They are not expecting anything. They are introverted. They are absorbed into their divine Self through \textit{japa}, through their noble acts, through their noble sacrifices, penance, through singing of the Divine Name, and through singing glory of the Lord, in everything they do.}

\section{Verses 33-35: Determination and the Guṇas}\label{sec-verses-33-35-determination-and-the-gunas}

\Verse[18.33]
{dhṛtyā yayā dhārayate manaḥ-prāṇendriya-kriyāḥ \\
yogenāvyabhicāriṇyā dhṛtiḥ sā pārtha sāttvikī}
{That determination by which one steadies the actions of the mind, breath, and senses with undeviating \textit{yoga} practice is \textit{sattvic}, O Pārtha.}


\Verse[18.34]
{yayā tu dharma-kāmārthān dhṛtyā dhārayate ’rjuna \\
prasaṅgena phalākāṅkṣī dhṛtiḥ sā pārtha rājasī}
{But the outcome-seeking determination, O Arjuna, by which one holds onto righteousness, desire, and wealth with great attachment, is \textit{rajasic}, O Pārtha.}

\Verse[18.35]
{yayā svapnaṁ bhayaṁ śokaṁ viṣādaṁ madam eva ca \\
na vimuñcati durmedhā dhṛtiḥ sā pārtha tāmasī}
{That determination by which a dull-witted person does not give up sleep, fear, grief, depression, and egotism is \textit{tamasic}, O Pārtha.}

\enquote{If people have this determination, this motivation to achieve their goal, they don’t waver or deviate left or right. They are so focused on their goal; they are fixed on their goal like the lion. When the lion fixes its gaze on a certain prey, there may be hundreds dancing around, but the lion is not distracted; the lion is focused, thinking, \enquote{I will have this delicacy today.} And the lion will go only for the one it has fixed its attention on. To its left or right might be a weak prey---dancing, tumbling around---but it would not bother about that. That’s the aim! To put the ball in the goal! When you watch the World Cup, you watch all the players running; they are determined, they know their aim, they know their goal! They fix their minds towards that the goal. Their vision becomes focused and constant.}

\enquote{When one has this one objective, which is God-realization, and one persists by any means to attain this control of the mind, the senses, control of life; when the mind is always directed towards God for the sake of serving Him, for the sake of attaining His grace, for the sake of seeing Him, for the sake of loving Him, automatically one has control. The mind gets attached to Him, to His form, to His supreme reality. And the mind does not jump left or right when the mind is fully focused on the form of the Lord. This enables man to attain the grace of God quickly! Because of this persistence, God will run towards you.}

\newpage
\section{Verses 36-40: Happiness and the Guṇas}\label{sec-verses-36-40-happiness-and-the-gunas}

\Verse[18.36–37]
{sukhaṁ tv idānīṁ tri-vidhaṁ śṛṇu me bharatarṣabha \\
abhyāsād ramate yatra duḥkhāntaṁ ca nigacchati \\
yat tad agre viṣam iva pariṇāme ’mṛtopamam \\
tat sukhaṁ sāttvikaṁ proktam ātma-buddhi-prasāda-jam}
{O best of the Bharatas, now hear from Me about the three kinds of happiness. The happiness in which one experiences joy as a result of continued practice and which brings an end to suffering, which is like poison in the beginning but in the end like nectar---this happiness is said to be \textit{sattvic}, arising from a serene intellect fixed on the Self.}

\Verse[18.38]
{viṣayendriya-saṁyogād yat tad agre ’mṛtopamam \\
pariṇāme viṣam iva tat sukhaṁ rājasaṁ smṛtam}
{That which is like nectar in the beginning but like poison in the end, stemming from the contact between the senses and their objects---this happiness is said to be \textit{rajasic}.}

\Verse[18.39]
{yad agre cānubandhe ca sukhaṁ mohanam ātmanaḥ \\
nidrālasya-pramādotthaṁ tat tāmasam udāhṛtam}
{The happiness which deludes the Self both at the beginning and at the end and which is based on sleep, laziness, and heedlessness, is declared as \textit{tamasic}.}

\Verse[18.40]
{na tad asti pṛthivyāṁ vā divi deveṣu vā punaḥ \\
sattvaṁ prakṛti-jair muktaṁ yad ebhiḥ syāt tribhir guṇaiḥ}
{There is no being on Earth, in heaven, or among the \textit{devas} who is liberated from these three \textit{guṇas} born of \textit{prakṛti}.}

\enquote{Here Bhagavān says that surely, one who has this deep yearning, listening to the glory of the Lord, chanting His Name, becomes free from suffering. They have endurance, they have strength and power within. They don’t cause harm to anyone, not even to an animal; they let everything take its course in life. They are fully absorbed in eternal happiness, true happiness, and once that happiness is attained, it will never leave.}

\enquote{The mind runs always towards the outside as a result of expectation, of desire: \enquote{I want a car,} \enquote{I want a wife,} \enquote{I want a husband.} \enquote{I want money,} \enquote{I want this,} \enquote{I want that.} When you first get these things, it’s great fun, you feel great joy, happiness. How long do these feelings last? Not for too long. Only until the sense organs meet with the sense object; only for the short-term. And when these feelings subside, the mind jumps to something else. Due to this attachment to sense objects, if one fails to get something the mind desires, one goes into deep depression, one becomes aggressive, one becomes deluded.}

\enquote{In this state, one creates jealousy, one creates a state of competitiveness\ldots So judgment arises, and when they start to judge, compare, compete, they lose all sense of happiness; they lose all their intelligence; they lose all their strength, vitality; they lose their energy, they become exhausted, drained.}

\enquote{Here Bhagavān says that you enjoy short-term happiness, but long-term suffering! This is how people with \textit{rajasic} qualities live; enjoying, and then undergoing terrible suffering of various kinds, in numerous ways.}

\enquote{Bhagavān says, \enquote{As I am eternally present inside you, the realized soul---the one whose mind is constantly absorbed in the Lord---rises above the \textit{guṇas}}\ldots Who is above the \textit{guṇas}? Only Him! No one else. The mind is not above the \textit{guṇas}, even the consciousness is not above the \textit{guṇas}. Here is supra consciousness. So, these realized souls are part of this world, they live their life, but\ldots they are in constant, eternal, union with \textit{Bhagavān Nārāyaṇa}, so all their qualities are transformed into only Him. It’s only Nārāyaṇa Kṛṣṇa that they perceive; wherever they look, they perceive only the Lord. They perceive only God everywhere, in each creature, each being\ldots Their mind, intellect, senses---everything---is in God consciousness constantly. This is what is called God-realization, this surrender to the Divine; rising above the \textit{guṇas}; elevating to God consciousness; leaving this human plane or terrestrial plane and rising even above the celestial plane itself. So by leaving the Earth plane, leaving the celestial plane, leaving the lower planes, one attains the grace of the Lord within one’s heart.}

\section{Verses 41-44: The Four Varṇas}\label{sec-verses-41-44-the-four-varnas}

\Verse[18.41]
{brāhmaṇa-kṣatriya-viśāṁ śūdrāṇāṁ ca parantapa \\
karmāṇi pravibhaktāni svabhāva-prabhavair guṇaiḥ}
{O Parantapa, the duties of the \textit{brāhmaṇas}, \textit{kṣatriyas}, \textit{vaiśyas}, and \textit{śūdras} are divided according to the \textit{guṇas} arising from their inherent nature.}

\Verse[18.42]
{śamo damas tapaḥ śaucaṁ kṣāntir ārjavam eva ca \\
jñānaṁ vijñānam āstikyaṁ brahma-karma svabhāva-jam}
{Calmness of mind and restraint of the senses, austerity, purity, patience, integrity, knowledge, wisdom, and faith in the \textit{Vedas}---this is the duty of \textit{brāhmaṇas}, according to their nature.}

\Verse[18.43]
{śauryaṁ tejo dhṛtir dākṣyaṁ yuddhe cāpy apalāyanam \\
dānam īśvara-bhāvaś ca kṣātraṁ karma svabhāva-jam}
{Heroism, power, determination, skill, not fleeing even in battle, generosity, and leadership---this is the duty of \textit{kṣatriyas}, according to their nature.}

\Verse[18.44]
{kṛṣi-go-rakṣya-vāṇijyaṁ vaiśya-karma svabhāva-jam \\
paricaryātmakaṁ karma śūdrasyāpi svabhāva-jam}
{Agriculture, tending cows, and trade are the natural work of \textit{vaiśyas}; the natural duty of the \textit{śūdras} is service to others.}

\enquote{Some are born into the \textit{Brāhmaṇa} caste. \textit{Brāhmaṇas} here doesn’t mean only priests, it means people who have \textit{sattvic} qualities, who are always aware of God consciousness, always helping other people and so on. Then the \textit{kṣatriyas} are the protectors, the ones who maintain the balance, and they have a mixture of \textit{rajasic} and \textit{sattvic} qualities. The \textit{vaiśyas} the traders, are \textit{rajasic}. And the \textit{śūdras}, of course, people would say they are \textit{tamasic}, but they have a mixture of \textit{rajas} and \textit{tamas}.}

\enquote{Here Bhagavān is not talking only about \textit{Brāhmaṇa} caste as it is understood nowadays. No, He refers to a category of people who have this kind of qualities inside of them---they are \textit{brāhmaṇas}\ldots A \textit{brāhmaṇa} should not have any judgment. A \textit{brāhmaṇa} should respect everyone. No matter what color you are, what creed you are, what faith you belong to. They don’t judge anyone: this is the quality of a \textit{brāhmaṇa}. They perceive the Lord and know that everything is by the grace of God. Bhagavān Kṛṣṇa made this very clear in this verse. This type of purity of mind, senses, body, activities, is what makes one a \textit{brāhmaṇa}.}

\enquote{When Bhagavān Kṛṣṇa speaks here about caste, He is not suggesting that they are either higher or lower in rank. If you listen properly, this is not a caste system as nowadays people understand a caste system. Here, He talks about certain grades within society, which are required to maintain balance across society. But as a result of pride and arrogance, the heads of these castes have corrupted everything; they have placed themselves as being superior. Bhagavān Kṛṣṇa is not talking about the superiority of anyone. He speaks about the distinction of different types of work required to create a balance. Just as the work of the hands relates to everything that the hands do; and the work of the feet is to walk. The feet can’t say, \enquote{Today, I will not walk} and the hands can’t say, \enquote{Okay, now I will start walking.} Then when you start walking, the hands start jumping around; or your head says, \enquote{I will start walking.} No, it doesn’t. If this happened, then complete imbalance would result.}

\section{Verses 45-49: Final Instructions About Karma-yoga}\label{sec-verses-45-49-final-instructions-about-karma-yoga}
\Verse[18.45]
{sve sve karmaṇy abhirataḥ saṁsiddhiṁ labhate naraḥ \\
sva-karma-nirataḥ siddhiṁ yathā vindati tac chṛṇu}
{A person content with their own respective duty attains perfection. Now listen how a person devoted to their own duty achieves perfection.}

\Verse[18.46]
{yataḥ pravṛttir bhūtānāṁ yena sarvam idaṁ tatam \\
sva-karmaṇā tam abhyarcya siddhiṁ vindati mānavaḥ}
{Worshiping Him from whom all beings manifest and by whom this entire cosmos is pervaded, by one’s own duty, a person attains perfection.}

\Verse[18.47]
{śreyān sva-dharmo viguṇaḥ para-dharmāt sv-anuṣṭhitāt \\
svabhāva-niyataṁ karma kurvan nāpnoti kilbiṣam}
{Better is one’s own \textit{dharma}, even if imperfectly done, than the \textit{dharma} of another done perfectly. Performing duty as established by one’s own nature, one doesn't incur sin.}

\Verse[18.48]
{saha-jaṁ karma kaunteya sa-doṣam api na tyajet \\
sarvārambhā hi doṣeṇa dhūmenāgnir ivāvṛtāḥ}
{O Kaunteya, one should not give up one's natural duty even if it has faults, since all endeavors are covered by fault, just as fire is covered by smoke.}

\Verse[18.49]
{asakta-buddhiḥ sarvatra jitātmā vigata-spṛhaḥ \\
naiṣkarmya-siddhiṁ paramāṁ sannyāsenādhigacchati}
{One whose intellect is detached at all times, who has mastered the mind, and is free from desires, attains the supreme perfection of freedom from \textit{karma} through renunciation.}


\enquote{People have to love and respect where God has put them. Whatever your work is in the outside world, that’s your \textit{dharma}. And if you do this properly, your greater \textit{dharma} in life will come later on, when the right moment arises.}

\enquote{Only \textit{tamasic} people believe in coincidence. People who don’t believe in the Lord, they believe in coincidence. But coincidence is all a divine arrangement. He places you where you have to be.}

\enquote{When people do their work with deep gratitude and sincerity, they will attain liberation; they will attain God-realization.}

\enquote{Bhagavān says that wherever you are, it doesn’t matter what your duty is, as long as you worship Him and realize that whatever you do in life is a worship to Him, is a service to Him. If you work in a bank, you are serving the Lord. Just remove greediness and expectation from the mind, and bear in mind deeply that one is serving the Lord.}

\enquote{Bhagavān doesn’t say you have to change your job, no. Work there, but change your attitude, change the way you do it, change your intention! And if the intention is right, it becomes a worship. The action itself transforms into worship, into prayer. So here, Bhagavān says that God-realization is open to all. All are born from Him, all originate from Him, and all are pervaded only by Him. So if a \textit{kṣatriya} worships God with his qualities of valor, strength, and power, and does his duty, he will reach the same state as a \textit{brāhmaṇa} reaches by worshiping the Lord through \textit{mantras} and religious practices. If the mind and the senses are controlled, all practices can be dedicated to God.}

\newpage
\section{Verses 50-53: Final Instructions About Jñāna-yoga}\label{sec-verses-50-53-final-instructions-about-jnana-yoga}

\Verse[18.50]
{siddhiṁ prāpto yathā brahma tathāpnoti nibodha me \\
samāsenaiva kaunteya niṣṭhā jñānasya yā parā}
{O Kaunteya, now learn from Me briefly how one who has achieved this perfection attains Brahman, which is the supreme culmination of knowledge.}

\Verse[18.51–53]
{buddhyā viśuddhayā yukto dhṛtyātmānaṁ niyamya ca \\
śabdādīn viṣayāṁs tyaktvā rāga-dveṣau vyudasya ca \\
vivikta-sevī laghv-āśī yata-vāk-kāya-mānasaḥ \\
dhyāna-yoga-paro nityaṁ vairāgyaṁ samupāśritaḥ \\
ahaṅkāraṁ balaṁ darpaṁ kāmaṁ krodhaṁ parigraham \\
vimucya nirmamaḥ śānto brahma-bhūyāya kalpate}
{Using the purified intellect; subduing the mind with determination; giving up the objects of the senses such as sound and casting aside attraction and hatred; living in solitude; eating little; restraining ones speech, body, and mind; constantly engaged in the yoga of meditation and taking full shelter of dispassion; giving up ego, power, arrogance, desire, anger, and possessiveness, peaceful and free from the sense of ownership, one becomes fit for the state of Brahman.}

\enquote{Those whose minds and intellects no longer entertain attraction towards the outside, whose minds are not longing and running towards sense pleasures, needs, wants; whose thirst for outside enjoyment has ceased; for them, there is cessation of \textit{karma}. Their minds are subdued, controlled, free from all kinds of desires. Here Bhagavān says that such a mind, with such surrender, true discipline, and fully absorbed into \textit{sādhana}, fully absorbed into Love, is free from the bondage of \textit{karma}. The ones whose minds are fully absorbed in the Divine attain God consciousness. They attain the consciousness of the Lord and they start to reflect divine qualities, the divine light.}

\enquote{In whatever one does, wherever one is, one is free from personal gain, one is free from likes and dislikes, and one rises above duality.}

\enquote{When you are above duality, you don’t perceive either good or bad. As I said before, when you sit in meditation, you see the Observer inside of you; this Observer doesn’t feel as the mind perceives you to understand and feel. It’s free. Through your \textit{sādhana}, you are immersed, deep within the true Self. And within the true Self, one sees Nārāyaṇa.}

\section{Verses 54-57: Realizing Pure Bhakti}\label{sec-verses-54-57-realizing-pure-bhakti}

\Verse[18.54]
{brahma-bhūtaḥ prasannātmā na śocati na kāṅkṣati \\
samaḥ sarveṣu bhūteṣu mad-bhaktiṁ labhate parām}
{Having realized Brahman and being tranquil, one neither grieves nor craves for anything. Being equal to all beings, one attains supreme devotion to Me.}

\Verse[18.55]
{bhaktyā mām abhijānāti yāvān yaś cāsmi tattvataḥ \\
tato māṁ tattvato jñātvā viśate tad-anantaram}
{By devotion, one comes to know who and what I truly am. Having known Me in truth in this way, one enters My nature thereafter.}

\Verse[18.56]
{sarva-karmāṇy api sadā kurvāṇo mad-vyapāśrayaḥ \\
mat-prasādād avāpnoti śāśvataṁ padam avyayam}
{Despite constantly performing all actions, the one who has taken refuge in Me, attains the eternal, changeless state by My grace.}

\Verse[18.57]
{cetasā sarva-karmāṇi mayi sannyasya mat-paraḥ \\
buddhi-yogam upāśritya mac-cittaḥ satataṁ bhava}
{Mentally surrendering all actions to Me, regarding Me as the supreme goal, and taking refuge in the \textit{yoga} of discernment, always be mindful of Me.}

\enquote{Here Bhagavān Kṛṣṇa says that this supreme Love doesn’t awaken just like that. It will never awaken in somebody whose mind is polluted, full of judgment, and criticism; who is always running towards the outside; who always sees differences and is always in competition. They will never be calm. One whose mind is always flickering around doesn’t attain the state of calmness, doesn’t attain the state of bliss. But whoever surrenders to Him with full devotion and love; whoever performs their daily work, knowing that whatever they do they are serving only Him; whoever is so absorbed in \textit{bhakti} that nothing distracts them on the outside, there is no entertainment of the \enquote{I} and \enquote{mine}; they are fit to receive the supreme Love.}

\enquote{This is the connection that we all have. We are not connected outside, through the body or through the blood. Different people have come from different wombs of different mothers, but what connects us all is what we have within the heart. It’s the same Lord, the Supreme Lord, seated in all hearts. And when you love, you perceive this. You are not in competition.}

\enquote{Your right hand will not compete with your left hand. Otherwise it would be dramatic, the right hand would go this way; the left hand would go that way. Your feet---imagine your two legs starting to compete with each other. It would be catastrophic! Imagine your right leg moving towards the right and the left going the other way; you will be split into two! This is what Bhagavān is saying: to realize that the whole of creation is in complete union in His cosmic body, and His Supreme Self is seated in the core of everyone’s heart. One who perceives this supreme reality, with Love, will see Nārāyaṇa Kṛṣṇa sitting in the heart of everyone.}

\enquote{When one longs for God, He Himself will come and reveal Himself, and this is God-realization. He reveals Himself in many aspects, and the greatest of these is that He will reveal Himself in the heart of the \textit{bhakta}. When one perceives this divine nature, the Absolute, the supreme aspect, one knows that in His \textit{nirguṇa} aspect the Lord pervades everything. However, when He reveals His supreme aspect to the \textit{bhakta}, it is with the \textit{saguṇa} aspect of Himself.}

\enquote{Bhagavān is saying here that this is the highest form of \textit{bhakti}, \textit{para-bhakti}. When one is fully immersed in the Lord, Bhagavān reveals Himself in one’s own heart, and then one attains one’s spiritual body and attains the Lord Himself.}

\enquote{One will never see the outer vision of the Lord if one can’t perceive the inner vision of the Lord. It is through the inner vision that Bhagavān reveals Himself, reveals His connection. It is through this inner vision that one is cleansed of all impurities. It is through this inner vision that you see His humbleness, and how much He cares for His devotees. Because the only thing He longs for is the Love of the devotee, the Love that you have. And He’s waiting eagerly for you! Imagine how patient He must be. He must be very, very, very, very patient, to wait for you for so many lives! Not just one life, but until you are ready for this Love to awaken. This is how much He loves you. And this is how much you have to learn to love. You have to try your best to awaken this Love.}

\section{Verses 58-63: Act as You See Fit}\label{sec-verses-58-63-act-as-you-see-fit}

\Verse[18.58]
{mac-cittaḥ sarva-durgāṇi mat-prasādāt tariṣyasi \\
atha cet tvam ahaṅkārān na śroṣyasi vinaṅkṣyasi}
{With a mind focused on Me, you will overcome all obstacles by My grace. If, however, out of pride, you do not listen, you will be ruined.}

\Verse[18.59]
{yad ahaṅkāram āśritya na yotsya iti manyase \\
mithyaiṣa vyavasāyas te prakṛtis tvāṁ niyokṣyati}
{If you give in to pride, thinking that you will not fight, your resolve will be in vain---your own nature will compel you.}

\Verse[18.60]
{svabhāva-jena kaunteya nibaddhaḥ svena karmaṇā \\
kartuṁ necchasi yan mohāt kariṣyasy avaśo ’pi tat}
{O Kaunteya, whatever you wish not to do out of delusion, that you will have to do, bound by the duty born from your own nature, despite being unwilling.}

\Verse[18.61]
{īśvaraḥ sarva-bhūtānāṁ hṛd-deśe ’rjuna tiṣṭhati \\
bhrāmayan sarva-bhūtāni yantrārūḍhāni māyayā}
{The Lord abides in the hearts of all beings, O Arjuna. By His power of \textit{māyā}, He causes them to move through life, as if they were seated on a machine.}

\Verse[18.62]
{tam eva śaraṇaṁ gaccha sarva-bhāvena bhārata \\
tat-prasādāt parāṁ śāntiṁ sthānaṁ prāpsyasi śāśvatam}
{O descendant of Bharata, surrender to Him alone with your entire being. By His grace, you shall attain the supreme peace and eternal abode.}

\Verse[18.63]
{iti te jñānam ākhyātaṁ guhyād guhya-taraṁ mayā \\
vimṛśyaitad aśeṣeṇa yathecchasi tathā kuru}
{In this way I have taught you the knowledge which is more secret than secret. After reflecting on it completely, act as you wish.}

\enquote{Here, Bhagavān Kṛṣṇa, as the supreme \textit{guru}, the \textit{jagadguru}, is giving an order to His disciple. He says, \enquote{If you listen to Me, nothing will go wrong. But if you don’t listen to Me, you are at risk.} Then you do what your mind perceives you should do, and for sure you would be doomed. Then you would not have real peace. You would not have real joy in this world or the next. You would fall from your spiritual level and you will not attain Him. Here Bhagavān says if one doesn’t follow what the \textit{guru} says, wholeheartedly, and full of devotion and love, grace will not be granted. It is only through the grace of the master, \textit{guru-kṛpā}, that one achieves complete, ultimate success in life, and one gets the grace of the Lord Himself.}

\enquote{He is reminding not only Arjuna, He is reminding everyone, every \textit{bhakta} on the spiritual path not to go into the fantasy of the mind. The fantasy that the mind creates can be very dangerous. That’s why it is very, very important to take shelter at the feet of the master. Because only the master can free one from one’s self-created illusion. When one creates one’s own illusion, one starts listening to one’s own reason. And then the \textit{tamasic} qualities dominate. And when the \textit{tamasic} qualities dominate, the \textit{bhaktas} are not fit to do their service. They pull themselves away. They will not have this Love, they will not have this feeling. And in place of having this Love and this deepness of surrender, they will develop pride and ego, and all kinds of judgment, all kinds of criticism. Here Bhagavān says to Arjuna, \enquote{Your duty is to be obedient, and just listen to what I am saying! If you listen to Me, to what I’m telling you, you will pass safely through all difficulties, and you won’t worry about anything.} But, Bhagavān also says that if there is egoism, the result will be the opposite.}

\newpage

\section{Verses 64-65: Kṛṣṇa's Eternal Promise}\label{sec-verses-64-65-krsna-s-eternal-promise}

\Verse[18.64]
{sarva-guhyatamaṁ bhūyaḥ śṛṇu me paramaṁ vacaḥ \\
iṣṭo ’si me dṛḍham iti tato vakṣyāmi te hitam}
{Once again hear My supreme teaching, the most secret of all; because you are so dear to Me, I will declare what is beneficial for you.}

\Verse[18.65]
{man-manā bhava mad-bhakto mad-yājī māṁ namaskuru \\
mām evaiṣyasi satyaṁ te pratijāne priyo ’si me}
{Fix your mind on Me, be devoted to Me, worship Me, bow down before Me, and you shall certainly come to Me. I promise you this for certain, since you are dear to Me.}

\section{Verse 66: Complete Surrender to God}\label{sec-verse-66-complete-surrender-to-god}

\Verse[18.66]
{sarva-dharmān parityajya mām ekaṁ śaraṇaṁ vraja \\
ahaṁ tvāṁ sarva-pāpebhyo mokṣayiṣyāmi mā śucaḥ}
{Abandon all \textit{dharmas} and surrender to Me alone. I will liberate you from all wrongdoings; do not worry.}

\enquote{Bhagavān is reminding all on the spiritual path who are surrendered, all who are surrendered to the \textit{guru’s} Feet, that they will attain the grace of the master, the grace of the \textit{guru}. And when one attains the grace of the \textit{guru}, one will attain the grace of Lord Kṛṣṇa Himself. Here He says, \enquote{Abandon all \textit{dharmas}.} One who is free, one who does not hang on duality, one who does not hang onto judgment, will attain this grace. With such judgment---\enquote{I am better than this one,} or \enquote{My way is better}---how would one awaken this Love? How can you love with judgment? You can’t. If you love Bhagavān, you must be fully aware. You have to love Him wholeheartedly, not half-heartedly.}

\enquote{This is the main teaching of the \textit{Vaiṣṇavism}: \enquote{Abandon all \textit{dharmas} and take refuge in Me alone}\ldots For those who are truly surrendered, whose minds are fully absorbed in surrendering to the feet of the master, all the rules are gone; all the regulations are gone, because that abandonment itself, the letting go of every concept and taking refuge at the Feet of the Lord, will awaken the supreme realization.}

\section{Verses 67-71: Spreading the Bhagavad Gītā}\label{sec-verses-67-71-spreading-the-bhagavad-gita}

\Verse[18.67]
{idaṁ te nātapaskāya nābhaktāya kadācana \\
na cāśuśrūṣave vācyaṁ na ca māṁ yo ’bhyasūyati}
{This teaching should never be revealed to one who is devoid of self-discipline, who is not devoted, who has no desire to listen, nor to one who envies Me.}

\Verse[18.68]
{ya idaṁ paramaṁ guhyaṁ mad-bhakteṣv abhidhāsyati \\
bhaktiṁ mayi parāṁ kṛtvā mām evaiṣyaty asaṁśayaḥ}
{There is no doubt that the one who has cultivated supreme devotion to Me and gives this highest mystery to My devotees shall come to Me.}

\Verse[18.69]
{na ca tasmān manuṣyeṣu kaścin me priya-kṛttamaḥ \\
bhavitā na ca me tasmād anyaḥ priya-taro bhuvi}
{There is nobody among people who can perform a service more pleasing to Me than such a person, nor shall there be another on Earth dearer to Me.}

\Verse[18.70]
{adhyeṣyate ca ya imaṁ dharmyaṁ saṁvādam āvayoḥ \\
jñāna-yajñena tenāham iṣṭaḥ syām iti me matiḥ}
{And whoever studies this sacred dialogue of ours, worships Me through the sacrifice of knowledge---this is My conviction.}

\Verse[18.71]
{śraddhāvān anasūyaś ca śṛṇuyād api yo naraḥ \\
saḥ api muktaḥ śubhān lokān prāpnuyāt puṇya-karmaṇām}
{The one who even merely listens to it with faith and free from envy, shall also be liberated and will attain the auspicious realms of those who have performed virtuous deeds.}

\enquote{Bhagavān gives His assurance that it’s not only one who studies or who meditates on it, that benefits; He says that by merely listening to the \textit{Gītā}, one’s sins are eradicated. One is cleansed from all past \textit{karmas}.}

\enquote{Bhagavān has said that not everybody can understand this glory. Some people will just read the \textit{Bhagavad Gītā} as a novel. They will enjoy the words, and then it will disappear. It’s like throwing the seeds on the stone. They will hear it, they will read about it, but it will not make any sense. It will dry up. Some people will hear the glory of the Lord and it will make some sense to them, but they will always look at how they will use it for their own benefit. And those devotees who are surrendered they will drink this nectar; they will meditate continuously upon it and they will grow inwardly in Love and devotion.}

\newpage
\section{Verses 72-73: Arjuna's Delusion is Overcome}\label{sec-verses-72-73-arjuna-s-delusion-is-overcome}

\Verse[18.72]
{kaccid etac chrutaṁ pārtha tvayaikāgreṇa cetasā \\
kaccid ajñāna-sammohaḥ praṇaṣṭas te dhanañjaya}
{O Pārtha, have you listened to this teaching with a single-pointed mind? Has your delusion of ignorance vanished, O Dhanañjaya?}

\Verse[18.73]
{\hspace*{1em}arjuna uvāca \\
naṣṭo mohaḥ smṛtir labdhā tvat-prasādān mayācyuta \\
sthito ’smi gata-sandehaḥ kariṣye vacanaṁ tava}
{\hspace*{1em}Arjuna said: \\
By Your grace, O Acyuta, my delusion has been destroyed and I have regained my understanding. I stand firm, my doubt has left me, and I will carry out Your word.}

\enquote{Realization is not given to everybody; it is only given when one is worthy of it. But when one is worthy of it, even by hearing the \textit{Gītā}, one will be free, one will attain Him. You see how sweet He is? He makes it easy. Earlier He says, \enquote{If you can have devotion, if you can do this and do that\ldots} and finally He says, \enquote{Listen! If you don’t have time to do all this, don’t worry. Offer one \textit{Tulsī} leaf and a few drops of water with love, I will accept this.} This is what Love does. For Love, He left \textit{Vaikuṇṭha}. For His Love for all, He manifests several times. For the Love of His \textit{bhaktas}, He comes to their rescue. It’s due to this Love, nothing else.}

\enquote{Here Arjuna clearly, without any doubt, without any confusion, without any delusion, said, \enquote{Lord, all has been melted away. That knowledge and wisdom which You have placed inside of me, which you have revealed to me, has transformed me into a new person.}}

\enquote{He understands his duty and the \textit{dharma} that is demanded from him. He has attained mental stability. Arjuna is in this state where he can fully accept what Kṛṣṇa has taught him and he is ready to fight.}

\section{Verses 74-78: Final Conclusion}\label{sec-verses-74-78-final-conclusion}

\Verse[18.74]
{\hspace*{1em}sañjaya uvāca \\
ity ahaṁ vāsudevasya pārthasya ca mahātmanaḥ \\
saṁvādam imam aśrauṣam adbhutaṁ roma-harṣaṇam}
{\hspace*{1em}Sañjaya said: \\
Thus have I heard this astounding dialogue between Vāsudeva and the great-minded Pārtha, which makes my hair stand on end.}

\Verse[18.75]
{vyāsa-prasādāc chrutavān etad guhyam ahaṁ param \\
yogaṁ yogeśvarāt kṛṣṇāt sākṣāt kathayataḥ svayam}
{By the grace of Vyāsa, I have heard this supreme mystery of \textit{yoga} directly from Kṛṣṇa, the Lord of \textit{yoga}, as personally spoken by Him.}

\Verse[18.76]
{rājan saṁsmṛtya saṁsmṛtya saṁvādam imam adbhutam \\
keśavārjunayoḥ puṇyaṁ hṛṣyāmi ca muhur muhuḥ}
{O king, repeatedly remembering this astounding and auspicious dialogue between Keśava and Arjuna, I rejoice again and again.}

\Verse[18.77]
{tac ca saṁsmṛtya saṁsmṛtya rūpam aty-adbhutaṁ hareḥ \\
vismayo me mahān rājan hṛṣyāmi ca punaḥ punaḥ}
{And as I repeatedly remember that most astounding form of Hari, my amazement is great, O king, and once more I rejoice again and again.}

\Verse[18.78]
{yatra yogeśvaraḥ kṛṣṇo yatra pārtho dhanurdharaḥ \\
tatra śrīr vijayo bhūtir dhruvā nītir matir mama}
{It is my firm conviction that wherever there is Kṛṣṇa, the Lord of \textit{yoga}, and wherever there is Pārtha the bearer of the bow, there will be prosperity, victory, success, and morality.}

\enquote{What Bhagavān Kṛṣṇa speaks to Arjuna in the \textit{Gītā} is very rich in information; it is rich with advice on how one has to love God. It’s about how to love God and let go of the negative qualities. All these qualities, which Bhagavān Kṛṣṇa talks about in this chapter, are not outside; you are born with them. Some of the qualities are stronger than the others, but nevertheless if you are not aware, if you are not attentive, or if you are not conscious of them, then you let yourself be overruled by these qualities. The best of all qualities is the Love that Bhagavān has put inside your heart and that’s true Love\ldots When you anchor yourself in that Love, nothing can deviate you, nothing can move you! That’s the stability of life!}

\enquote{For those who surrender to the Lord with faith and devotion, the Lord shall reside with them continuously\ldots Only by complete surrender does one attain to the highest perfection. Only by constantly remembering of the Lord and His \textit{līlās}, by associating yourself with the devotees of the Lord and by complete surrender and dedication to the feet of the master, is it possible. Nothing else.}
