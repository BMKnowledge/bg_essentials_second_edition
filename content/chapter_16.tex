\chapter{Chapter 16: Daivāsura-sampad-vibhāga-yoga}\label{chap-daivasura-sampad-vibhaga-yoga}

\paragraph*{\MakeUppercase{The Divine and Demonic Qualities}}
%%%%%%%%%%%%%%%%%%%%%%%%%%%%%%%%%%%%%%%%%%%%%%%%%%%%%%%%%
\noindent In this chapter, the \textit{Gītā} moves away from explanations about the soul and the Lord and instead outlines the various qualities and virtues that make up one’s character. The \enquote{\textit{daivi-sampad}} or the way of the gods, is all about fostering noble and righteous values. Truthfulness, renunciation, not harming, and kindness all fall into this category. The \enquote{\textit{asuri-sampad}} is the way of the \textit{asuras} or unrighteous persons. Qualities such as deceit, arrogance, and ignorance belong to this path.

Those who have an \textit{asuric} nature are devoid of wisdom and refuse to acknowledge the Lord. They engage in cruel acts that bring destruction to the world. They are enslaved by material desires and endlessly try to gather wealth and kill their enemies. They arrogantly enjoy life while performing empty rituals. Such people ignore Kṛṣṇa within and resign themselves to lower realms after death.

Caught in the cycle of rebirth, they fail to attain Kṛṣṇa. Greed, anger, and desire are the three gateways which lead to this path. One must overcome them to attain higher levels of realization.

\section{Verses 1-6: Qualities of the Righteous}\label{sec-verses-1-6-qualities-of-the-righteous}

\Verse[16.1–3]
{\hspace*{1em}śrī-bhagavān uvāca \\
abhayaṁ sattva-saṁśuddhir jñāna-yoga-vyavasthitiḥ \\
dānaṁ damaś ca yajñaś ca svādhyāyas tapa ārjavam \\
ahiṁsā satyam akrodhas tyāgaḥ śāntir apaiśunam \\
dayā bhūteṣv aloluptvaṁ mārdavaṁ hrīr acāpalam \\
tejaḥ kṣamā dhṛtiḥ śaucam adroho nāti-mānitā \\
bhavanti sampadaṁ daivīm abhijātasya bhārata}
{\hspace*{1em}Bhagavān Kṛṣṇa said: \\
Fearlessness, purity of being, the steadfast pursuit of knowledge, charity, self-control, worship, study of the \textit{Vedas}, austerity, honesty; nonviolence, truthfulness, freedom from anger, renunciation, tranquility, freedom from fault-finding; compassion toward all beings, freedom from greed, gentleness, modesty, steadiness; energy, patience, fortitude, purity, freedom from malice, and absence of excessive pride---these are qualities of one born with a divine nature, O Bhārata}

\Verse[16.4]
{dambho darpo ’bhimānaś ca krodhaḥ pāruṣyam eva ca \\
ajñānaṁ cābhijātasya pārtha sampadam āsurīm}
{Hypocrisy, pride, arrogance, harshness, and ignorance, O Pārtha, belong to one born with a demonic nature.}

\Verse[16.5]
{daivī sampad vimokṣāya nibandhāyāsurī matā \\
mā śucaḥ sampadaṁ daivīm abhijāto ’si pāṇḍava}
{A divine disposition leads to liberation, whereas a demonic disposition is considered to lead to bondage. Do not worry, O Pāṇḍava, for you have been born with a divine disposition.}

\Verse[16.6]
{dvau bhūta-sargau loke ’smin daiva āsura eva ca \\
daivo vistaraśaḥ prokta āsuraṁ pārtha me śṛṇu}
{There are two types of beings in this world---the divine and the demonic. The divine has been described in detail; now hear from Me, O Pārtha, about the demonic.}

\enquote{The \textit{bhaktas} who are surrendered are standing in the Lord. It doesn’t matter what happens on the outside, they are not touched by it; not by the calamities, by the sorrow, by the danger, by what is right or what is wrong. Nothing. They are just absorbed into the Divine inside of them. They don’t have any grudge towards anyone, especially not to the ones who are ill-treating them. \textit{Bhaktas} also don’t consider themselves superior to others. They are humble. They don’t run after fame, name, prestige, or whether somebody gives them respect or not; they don’t care. Whether or not somebody praises them, they don’t bother about it. They are not running after that. Because people who run after that, they are into it---they want to be praised, they want to get the glory, the glory of \enquote{I, I, I.} But Bhagavān says that a \textit{bhakta} doesn’t have all this and doesn’t long for all this.}

\enquote{A \textit{bhakta} is humble! Humility is the greatest treasure for the \textit{bhaktas}. And due to that humility, they don’t judge or criticize anyone. They accept everything coming from the Lord Himself. So, this is one of the most important noble virtues of a \textit{bhakta}: to be humble. If one takes a spiritual path, one must be humble to accept what the master says. Only due to that humbleness, one will rise above the duality and attain the divine nature.}

\enquote{Whatever path you go on which is good, which leads you to God-realization, surrender to that path with all humbleness. For the ones who have surrendered with all humbleness, all the other negative qualities disappear from them, little by little. The deeper they go into spirituality, the more they are surrendered, their own will disappears more and more. Only the will of the master and the will of God will pervade them.}

\enquote{So this is the awakening of the divine characteristics. All the characteristics, which were mentioned here earlier, all these divine qualities awake in a \textit{bhakta} who is truly surrendered. The mind doesn’t dwell on the outside but it dwells on advancing deeper and deeper. And that brings one to God-realization.}

\enquote{Those who concentrate upon the negativity, the harshness, pride and arrogance and sing the glory of \enquote{\textit{mea culpa}, \textit{mea culpa},} will never be free. You see, if you take a rope and beat yourself a million times and say, \enquote{I am a sinner, I am a sinner, I am a sinner,} constantly reminding yourself that you are a sinner, you will become a sinner. But if you say you are humble and you remind yourself of the Love of God, you will be free.}

\section{Verses 7-20: Qualities of the Unrighteous}\label{sec-verses-7-20-qualities-of-the-unrighteous}
\Verse[16.7]
{pravṛttiṁ ca nivṛttiṁ ca janā na vidur āsurāḥ \\
na śaucaṁ nāpi cācāro na satyaṁ teṣu vidyate}
{Demonic people know neither right action nor renunciation. Neither purity, proper conduct, nor truthfulness is found in them.}

\Verse[16.8]
{asatyam apratiṣṭhaṁ te jagad āhur anīśvaram \\
aparaspara-sambhūtaṁ kim anyat kāma-haitukam}
{They say that this world is unreal, without any foundation, and without God; that it has arisen from mutual union, with sensual desire as its sole cause}

\Verse[16.9]
{etāṁ dṛṣṭim avaṣṭabhya naṣṭātmāno ’lpa-buddhayaḥ \\
prabhavanty ugra-karmāṇaḥ kṣayāya jagato ’hitāḥ}
{Those who have ruined themselves and possess meager intelligence hold such a view. Hostile to the world, they take birth for its destruction and engage in vile deeds.}

\Verse[16.10]
{kāmam āśritya duṣpūraṁ dambha-māna-madānvitāḥ \\
mohād gṛhītvāsad-grāhān pravartante ’śuci-vratāḥ}
{Resorting to insatiable desire and filled with deceit, pride, and arrogance, these people of impure resolve stubbornly cling to false notions out of delusion and thus act in the world.}

\Verse[16.11–12]
{cintām aparimeyāṁ ca pralayāntām upāśritāḥ \\
kāmopabhoga-paramā etāvad iti niścitāḥ}
{Convinced that sensual enjoyment is the supreme goal and thinking it to be all there is, embracing endless anxieties lasting until the moment of death, bound by hundreds of desires and attachments, devoted to lust and anger, they strive to accumulate wealth through unjust means for the fulfillment of their desires.}

\Verse[16.13–15]
{idam adya mayā labdham imaṁ prāpsye manoratham \\
idam astīdam api me bhaviṣyati punar dhanam \\
asau mayā hataḥ śatrur haniṣye cāparān api \\
īśvaro ’ham ahaṁ bhogī siddho ’haṁ balavān sukhī \\
āḍhyo ’bhijanavān asmi ko ’nyo ’sti sadṛśo mayā \\
yakṣye dāsyāmi modiṣya ity ajñāna-vimohitāḥ}
{Deluded by ignorance, such people declare, \enquote{I have gained this today, and I shall fulfill this [other] desire. This is mine, and that wealth shall also be mine. This enemy has been slain by me, and I shall kill the others as well. I am the lord, I am the enjoyer, I am successful, powerful, and happy. I am wealthy and of noble birth; who else can equal me? I will perform sacrifices, I will give charity, I will rejoice!}}

\Verse[16.16]
{aneka-citta-vibhrāntā moha-jāla-samāvṛtāḥ \\
prasaktāḥ kāma-bhogeṣu patanti narake ’śucau}
{Bewildered by many fantasies, caught by the net of delusion, and addicted to sensual enjoyments, they fall into a filthy hell.}

\Verse[16.17]
{ātma-sambhāvitāḥ stabdhā dhana-māna-madānvitāḥ \\
yajante nāma-yajñais te dambhenāvidhi-pūrvakam}
{Being self-glorifying, stubborn, and intoxicated with wealth, pride, and arrogance, they hypocritically perform sacrifices in name only, contrary to the injunctions of scripture.}

\Verse[16.18]
{ahaṅkāraṁ balaṁ darpaṁ kāmaṁ krodhaṁ ca saṁśritāḥ \\
mām ātma-para-deheṣu pradviṣanto ’bhyasūyakāḥ}
{Overcome with ego, power, pride, desire, and anger, these envious people hate Me, who is present in their own bodies and in those of others.}

\Verse[16.19]
{tān ahaṁ dviṣataḥ krūrān saṁsāreṣu narādhamān \\
kṣipāmy ajasram aśubhān āsurīṣv eva yoniṣu}
{These cruel vicious people, filled with hatred and the lowest of mankind, I continuously cast into the wombs of demonic beings in this \textit{saṁsāra}.}

\Verse[16.20]
{āsurīṁ yonim āpannā mūḍhā janmani janmani \\
mām aprāpyaiva kaunteya tato yānty adhamāṁ gatim}
{Thrown into a demonic womb life after life, these fools fail to attain Me, O Kaunteya, and thus sink to the lowest state of existence.}

\enquote{When you are born on Earth, when the \textit{ātmā} enters the fetus in a womb of a mother, it’s very powerful energy. This energy is like an atomic bomb, or even more powerful than an atomic bomb. So, imagine somebody who is fully negative. Each atom inside of them beams just negativity, they are like \enquote{a power of destruction} for this world. They destroy. Here it doesn’t mean the world as we see it, but they infuse the negativity into the minds of others, destroying the world of other people, destroying the path of other people. They become \enquote{a fount of injury,} they become an instrument \enquote{of injury and evil.} This is where evilness come from; this is where the demons and all the negative qualities come from. They are not out there, they are inside the people themselves. And sometimes, due to that negativity, they want many people to become negative. They have a big agenda: whatever people do, they are doing wrong. That’s their constant judgment of the world and for that, they try to get more and more people into their boat, which will sink them all together.}

\enquote{they have life inside of them, they carry on with the same kind of behavior, attachment, longing, enjoyment of all various kinds of luxurious things, and thinking that this is the sole aim of life.}

\enquote{They perform sacrifices, they perform charities, they do good work just to show off. Because in reality they don’t care. They don’t have this love in sharing, they don’t have the love when they go and do charitable work, service, or prayer to the Lord. To do something really worthy, you have to have this Love inside of you. When one is devoid of this Love, one can’t really do anything good; even if one is appearing to do it, it is not from the heart.}

\enquote{It’s very good to do charity! But do it full-heartedly. And when you do it, put yourself always down, hidden in the corner. I tell you frankly, I know people who do charity and you don’t even know that they have done this charity. Here Bhagavān Kṛṣṇa says that people who act sincerely don’t need to be known. They don’t need any publicity or any award given to them. Because God sees. They don’t need the gratification of the outside.}

\enquote{Of course, the Love of God awakens in everybody’s life once. You can’t say that someone who is living in a \enquote{hellish way} has never experienced the Love of God. Impossible! God is merciful. He reveals throughout and He gives experiences to people. But when people are so much into \textit{māyā}, so much into the negative qualities, so much into this \textit{asuric} nature, they don’t perceive it. It passes and goes unnoticed.}

\enquote{That’s how people get into habits, putting themselves into all kinds of desires and enjoyments. When one has imprinted this in themselves, and solidified in the mind such a reality, they can’t perceive the goodness. It’s not that it’s not possible to help them. It is very possible to help them. But are they willing to listen? No. Are they willing to change their life? No, they are not. But in that case, when you go towards them, you yourself have to be strong in your qualities. You have to be strong on your path. You have to be like a pillar, a house which is built on a rock. Your faith should not be flickering, because if your faith is flickering, you’ll go to people who have these \textit{asuric} qualities and get contaminated. Whereas if your faith is strong and you are well-rooted in your faith and in your inner strength, you can go to them. Then they will not affect you; nothing will touch you. But until you are strong, don’t venture into that. Make yourself strong in your path! Make yourself strong in your belief! Awaken all these divine qualities inside of you! And let these divine qualities reflect through you. Then you will be a light in this world of darkness.}

\section{Verse 21: The Three Gateways to Hell}\label{sec-verse-21-the-three-gateways-to-hell}

\Verse[16.21]
{tri-vidhaṁ narakasyedaṁ dvāraṁ nāśanam ātmanaḥ \\
kāmaḥ krodhas tathā lobhas tasmād etat trayaṁ tyajet}
{Desire, anger, and greed---this is the threefold gate to hell leading to one's destruction. Therefore, one should abandon these three.}

\enquote{There are three qualities that represent the doors of Hell. These doors are desire, anger and greed.}

\enquote{When these three drive man to action, one starts to ruin oneself, ruin the body, ruin the mind. The mind becomes filled with negative evil thoughts, all kinds of duality and judgment. Then, when there are evil thoughts, the evil qualities arise. So all this appears in people’s life when they take the path of \textit{tamasic} qualities. When they take the demonic path, they are robbed of joy, peace, purity. Their life becomes hell, and life after life they will have a similar pattern.}

\enquote{These three qualities---desire, anger and greed---are due to ignorance. Without knowledge of the Self, without any knowledge of the Divine, one ventures downward, pulled to the lower entities and the lower incarnations.}

\section{Verses 22-24: The Importance of Scripture}\label{sec-verses-22-24-the-importance-of-scripture}

\Verse[16.22]
{etair vimuktaḥ kaunteya tamo-dvārais tribhir naraḥ \\
ācaraty ātmanaḥ śreyas tato yāti parāṁ gatim}
{Liberated from these three gates of darkness, a person practices what is favorable to the Self and thereafter reaches the Supreme Abode, O Kaunteya.}

\Verse[16.23]
{yaḥ śāstra-vidhim utsṛjya vartate kāma-kārataḥ \\
na sa siddhim avāpnoti na sukhaṁ na parāṁ gatim}
{Whoever abandons the injunctions of \textit{śāstra} and acts under the influence of personal desire attains neither perfection, happiness, nor the Supreme Abode.}

\Verse[16.24]
{tasmāc chāstraṁ pramāṇaṁ te kāryākārya-vyavasthitau \\
jñātvā śāstra-vidhānoktaṁ karma kartum ihārhasi}
{Therefore, let \textit{śāstra} be your authority in determining what should and should not be done. Knowing the scriptural injunctions, you should perform your duty in this world.}

\enquote{They [righteous persons] don’t let themselves be filled with other things of this world. They drink the nectar of the \textit{Vedas} and meditate upon the essence of the Vedas. \textit{They} do everything to realize the Lord. Studying the \textit{Vedas} is also about chanting, about reciting different hymns in the praise of the Lord and about reflection upon them; not just about chant and finish, but about going into the deepness of it.}

\enquote{The holy scriptures were laid out to perfect yourself, to awaken the divine qualities. But following one’s own reason, following the mind, one can’t think. Following the mind, people will be tempted to desire, and they start to follow all the desires, and through that, the desire wins. When the desire wins, one loses perfection. One loses happiness and one dwells into the lowest spheres. Bhagavān Kṛṣṇa says that one doesn’t attain the \textit{sattvic} qualities.}

\enquote{However, those who are not ruled by their own reason or pride, who take the shelter of the master, seek His advice and blessing, then due to this humbleness, due to this humility, all these other qualities disappear. Then they are following their heart, not their reason, not their mind. When the master, or the Lord Himself, reveals Himself within their heart, they become humble. Only a humble soul, with a humble quality, can surrender to the master.}

\enquote{Taking the shelter of the master, following the scriptures, following what the master says, should be done in a selfless spirit, with a humble heart. This must be done with a heart full of Love. And that’s what will bring the grace upon oneself. It is through this that one attains God-realization.}
