\chapter{Viśvarūpa-darśana-yoga}\label{chap-visvarupa-darsana-yoga}
\chaptersubtitle{The Cosmic Form}
%%%%%%%%%%%%%%%%%%%%%%%%%%%%%%%%%%%%%%%%%%%%%%%%%%%%%%%%%

\noindent Up until now, Kṛṣṇa has only described His identity as the Supreme Lord. But this description has awakened Arjuna’s longing to experience this divinity first-hand. As a result, Arjuna requests Kṛṣṇa to actually show him that which he has heard so much about. But this cosmic form (Viśvarūpa), is beyond the material world. As an act of grace therefore, Kṛṣṇa gifts Arjuna divine vision through which he can behold this inconceivable form.

Because of this grace, Arjuna can suddenly see all creation and all living beings present in this one Person. There are endless faces, weapons, and ornaments with unimaginable effulgence. All the \textit{Vedic} gods and \textit{ṛṣis} are glorifying Kṛṣṇa in this infinite space. Among the many arms, bellies and mouths, there are gnashing teeth and blazing fires. In front of this terrible form, Arjuna is unable to compose himself. Among everything else, he sees the enemies being sucked into the mouths of the Lord. Trembling with fear, Arjuna bows down and begs to know who Kṛṣṇa really is.

Kṛṣṇa replies that He is Time, the great destroyer of worlds. He assures Arjuna that the result of the war has already been decided, and that he is merely an instrument. Overwhelmed with emotion, Arjuna utters prayers to the Lord and begs forgiveness for any offenses done out of ignorance. Arjuna asks Kṛṣṇa to relieve him of this experience and show him instead His four-handed form.

The Lord explains that this cosmic form which has never been seen before, was given out of grace and cannot be obtained through any ritual or austerity. It is only by devotion that one can know, see and enter into Him.


\section{Verses 1-4: Arjuna's Request to See Kṛṣṇa's Cosmic Form}\label{sec-verses-1-4-arjuna-s-request-to-see-krsna-s-cosmic-form}
\Verse[11.1]
{\hspace*{1em}arjuna uvāca \\
mad-anugrahāya paramaṁ guhyam adhyātma-saṁjñitam \\
yat tvayoktaṁ vacas tena moho ’yaṁ vigato mama}
{\hspace*{1em}Arjuna said: \\
My delusion has been dispelled by the supremely secret discourse regarding spiritual knowledge, which You have spoken out of compassion for me.}

\Verse[11.2]
{bhavāpyayau hi bhūtānāṁ śrutau vistaraśo mayā \\
tvattaḥ kamala-patrākṣa māhātmyam api cāvyayam}
{O lotus-eyed Lord, I have heard in great detail about the beginning and end of all beings, and also about Your inexhaustible glory.}

\Verse[11.3]
{evam etad yathāttha tvam ātmānaṁ parameśvara \\
draṣṭum icchāmi te rūpam aiśvaraṁ puruṣottama}
{O Supreme Lord, I long to see Your glorious form, just as You Yourself have described it, O Puruṣottama.}

\Verse[11.4]
{manyase yadi tac chakyaṁ mayā draṣṭum iti prabho \\
yogeśvara tato me tvaṁ darśayātmānam avyayam}
{O Lord, if You deem it possible that this form is seen by me in this way, O Lord of \textit{yoga}, then show me Your imperishable Self.}

\enquote{Here, Arjuna is expressing his gratitude. As long as a \textit{bhakta}, a devotee, hangs onto his own ideas, and as long as he relies on his own will, or effort or practices, but doesn’t count on the divine grace of the Lord, he will not achieve a higher state. By one’s own effort one can only reach a certain level. But\ldots when someone has completely surrendered to Him, one counts not on one’s own effort but on the grace of the Lord, the grace of Kṛṣṇa, and knows that everything is His grace. Whatever happens, whatever is given, is the grace of God. Life itself is the grace of God. It is the grace of the Lord Himself that brings each \textit{bhakta} on each step of their path, to the point where each \textit{bhakta} has to be.}

\enquote{Of course, for a long time he has heard from others---he has heard from Bhīṣma, he has heard from all the sages---how great Kṛṣṇa is. He has heard all the \textit{līlās} of Kṛṣṇa. But at this point, he realizes that due to his pride, due to a lack of this true knowledge, he could not perceive this---he used to see Kṛṣṇa only as his friend, his best friend.}

\enquote{As Arjuna is asking the Lord to see His divine form, it may appear as if there is still doubt. If somebody asks this, you will say, \enquote{You want to see because you don’t trust fully in me!} But it’s not like this\ldots there is no trace of doubt in him. He had full trust in the virtues of the Lord, full trust in His glory, His reality, His power. But due to this yearning (which the Lord Himself put inside of Arjuna), he asks Him for this.}

\enquote{Let’s say that you know someone who is truthful---he always speaks the truth, nothing else. And you have heard from some friends that this man has a philosopher’s stone. You know this person very well, and you know that whatever he says is not untrue. You have full trust in this person. You know his whole life is based on the truth. But you have heard that he has a philosopher’s stone. What would your reaction be? First of all, you would not say directly, \enquote{Let me see!} Firstly, to be polite, you would say, \enquote{I have heard that you have the philosopher’s stone?} And the person would say, \enquote{Yes, I do.} What would be your next question? It would be \enquote{Let me see!} You would ask this question because you want to see it---if you didn’t want to see it, you would not ask the question. It’s not about doubting the person. But this is the curiosity of humans, the excitement. Due to this excitement, you ask to see it. It is the same with Arjuna here. It’s not because he doubts the Lord; he has full trust in the Lord. It is due to this excitement, due to this faith, and the trust that he has in the word of Bhagavān Kṛṣṇa. There is no slight doubt at all. And due to this child-like excitement, he says with folded hands, \enquote{I desire to see Your cosmic form!} This is full trust, full love!}

\section{Verses 5-8: Arjuna is Given Divine Vision}\label{sec-verses-5-8-arjuna-is-given-divine-vision}

\Verse[11.5]
{\hspace*{1em}śrī-bhagavān uvāca \\
paśya me pārtha rūpāṇi śataśo ’tha sahasraśaḥ \\
nānā-vidhāni divyāni nānā-varṇākṛtīni ca}
{\hspace*{1em}Bhagavān Kṛṣṇa said: \\
Pārtha, behold My hundreds and thousands of divine forms of various kinds and of different shapes, colors, and sizes.}

\Verse[11.6]
{paśyādityān vasūn rudrān aśvinau marutas tathā \\
bahūny adṛṣṭa-pūrvāṇi paśyāścaryāṇi bhārata}
{O descendant of Bharata, see the Ādityas, the Vasus, the Rudras, the Aśvin-kumāras,\footnote{Twin celestial deities responsible for medicine.} and the Maruts. Behold many wondrous things which have never been seen before.}

\Verse[11.7]
{ihaika-sthaṁ jagat kṛtsnaṁ paśyādya sa-carācaram \\
mama dehe guḍākeśa yac cānyad draṣṭum icchasi}
{O Guḍākeśa, witness the whole universe with moving and non-moving, and whatever else you wish to see, all here in one place within My body.}

\Verse[11.8]
{na tu māṁ śakyase draṣṭum anenaiva sva-cakṣuṣā \\
divyaṁ dadāmi te cakṣuḥ paśya me yogam aiśvaram}
{But you cannot see Me with your own eyes. Therefore, I give you divine eyes. Behold My majestic power.}

\enquote{The Lord, who is inside the heart of all, the Lord that dwells within the core of the \textit{ātmā} itself, knew why Arjuna couldn’t see. Because Arjuna, even though he had transcended this doubt, is still searching in the outside reality. Even if his mind has no doubt, yet he has this one desire to see Him in His cosmic form. It’s not that he doesn’t believe what Kṛṣṇa is saying to him. He is drinking it fully. Kṛṣṇa is standing in front of him saying, \enquote{This body itself, this body here, has everything here in front of you! All creation, all universes, all the stars, everything ‘moving and unmoving,’ all the Ādityas, Rudras, everything!} And Arjuna is standing there looking at Him. Without any doubt, he is looking at His form that he knows. All this time, Arjuna has known Kṛṣṇa as his dear friend.}

\enquote{The physical eyes lack the faculty due to their limitation in seeing only matter. They hang only on matter, only on the limitation, so they can’t grasp this reality. For this, other eyes must be given to see and perceive the cosmic form of the Lord. So the Lord endowed Arjuna with the cosmic power, the supernatural power, to have the real vision of the Lord Himself. You need divine eyes to perceive an innermost seeing. With these divine eyes, even if you appear as though you are looking outside, the vision is the inner vision, which projects itself outwardly.}

\newpage
\section{Verses 9-14: The Cosmic Form of Kṛṣṇa Revealed}\label{sec-verses-9-14-the-cosmic-form-of-krsna-revealed}

\Verse[11.9–11]
{\hspace*{1em}sañjaya uvāca \\
evam uktvā tato rājan mahā-yogeśvaro hariḥ \\
darśayām āsa pārthāya paramaṁ rūpam aiśvaram \\
aneka-vaktra-nayanam anekādbhuta-darśanam \\
aneka-divyābharaṇaṁ divyānekodyatāyudham \\
divya-mālyāmbara-dharaṁ divya-gandhānulepanam \\
sarvāścarya-mayaṁ devam anantaṁ viśvato-mukham}
{\hspace*{1em}Sañjaya said: \\
O king, having spoken thus, the great Lord of \textit{yoga}, Śrī Hari, then revealed to Pārtha His supreme majestic form possessing many mouths and eyes, innumerable astonishing features, numerous divine ornaments, having many divine weapons raised aloft, wearing celestial garlands and garments, being anointed with divine perfumes---the limitless Lord facing all directions, full of wonders.}

\Verse[11.12]
{divi sūrya-sahasrasya bhaved yugapad utthitā \\
yadi bhāḥ sadṛśī sā syād bhāsas tasya mahātmanaḥ}
{If the radiance of a thousand suns were to simultaneously rise in the sky, that brilliance might be comparable to the radiance of that Supreme Being.}

\Verse[11.13]
{tatraika-sthaṁ jagat kṛtsnaṁ pravibhaktam anekadhā \\
apaśyad deva-devasya śarīre pāṇḍavas tadā}
{Then the son of Pāṇḍu saw there the entire universe with many divisions, situated in one place, all within the body of the God of gods.}

\Verse[11.14]
{tataḥ sa vismayāviṣṭo hṛṣṭa-romā dhanañjayaḥ \\
praṇamya śirasā devaṁ kṛtāñjalir abhāṣata}
{Then Arjuna, overcome with amazement and his hair standing on end, bowed his head to the Lord, and began to speak with hands joined together.}

\enquote{With this spiritual vision, Arjuna is baffled; he doesn’t know what to express, or how to express himself, because he has the ultimate sense of devotion. He just kneels in deep devotion in front of the Lord. And seeing this magnitude aspect, he can’t do anything. At that moment he realizes how small he is in the universe. He looks at the Lord with His cosmic body in awe, wonder, and amazement. \enquote{He can’t say this is beautiful. This is the beauty of beauties, it is beyond beautiful; the greatest of the great; and everything is sublime in itself.} He even feels fear looking at this vision, because he realizes at that moment how small he is in comparison to the Lord. Arjuna realizes, \enquote{I, who have been looking at the greatness of the Lord Himself, have realized that I wanted to understand Him, but how could I?} He is trembling, thinking, \enquote{He, who I took as my friend, who was related to me, my goodness!} This is when ignorance is removed; one perceives the Lord in His glory everywhere. But this is more than His glory. This is His cosmic form that never, ever, had God manifested and shown to anyone.}


\section{Verses 15-31: Arjuna Describes the Lord's Form}\label{sec-verses-15-31-arjuna-describes-the-lord-s-form}


\Verse[11.15]
{\hspace*{1em}arjuna uvāca \\
paśyāmi devāṁs tava deva dehe sarvāṁs tathā bhūta-viśeṣa-saṅghān \\
brahmāṇam īśaṁ kamalāsana-stham ṛṣīṁś ca sarvān uragāṁś ca divyān}
{\hspace*{1em}Arjuna said: \\
O Lord, in Your body I see all the \textit{devas} and all the various species of beings, Brahmā seated on a lotus, Śiva, all the sages, as well as celestial serpents.}

\Verse[11.16]
{aneka-bāhūdara-vaktra-netraṁ paśyāmi tvāṁ sarvato ’nanta-rūpam \\
nāntaṁ na madhyaṁ na punas tavādiṁ paśyāmi viśveśvara viśva-rūpa}
{I see Your boundless form everywhere, having innumerable arms, torsos, mouths, and eyes. I see neither end, middle, nor beginning to You, O Lord of the universe, O universal form.}

\Verse[11.17]
{kirīṭinaṁ gadinaṁ cakriṇaṁ ca tejo-rāśiṁ sarvato dīptimantam \\
paśyāmi tvāṁ durnirīkṣyaṁ samantād dīptānalārka-dyutim aprameyam}
{I behold You with a crown, mace, and discus---a mass of brilliance, radiant in all directions, completely unbearable to gaze upon, and possessing an unfathomable brilliance like a burning fire or the sun.}

\Verse[11.18]
{tvam akṣaraṁ paramaṁ veditavyaṁ \\
tvam asya viśvasya paraṁ nidhānam \\
tvam avyayaḥ śāśvata-dharma-goptā \\
sanātanas tvaṁ puruṣo mato me}
{You are the imperishable supreme, the One to be realized. You are the supreme refuge of this universe. You are unchanging and the protector of the eternal \textit{dharma}; I perceive You to be the eternal Supreme Being.}

\Verse[11.19]
{anādi-madhyāntam ananta-vīryam ananta-bāhuṁ śaśi-sūrya-netram \\
paśyāmi tvāṁ dīpta-hutāśa-vaktraṁ sva-tejasā viśvam idaṁ tapantam}
{I see You as being without beginning, middle, and end, possessing limitless creative power, having countless arms, with the sun and moon as eyes, and a mouth like blazing fire, engulfing the entire universe in its radiance.}

\Verse[11.20]
{dyāv ā-pṛthivyor idam antaraṁ hi \\
vyāptaṁ tvayaikena diśaś ca sarvāḥ \\
dṛṣṭvādbhutaṁ rūpam ugraṁ tavedaṁ \\
loka-trayaṁ pravyathitaṁ mahātman}
{The space between Heaven and Earth, and all the directions, are pervaded by You alone. Beholding Your astounding and terrible form, O Great Soul, the three worlds\footnote{Referenced in \textit{Vedic} texts as Heaven, Earth, and the netherworld.} are trembling.}

\Verse[11.21]
{amī hi tvāṁ sura-saṅghā viśanti \\
kecid bhītāḥ prāñjalayo gṛṇanti \\
svastīty uktvā maharṣi-siddha-saṅghāḥ \\
stuvanti tvāṁ stutibhiḥ puṣkalābhiḥ}
{These multitudes of \textit{devas} are entering into You, and some out of fear pray with joined palms. The collective of great sages and perfected beings proclaim, \enquote{Let there be prosperity,} and glorify You with abundant hymns of praise.}

\Verse[11.22]
{rudrādityā vasavo ye ca sādhyā viśve ’śvinau marutaś coṣmapāś ca \\
gandharva-yakṣāsura-siddha-saṅghā vīkṣante tvāṁ vismitāś caiva sarve}
{The Rudras, the Ādityas, the Vasus, the Sādhyas,\footnote{Class of divine beings dwelling in the celestial planets.} the Viśvadevas,\footnote{A term used to describe all the celestial deities as a whole.} the Aśvinīs, the Maruts, the ancestors, and the hosts of Gandharvas, Yakṣas, \textit{asuras}, and Siddhas---all gaze upon You in amazement.}

\Verse[11.23]
{rūpaṁ mahat te bahu-vaktra-netraṁ mahā-bāho bahu-bāhūru-pādam \\
bahūdaraṁ bahu-daṁṣṭrā-karālaṁ dṛṣṭvā lokāḥ pravyathitās tathāham}
{O mighty-armed One! Beholding Your great form with many faces and eyes, many arms, thighs, and feet, having many bellies and dreadful fangs, the worlds, as well as myself, are terrified.}

\Verse[11.24]
{nabhaḥ-spṛśaṁ dīptam aneka-varṇaṁ \\
vyāttānanaṁ dīpta-viśāla-netram \\
dṛṣṭvā hi tvāṁ pravyathitāntar-ātmā \\
dhṛtiṁ na vindāmi śamaṁ ca viṣṇo}
{O Viṣṇu, seeing You touching the sky, being radiant and multi-colored, with Your gaping mouths and huge glaring eyes, my heart is terrified and I can find neither composure nor peace.}

\Verse[11.25]
{daṁṣṭrā-karālāni ca te mukhāni dṛṣṭvaiva kālānala-sannibhāni \\
diśo na jāne na labhe ca śarma prasīda deveśa jagan-nivāsa}
{O Lord of the \textit{devas}! O shelter of the universe! Seeing Your mouths and awesome teeth that resemble the flames of cosmic destruction, I am disoriented and don't find any peace. Have mercy on me!}

\Verse[11.26–27]
{amī ca tvāṁ dhṛtarāṣṭrasya putrāḥ sarve sahaivāvani-pāla-saṅghaiḥ \\
bhīṣmo droṇaḥ sūta-putras tathāsau sahāsmadīyair api yodha-mukhyaiḥ \\
vaktrāṇi te tvaramāṇā viśanti daṁṣṭrā-karālāni bhayānakāni \\
kecid vilagnā daśanāntareṣu sandṛśyante cūrṇitair uttamāṅgaiḥ}
{All the sons of Dhṛtarāṣṭra, together with the host of other kings, Bhīṣma, Droṇa, and Karṇa, along with even our principal warriors, are all rushing into Your fearsome mouths with terrible fangs. Some are seen to be caught between Your teeth with their heads crushed.}

\Verse[11.28]
{yathā nadīnāṁ bahavo ’mbu-vegāḥ samudram evābhimukhā dravanti \\
tathā tavāmī nara-loka-vīrā viśanti vaktrāṇy abhivijvalanti}
{Just as the countless torrents of rivers rush toward the ocean, so do these heroes of the world enter Your blazing mouths.}

\Verse[11.29]
{yathā pradīptaṁ jvalanaṁ pataṅgā viśanti nāśāya samṛddha-vegāḥ \\
tathaiva nāśāya viśanti lokās tavāpi vaktrāṇi samṛddha-vegāḥ}
{As moths rush with great speed into a blazing fire for their annihilation, so too do these people swiftly enter Your mouths to meet their destruction.}

\Verse[11.30]
{lelihyase grasamānaḥ samantāl lokān samagrān vadanair jvaladbhiḥ \\
tejobhir āpūrya jagat samagraṁ bhāsas tavogrāḥ pratapanti viṣṇo}
{O Viṣṇu! Devouring all the worlds on every side, You incessantly lick them up with flaming mouths. Your fierce radiance scorches the entire universe, filling it with brilliant rays.}

\Verse[11.31]
{ākhyāhi me ko bhavān ugra-rūpo namo ’stu te deva-vara prasīda \\
vijñātum icchāmi bhavantam ādyaṁ na hi prajānāmi tava pravṛttim}
{Tell me who You are, O Lord possessing a fearsome form! Glories to You, O best of \textit{devas}. You are the Original Person whom I wish to know, be merciful to me. I cannot comprehend Your actions.}

\enquote{Here Arjuna is in front of God Himself, shining like 1000 suns all together. What he sees at that moment is incomparable. But yet, due to the divine vision, he could perceive this transcendent infinite sun, this effulgence of the Lord as the cosmic Light itself.}

\enquote{Arjuna beholds in the person of the Supreme Deity the entire universe: many visions and shapes; each of the gods that Bhagavān Kṛṣṇa has mentioned earlier are inside of Him in their fullness; all the beasts, animals, demigods, plants; all creation---from the smallest one to the biggest one; are all present inside of Him. All the pleasures; all the pains; all the joys appear in His cosmic body.}

\enquote{Here, note one thing---this cosmic form of the Lord in front of him on the battlefield, nobody else can see. For those on the outside, the Kauravas and the \textit{Pāṇḍavas}, they can only see Arjuna standing in front of Kṛṣṇa full of deep emotion. They can feel that something else is happening, but they can’t understand what it is. They are wondering, \enquote{What is happening?} Because this vision doesn’t happen on this field, in this reality; it surpasses this reality. With the divine sight that Arjuna has received, he perceives the Lord in His glory. Everything on the battlefield disappears. He sees only the Lord there in His cosmic form.}

\section{Verses 32-34: Kṛṣṇa is Time, The Great Destroyer}\label{sec-verses-32-34-krsna-is-time-the-great-destroyer}

\Verse[11.32]
{\hspace*{1em}śrī-bhagavān uvāca \\
kālo ’smi loka-kṣaya-kṛt pravṛddho lokān samāhartum iha pravṛttaḥ \\
ṛte ’pi tvāṁ na bhaviṣyanti sarve ye ’vasthitāḥ praty-anīkeṣu yodhāḥ}
{\hspace*{1em}Bhagavān Kṛṣṇa said: \\
I am Time, the mighty destroyer of the world, engaged in annihilating all these worlds. Even without you, none of the warriors gathered here for battle shall survive.}

\Verse[11.33]
{tasmāt tvam uttiṣṭha yaśo labhasva  \\
jitvā śatrūn bhuṅkṣva rājyaṁ samṛddham \\
mayaivaite nihatāḥ pūrvam eva  \\
nimitta-mātraṁ bhava savya-sācin}
{O great Archer! Therefore rise and win glory! After conquering the enemies, enjoy a prosperous kingdom. They have already been slain by Me; become merely My instrument!}

\Verse[11.34]
{droṇaṁ ca bhīṣmaṁ ca jayadrathaṁ ca \\
karṇaṁ tathānyān api yodha-vīrān \\
mayā hatāṁs tvaṁ jahi mā vyathiṣṭhā  \\
yudhyasva jetāsi raṇe sapatnān}
{Droṇa, Bhīṣma, Jayadratha, Karṇa, as well as other heroic warriors, are already slain by Me, thus do not hesitate and kill them. Fight and you shall defeat your enemies in battle.}

\newpage
\section{Verses 35-46: Arjuna is Overcome With Fear}\label{sec-verses-35-46-arjuna-is-overcome-with-fear}

\Verse[11.35]
{\hspace*{1em}sañjaya uvāca \\
etac chrutvā vacanaṁ keśavasya kṛtāñjalir vepamānaḥ kirīṭī \\
namaskṛtvā bhūya evāha kṛṣṇaṁ sa-gadgadaṁ bhīta-bhītaḥ praṇamya}
{\hspace*{1em}Sañjaya said: \\
Hearing these words of Keśava, Arjuna shuddered and placed his palms together. Bowing down to Kṛṣṇa with fear and awe, he paid his respects and spoke in a choked voice.}

\Verse[11.36]
{\hspace*{1em}arjuna uvāca \\
sthāne hṛṣīkeśa tava prakīrtyā jagat prahṛṣyaty anurajyate ca \\
rakṣāṁsi bhītāni diśo dravanti sarve namasyanti ca siddha-saṅghāḥ}
{\hspace*{1em}Arjuna said: \\
O Kṛṣṇa, it is proper that the world is delighted and becomes filled with affection by glorifying You. While the terrified demons flee in all directions, the hosts of perfected beings pay homage to You.}

\Verse[11.37]
{kasmāc ca te na nameran mahātman garīyase brahmaṇo ’py ādi-kartre \\
ananta deveśa jagan-nivāsa tvam akṣaraṁ sad-asat tat paraṁ yat}
{And why should they not bow down to You, for You are superior to even Brahmā, the first creator. O Infinite One, Lord of \textit{devas}, O refuge of the universe! You are the Imperishable One, the manifest and the unmanifest, and that which is beyond both.}

\Verse[11.38]
{tvam ādi-devaḥ puruṣaḥ purāṇas tvam asya viśvasya paraṁ nidhānam \\
vettāsi vedyaṁ ca paraṁ ca dhāma tvayā tataṁ viśvam ananta-rūpa}
{You are the original deity, the Lord, the most ancient being. You are the ultimate support of the universe. You are the knower and the knowable. You are the supreme abode. This whole universe is pervaded by You, O Lord of infinite forms.}

\Verse[11.39]
{vāyur yamo ’gnir varuṇaḥ śaśāṅkaḥ prajāpatis tvaṁ prapitāmahaś ca \\
namo namas te ’stu sahasra-kṛtvaḥ punaś ca bhūyo ’pi namo namas te}
{You are Vāyu,\footnote{Deity responsible for the winds.} Yama, Agni, Varuṇa, the moon god, Prajāpati,\footnote{Personification of creative power.} and the great-grandfather of creation. Glories be to You a thousand times! Again glories be to You. Yet again, repeated glories to You.}

\Verse[11.40]
{namaḥ purastād atha pṛṣṭhatas te namo ’stu te sarvata eva sarva \\
ananta-vīryāmita-vikramas tvaṁ sarvaṁ samāpnoṣi tato ’si sarvaḥ}
{I bow to You from the front and the back! I bow down to You from all directions, O all-pervading One. O Lord of limitless power and boundless might, You pervade everything and therefore, You are everything.}

\Verse[11.41–42]
{sakheti matvā prasabhaṁ yad uktaṁ he kṛṣṇa he yādava he sakheti \\
ajānatā mahimānaṁ tavedaṁ mayā pramādāt praṇayena vāpi \\
yac cāvahāsārtham asat-kṛto ’si vihāra-śayyāsana-bhojaneṣu \\
eko ’tha vāpy acyuta tat-samakṣaṁ tat kṣāmaye tvām aham aprameyam}
{Considering You as my dear friend and being unaware of Your majesty, whatever was rashly spoken by me out of delusion or affection, such as \enquote{O Kṛṣṇa, O Yādava,\footnote{Kṛṣṇa is a descendant of King Yadu, and is therefore also referred to as Yādava.} O friend}  and whenever You were jokingly shown disrespect while playing, resting, while sitting or eating together, whether alone or in the presence of others, O Acyuta---for all that, I beg Your forgiveness, O incomprehensible One.}

\Verse[11.43]
{pitāsi lokasya carācarasya tvam  \\
asya pūjyaś ca gurur garīyān \\
na tvat-samo ’sty abhyadhikaḥ kuto ’nyo  \\
loka-traye ’py apratima-prabhāva}
{You are the father of this world, of both the moving and the non-moving beings. You are the most worshipable and the greatest teacher. There is no one equal to You; how could there be someone greater in even all the three worlds, O incomparable power?}

\Verse[11.44]
{tasmāt praṇamya praṇidhāya kāyaṁ prasādaye tvām aham īśam īḍyam \\
piteva putrasya sakheva sakhyuḥ priyaḥ priyāyārhasi deva soḍhum}
{Therefore, bowing down and prostrating this body, I beg Your pardon, O venerable Lord. Please forgive me, just as a father pardons his son, a friend their friend, and a lover their beloved, O Lord.}

\Verse[11.45]
{adṛṣṭa-pūrvaṁ hṛṣito ’smi dṛṣṭvā bhayena ca pravyathitaṁ mano me \\
tad eva me darśaya deva rūpaṁ prasīda deveśa jagan-nivāsa}
{O Lord! Having seen that form which has never been seen before, I am gladdened, but my mind is overwhelmed with anxiety. Be gracious and show me Your original form, O Lord of the \textit{devas}! O abode of the universe!}

\Verse[11.46]
{kirīṭinaṁ gadinaṁ cakra-hastam icchāmi tvāṁ draṣṭum ahaṁ tathaiva \\
tenaiva rūpeṇa catur-bhujena sahasra-bāho bhava viśva-mūrte}
{I wish to see You wearing a crown, holding a mace and discus in hand. O universal form! O thousand-armed one! Manifest that four-armed form exactly in this manner.}

\enquote{Because of all the mixed emotions Arjuna feels, he doesn’t know what to do or what to say. He is super-shocked, his heart has shrunk inside him. He feels his body is full of electric currents, flowing everywhere inside his body. All his hairs are standing on end. He doesn’t know what to do apart from bowing his head down, and with \textit{śaraṇāgati}, with folded hands, falling at the Feet of the Lord. Overwhelmed with joy and wonder, he has seen that which sages and saints have not perceived. By the grace of the Lord Himself, because of the Love He has for His devotee, He has not denied him the ultimate vision.}

\enquote{Arjuna says, \enquote{There is nothing which is not You. You surpass the whole universe itself; everything in, out, left, right, up, down, is only You.} Wherever Arjuna looks, whatever he does, there is not a small space where he cannot perceive the Lord. He is saying that in every single atom is the manifestation of Nārāyaṇa. And there is nothing other than Him. Although he praises Him, salutation upon salutation, this can’t do justice to His greatness. Arjuna is filled with reverence, filled with gratitude that the Lord has permitted him to perceive this cosmic vision.}

\enquote{Arjuna asks the Lord for forgiveness for anything he has said to Kṛṣṇa in a certain tone, with arrogance or pride. Due to his ignorance, he has thought of Him as a friend and sometimes as a normal human being, and forgetting who He is, he has said many things. He has never had in mind that He is the Lord of lords, until he has seen this cosmic form of Lord Kṛṣṇa. And due to this intimacy---this thoughtless sense of fun that he had towards his friend---he didn’t perceive the reality of who He is.}

\enquote{It is very important to know where the \textit{guru} stands and where the disciple stands. Even if the \textit{guru} allows the disciple to be close, to be friends and dear, the \textit{guru} always remains the \textit{guru}. One must always know where one stands and not be fooled by the outside}

\enquote{Bhagavān has not withdrawn his mind completely, He is still letting him feel his individual Self. Even if he knew that the Lord is all prevailing, yet the individual Self is separated. There is this intimacy; there is still this separation. This is how one can enjoy the beauty of the Lord. This is how one can drink this nectar! This is how one can enjoy the \textit{līlā} of the Lord, the Love that emanates from God! If one would become fully merged with Him, one could not enjoy that. This is the sweetest thing! This relationship that one can have with Bhagavān, when one can have a personal relationship with God, and can love Him! Then you can say, \enquote{Yes, I love God!} And not just by superficially saying, \enquote{I love God, He is everywhere.} But this is when you truly love Him; when all the \textit{bhāva}\footnote{An intense feeling of Love and self-surrender directed towards God.} awakens inside of you; when you have this longing; your heart is in pain; you can’t breathe; you can’t for a single moment divert your mind left or right. Such a \textit{bhāva} when –just by thinking of Him, by chanting His Name, wherever you are in the world---your mind, your heart is always with Him! It always runs to Him, as your heart always runs here, to the \textit{āśrama}. Such devotion! Such surrender! Arjuna is rejoicing inside himself, but it is this separation which allows him to experience this wonder of the cosmic form of the Lord; that is how he could enjoy it. And this is most beautiful! Without this separation, there would not be this enjoyment! And this is the greatness of \textit{bhakti}: you are not completely separated from the Lord but you enjoy the \textit{līlā}, the love, the \textit{bhāva} of serving Him, longing for Him, wanting to be with Him at all times, driving you crazy! That is the sweetness of this.}

\enquote{\enquote{Be gracious and show me Your other form, O Lord of the gods!}---Here again he is asking for grace, because it is only through grace that everything is possible. Through \textit{kṛpā}, Bhagavān can take people from Hell and put them in Heaven. Through \textit{kṛpā}, He can give Himself fully to His \textit{bhakta}---just through grace, nothing else. This is the power of grace! There is nothing else; everything is His grace.}

\section{Verses 47-51: The Uniqueness of this Cosmic Form}\label{sec-verses-47-51-the-uniqueness-of-this-cosmic-form}

\Verse[11.47]
{\hspace*{1em}śrī-bhagavān uvāca \\
mayā prasannena tavārjunedaṁ  \\
rūpaṁ paraṁ darśitam ātma-yogāt \\
tejo-mayaṁ viśvam anantam ādyaṁ  \\
yan me tvad anyena na dṛṣṭa-pūrvam}
{\hspace*{1em}Bhagavān Kṛṣṇa said: \\
O Arjuna, by My grace this supreme resplendent, universal, infinite, beginningless form of Mine, which has never been seen before by anyone apart from you, was revealed to you through My own potency.}

\Verse[11.48]
{na veda-yajñādhyayanair na dānair na ca kriyābhir na tapobhir ugraiḥ \\
evaṁ-rūpaḥ śakya ahaṁ nṛ-loke draṣṭuṁ tvad anyena kuru-pravīra}
{Neither through the study of the \textit{Vedas}, sacrifices, nor by \textit{Vedic} recital, nor by charity, nor by rituals, nor by strict austerities, can I be seen like this by anyone other than you in this human world, O hero of the Kurus!}

\Verse[11.49]
{mā te vyathā mā ca vimūḍha-bhāvo  \\
dṛṣṭvā rūpaṁ ghoram īdṛṅ mamedam \\
vyapeta-bhīḥ prīta-manāḥ punas tvaṁ  \\
tad eva me rūpam idaṁ prapaśya}
{Do not fear or be confused by seeing this terrifying form of Mine. Free from fear and with a pacified mind, again behold that very form of Mine.}

\Verse[11.50]
{\hspace*{1em}sañjaya uvāca \\
ity arjunaṁ vāsudevas tathoktvā svakaṁ rūpaṁ darśayām āsa bhūyaḥ āśvāsayām āsa ca bhītam enaṁ bhūtvā punaḥ saumya-vapur mahātmā}
{\hspace*{1em}Sañjaya said: \\
Having thus spoken to Arjuna, Vāsudeva again showed His own form. Having once more assumed His gentle form, that great being reassured the frightened Arjuna.}

\Verse[11.51]
{\hspace*{1em}arjuna uvāca \\
dṛṣṭvedaṁ mānuṣaṁ rūpaṁ tava saumyaṁ janārdana \\
idānīm asmi saṁvṛttaḥ sa-cetāḥ prakṛtiṁ gataḥ}
{\hspace*{1em}Arjuna said: \\
Seeing this mild human form of Yours, O Janārdana, I am now composed. My mind has returned to its natural composure.}

\enquote{The cosmic form that the Lord took was very terrible for Arjuna to perceive or to understand. Very often people ask for something that they think would be good for them. And when they have it, they don’t know how to handle it. When someone doesn’t truly know what one is asking for, when one receives it---in whichever form, and however it is given---it is very difficult to deal with. In daily life, when we ask for things without knowing what we are asking for, or how it will come, it’s hard to handle.}

\enquote{I remember once somebody asked me, \enquote{I want to realize myself straight away.} And I asked this person, \enquote{Do you know what you are asking for?} He said, \enquote{Yes I know!} I said, \enquote{Are you sure that you know what you are asking for?} He said, \enquote{Yes, yes,} with deep conviction. I said, \enquote{Okay.} Very often people think of themselves as being very superior but in reality, they are not superior at all. If people think of themselves as superior, they are not superior. Only when one is humble, one attains the state of surrender, but not when one is proud. So this person said, \enquote{Yes, yes, I know very well what it is.} I said, \enquote{Okay, fine, you know what you are asking for. So may God bless you for it.} \enquote{And at that moment his life turned upside down, everything turned around. He cried, ‘I didn’t ask for this. I am suffering, I have all this suffering.’ And I replied, ‘Sure, you did not ask for this suffering but you asked for realization, for liberation, and this is the cost of it; it comes together with it!’ When nature is forced into something, everything gets accelerated. This is why it is not good to ask. If you really want to ask one thing, ask God for His Love. Of this, you are fully guaranteed! He signs and seals all the documents! This is the only thing you can be sure of. Don’t ask for something when you don’t know what you are asking for.}}

\enquote{When this Viśvarūpa disappeared, in front of him stood the \textit{Caturbhuja}---the four-armed Mahā-Viṣṇu with the \textit{Cakra}, the \textit{Śaṅkha}, the \textit{Gadā} and the \textit{Padmā}. And this form is so sweet and calm that the whole atmosphere completely changed. Even the \textit{bhāva} of Arjuna inside himself; he is not trembling, he is not shaking, he is calm, he is at ease, and he can understand. His mind has automatically subdued itself.}

\enquote{Here Bhagavān Kṛṣṇa as a great Teacher, as the \textit{guru}, consents to let Arjuna have a glimpse of His cosmic form. He wants him to have this glimpse so that he can put his mind at peace about who is with him, telling him to go and fight. The Lord has a plan for each individual soul. Nothing that you do is really you doing it.}


\section{Verses 52-55: The Supremacy of Bhakti}\label{sec-verses-52-55-the-supremacy-of-bhakti}


\Verse[11.52]
{\hspace*{1em}śrī-bhagavān uvāca \\
sudurdarśam idaṁ rūpaṁ dṛṣṭavān asi yan mama \\
devā apy asya rūpasya nityaṁ darśana-kāṅkṣiṇaḥ}
{\hspace*{1em}Bhagavān Kṛṣṇa said: \\
This form of Mine that you have seen is extremely difficult to behold. Even the gods are ever eager to have a vision of this form.}

\Verse[11.53]
{nāhaṁ vedair na tapasā na dānena na cejyayā \\
śakya evaṁ-vidho draṣṭuṁ dṛṣṭavān asi māṁ yathā}
{Neither through the \textit{Vedas}, nor by austerities, nor by charity, nor by the performance of sacrifices, is it possible to see Me in the way You have seen Me.}

\Verse[11.54]
{bhaktyā tv ananyayā śakya aham evaṁ-vidho ’rjuna \\
jñātuṁ draṣṭuṁ ca tattvena praveṣṭuṁ ca parantapa}
{By single-minded devotion, however, it is possible to truly know, see, and become absorbed in Me in this way, O Arjuna, vanquisher of enemies.}

\Verse[11.55]
{mat-karma-kṛn mat-paramo mad-bhaktaḥ saṅga-varjitaḥ \\
nirvairaḥ sarva-bhūteṣu yaḥ sa mām eti pāṇḍava}
{Whoever performs action for My sake, regards Me as the highest, is devoted to Me, free from attachment, and without malice toward any living being---such a person comes to Me, O son of Pāṇḍu.}

\enquote{The fool regards Him [Kṛṣṇa] as just being a normal human being. When they see Him dressed normally, acting normally, eating, drinking, sleeping, having fun, they think that He is just a normal person. But He is not. This is just to bewilder the mind. This is His \textit{māyā}, which He has put over His creation so that no one perceives Him. Only the one who is worthy will perceive this. From time to time He reveals Himself through His \textit{līlā}. From time to time He reveals Himself through the heart of the \textit{bhakta}, through intuition and feeling.}

\enquote{Here Bhagavān is saying, \enquote{The demigods themselves desire to look upon that cosmic form. All the angels, sages, saints, all the prophets, want to see that cosmic form, but none have ever seen it. They have seen only a partial manifestation of Me. They realized only a part of Me, not the fullness of My manifestation. It is not possible even if one performs austerities. It is not possible even if one studies the \textit{Vedas} and chants all kinds of \textit{mantras}. It is not possible if one is performing every kind of ritual. But to those who have single-pointed devotion towards Me, those who are surrendered to Me completely, I raise them to that state where they come face to face with Me. And I do that due to the Love they have for me. Due to their devotion and surrender, out of grace, I reveal a different aspect of Myself.}}