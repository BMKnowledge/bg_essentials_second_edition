\chapter{Arjuna-viṣāda-yoga}\label{chap-arjuna-visada-yoga}
\chaptersubtitle{The Lamentation of Arjuna}
%%%%%%%%%%%%%%%%%%%%%%%%%%%%%%%%%%%%%%%%%%%%%%%%%%%%%%%%%

\noindent The \textit{Gītā} begins with a vivid description of the scene on the Kurukṣetra battlefield. Key warriors on both sides are named. The blowing of conches and the beating of drums signal the start of this terrible war. It is at this point that the focus shifts to Arjuna and Kṛṣṇa. Eager to assess the enemy ranks, Arjuna asks Kṛṣṇa to take him to the middle of the battlefield so he can take a closer look at those he is about to fight.

Among the opposing army, he sees relatives, friends, and revered elders. The sight of those who were once so dear to him causes Arjuna to lose his resolve. He can no longer see the point in engaging in this battle which will inevitably destroy his family. Confused about his duty and overwhelmed with compassion for his enemies, he begins to pour out his heart to Kṛṣṇa. Surely, he argues, this war cannot be based on righteousness. He repeatedly makes the point that killing one’s own family for the sake of a kingdom will only produce dire consequences for the future. After making his case, the chapter ends with Arjuna casting aside his bow in grief.

\section{Verse 1: Dhṛtarāṣṭra's Inquiry}\label{sec-verse-1-dhrtarastra-s-inquiry}

\Verse[1.1]
{\hspace*{1em}dhṛtarāṣṭra uvāca \\
dharma-kṣetre kuru-kṣetre samavetā yuyutsavaḥ \\
māmakāḥ pāṇḍavāś caiva kim akurvata sañjaya}
{\hspace*{1em}Dhṛtarāṣṭra said: \\
O Sañjaya, having gathered on the holy field of Kurukṣetra, eager for battle, what did my sons and the Pāṇḍavas do?}

\enquote{This verse starts with the word \enquote{\textit{dharma-kṣetra}.} \enquote{\textit{Dharma}} means righteous, \enquote{\textit{kṣetra}} means the field---so this is the field of righteousness.}

\enquote{One of the meanings of this war is life, where the \enquote{good} side fights with the \enquote{not good} side. This war is not outside, it is also happening inside the human body. Your physical body is the \textit{dharma-kṣetra}. You have incarnated to do your \textit{dharma} (duty) in this field.}

\enquote{Life is also a \textit{dharma-kṣetra}. You have come to fulfill your divine purpose. When you are in tune with your true Self, you realize your purpose in life\ldots and that’s what the word \enquote{\textit{dharma-kṣetra}} is reminding you of. Do your \textit{dharma}! Awaken! This \textit{dharma} can be done with the greatest gift which God has given: this field, this body. And when you start doing your \textit{dharma}, you’ll get good merit! But, if you run away from your \textit{dharma}, then you turn towards the dark side.}

\enquote{This blind king, Dhṛtarāṣṭra, represents the mind---the mind which is blind and wants to always stay blind. The mind is hanging on to the outside so much that it has power only when it is focused on something exterior: on the material, on relationships, on gaining this or that. This is the nature of the mind. The mind is blind.}

\enquote{Both families were from the Kuru dynasty. But the king refused to recognize the Pāṇḍavas. The mind doesn’t recognize the good qualities which are present in oneself. The mind can only look towards the senses, looking always towards the outside. The Self, and the positive qualities which are present inside, are not comprehended by it.}

\section{Verses 2-13: The Warriors are Introduced}\label{sec-verses-2-13-the-warriors-are-introduced}


\Verse[1.2]
{\hspace*{1em}sañjaya uvāca \\
dṛṣṭvā tu pāṇḍavānīkaṁ vyūḍhaṁ duryodhanas tadā \\
ācāryam upasaṅgamya rājā vacanam abravīt}
{\hspace*{1em}Sañjaya said: \\
Having seen the Pāṇḍava army arranged in military formation, Duryodhana then approached his teacher Droṇa and spoke these words:}


\Verse[1.3]
{paśyaitāṁ pāṇḍu-putrāṇām ācārya mahatīṁ camūm \\
vyūḍhāṁ drupada-putreṇa tava śiṣyeṇa dhīmatā}
{O teacher, behold this mighty army of the Pāṇḍavas, arrayed by the son of Drupada, your intelligent disciple.}

\Verse[1.4]
{atra śūrā maheṣv-āsā bhīmārjuna-samā yudhi \\
yuyudhāno virāṭaś ca drupadaś ca mahā-rathaḥ}
{In that army are heroes and great archers, equal to Bhīma and Arjuna in battle; there are mighty warriors like Yuyudhāna, Virāṭa, and Drupada.}

\Verse[1.5]
{dhṛṣṭaketuś cekitānaḥ kāśirājaś ca vīryavān \\
purujit kuntibhojaś ca śaibyaś ca nara-puṅgavaḥ}
{There are Dhṛṣṭaketu, Cekitāna, and the valiant king of Kāśī, Purujit, Kuntibhoja, and Śaibyā, the best among men.}

\Verse[1.6]
{yudhāmanyuś ca vikrānta uttamaujāś ca vīryavān \\
saubhadro draupadeyāś ca sarva eva mahā-rathāḥ}
{There are the mighty Yudhāmanyu, the strong Uttamaujā, the son of Subhadrā, as well as the sons of Draupadī, all of whom are mighty warriors.}

\Verse[1.7]
{asmākaṁ tu viśiṣṭā ye tān nibodha dvijottama \\
nāyakā mama sainyasya saṁjñārthaṁ tān bravīmi te}
{O best of \textit{brāhmaṇas}, now hear about our distinguished warriors. For your understanding, I shall tell you the leaders of my army.}

\Verse[1.8]
{bhavān bhīṣmaś ca karṇaś ca kṛpaś ca samitiñjayaḥ \\
aśvatthāmā vikarṇaś ca saumadattis tathaiva ca}
{There are yourself, Bhīṣma, Karṇa, the victorious Kṛpa, Aśvatthāmā, Vikarṇa, and the son of Somadatta.}

\Verse[1.9]
{anye ca bahavaḥ śūrā mad-arthe tyakta-jīvitāḥ \\
nānā-śastra-praharaṇāḥ sarve yuddha-viśāradāḥ}
{There are also many other heroes who have offered their lives for my sake, all wielding various weapons and experienced in the art of warfare.}

\Verse[1.10]
{aparyāptaṁ tad asmākaṁ balaṁ bhīṣmābhirakṣitam \\
paryāptaṁ tv idam eteṣāṁ balaṁ bhīmābhirakṣitam}
{This force of our army marshalled by Bhīṣma is immeasurable, while the strength of their army guarded by Bhīma is limited.}

\Verse[1.11]
{ayaneṣu ca sarveṣu yathā-bhāgam avasthitāḥ \\
bhīṣmam evābhirakṣantu bhavantaḥ sarva eva hi}
{Indeed all of you should just guard Bhīṣma, while stationed at all strategic positions in the army.}

\Verse[1.12]
{tasya sañjanayan harṣaṁ kuru-vṛddhaḥ pitāmahaḥ \\
siṁha-nādaṁ vinadyoccaiḥ śaṅkhaṁ dadhmau pratāpavān}
{\hspace*{1em}(Sañjaya said:) \\
The powerful grandsire Bhīṣma, eldest of the Kuru clan, roaring loudly like a lion, blew his conch to incite Duryodhana's cheerfulness.}

\Verse[1.13]
{tataḥ śaṅkhāś ca bheryaś ca paṇavānaka-gomukhāḥ \\
sahasaivābhyahanyanta sa śabdas tumulo ’bhavat}
{Consequently, conches, kettle drums, small and big drums, and horns immediately sounded forth and the sound was terrific.}


\enquote{Duryodhana represents this great pride that is born from the mind. When the mind is very active, one becomes proud of many things: proud of knowledge, proud of what one has.}

\enquote{The army of the Pāṇḍavas was arrayed in a very special formation. Seeing this orderly formation, Duryodhana felt much nervousness and anxiety inside himself. Anxiety appears when one is proud. Even if pride appears very strong on the outside, in reality, it has a lot of weaknesses in it. Why does pride arise? Do you think it is out of strength? No! In reality, pride arises due to the weakness that one has inside. Even if somebody says, \enquote{Ah yes, I am very proud of this and I am very proud of that,} you can feel that this pride is actually weakness. When pride arises, people think, \enquote{Yes, I am very confident!} No. It’s the mind that perceives pride as confidence. In reality, one is running away from something, from the opposite of pride, which is humility. When one is running away from humility, one only appears to be very grand and confident.}

\enquote{When you start on the spiritual path, your pride sees all your good qualities, but then the mind becomes anxious. This pride tries to make you reason, tries to make you go sideways in a cunning way. That’s why Duryodhana rushes to Droṇācārya, the great teacher of both the Kauravas and the Pāṇḍavas.}

\enquote{Droṇācārya represents attachment to the material. He represents the greed in man. Droṇācārya also had good qualities. He was a great teacher of military science. Sometimes he would even advise Bhīṣma. He was the royal guru. But when the pride of Duryodhana saw the greed in Droṇācārya, he said to himself, \enquote{Let me go and feed his greed. Let me corrupt him.}}

\enquote{Then he praises Bhīṣma. Bhīṣma represents the ego. He was very powerful. He was the great-uncle of the Kauravas and the Pāṇḍavas, and was considered the greatest of all in the Kuru dynasty. He was very virtuous. He was a renunciate\ldots   He was very devoted to his parents. He knew about the scriptures. He was devoted to his teachers. Above all, he was very dedicated to God. Due to this, there was a great ego.}

\enquote{The ego makes you think and feel that you are the best among men. You are the most knowledgeable of all. You have knowledge of everything. And that blinds you. Even if you have many good qualities, if you are egoistic, all these good qualities are nothing because it is all self-centered.}

\enquote{Duryodhana continues, \enquote{There are many more heroes who have sacrificed their lives for my sake.} You see, pride has many friends, and most of these friends have qualities similar to him. Because of his arrogance, Duryodhana attracted similar people with similar qualities. Most of these qualities that were supporting him were in the form of his own ninety-nine brothers.}

\enquote{The Kauravas stand for this outside reality which is looking, fighting, always wanting something, but it’s all material, it’s all external, and that brings superficial joy: joy for a very short time and misery for a longer time.}

\enquote{When you start on your spiritual path, you have a battle: you perceive all the negative qualities within more strongly. Sometimes you’ll awaken a quality which has never been there before. But this is the purification that one goes through; this is the Kurukṣetra that you go through, the \textit{dharma-kṣetra} that you go through. You uproot, one by one, all these qualities and transform them. You transform them until finally you have the Love of God that stays. That’s realization! To receive His grace. To manifest His Love and to beam His Love. And that's the duty of each human being.}

\section{Verses 14-20: Kṛṣṇa is Introduced; The Conches are Blown}\label{sec-verses-14-20-krsna-is-introduced-the-conches-are-blown}

\Verse[1.14]
{tataḥ śvetair hayair yukte mahati syandane sthitau \\
mādhavaḥ pāṇḍavaś caiva divyau śaṅkhau pradadhmatuḥ}
{Then Mādhava and the son of Paṇḍu, seated in a great chariot harnessed by white horses, blew their divine conches.}

\Verse[1.15–16]
{pāñcajanyaṁ hṛṣīkeśo devadattaṁ dhanañjayaḥ \\
pauṇḍraṁ dadhmau mahā-śaṅkhaṁ bhīma-karmā vṛkodaraḥ \\
anantavijayaṁ rājā kuntī-putro yudhiṣṭhiraḥ \\
nakulaḥ sahadevaś ca sughoṣa-maṇipuṣpakau}
{Śrī Kṛṣṇa blew His conch, Pāñcajanya, Arjuna blew his conch named Devadatta, and Bhīma, the performer of terrible deeds, blew the great conch Pauṇḍra. King Yudhiṣṭhira, the son of Kuntī, blew his conch Ananta-vijaya, and Nakula and Sahadeva blew their conches Sughoṣa and Maṇi-puṣpaka.}

\Verse[1.17–18]
{kāśyaś ca parameṣv-āsaḥ śikhaṇḍī ca mahā-rathaḥ \\
dhṛṣṭadyumno virāṭaś ca sātyakiś cāparājitaḥ \\
drupado draupadeyāś ca sarvaśaḥ pṛthivī-pate \\
saubhadraś ca mahā-bāhuḥ śaṅkhān dadhmuḥ pṛthak pṛthak}
{The great archer, king of Kāśī and the mighty warrior Śikhaṇḍī, Dhṛṣṭadyumna, Virāṭa, and the invincible Sātyaki, King Drupada, the sons of Draupadī, and the strong-armed son of Subhadrā, all blew their respective conches.}

\Verse[1.19]
{sa ghoṣo dhārtarāṣṭrāṇāṁ hṛdayāni vyadārayat \\
nabhaś ca pṛthivīṁ caiva tumulo ’bhyanunādayan}
{That tumultuous sound, reverberating through heaven and Earth, tore apart the hearts of Dhṛtarāṣṭra’s sons.}

\Verse[1.20]
{atha vyavasthitān dṛṣṭvā dhārtarāṣṭrān kapi-dhvajaḥ \\
pravṛtte śastra-sampāte dhanur udyamya pāṇḍavaḥ \\
hṛṣīkeśaṁ tadā vākyam idam āha mahī-pate}
{Then, O king, having seen the sons of Dhṛtarāṣṭra arranged in formation, the son of Pāṇḍu, who had Hanumān on his banner, took up his bow in preparation for the clash of weapons. He then spoke these words to Hṛṣīkeśa.}


\enquote{Kṛṣṇa is seated in His chariot holding the reins, controlling the five horses. He is the Controller of all. Arjuna is seated with Him in the chariot and they blow their divine conches. This denotes that when one shows interest in changing, when the Lord perceives that one is making an effort to change, then the Lord Himself gives strength, power, energy and faith to that person who is willing to change. He gives one power to control the senses: the five horses pulling the chariot represent the five senses.}

\enquote{Imagine all these people blowing conches: so powerful and so strong must have been this sound at that time, that everything around started to tremble! And not only there on the battlefield, but also in the heavens. It didn’t happen just in this physical world. It also happened in the spiritual world.}


\enquote{They used the conch to announce the beginning and the end of war. The sound is \textit{oṁ}. This sound which vibrates, shows that at the end, no matter how big the battle is, how strong or difficult it is, one finds one’s way. At the end, everything resounds in the cosmic sound. At the end, it is through this cosmic sound that one attains realization. This cosmic sound is the word of God Himself. The conch and the bell are not just mere instruments. When one blows the conch, it awakens the divinity inside. It awakens clarity inside. It vibrates and the vibration also vibrates in your brain. The same is true of the bell.}

\enquote{When you go with full force on your spiritual path, nothing can move you! Nothing that people can say, nothing that people can do, can make you move from your path. That’s what Christ says about having faith. About building your faith on a rock so that nothing can move you! If you build your house on sand, it will break. If you build your house on a rock, it will be very strong. This sound that we are talking about is the inner strength. When you have inner strength, it’s not only in your heart. Your whole being will be full of energy! From head to toe, you’ll be full of energy, because that energy is not from the outside, it is from deep within. This energy is from God Himself, from Kṛṣṇa Himself, and it beams out and gives you the support and the strength to move forward. But you should not look at your weakness, even if you see the weakness. It is part of you, yes! But hang onto your strength!}

\newpage
\section{Verses 21-25: Between the Two Armies}\label{sec-verses-21-25-between-the-two-armies}

\Verse[1.21–22]
{\hspace*{1em}arjuna uvāca \\
senayor ubhayor madhye rathaṁ sthāpaya me ’cyuta \\
yāvad etān nirīkṣe ’haṁ yoddhu-kāmān avasthitān \\
kair mayā saha yoddhavyam asmin raṇa-samudyame}
{\hspace*{1em}Arjuna said: \\
O Acyuta, station my chariot between both armies until I may see those standing near, eager for combat and with whom I must fight in this impending battle.}

\Verse[1.23]
{yotsyamānān avekṣe ’haṁ ya ete ’tra samāgatāḥ \\
dhārtarāṣṭrasya durbuddher yuddhe priya-cikīrṣavaḥ}
{I wish to see those who have gathered here and will fight in battle, wishing to please the evil-minded son of Dhṛtarāṣṭra}

\Verse[1.24–25]
{\hspace*{1em}sañjaya uvāca \\
evam ukto hṛṣīkeśo guḍākeśena bhārata \\
senayor ubhayor madhye sthāpayitvā rathottamam \\
bhīṣma-droṇa-pramukhataḥ sarveṣāṁ ca mahī-kṣitām \\
uvāca pārtha paśyaitān samavetān kurūn iti}
{\hspace*{1em}Sañjaya said: \\
O descendant of Bharata, having thus been addressed by Guḍākeśa, Hṛṣīkeśa stationed that best of chariots between the two armies. In front of Bhīṣma, Droṇa, and all the other kings, He said: \enquote{O Pārtha, behold all the Kurus who have assembled here.}}

\enquote{Arjuna said to Kṛṣṇa, \enquote{move my chariot in between.} So, this \enquote{in-between state} stands for neutral. You are neither on one side, nor the other. You are neither on the good side, nor on the bad side. For you to be able to observe, you have to be in that neutral state, that neutral point. Very often people take decisions in life being on one side. If you are on one side, there is always judgment, there is always confusion.}

\enquote{Testing times may arise in the form of troubles. Often one doesn’t want to go through it, and so one tries to bypass it, and go sideways. But life is like this. Life is a great lesson. If you don’t face your problem, if you don’t face your negative quality and look at it in the eyes, you’ll never become strong. If you always try to go sideways, you will not learn anything. Arjuna says, \enquote{Place me between them. Let me see---face to face, eye to eye---what are these qualities, what is this pain. Let me go above it, rise over, master it! I wish to see who I am to fight.} Here it is not about fighting, it is about transcending. \enquote{What will I transcend? Which quality?} This is self-analysis. This is the point where Arjuna is analysing. Arjuna here represents self-observance. Observe all your qualities---your \enquote{good} qualities and your \enquote{not good} qualities---and then you can know how to overcome them, transcend them and transform them.}

\section{Verses 26-47: Arjuna Refuses to Fight}\label{sec-verses-26-47-arjuna-refuses-to-fight}
\Verse[1.26–27]
{tatrāpaśyat sthitān pārthaḥ pitṝn atha pitāmahān \\
ācāryān mātulān bhrātṝn putrān pautrān sakhīṁs tathā \\
śvaśurān suhṛdaś caiva senayor ubhayor api}
{Pārtha saw standing there, fathers and grandfathers,
teachers, maternal uncles, brothers, sons, grandsons and friends, as well as fathers-in-law and well-wishers among both armies.}

\Verse[1.27–28]
{tān samīkṣya sa kaunteyaḥ sarvān bandhūn avasthitān \\
kṛpayā parayāviṣṭo viṣīdann idam abravīt}
{Having seen all his friends and relatives there, Kaunteya became filled with deep compassion and grievingly spoke these words:}

\Verse[1.28–29]
{\hspace*{1em}arjuna uvāca \\
dṛṣṭvemaṁ sva-janaṁ kṛṣṇa yuyutsuṁ samupasthitam \\
sīdanti mama gātrāṇi mukhaṁ ca pariśuṣyati \\
vepathuś ca śarīre me roma-harṣaś ca jāyate}
{\hspace*{1em}Arjuna said: \\
O Kṛṣṇa, after seeing my own family and friends assembled here desiring to fight, my limbs are becoming weak and my mouth is drying up. My body is trembling, and my hair is standing on end.}

\Verse[1.30]
{gāṇḍīvaṁ sraṁsate hastāt tvak caiva paridahyate \\
na ca śaknomy avasthātuṁ bhramatīva ca me manaḥ}
{My bow, the Gāṇḍīva, is slipping from my hands and my skin is burning. I am unable to hold myself together and my mind appears as if reeling.}

\Verse[1.31]
{nimittāni ca paśyāmi viparītāni keśava \\
na ca śreyo ’nupaśyāmi hatvā sva-janam āhave}
{O Keśava, moreover I perceive sinister omens and cannot see any benefit in killing my own family in battle.}

\Verse[1.32]
{na kāṅkṣe vijayaṁ kṛṣṇa na ca rājyaṁ sukhāni ca \\
kiṁ no rājyena govinda kiṁ bhogair jīvitena vā}
{O Kṛṣṇa, neither do I desire victory, the kingdom, or comforts. Of what use to us is a kingdom, enjoyments, or even life itself, Govinda?}

\Verse[1.33–34]
{yeṣām arthe kāṅkṣitaṁ no rājyaṁ bhogāḥ sukhāni ca \\
ta ime ’vasthitā yuddhe prāṇāṁs tyaktvā dhanāni ca \\
ācāryāḥ pitaraḥ putrās tathaiva ca pitāmahāḥ \\
mātulāḥ śvaśurāḥ pautrāḥ śyālāḥ sambandhinas tathā}
{Those for whose sake we desire the kingdom, enjoyments, and comforts stand ready to fight and have given up their lives and wealth for this war---teachers, fathers, sons and also grandfathers, uncles, fathers-in-law and grandsons, brothers-in-law and other relatives.}

\Verse[1.35]
{etān na hantum icchāmi ghnato ’pi madhusūdana \\
api trailokya-rājyasya hetoḥ kiṁ nu mahī-kṛte}
{O Madhusūdana, they may be intent on slaying me, but I have no desire to kill them, even for reign over the three worlds, much less for sovereignty over this Earth.}

\Verse[1.36]
{nihatya dhārtarāṣṭrān naḥ kā prītiḥ syāj janārdana \\
pāpam evāśrayed asmān hatvaitān ātatāyinaḥ}
{After killing the sons of Dhṛtarāṣṭra, what joy would be ours, O Janārdana? Sin alone would stain us if we killed these aggressors.}

\Verse[1.37]
{tasmān nārhā vayaṁ hantuṁ dhārtarāṣṭrān sva-bāndhavān \\
sva-janaṁ hi kathaṁ hatvā sukhinaḥ syāma mādhava}
{Therefore, it is not right that we slay the sons of Dhṛtarāṣṭra, our own relatives. Indeed how can we rejoice, O Mādhava, by killing our kinsmen?}

\Verse[1.38–39]
{yady apy ete na paśyanti lobhopahata-cetasaḥ \\
kula-kṣaya-kṛtaṁ doṣaṁ mitra-drohe ca pātakam \\
kathaṁ na jñeyam asmābhiḥ pāpād asmān nivartitum \\
kula-kṣaya-kṛtaṁ doṣaṁ prapaśyadbhir janārdana}
{Even though these people, whose minds are overpowered by greed, see no vice in destroying their family and betraying their friends, why should we, O Kṛṣṇa, who do recognize this evil of destroying the family, not know to turn away from this evil?}

\Verse[1.40]
{kula-kṣaye praṇaśyanti kula-dharmāḥ sanātanāḥ \\
dharme naṣṭe kulaṁ kṛtsnam adharmo ’bhibhavaty uta}
{Once the family lineage is destroyed, ancient family traditions perish, and when traditions perish, unrighteousness certainly overtakes the whole clan.}

\Verse[1.41]
{adharmābhibhavāt kṛṣṇa praduṣyanti kula-striyaḥ \\
strīṣu duṣṭāsu vārṣṇeya jāyate varṇa-saṅkaraḥ}
{O Kṛṣṇa, from the rise of unrighteousness, the women of the clan become corrupt; when women become corrupt, O descendant of Vṛṣṇi, the mixing of \textit{varṇas} ensues.}

\Verse[1.42]
{saṅkaro narakāyaiva kula-ghnānāṁ kulasya ca \\
patanti pitaro hy eṣāṁ lupta-piṇḍodaka-kriyāḥ}
{This mixing of social classes verily results in hell for the family and for those who destroy it. Consequently, the ancestors of such a family lineage certainly fall due to being deprived of their ritual offerings.}

\Verse[1.43]
{doṣair etaiḥ kula-ghnānāṁ varṇa-saṅkara-kārakaiḥ \\
utsādyante jāti-dharmāḥ kula-dharmāś ca śāśvatāḥ}
{By these sinful deeds of the destroyers of the family lineage which cause the mixing of social classes, the everlasting \textit{dharma} of the society and family traditions are ruined.}

\Verse[1.44]
{utsanna-kula-dharmāṇāṁ manuṣyāṇāṁ janārdana \\
narake niyataṁ vāso bhavatīty anuśuśruma}
{We have heard repeatedly, O Janārdana, that there is inevitably a place in Hell for those whose family traditions are destroyed.}

\Verse[1.45]
{aho bata mahat pāpaṁ kartuṁ vyavasitā vayam \\
yad rājya-sukha-lobhena hantuṁ sva-janam udyatāḥ}
{How unfortunate! We have resolved to commit a great crime by being ready to kill our own family, driven by greed for the pleasures of a kingdom.}

\Verse[1.46]
{yadi mām apratīkāram aśastraṁ śastra-pāṇayaḥ \\
dhārtarāṣṭrā raṇe hanyus tan me kṣemataraṁ bhavet}
{It would be better for me if the sons of Dhṛtarāṣṭra with weapons in their hands were to kill me in battle, unresisting and unarmed.}


\Verse[1.47]
{\hspace*{1em}sañjaya uvāca \\
evam uktvārjunaḥ saṅkhye rathopastha upāviśat \\
visṛjya sa-śaraṁ cāpaṁ śoka-saṁvigna-mānasaḥ}
{\hspace*{1em}Sañjaya said: \\
Having spoken these words on the battlefield, Arjuna threw aside his bow and arrows, and sat down on the seat of his chariot, his mind overwhelmed with grief.}

\enquote{You were born for a reason and that reason is to realize your Self, to awaken the divinity inside of you and to bring it to others; not merely to say, \enquote{Okay, I have a good life.}}

\enquote{You are born with all your negative qualities. They are dormant inside you. Throughout many lifetimes you have carried them with you, but when the time comes for you to remove them, you just sit there and say, \enquote{No, I can’t do it!} The same was true for Arjuna. He is just sitting there saying, \enquote{I can’t do it!,} because he is still holding onto his weaknesses.}

\enquote{Arjuna is overtaken by emotions which will stop him from fighting. So he becomes cowardly, very soft. If you are soft, you can’t help anyone; you have to be strong. Here Kṛṣṇa reminds Arjuna that he will not move forward if he is weak. Yet Kṛṣṇa allows this feeling to awaken in Arjuna because He is purifying him. He is showing him, \enquote{Look at this clearly. It is not outside; it is inside of you. These people who you see outside of you, why do you feel this connection with them? Because they are inside your heart.} These negative qualities that you see outside, in other people, they are not outside: they are inside of you. If you want to transcend them, you have to dig them out from the inside. You have to remove them from deep within. Not just superficially, by the mind saying, \enquote{Ah yes, I have changed, it is finished! Good! Now, God loves me! And I love God!} No, it doesn’t work like that! Because loving God has to come from inside. For Him to manifest Himself, for Him to come to you, to run to you, a great amount of sincerity, strength and power has to be there!}

\enquote{In his confusion, Arjuna tries to find all kinds of excuses not to fight. He sees that Kṛṣṇa is just looking at him, not bothered about these things. Arjuna is using all kinds of words to please Kṛṣṇa, to make Kṛṣṇa agree with him. Like I explained earlier, when one is in a depressed state, one will look to everybody else to acknowledge one’s depression. Arjuna is depressed and trying to make Kṛṣṇa say, \enquote{Poor you! We will not fight.} He is trying to make Kṛṣṇa acknowledge that whatever he is saying is right. He is in a state of deep confusion and the Great Doctor is sitting there with him! Do you think that the Great Doctor would just sit there, start crying with him and say, \enquote{Oh, my God! Oh, Arjuna, you are right, let’s go back!} Not a chance!}

\enquote{Kṛṣṇa is just listening and waiting for Arjuna to finish crying. It is like this in life. If you try to reason with a person who is in this state, it’s of no use. The Lord is just watching and saying, \enquote{Okay, carry on. Do you have more? Send it! I am listening. I am patient.}}

\enquote{In life, your \textit{guru} is always with you, whether you are on the left side or on the right side, but you have to learn to listen. Here, Kṛṣṇa is standing with him in the battle. In the battle of life, the master is with you, helping you to find your way out of this confusion.}

\enquote{This inner confusion which Arjuna is going through, is in all of us\ldots You can’t change the world outside. What you can change is yourself. But the willingness to change must be there. Because if you don’t have the willingness to change, even reading or listening to the \textit{Bhagavad Gītā} will not do anything inside of you. But if you have just a little percentage of that willingness, then the change will happen.}

\enquote{Arjuna is reduced to a very terrible state\ldots In daily life, you see many people who go through that state. But it is a very important state, because in this state you are building your base in spirituality. If you are weak, you can’t stand on the spiritual path. You have to become strong\ldots Don’t look only at the situation; often certain situations happen in life that we don’t want to see. Here Kṛṣṇa is saying, \enquote{Look at it! Face it and go beyond it!}}

\enquote{At the beginning of your spiritual path, it’s not easy. You have to fight that mind.}
\enquote{In this battlefield of life, if you create an illusion to make yourself feel happy, then if anything appears in front of you, you try to find ways to run away from it. This is cowardice, going sideways, \enquote{putting it in a drawer.} You will never be free, because sooner or later you will be facing the same thing again. But once you have faced it, it will never come back to you again.}
