\chapter{Chapter 10: Vibhūti-yoga}\label{chap-vibhuti-yoga}

\paragraph*{\MakeUppercase{The Glories of Kṛṣṇa}}
%%%%%%%%%%%%%%%%%%%%%%%%%%%%%%%%%%%%%%%%%%%%%%%%%%%%%%%%%

\noindent In this chapter, Kṛṣṇa continues to reveal to Arjuna His endless glories. As the Lord He is the origin of creation. The \textit{ṛṣis}, the gods, and all qualities come from Him. Knowing this truth, the pure devotees worship Him with love. They rejoice in talking and glorifying Him. To such people, Kṛṣṇa Himself gives the knowledge by which they can come to Him. From within their hearts, He destroys the darkness of ignorance and raises the light of realization.

Thrilled by these statements, Arjuna fully accepts Kṛṣṇa as the Supreme Lord. Eager to taste more of His glory, Arjuna urges Him to describe exactly how His presence can be known in this world. Kṛṣṇa replies that there is no end to His attributes, but in order to appease Arjuna the most prominent will be described.

What follows is a series of poetic descriptions where Kṛṣṇa identifies with various deities, beings, principles, and values. The overriding message is that whatever we can perceive with beauty and wonder in this world is simply a manifestation of the Lord’s divinity. At the same time, Kṛṣṇa stresses that what has been described is only a small part of His nature. With just a fraction of His being, He sustains all that is.

\section{Verses 1-7: Kṛṣṇa is the Origin of All}\label{sec-verses-1-7-krsna-is-the-origin-of-all}

\Verse[10.1]
{\hspace*{1em}śrī-bhagavān uvāca \\
bhūya eva mahā-bāho śṛṇu me paramaṁ vacaḥ \\
yat te ’haṁ prīyamāṇāya vakṣyāmi hita-kāmyayā}
{\hspace*{1em}Bhagavān Kṛṣṇa said: \\
O mighty-armed Arjuna, listen once more to My supreme teachings. Since you take great pleasure in them, I shall instruct you for your benefit.}

\Verse[10.2]
{na me viduḥ sura-gaṇāḥ prabhavaṁ na maharṣayaḥ \\
aham ādir hi devānāṁ maharṣīṇāṁ ca sarvaśaḥ}
{Neither the hosts of celestials nor the great sages know My origin. Indeed, in all respects I alone am the primordial source of the \textit{devas} and the great sages.}

\Verse[10.3]
{yo mām ajam anādiṁ ca vetti loka-maheśvaram \\
asammūḍhaḥ sa martyeṣu sarva-pāpaiḥ pramucyate}
{One who knows Me as the unborn, beginningless Lord of the worlds is undeluded among mortals and liberated from all sin.}

\Verse[10.4–5]
{buddhir jñānam asammohaḥ kṣamā satyaṁ damaḥ śamaḥ \\
sukhaṁ duḥkhaṁ bhavo ’bhāvo bhayaṁ cābhayam eva ca \\
ahiṁsā samatā tuṣṭis tapo dānaṁ yaśo ’yaśaḥ \\
bhavanti bhāvā bhūtānāṁ matta eva pṛthag-vidhāḥ}
{Intellect, knowledge, freedom from delusion, patience, truthfulness, control of the senses, and control of the mind; happiness and sorrow, existence and non-existence, fear and fearlessness; non-violence, equanimity, contentment, austerity, charity, fame, and infamy---all these diverse states of being arise in living entities solely from Me.}

\Verse[10.6]
{maharṣayaḥ sapta pūrve catvāro manavas tathā \\
mad-bhāvā mānasā jātā yeṣāṁ loka imāḥ prajāḥ}
{The seven great seers\footnote{Created by Brahmā and known as the \textit{Sapta-ṛṣis}, they are responsible for guiding the human race during the four \textit{yugas} (ages). They are in complete communion with the Lord at all times and work towards maintaining a balance on Earth. Before them, and also born from Brahmā, were the four Kumāras: Sanaka, Sanātana, Sanandana, Sanat-kumāra.} and before them the four Kumāras and the Manus\footnote{A title given to the first man of every age (known as a \textit{manvantara}). All humans are believed to have descended from his lineage.
}---all are manifestations of My being, born from My mind. From them, all beings in the world descend.}

\Verse[10.7]
{etāṁ vibhūtiṁ yogaṁ ca mama yo vetti tattvataḥ \\
so ’vikalpena yogena yujyate nātra saṁśayaḥ}
{One who knows in truth My glory and power undoubtedly becomes endowed with the unshakable \textit{yoga} of devotion.}

\enquote{Bhagavān Kṛṣṇa is saying that even the \textit{ṛṣis}, the sages, the \textit{śāstras}, the holy scriptures don’t know anything of Him. They can speak only a little of Him, but they can’t speak about the full revelation of Himself. He has assumed various forms, more than it is written in the \textit{śāstras} and the Vedas. Whenever His devotees have called upon Him, He has manifested Himself. Even if for a short while, He has taken a form and manifested Himself into this world.}

\enquote{He assumes various aspects, various forms, as desired by His devotees. Whoever calls Him, in whatever form they call Him, He will come to the \textit{bhakta} in that aspect. To a worshiper of Lord Śiva, He will appear as Śiva. To a worshiper of Devī, He will appear as Devī. To a worshiper of Nārāyaṇa, He will appear as Nārāyaṇa.}

\enquote{If you don’t perceive the divinity within yourself, if you don’t perceive God inside yourself, you will never find Him outside. This is true knowledge, the knowledge of the Self. When this wisdom arises, when one gets this revelation, and dives into the ocean within oneself, then one perceives God within. The saints perceive the Divine within themselves, and they see the same Lord everywhere. Then they don’t judge anyone. Why should they judge someone when they see that what is inside them is also in the other person, who performs a different role? But this is high knowledge.}

\enquote{The creation will never understand the Creator. The only thing that the creation can do is to surrender to the Creator, nothing else. That’s why He says that even if the \textit{ṛṣis} try to understand, even if the \textit{yogīs} try to understand, even if the demigods try to understand, their knowledge is limited.}

\newpage
\section{Verses 8-11: Kṛṣṇa Gives Knowledge to His Devotees}\label{sec-verses-8-11-krsna-gives-knowledge-to-his-devotees}

\Verse[10.8]
{ahaṁ sarvasya prabhavo mattaḥ sarvaṁ pravartate \\
iti matvā bhajante māṁ budhā bhāva-samanvitāḥ}
{I am the source of all; everything originates from Me. Realizing this, the wise worship Me full of devotion.}

\Verse[10.9]
{mac-cittā mad-gata-prāṇā bodhayantaḥ parasparam \\
kathayantaś ca māṁ nityaṁ tuṣyanti ca ramanti ca}
{With their minds focused on Me and their lives centered upon Me, they inspire one another. By continually speaking of Me, they attain contentment and delight.}

\Verse[10.10]
{teṣāṁ satata-yuktānāṁ bhajatāṁ prīti-pūrvakam \\
dadāmi buddhi-yogaṁ taṁ yena mām upayānti te}
{To those who are in constant union with Me and worship Me with intense affection, I grant the yoga of understanding by which they attain Me.}

\Verse[10.11]
{teṣām evānukampārtham aham ajñāna-jaṁ tamaḥ \\
nāśayāmy ātma-bhāva-stho jñāna-dīpena bhāsvatā}
{Out of compassion for them, I---abiding within the Self---disperse their darkness born of ignorance with the radiant lamp of knowledge.}

\enquote{Those who are deluded---the sensually minded people---are engrossed in all kinds of worldly enjoyments, and they find great delight in them. People who work the whole week, and look forward to getting drunk at the weekend, don’t realize when the weekend has passed. What have they done? They don’t remember. They call this \enquote{life.} I know many of you have gone through this stage, when you were young. Not all of you had the grace to be on the spiritual path as quickly as others. You experienced \textit{māyā}\ldots Now you’re drinking the \textit{amṛta}, which in turn helps you to become divine. The one who drinks the nectar that is derived from hearing the glory of the Lord, is purified. Even just by hearing the glory of the Lord, all the sins of one’s past---not only of this life, but also of past lives---are washed away. And \textit{bhakti} awakens in the heart of one who is not deluded by the world. His ultimate reality is revealed only to the one who is ready. Many are not ready, but you experience this in your daily life.}

\enquote{One who is fully surrendered in devotion to Him is full of great joy\ldots Through this surrender and this Love, one becomes like a lantern and starts to shine the light for others in the darkness. One becomes a beam of light, a supporter. The Lord says that a \textit{bhakta} is one who helps others to come on the way, to find the Divine, to become spiritual. A \textit{bhakta} is not egoistic. You don’t keep what you get for yourself; you share it, you spread it.}

\enquote{There are saints who don’t speak one word. But the \textit{bhaktas} find great delight, and everything else, in silence---just by being around the saints. Mahāvatāra Bābājī doesn’t speak as I’m speaking right now. Everything is given in silence and a \textit{bhakta} finds great delight in Him.}

\newpage
\section{Verses 12-18: Arjuna Accepts Kṛṣṇa's Divinity}\label{sec-verses-12-18-arjuna-accepts-krsna-s-divinity}

\Verse[10.12–13]
{\hspace*{1em}arjuna uvāca \\
paraṁ brahma paraṁ dhāma pavitraṁ paramaṁ bhavān \\
puruṣaṁ śāśvataṁ divyam ādi-devam ajaṁ vibhum \\
āhus tvām ṛṣayaḥ sarve devarṣir nāradas tathā \\
asito devalo vyāsaḥ svayaṁ caiva bravīṣi me}
{\hspace*{1em}Arjuna said: \\
You are the Parabrahman, the Supreme Abode, the supreme purifier. All the sages, such as the divine sage Nārada, Asita, Devala, and Vyāsa call you the eternal, transcendent Supreme Being, the primordial deity, unborn and all-pervading. And now You, Yourself, declare it to me.}

\Verse[10.14]
{sarvam etad ṛtaṁ manye yan māṁ vadasi keśava \\
na hi te bhagavan vyaktiṁ vidur devā na dānavāḥ}
{O Keśava, I consider all that You speak to me to be true. Indeed, O Lord, neither the \textit{devas} nor the \textit{asuras} comprehend Your manifestation.}

\Verse[10.15]
{svayam evātmanātmānaṁ vettha tvaṁ puruṣottama \\
bhūta-bhāvana bhūteśa deva-deva jagat-pate}
{O Supreme Person, originator and controller of all beings, God of gods, and Lord of the worlds, only You know Yourself by Your own Self}

\Verse[10.16]
{vaktum arhasy aśeṣeṇa divyā hy ātma-vibhūtayaḥ \\
yābhir vibhūtibhir lokān imāṁs tvaṁ vyāpya tiṣṭhasi}
{Thus, only You are capable of recounting in full Your divine glories. Indeed, by which manifestations You pervade and abide within all these worlds.}

\Verse[10.17]
{kathaṁ vidyām ahaṁ yogiṁs tvāṁ sadā paricintayan \\
keṣu keṣu ca bhāveṣu cintyo ’si bhagavan mayā}
{O Lord of \textit{yogīs}, how may I know You and constantly meditate upon You? In which objects, O Lord, are You to be contemplated by me?}

\Verse[10.18]
{vistareṇātmano yogaṁ vibhūtiṁ ca janārdana \\
bhūyaḥ kathaya tṛptir hi śṛṇvato nāsti me ’mṛtam}
{O Janārdana, describe again in detail Your divine power and glories. Indeed, I find no satisfaction in hearing Your words which are the nectar of immortality.}



\enquote{Arjuna is fully absorbed in the Love of Kṛṣṇa. His mind is also fully absorbed, so Arjuna is fully convinced about everything that Kṛṣṇa is saying.}

\enquote{He realizes the supremacy of the Lord, praises the Lord, and bows to Him, saying, \enquote{You are the One that all \textit{ṛṣis} and the \textit{devas} have praised. And here, You---the Lord who all have praised---stands in front of me.}}

\enquote{Kṛṣṇa has given Arjuna one drop of \textit{amṛta}, and once one has had one drop of \textit{amṛta}, one wants more. The Love that is awakening in Arjuna’s heart, the longing, is the \textit{amṛta} which will lead him to immortality, from ignorance of the Self to realization of the Self. Arjuna says, \enquote{I am not satisfied with what I have heard, it’s not enough, I want more!} This feeling is not from greed. Out of greediness, someone longs for material things; to get more and more for self-enjoyment; to satisfy the senses with objects of the senses; all just to be pleased on the outside. But here, it’s \textit{bhakti} itself; it’s the awakening of devotion.}

\enquote{Now, Arjuna is not thinking of war; he is not thinking of anything. In the first chapter, his mind is completely destroyed. He doesn’t want to fight, he is fully absorbed in a state of delusion, and he is deluding himself completely, even thinking about what will happen to his ancestors later on when they die. When Kṛṣṇa tells him to fight, he finds hundreds of excuses not to fight. But here, he finds hundreds of excuses to listen to the glory of the Lord. He doesn’t want Him to stop talking. When Kṛṣṇa stops talking, he says, \enquote{Please!} He’s finding excuses to ask Kṛṣṇa to reveal more and more of Himself.}

\section{Verses 19-42: Kṛṣṇa Describes His Glories}\label{sec-verses-19-42-krsna-describes-his-glories}

\Verse[10.19]
{\hspace*{1em}śrī-bhagavān uvāca \\
hanta te kathayiṣyāmi divyā hy ātma-vibhūtayaḥ \\
prādhānyataḥ kuru-śreṣṭha nāsty anto vistarasya me}
{\hspace*{1em}Bhagavān Kṛṣṇa said: \\
My dear, best of the Kurus. Indeed, I will tell you only about my prominent divine glories, since there is no limit to My manifestations.}

\Verse[10.20]
{aham ātmā guḍākeśa sarva-bhūtāśaya-sthitaḥ \\
aham ādiś ca madhyaṁ ca bhūtānām anta eva ca}
{I am the Paramātmā, O Guḍākeśa, dwelling in the hearts of all beings. I am verily their beginning, middle, and end.}

\Verse[10.21]
{ādityānām ahaṁ viṣṇur jyotiṣāṁ ravir aṁśumān \\
marīcir marutām asmi nakṣatrāṇām ahaṁ śaśī}
{Among the Ādityas,\footnote{These are the twelve deities which take the role of the sun. Every month these deities change, and out of them, Viṣṇu is the most radiant.} I am Viṣṇu; among luminous objects, I am the radiant sun. Among the Maruts,\footnote{Identified as wind deities, each with a specific function. All of them are presided over by Marīci.} I am Marīci, and among the stars, I am the Moon.}

\Verse[10.22]
{vedānāṁ sāma-vedo ’smi devānām asmi vāsavaḥ \\
indriyāṇāṁ manaś cāsmi bhūtānām asmi cetanā}
{Among the \textit{Vedas}, I am the \textit{Sāma Veda};\footnote{One of the four \textit{Vedas}. The \textit{Sāma Veda} contains melodies and chants.} among the \textit{devas}, I am Indra. Among the senses, I am the mind, and among living beings, I am consciousness.}

\Verse[10.23]
{rudrāṇāṁ śaṅkaraś cāsmi vitteśo yakṣa-rakṣasām \\
vasūnāṁ pāvakaś cāsmi meruḥ śikhariṇām aham}
{Among the Rudras,\footnote{Described as loyal companions of Rudra, also identified as Lord Śiva. Rudras are messengers and different aspects of Śiva and are fearsome in nature. Śaṅkara is commonly known as the ascetic form of Śiva who dwells in Mount Kailāśa.} I am Śaṅkara; among the Yakṣas\footnote{Class of nature spirits who serve Kubera (deity responsible for wealth).} and Rākṣasas,\footnote{A race of beings who are generally well known for their aggressive and bloodthirsty nature.} I am Kubera.\footnote{Deity responsible for material wealth, and king of the Yakṣas (nature spirits).} Among the Vasus,\footnote{Eight elemental deities representing different aspects of nature. The element of fire is presided over by Agni.} I am Agni, and among mountains, I am Meru.\footnote{The sacred five-peaked mountain, considered to be the center of the physical, metaphysical, and spiritual universe. It resides in the spiritual realm and cannot be perceived with the physical eyes.}}

\Verse[10.24]
{purodhasāṁ ca mukhyaṁ māṁ viddhi pārtha bṛhaspatim \\
senānīnām ahaṁ skandaḥ sarasām asmi sāgaraḥ}
{O Pārtha, among the priests, know Me to be the chief: Bṛhaspati.\footnote{Deity responsible for wisdom and is the preceptor of the \textit{devas}.} Among army generals, I am Skanda.\footnote{Born from the third eye of Śiva, he is the brother of Gaṇeśa, and leads the celestial armies against the Asuras.} Among reservoirs of water, I am the ocean.}

\Verse[10.25]
{maharṣīṇāṁ bhṛgur ahaṁ girām asmy ekam akṣaram \\
yajñānāṁ japa-yajño ’smi sthāvarāṇāṁ himālayaḥ}
{Among the great sages, I am Bhṛgu.\footnote{One of the seven great \textit{Sapta-ṛṣis} born from Brahmā.} Among words, I am the single syllable \enquote{\textit{oṁ}.} Among sacrifices, I am the sacrifice of \textit{japa}.\footnote{Meditative repetition of a \textit{mantra} or a Divine Name.} Among immovable things, I am the Himālayas.}

\Verse[10.26]
{aśvatthaḥ sarva-vṛkṣāṇāṁ devarṣīṇāṁ ca nāradaḥ \\
gandharvāṇāṁ citrarathaḥ siddhānāṁ kapilo muniḥ}
{Among all trees, I am the Aśvattha.\footnote{A sacred fig tree mentioned in various Hindu texts.} Among celestial seers, I am Nārada.\footnote{A great \textit{Vedic} sage born from Brahmā and one the foremost devotees of the Lord. He wanders the universe imparting the knowledge of devotion.} Among the Gandharvas,\footnote{A group of celestial deities skilled in various arts such as music and dancing. Among the Gandharvas, Citraratha is said to be highly skilled in the art of singing.} I am Citraratha, and among perfected beings, I am the sage Kapila.\footnote{Incarnation of the Lord and a great sage who taught his mother Devahūtī the knowledge of the Self, as mentioned in the \textit{Śrīmad Bhāgavatam}.}}

\Verse[10.27]
{uccaiḥśravasam aśvānāṁ viddhi mām amṛtodbhavam \\
airāvataṁ gajendrāṇāṁ narāṇāṁ ca narādhipam}
{Among horses, know Me to be Uccaiḥśravas,\footnote{A beautiful white horse produced from the Ocean of Milk and claimed by Bali, the leader of the Asuras.} born from the churning of the milky ocean.\footnote{One of the central events in the eternal struggle between the \textit{devas} and \textit{asuras}. Both groups churned the Ocean of Milk to obtain the nectar of immortality (\textit{amṛta})} Among mighty elephants, I am Airāvata,\footnote{Considered to be the best among elephants, and the vehicle of the king of the demigods (Indra).} and among humans, I am the king.}

\Verse[10.28]
{āyudhānām ahaṁ vajraṁ dhenūnām asmi kāma-dhuk \\
prajanaś cāsmi kandarpaḥ sarpāṇām asmi vāsukiḥ}
{Among weapons, I am the Vajra.\footnote{The main weapon of the king of the demigods (Indra) in the form of a thunderbolt.} Among cows, I am Kāmadhenu.\footnote{The mother of all cows who provides her owners with whatever they desire.} Among the causes of procreation, I am Kandarpa,\footnote{Deity responsible for human love. Most commonly associated with Cupid in the West.} and among serpents, I am Vāsuki.\footnote{King of the serpents, who helped in the Churning of the Ocean of Milk.}}

\Verse[10.29]
{anantaś cāsmi nāgānāṁ varuṇo yādasām aham \\
pitṝṇām aryamā cāsmi yamaḥ saṁyamatām aham}
{Among the \textit{nāgas}, I am Ananta.\footnote{Half human/half serpent beings that reside in the netherworld. Ananta is the king of the Nāgas and eternal servant of Viṣṇu. Viṣṇu is often depicted resting on Ananta.} Among aquatics, I am Varuṇa.\footnote{Deity responsible for the oceans.} Among the ancestors, I am Aryamā, and among those who enforce the law, I am Yama.\footnote{The lord of death who rules the netherworld. For those bound to this material world, he delivers justice for each soul.}}

\Verse[10.30]
{prahlādaś cāsmi daityānāṁ kālaḥ kalayatām aham \\
mṛgāṇāṁ ca mṛgendro 'haṁ vainateyaś ca pakṣiṇām}
{Among the Daityas,\footnote{A clan of \textit{asuras} who are eternally engaged in a fight with the \textit{devas}. Prahlāda was a great king of the Asuras and devotee of the Lord. He played a central role in the incarnation of Nārasiṁha.} I am Prahlāda, and among those who subjugate, I am time. Among beasts, I am the lion, and among birds, I am Garuḍa.\footnote{Considered to be the best among birds, in the form of an eagle. He is the vehicle and eternal servant of Viṣṇu.}}

\Verse[10.31]
{pavanaḥ pavatām asmi rāmaḥ śastra-bhṛtām aham \\
jhaṣāṇāṁ makaraś cāsmi srotasām asmi jāhnavī}
{Among moving things, I am the wind. Among those who wield weapons, I am Rāma. Among fish, I am the \textit{makara}, and among rivers, I am Gaṅgā.}

\Verse[10.32]
{sargāṇām ādir antaś ca madhyaṁ caivāham arjuna \\
adhyātma-vidyā vidyānāṁ vādaḥ pravadatām aham}
{O Arjuna, among beings, I am the beginning, the end, and also the middle. Among kinds of knowledge, I am the knowledge of the Self, and among debates, I am the ultimate conclusion.}

\Verse[10.33]
{akṣarāṇām a-kāro ’smi dvandvaḥ sāmāsikasya ca \\
aham evākṣayaḥ kālo dhātāhaṁ viśvato-mukhaḥ}
{Among letters, I am the letter \enquote{A.} Among compounds, I am the dual compound. I am indeed imperishable time, and I am Brahmā, facing every direction.}

\Verse[10.34]
{mṛtyuḥ sarva-haraś cāham udbhavaś ca bhaviṣyatām \\
kīrtiḥ śrīr vāk ca nārīṇāṁ smṛtir medhā dhṛtiḥ kṣamā}
{I am death, the destroyer of all, as well as the origin of all that shall be born. Among feminine qualities, I am fame, prosperity, eloquence, memory, intelligence, patience, and forgiveness.}

\Verse[10.35]
{bṛhat-sāma tathā sāmnāṁ gāyatrī chandasām aham \\
māsānāṁ mārga-śīrṣo ’ham ṛtūnāṁ kusumākaraḥ}
{Among the hymns of the \textit{Sāma Veda} , I am the \textit{Bṛhat-sāma},\footnote{A hymn which praises the Lord. The \textit{Bṛhat-sāma} is chanted at the conclusion of auspicious \textit{Vedic} ceremonies.} and among \textit{Vedic} meters, I am the Gāyatrī.\footnote{Highly revered \textit{mantra} from the \textit{Ṛg Veda} dedicated to the Sun deity and is chanted widely by Hindus.} Among months, I am Mārgaśīrṣa,\footnote{The ninth month in the Hindu lunar calendar when grains from agriculture fields are collected by farmers to earn money and make their living in India.} and among seasons, I am spring.}

\Verse[10.36]
{dyūtaṁ chalayatām asmi tejas tejasvinām aham \\
jayo ’smi vyavasāyo ’smi sattvaṁ sattvavatām aham}
{Among fraudulent activities, I am gambling. I am the brilliance of the brilliant, I am victory, I am determined effort, and I am the strength of the strong.}

\Verse[10.37]
{vṛṣṇīnāṁ vāsudevo ’smi pāṇḍavānāṁ dhanañ-jayaḥ \\
munīnām apy ahaṁ vyāsaḥ kavīnām uśanā kaviḥ}
{Among the Vṛṣṇis,\footnote{Ancient \textit{Vedic} clan which Kṛṣṇa was born into. Vāsudeva, who belonged to the Vṛṣṇi clan was the father of Kṛṣṇa.} I am Vāsudeva. Among the Pāṇḍavas, I am Arjuna. Among sages, I am Vyāsa,\footnote{A great \textit{Vedic} sage who organized and divided the \textit{Vedas}. He authored many of the \textit{puranic} scriptures, but is perhaps most famous for authoring the \textit{Mahābhārata} and the \textit{Śrīmad Bhāgavatam}.} and among great thinkers, I am Śukrācārya.}

\Verse[10.38]
{daṇḍo damayatām asmi nītir asmi jigīṣatām \\
maunaṁ caivāsmi guhyānāṁ jñānaṁ jñānavatām aham}
{Among rulers I am the principle of punishment. Among conquerors, I am diplomacy. Of secrets, I am silence, and I am the wisdom of the wise.}

\Verse[10.39]
{yac cāpi sarva-bhūtānāṁ bījaṁ tad aham arjuna \\
na tad asti vinā yat syān mayā bhūtaṁ carācaram}
{O Arjuna, whatever is the seed of all living beings, that I am. There is nothing, whether moving or unmoving, that can exist without Me.}

\Verse[10.40]
{nānto ’sti mama divyānāṁ vibhūtīnāṁ parantapa \\
eṣa tūddeśataḥ prokto vibhūter vistaro mayā}
{O scorcher of foes, there is no end to My divine glories. This description spoken by Me is merely a brief indication of My glories.}

\Verse[10.41]
{yad yad vibhūtimat sattvaṁ śrīmad ūrjitam eva vā \\
tat tad evāvagaccha tvaṁ mama tejo-’ṁśa-sambhavam}
{Whatever exists that is splendid, glorious, or powerful, know that it has manifested from just a part of My splendor.}

\Verse[10.42]
{atha vā bahunaitena kiṁ jñātena tavārjuna \\
viṣṭabhyāham idaṁ kṛtsnam ekāṁśena sthito jagat}
{But of what use is it to know all these details, Arjuna? With just a fraction of Myself, I sustain and pervade this entire universe.}

\enquote{The Lord is not just a normal ordinary person, even if He appears very human. By being among the people on the battlefield in the \textit{Kurukṣetra}, everybody could see Kṛṣṇa as normal, but in reality, He was God Himself. It is through Him that creation happens. It is through Him that creation is preserved, and it is through Him that dissolution happens.}

\enquote{He is revealing to Arjuna His omnipresence and His identity with all. All creatures which have been created, whatever we see around us, the \enquote{moving or unmoving,} the \enquote{animate or inanimate} are all pervaded by Him alone. So nothing is devoid of the presence of God. The one who looks for Him in their own consciousness will find Him. The one who looks for Him in the stone, will find Him. The one who looks for Him in the tree, will find Him. But the true knowledge of this is to long for God only, not to long for the tree or the creatures of the trees. It’s to go beyond this! And then with true knowledge, one realizes the presence of the Lord, and that everything is pervaded only by Him. All existence is only Him. So when one has this true knowledge, this true wisdom, one should think only of God, in whatever one is doing. Wherever the mind runs, let it come back, and be focused on the Lord Himself.}

\enquote{Here Bhagavān says that the glories are endless and countless, they are so numerous, that what He can say about the glories is just a summary. In all universes, there’s no limit to His glory. He says, \enquote{What I have given you here is just a fraction, a portion, of these glories. And it is very difficult for a normal human mind to understand these small splinters I have given you.}}

\enquote{The realized soul, the one who has ascended towards God consciousness, who has attained God-realization, perceives the eternal Supreme Lord with full knowledge and full awareness.}

\enquote{So, all these manifestations are sustained by just a fraction of Himself alone, it’s not His fullness. But, in that fraction resides the fullness. However, the fullness doesn’t reside in that fraction.}
