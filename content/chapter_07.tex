\chapter{Jñāna-vijñāna-yoga}\label{chap-jnana-vijnana-yoga}
\chaptersubtitle{Knowledge of Kṛṣṇa}
%%%%%%%%%%%%%%%%%%%%%%%%%%%%%%%%%%%%%%%%%%%%%%%%%%%%%%%%%

\noindent The first six chapters have focused on how one can attain realization of the Self through desireless action. The practice of \textit{karma-yoga} allows one to internally withdraw from the world and gain mastery over the senses. Eventually, through \textit{dhyāna-yoga} one gains knowledge of the \textit{ātmā} (\textit{Brahma-jñāna}) and becomes firmly established in this state. But throughout Kṛṣṇa's discourse, Arjuna has been both confused and doubtful about just how successful one can be on this path.

Chapter seven marks the beginning of a new phase in the \textit{Gītā’s} teaching, where the next six chapters shift the emphasis to devotion (\textit{bhakti}) on Kṛṣṇa as a means of freeing oneself from the bondage of \textit{karma}. It is where Arjuna gradually becomes aware of the divine personality who creates, sustains, and destroys all that exists. So far there have been incidental references to Kṛṣṇa as the Supreme Lord, but now this truth is revealed explicitly to Arjuna.

In this chapter, Arjuna learns how this world and the individual souls rest upon Kṛṣṇa, forming a part of His cosmic form. Not only does Kṛṣṇa support creation, but He is actually immanent within each and every aspect of it. He is the fragrance of the earth, the brilliance of fire and the life of life itself.

Kṛṣṇa states that this material world is very difficult to cross over, but by surrendering to Him one can easily break free of it. Various people approach Kṛṣṇa, but out of these it is the pure devotee who is most dear. Such a person enjoys a mutual loving relationship and sees the Truth that Kṛṣṇa is all things. But those without this knowledge, worship other gods, failing to see that they are empowered by Kṛṣṇa alone. Kṛṣṇa points out that the ignorant cannot recognize His divinity and are fooled by His external form. Only those who are pure and engage in virtuous acts can sincerely worship Him as the Lord.

\section{Verses 1-3: The Rarity of Knowing Kṛṣṇa}\label{sec-verses-1-3-the-rarity-of-knowing-krsna}

\Verse[7.1]
{\hspace*{1em}śrī-bhagavān uvāca \\
mayy āsakta-manāḥ pārtha yogaṁ yuñjan mad-āśrayaḥ \\
asaṁśayaṁ samagraṁ māṁ yathā jñāsyasi tac chṛṇu}
{\hspace*{1em}Bhagavān Kṛṣṇa said: \\
Listen Pārtha, how with your mind attached to Me, practicing \textit{yoga}, and taking shelter in Me, you will without doubt know Me completely.}

\Verse[7.2]
{jñānaṁ te ’haṁ sa-vijñānam idaṁ vakṣyāmy aśeṣataḥ \\
yaj jñātvā neha bhūyo ’nyaj jñātavyam avaśiṣyate}
{I will explain that knowledge to you, along with the process of its immediate realization. Once understood, nothing more remains to be known in this world.}

\Verse[7.3]
{manuṣyāṇāṁ sahasreṣu kaścid yatati siddhaye \\
yatatām api siddhānāṁ kaścin māṁ vetti tattvataḥ}
{Among thousands of people, perhaps one strives for perfection; even among those who strive for perfection as well as those who have reached perfection, perhaps one alone knows Me in reality.}

\enquote{How great it is to be born as a human. Because no other creature has the faculty of longing for the Lord, of attaining the Lord, of achieving God-realization. No other creature has this choice. You will not see a bird saying, \enquote{Oh, today I will not do anything. I will just sit and meditate on God.} Rarely is one given the gift to realize God.}

\enquote{Sadly, human beings don’t realize the greatness of the blessing, the grace of being born on Earth---even if they are reminded thousands of times, \enquote{Rise above this illusion, don’t attach yourself to this illusion.} The attachment to worldly enjoyment lasts only during this life and then you are bound by imperfection. Kṛṣṇa says, \enquote{Love God.} Yet, out of the thousands who love God, only a handful, only a few, can really reach perfection. As it is said in the Bible, \enquote{Many are called; few are chosen.}}

\enquote{When one sees Him within the core of one’s own heart, within one’s own Self, through this supreme knowledge, one will perceive the Supreme Lord everywhere. The \textit{bhaktas}, the devotees, who have achieved such perfection, see that there is no difference anywhere. Wherever they go, whichever form of the Divine they worship, they perceive only one reality\ldots Kṛṣṇa says that this is very rare. This is why there are so many groups, so many denominations, that bring abomination into the world, because of this lack of perception that it is the same Supreme Lord who has different Names and different forms, who is present everywhere. That’s why there is complete judgment, \enquote{My path is better than yours.}}

\enquote{Some devotees have the grace of becoming true \textit{Vaiṣṇavas}. However, one doesn’t become a true \textit{Vaiṣṇava} just by wearing a \textit{tilak} and a \textit{dhotī}. The true \textit{Vaiṣṇavas} who are striving to attain the Lord, change inside and let the divine qualities awaken inside them. They are completely detached from the outside world and are not attached to anything. Whatever they have, they offer it in devotion to the Lord. They don’t desire anything but Him. They have super faith, and when they sit in their \textit{sādhana}, they don’t think, \enquote{Oh my goodness, when will I finish?} The Lord will reveal Himself within these great ones.}

\enquote{When you look at the life of saints, you see that they are so fully absorbed into Kṛṣṇa that nothing else matters to them, nothing in the outside reality disturbs them. Bhagavān Kṛṣṇa says that there were many great kings who achieved such a state. They still continued doing their everyday duties, but their minds were fully focused on the Lord. However, Lord Kṛṣṇa says that even among those, only a few perceive and know Him \enquote{in reality.} One can know Him in the scriptures, yet only the one who is fully surrendered, can know Him as the supreme reality. And these blessed ones are truly very few.}

\section{Verses 4-11: The Higher and Lower Natures of God}\label{sec-verses-4-11-the-higher-and-lower-natures-of-god}
\Verse[7.4]
{bhūmir āpo ’nalo vāyuḥ khaṁ mano buddhir eva ca \\
ahaṅkāra itīyaṁ me bhinnā prakṛtir aṣṭadhā}
{Earth, water, fire, air, ether, mind, intellect, and ego---My \textit{prakṛti} is thus divided eightfold.}

\Verse[7.5]
{apareyam itas tv anyāṁ prakṛtiṁ viddhi me parām \\
jīva-bhūtāṁ mahā-bāho yayedaṁ dhāryate jagat}
{This is My lower nature, but know that distinct from it is My higher nature, the individual \textit{jīva} by which the cosmos is upheld, O mighty-armed Arjuna.}

\Verse[7.6]
{etad-yonīni bhūtāni sarvāṇīty upadhāraya \\
ahaṁ kṛtsnasya jagataḥ prabhavaḥ pralayas tathā}
{Know these two natures of Mine to be the source of all beings. In this way I am the origin and the dissolution of this entire universe.}

\enquote{The whole universe comes out of Kṛṣṇa and disappears back into Kṛṣṇa. This is a law of creation, which manifests through His two \textit{Śaktis}---\textit{aparā-prakṛti} and \textit{parā-prakṛti}. Everything is taking place in His cosmic body.}

\enquote{\textit{Prakṛti}, His \textit{Māyā}, makes everybody jump and dance to Her tune and think that this tune is the higher reality. However, this eightfold \textit{prakṛti} is just one aspect of His \textit{Māyā}: it is called \textit{aparā-prakṛti}, where the mind is veiled and can’t perceive the eternal reality. These \enquote{eightfold} principles are the Lord’s \enquote{material nature (\textit{prakṛti}).} It is through this \textit{prakṛti}, this \textit{māyā}, that the Lord manifests Himself and manifests His whole creation. It is because of this \textit{māyā}, which created the whole outside world, that human beings get trapped, carried away and rooted in ignorance, forgetting that there is a higher, supreme reality.}

\enquote{Who are you? This mind? This ego? This intellect? People who dwell in the lower nature of \textit{māyā}, perceive the difference between themselves and God. They don’t realize themselves. Then they stay in ignorance. And due to ignorance they have this pain, this sadness, this suffering of separation.}

\enquote{Everything is His \textit{māyā}, His own creation. Through these two \textit{prakṛtis}---\textit{aparā-prakṛti} and \textit{parā-prakṛti}---He manifests and sustains His own creation, where He plays all the different roles. The one who has realized such a state, doesn’t sing the \enquote{Glory Song}: \enquote{I am beautiful. I am the most important. I am the greatest. There’s nothing but me.} Rather he sings, \enquote{Kṛṣṇa, You are beautiful. Kṛṣṇa is everything. Kṛṣṇa is here, Kṛṣṇa is there and Kṛṣṇa is everywhere.} Such are the wise ones, who are always seated in this deep realization and perceive this reality, this true knowledge.}

\newpage
\section{Verses 7-12: Kṛṣṇa's Transcendence and Immanence}\label{sec-verses-7-12-krsna-s-transcendence-and-immanence}

\Verse[7.7]
{mattaḥ parataraṁ nānyat kiñcid asti dhanañjaya \\
mayi sarvam idaṁ protaṁ sūtre maṇi-gaṇā iva}
{There is nothing superior to Me, O Dhanañjaya. All this is strung on Me, like pearls on a thread.}

\Verse[7.8]
{raso ’ham apsu kaunteya prabhāsmi śaśi-sūryayoḥ \\
praṇavaḥ sarva-vedeṣu śabdaḥ khe pauruṣaṁ nṛṣu}
{O Kaunteya, I am the taste of water; I am the radiance of the moon and the sun. I am the \textit{oṁ} in all the \textit{Vedas}; I am the sound in ether, and the valor in men.}

\Verse[7.9]
{puṇyo gandhaḥ pṛthivyāṁ ca tejaś cāsmi vibhāvasau \\
jīvanaṁ sarva-bhūteṣu tapaś cāsmi tapasviṣu}
{I am the pure fragrance of the earth; I am the heat of fire; I am the life-force in all living beings, and the austerity of ascetics.}

\Verse[7.10]
{bījaṁ māṁ sarva-bhūtānāṁ viddhi pārtha sanātanam \\
buddhir buddhimatām asmi tejas tejasvinām aham}
{Pārtha, know Me to be the primordial seed of all beings. I am the intelligence of the intelligent, and the energy of the energetic.}

\Verse[7.11]
{balaṁ balavatāṁ cāhaṁ kāma-rāga-vivarjitam \\
dharmāviruddho bhūteṣu kāmo ’smi bharatarṣabha}
{Moreover, I am the strength of the strong, free from desire and passion. I am that desire in living beings which is compatible with \textit{dharma}, O best of the Bhāratas.}

\Verse[7.12]
{ye caiva sāttvikā bhāvā rājasās tāmasāś ca ye \\
matta eveti tān viddhi na tv ahaṁ teṣu te mayi}
{Know the states of \textit{sattva}, \textit{rajas}, and \textit{tamas} to originate from Me alone. But I am not in them; they are in Me.}

\enquote{All the universes are strung together like pearls on a necklace that He Himself wears. Nothing exists beyond Him. He is the sole reality. Like the thread holding the pearls, God is holding and sustaining His entire creation. And He is in each part of His creation.}

\enquote{He is the strength and the essence of men, because He is the Supersoul. He is the soul of souls. He is the one who pervades the soul itself and there is none beyond Him. If one reaches such a state, one fully realizes Kṛṣṇa, one attains Him fully.}

\enquote{\enquote{It is Me who inspires one to rise above desire. I am the one who gives the strength to rise above duality and see the oneness. I am also the one who is striving for awakening. It is Me who gives one the awareness to see the limitation and to divert from the limitation, so that one can perceive beyond \textit{prakṛti}. I am this eternal perfection seated beyond any duality.}}

\enquote{\enquote{But I am not in them, they are in Me.} Do you understand that? This means that even though God pervades everything, He is not touched by anything. Even in His cosmic body, He is ever-free from His own lower nature, from the qualities of \textit{rajasic}, \textit{tamasic}, and \textit{sattvic}. These qualities are said to be \enquote{secondary, subjective becomings of nature.} They evolve not from \textit{yoga-māyā}, \textit{parā-prakṛti}, but from\ldots \textit{aparā-prakṛti}, the \textit{prakṛti} that blinds men, the \textit{prakṛti} that makes one become attached to the outside reality. The supreme quality is beyond that. That’s why He says, \enquote{I am not in them.}}

\enquote{Even the \textit{ātmā} itself is a lower manifestation of the Lord. That’s why the \textit{ātmā} is always permanent and It is always there serving the Supreme. Have you ever seen great souls who have realized themselves? What do they do? They always pray and they always tell everybody else to pray. All the great sages are always constantly praying to the Supreme Lord. They realize that they are always next to Him and that they are here to serve Him. They don’t say, \enquote{Yes, I am the Ultimate. There is nothing to pray to.} They say, \enquote{There is this Supreme Lord that we all are turning towards.}\ldots Once one awakens, one will perceive that the Lord remains always the Lord. The servant remains always the servant.}

\section{Verses 13-15: Surrender Allows Us to Cross Over Māyā}\label{sec-verses-13-15-surrender-allows-us-to-cross-over-maya}

\Verse[7.13]
{tribhir guṇa-mayair bhāvair ebhiḥ sarvam idaṁ jagat \\
mohitaṁ nābhijānāti mām ebhyaḥ param avyayam}
{Deluded by these three states originating from the \textit{guṇas}, this whole world fails to recognize Me, who is unchanging and beyond them.}

\Verse[7.14]
{daivī hy eṣā guṇa-mayī mama māyā duratyayā \\
mām eva ye prapadyante māyām etāṁ taranti te}
{This divine \textit{māyā} of Mine consisting of the three \textit{guṇas} is certainly hard to overcome. But those who take refuge in Me alone, go beyond this \textit{māyā}.}

\Verse[7.15]
{na māṁ duṣkṛtino mūḍhāḥ prapadyante narādhamāḥ \\
māyayāpahṛta-jñānā āsuraṁ bhāvam āśritāḥ}
{However, the unrighteous, the foolish, the lowest of people, those whose wisdom has been stolen away by \textit{māyā}, and those who have taken on a demonic nature, do not seek shelter of Me.}

\enquote{The one who is full of sorrow, who is deluded by the glamour of the outside world, who rejoices in the objects of the senses, who is constantly focused on the outside reality, who is always longing for outward perfection, can’t perceive true perfection, true knowledge, the glory of God, the mystery of God, because the mind is so full of doubts and misconceptions. There are even people who are constantly hanging onto the qualities of the three \textit{guṇas}. Until they rise above the three \textit{guṇas}, they will never realize themselves and reach the goal of life. Bhagavān is emphasizing, \enquote{I am above that. When you realize that I am the cosmic soul, when you realize within the core of your Self that I am the essence of all, then you will rise above this kind of judgment; you will rise above the \textit{sattvic}, \textit{rajasic} and \textit{tamasic} qualities.} Then you can completely identify yourself with the Lord.}

\enquote{There is no judgment of others, because one knows that God is doing His work through each person. In one person, He is a banker; in another, He is a sweeper; in another, He is tapping on the \textit{iPad}; and in another, He is playing the drums. But the one who is a fool doesn’t realize this.}

\enquote{[The Lord is saying:] \enquote{As I am not touched by \textit{māyā}, so My devotees who are surrendered to Me are free. \textit{Māyā} stops controlling them. Those who chant My Name; those who by any means try to control their mind; those who are fully absorbed in My service, forgetting themselves completely; those who think only on Me; they are My devotees, and they cross beyond the \textit{guṇas}. The \textit{guṇas} don’t have any effect on them, because they become as Me. I start to shine upon them and they start to reflect My divine qualities, My cosmic qualities, and wherever My cosmic qualities are shining, \textit{māyā} doesn’t have any power.} Here Bhagavān is saying, \enquote{Even if it appears as if \textit{Māyā} controls everything, I am the Lord of \textit{Māyā} and I control Her.}}

\section{Verses 16-22: Four Categories of Devotees}\label{sec-verses-16-22-four-categories-of-devotees}

\Verse[7.16]
{catur-vidhā bhajante māṁ janāḥ sukṛtino ’rjuna \\
ārto jijñāsur arthārthī jñānī ca bharatarṣabha}
{Four kinds of virtuous people worship Me, O Arjuna: The distressed, the seeker of knowledge, the one who desires wealth, and the wise \textit{jñānī},\footnote{The pure devotee who possesses knowledge of who the Lord truly is.} O best of Bhāratas.}

\Verse[7.17]
{teṣāṁ jñānī nitya-yukta eka-bhaktir viśiṣyate \\
priyo hi jñānino ’tyartham ahaṁ sa ca mama priyaḥ}
{Of these, the \textit{jñānī} who is constantly united with Me and who is single-mindedly devoted to Me, excels. I am supremely dear to him and he is dear to Me.}

\Verse[7.18]
{udārāḥ sarva evaite jñānī tv ātmaiva me matam \\
āsthitaḥ sa hi yuktātmā mām evānuttamāṁ gatim}
{All those who approach Me are noble, but the \textit{jñānī} I consider to be My very essence. Such a person has turned toward Me, is absorbed in Me, and verily regards Me alone as the highest goal.}

\Verse[7.19]
{bahūnāṁ janmanām ante jñānavān māṁ prapadyate \\
vāsudevaḥ sarvam iti sa mahātmā sudurlabhaḥ}
{After many births, one who has true wisdom surrenders to Me, realizing that \enquote{Kṛṣṇa is everything.} Such a great soul is exceedingly rare.}

\Verse[7.20]
{kāmais tais tair hṛta-jñānāḥ prapadyante ’nya-devatāḥ \\
taṁ taṁ niyamam āsthāya prakṛtyā niyatāḥ svayā}
{Those whose wisdom has been veiled by numerous desires, take shelter of other deities and observe various rituals, enslaved by their own mental disposition.}

\Verse[7.21]
{yo yo yāṁ yāṁ tanuṁ bhaktaḥ śraddhayārcitum icchati \\
tasya tasyācalāṁ śraddhāṁ tām eva vidadhāmy aham}
{Whichever deity any \textit{bhakta} wishes to venerate with faith, it is I who grants them unshakable faith in it.}

\Verse[7.22]
{sa tayā śraddhayā yuktas tasyārādhanam īhate \\
labhate ca tataḥ kāmān mayaiva vihitān hi tān}
{Endowed with that faith, one engages in the worship of the chosen deity, obtaining from it desired objects, which are factually given by Me alone.}


\enquote{Lord Kṛṣṇa says that there are four kinds of \textit{bhaktas}, four kinds of devotees who come to Him. The first kind of devotees, \enquote{\textit{ārta},} are the suffering ones.}

\enquote{\enquote{\textit{Jijñāsu}} is the second kind of devotee, whose sole motive is to know God and to attain God-realization in this life. Unlike the first kind of devotees, they care not for the outside world. They are completely the opposite. They don’t care about the objects of the senses. Everything they do in their work, in their daily routine, is only for one thing: to attain the Lord, nothing else.}

\enquote{Here Lord Kṛṣṇa says that this kind of devotee is also very dear to Him. They worship the Lord, but they still perceive God as being outside themselves, separate from themselves. They don’t perceive the Divine, they don’t perceive Kṛṣṇa within themselves. They serve Him when He is present as a deity, but they can’t see beyond that. They don’t know that the Supreme Lord is beyond everything, is seated everywhere, in all His creation. Of course, they know it, but only in the mind; they have not perceived it.}

\enquote{The third kind of devotees, \enquote{\textit{arthārthī},} are those who know that everything comes from God, yet they are still very attached to wealth, honor, name, fame and glory. They seek God only to fulfill their desires. They pray, they do their \textit{sādhana}, they do everything with that one aim. These devotees have great faith and devotion and know deep within that only God can fulfill their desires. But they long only for God’s gifts and their faith is based on that. They do everything for personal gain, not selflessly. They know that God is the embodiment of Truth. They pray to Him for security, for they know that He is the only one who can protect and save them.}

\enquote{The fourth type of devotee, the \enquote{\textit{jñānī},} the realized souls, are the ones who abide completely in the reality of God consciousness. For them, there is nothing other than Him, Kṛṣṇa. All their desires have ceased, including the desire to meet Kṛṣṇa or to realize Kṛṣṇa, because they perceive Kṛṣṇa within themselves, in the core of their heart. Wherever they are, whatever they do, is an act of worship to the Lord. In such a state of devotion and surrender, they realize that there’s only the Lord Himself in them who is doing everything.}

\enquote{Here Bhagavān Kṛṣṇa Himself says that it is okay for ignorant people to worship other deities, not knowing His true nature. If they have faith, it is actually Him that makes that faith strong. For the sake of people’s faith, He manifests Himself in different aspects, He takes different forms. If the devotee worships Him through \textit{yajña}, due to the love and faith that they have, even if it is for the demigods, He limits Himself and manifests Himself into that aspect. Such is the mercy of God that by His own will, He happily limits Himself and takes any aspect, any form\ldots Through faith, you can transform a block of marble and make Bhagavān Himself install Himself fully in it. Then that marble statue is no longer just a statue. Your faith has transformed the statue and made it come alive: the Lord Himself has become alive in the statue. Many people may say, \enquote{In Hinduism, they pray to statues.} But people don’t know that they are not just statues. They are deities. They are alive. Through faith, they become alive.}

\section{Verse 23: What You Worship Determines Your Destination}\label{sec-verse-23-what-you-worship-determines-your-destination}

\Verse[7.23]
{antavat tu phalaṁ teṣāṁ tad bhavaty alpa-medhasām \\
devān deva-yajo yānti mad-bhaktā yānti mām api}
{But the reward of these people with limited understanding is temporary. The worshipers of the \textit{devas} will go to the \textit{devas}, whereas My devotees will attain Me.}

\enquote{Here Kṛṣṇa says people who pray or do big sacrifices just for the sake of selfish gain, whatever business they have done with the demigods, when they leave this plane, when they die, their soul migrates to the celestial world for a small holiday.}

\enquote{The demigods will give you what you ask for, but it’s a business. You go to those celestial realms for some time, and then you come back, you are born on Earth again because you missed the opportunity to realize God when you were here before. You missed the opportunity to attain Him through devotion to Him alone. Like that, many souls get trapped in this delusion. That’s why there is birth and death, death and birth; one continues in this cycle, one life after another. One stays in this dormant state of illusion. Even the Lord stays dormant within the hearts of these people, waiting for them to become conscious of Him. He is waiting, life after life, until they come face to face with God’s reality, until they have the chance to surrender to the \textit{guru} wholeheartedly, in body, mind and spirit.}

\section{Verses 24-30: Perceiving the Truth of Kṛṣṇa}\label{sec-verses-24-30-perceiving-the-truth-of-krsna}

\Verse[7.24]
{avyaktaṁ vyaktim āpannaṁ manyante mām abuddhayaḥ \\
paraṁ bhāvam ajānanto mamāvyayam anuttamam}
{Not knowing My higher state as imperishable and supreme, those who lack discrimination think of Me who is unmanifest to have assumed a limited manifestation.}

\Verse[7.25]
{nāhaṁ prakāśaḥ sarvasya yoga-māyā-samāvṛtaḥ \\
mūḍho ’yaṁ nābhijānāti loko mām ajam avyayam}
{I am not manifest to everybody, covered by My \textit{yoga-māyā}. This deluded world does not recognize Me as unborn and infallible.}

\Verse[7.26]
{vedāhaṁ samatītāni vartamānāni cārjuna \\
bhaviṣyāṇi ca bhūtāni māṁ tu veda na kaścana}
{O Arjuna, I know all beings of the past, those of the present, and those who are yet to come; but no one knows Me.}

\Verse[7.27]
{icchā-dveṣa-samutthena dvandva-mohena bhārata \\
sarva-bhūtāni sammohaṁ sarge yānti parantapa}
{O Bhārata, chastiser of enemies, due to the bewilderment of duality arising from desire and aversion, all beings in creation are subject to delusion.}

\Verse[7.28]
{yeṣāṁ tv anta-gataṁ pāpaṁ janānāṁ puṇya-karmaṇām \\
te dvandva-moha-nirmuktā bhajante māṁ dṛḍha-vratāḥ}
{But those people whose sins have ceased, who perform virtuous deeds, and who are freed from the delusion of the pairs of opposites, worship Me, steadfast in their determination.}

\Verse[7.29]
{jarā-maraṇa-mokṣāya mām āśritya yatanti ye \\
te brahma tad viduḥ kṛtsnam adhyātmaṁ karma cākhilam}
{Those who strive to be liberated from old age and death and have taken shelter of Me, come to fully know Brahman, \textit{adhyātma}, and the entirety of \textit{karma}.}

\Verse[7.30]
{sādhibhūtādhidaivaṁ māṁ sādhiyajñaṁ ca ye viduḥ \\
prayāṇa-kāle ’pi ca māṁ te vidur yukta-cetasaḥ}
{And those who realize Me in relation to the \textit{adhibhūta}, \textit{adhidaiva}, and the \textit{adhiyajña}, with their minds fixed, are aware of Me even at the hour of death.\footnote{These terms are defined by Kṛṣṇa in Chapter Eight.}}

\enquote{When the Lord manifests Himself in this world, He spreads His \textit{māyā} all over Himself and remains hidden; He covers Himself well with His own \textit{yoga-māyā}, wrapping it around Him like a shawl. However, even if He is covered by His own \textit{yoga-māyā}, He is never touched by \textit{yoga-māyā}.}

\enquote{Due to His \textit{yoga-māyā}, the vision of ordinary man cannot pierce this veil, so most people regard Him as just an ordinary human being. Such deluded ones cannot perceive the true identity of the unmanifest. They can only see what He is doing outwardly. The Kauravas said, \enquote{Oh, Kṛṣṇa is just a charioteer, sitting there and bringing Arjuna wherever Arjuna wants.} They could not see that He is bringing everyone; that it is He, the charioteer, in each one.}

\enquote{Due to his faith in Kṛṣṇa, Arjuna could humble himself in front of Him and say, \enquote{I take You now as my \textit{guru}. Instruct me. Show me the right way that will free me.} It is due to his faith that Kṛṣṇa is there with him. Without this faith, Kṛṣṇa would not be there.}

\enquote{Here Bhagavān Kṛṣṇa is saying, \enquote{I am the Lord Himself of the past, present and future. Even if I manifested Myself in a human form, even if I have limited Myself in this form in front of you, I am still the Supreme One. I know the past, I know the present and I know what will come, but no one knows Me.}}

\enquote{Those who are surrendered; the \textit{bhaktas} who have faith; those who are longing to know Him; those who want to get out of ignorance; those who have devotion and love; those who are free from judgment; free from criticizing others, free from delusion---such a devotee as this, has a glimpse of Him. Such a devotee knows a little bit of Him. And that little bit itself is everything, because their \enquote{knowing} is not in the mind. That knowing is deep inside of their heart, in their consciousness, in their soul itself.}

\enquote{[The Lord says:] \enquote{For those blessed souls, who throughout many lives, have done noble deeds, much penance, sacrifice, and charity, and practiced devotion, there is no sin. They are guided in this life to have good association, high association. Due to this good association, they are freed. They attain purity of heart. When they worship Me, when they surrender to Me, there is complete eradication of all negativity from all their past lives. They are freed from delusion, from the dualities and from judgment. They are freed from anger. In all circumstances, they stay balanced and ever focused on Me. Their minds never dwell in the world, even while they are doing their daily activities. Their minds are always focused on Me. They have attained the ultimate perfection.}}

\enquote{They live in Me and I live in them. They become a temple within which I am completely, fully present.\enquote{ They become a movable temple, a portable temple, within which the Lord is moving from one place to another. When you go to a temple, you always profit, isn’t it true? Like that, whoever goes to a temple, or whoever meets a temple on the way, gets a blessing, because temples always give this Love of God. Such is the one who has risen above duality, who has completely surrendered to Nārāyaṇa Kṛṣṇa.}}
