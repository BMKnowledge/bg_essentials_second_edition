\chapter{Chapter 6: Dhyāna-yoga}\label{chap-dhyana-yoga}

\paragraph*{\MakeUppercase{The Process of Meditation}}
%%%%%%%%%%%%%%%%%%%%%%%%%%%%%%%%%%%%%%%%%%%%%%%%%%%%%%%%%

\noindent Now that renunciation and the nature of action have been thoroughly explained, chapter six begins to describe \textit{dhyāna-yoga} or meditation. Kṛṣṇa up till now has provided a step-by-step path beginning with \textit{karma-yoga}. Once this is successfully practiced, one achieves inner tranquility. This creates the desired state for \textit{dhyāna-yoga}, where the mind is made to focus within and have the direct perception of the \textit{ātmā}.

Kṛṣṇa gives Arjuna the practical techniques upon which one can best attempt to master the mind through the practice of meditation. Gradually, through intense self-control and repeated practice, one is able to perceive the \textit{ātmā} and experience limitless joy, free from suffering. Responding to Arjuna’s fears of failure, Kṛṣṇa explains that the fruits of this \textit{yoga} are never lost. If one does not achieve the goal in this life, one will inevitably be given the right circumstances to carry on one’s spiritual development.

The chapter concludes by stating that the one who is absorbed in the path of \textit{yoga}, given by Kṛṣṇa thus far, is superior to all other spiritual seekers, but among all such \textit{yogīs}, the one who dwells on Kṛṣṇa is the highest of all. It is this last verse which marks the beginning of a new line of discourse in the \textit{Gītā}: one where Kṛṣṇa places Himself explicitly as the Supreme Lord and teaches Arjuna the way of \textit{bhakti} (devotion).

\section{Verses 1-4: The True Renunciate}\label{sec-verses-1-4-the-true-renunciate}

\Verse[6.1]
{\hspace*{1em}śrī-bhagavān uvāca \\
anāśritaḥ karma-phalaṁ kāryaṁ karma karoti yaḥ \\
sa sannyāsī ca yogī ca na niragnir na cākriyaḥ}
{\hspace*{1em}Bhagavān Kṛṣṇa said: \\
One who performs their prescribed work while being detached from the results of action is a true renunciate and a \textit{yogī}; not one who gives up the sacrificial fire nor the one who avoids their duty.}

\Verse[6.2]
{yaṁ sannyāsam iti prāhur yogaṁ taṁ viddhi pāṇḍava \\
na hy asannyasta-saṅkalpo yogī bhavati kaścana}
{O Pāṇḍava, know that what is called renunciation to be \textit{yoga}. For no one becomes a \textit{karma-yogī} without renouncing desire.}

\Verse[6.3]
{ārurukṣor muner yogaṁ karma kāraṇam ucyate \\
yogārūḍhasya tasyaiva śamaḥ kāraṇam ucyate}
{For the sage seeking to ascend to \textit{yoga}, action is said to be the way; for one who has ascended to \textit{yoga}, cessation of action is the way.}

\Verse[6.4]
{yadā hi nendriyārtheṣu na karmasv anuṣajjate \\
sarva-saṅkalpa-sannyāsī yogārūḍhas tadocyate}
{When one is neither attached to sense-objects nor to performing action, and has abandoned all desires, then one is said to have ascended to the heights of \textit{yoga}.}

\enquote{Very often you will see people calling themselves renunciates, but still they enjoy everything, they are bothered about everything. They are renunciates only on the outside, but inside they have a certain motive or a certain aim behind their actions.}

\enquote{On the other hand, people who live in the world, but are not attached to it, who dedicate their life to serving others, to helping people, helping the animals, protecting the animals, they are the true \textit{sannyāsīs}.}

\enquote{Those who are doing every action without expecting something in return, selflessly; those who love unconditionally; those who are devoid of pride and ego and expectation; those who have renounced the mind (because it is the mind that desires and always longs for recognition); they are the true \textit{sannyāsīs}. It's not the color or the clothes that they are wearing; that’s not \textit{sannyāsa}. It’s not the external \textit{sannyāsa}. It is the internal.}

\section{Verses 5-9: The Mind is Our Best Friend or Our Worst Enemy}\label{sec-verses-5-9-the-mind-is-our-best-friend-or-our-worst-enemy}

\Verse[6.5]
{uddhared ātmanātmānaṁ nātmānam avasādayet \\
ātmaiva hy ātmano bandhur ātmaiva ripur ātmanaḥ}
{One should raise oneself by one’s own mind and not allow oneself to degrade, since the mind is indeed one’s friend, as well as one’s enemy.}

\Verse[6.6]
{bandhur ātmātmanas tasya yenātmaivātmanā jitaḥ \\
anātmanas tu śatrutve vartetātmaiva śatru-vat}
{The mind is a friend to one who has conquered it. But for one whose mind is not subdued, it remains hostile, like an enemy.}

\Verse[6.7–8]
{jitātmanaḥ praśāntasya paramātmā samāhitaḥ \\
śītoṣṇa-sukha-duḥkheṣu tathā mānāpamānayoḥ \\
jñāna-vijñāna-tṛptātmā kūṭa-stho vijitendriyaḥ \\
yukta ity ucyate yogī sama-loṣṭrāśma-kāñcanaḥ}
{One who is self-controlled and tranquil, whose mind is fixed on the Supreme, who is perfectly composed in cold and heat, pleasure and pain, as well as in honor and dishonor, who is content with knowledge and direct experience, who dwells in a transcendent state, who has mastered the senses and regards earth, stone, and gold to be of the same value, is said to be a \textit{yogī} united with the Absolute.}

\Verse[6.9]
{suhṛn-mitrāry-udāsīna-madhyastha-dveṣya-bandhuṣu \\
sādhuṣv api ca pāpeṣu sama-buddhir viśiṣyate}
{The one who regards equally a well-wisher, friend, enemy, one who is indifferent, a mediator, a hateful person, a relative, the righteous, and even the unrighteous, truly excels.}

\enquote{The mind is the best friend of one who has conquered the mind. Who has conquered what? It is the Self that conquers the mind.}

\enquote{So, what happens? The thoughts and subsequent actions, which are always under the command of the mind, change. Now, the mind is being controlled by the soul. When the mind is running somewhere else, the soul says, \enquote{Come back here!,} and puts you on the right track.}

\enquote{If you don’t control the mind, if you don’t succeed in your endeavor, the mind will become your worst enemy. This is when, very often, people go into deep depression. They collapse. They can’t handle anything. Even if they have all knowledge, but yet they\ldots have let the mind poison them.}

\enquote{When the mind and the senses are controlled, they don’t run after the sense objects, so one can direct the senses to where one wants; this is \enquote{self-mastery.} If the mind and senses are under control, they can be focused on God, on the image of the Divine, which will lead one towards God-realization.}

\enquote{The ones who are completely surrendered to God,\ldots through the control of the mind, they don’t get disturbed by anything. They are not touched by pleasure or pain, honor and dishonor, or anything. These opposites don’t affect them at all, because they see the Divine in everything. This is the quality of someone who has merged into his own Self. He reflects calmness, peace, tranquility and joy. These qualities emanate from the ones who are not agitated by anything; they are free.}

\enquote{True \textit{yogīs} are here on Earth to love: they always stay in that loving state and in that loving state, there is no judgment. It’s not like human love, where you put someone high up on a pedestal, and then the moment you hear or see something that you don’t like about the person you think you love, then you become aggressive, and depressive---you become all the \enquote{sives}---expansive, expressive, oppressive, and possessive\ldots Whereas the ones who excel are above all this, as they look at everything equally and rejoice in God consciousness. They see that it is God who is doing this multitude of activities through His creation. It is God who has manifested Himself and is enjoying Himself while He is doing everything.}

\enquote{There is no need to judge when you are absorbed in Divine Love. The only thing to do is to love. That’s why I chose \enquote{Just Love} Otherwise I would tell you, \enquote{Think about Love, think Love.} No, I said, \enquote{Just Love,} don’t bother about anything. Don’t worry. Your aim is to love. This is your \textit{dharma}.}

\section{Verses 10-17: The Practice of Meditation}\label{sec-verses-10-17-the-practice-of-meditation}
\Verse[6.10]
{yogī yuñjīta satatam ātmānaṁ rahasi sthitaḥ \\
ekākī yata-cittātmā nirāśīr aparigrahaḥ}
{Remaining alone in a solitary place, controlling mind and body, devoid of any desire and sense of ownership, the \textit{yogī} should constantly engage the mind in meditation.}

\Verse[6.11]
{śucau deśe pratiṣṭhāpya sthiram āsanam ātmanaḥ \\
nāty-ucchritaṁ nāti-nīcaṁ cailājina-kuśottaram}
{In a pure place, one should establish a firm seat for oneself which is neither too high nor too low, with a cloth and deer-skin on top of \textit{kuśa} grass.\footnote{A type of grass considered sacred in India. It is generally used for purification or as a conduit for energy during \textit{pūjā} and ritual ceremonies. It can also be used to construct a seat for meditation.}}

\Verse[6.12]
{tatraikāgraṁ manaḥ kṛtvā yata-cittendriya-kriyaḥ \\
upaviśyāsane yuñjyād yogam ātma-viśuddhaye}
{Sitting on that seat, restraining the activities of the thoughts and senses and making the mind single-pointed, one should practice \textit{yoga} for the purification of the mind.}

\Verse[6.13–14]
{samaṁ kāya-śiro-grīvaṁ dhārayann acalaṁ sthiraḥ \\
samprekṣya nāsikāgraṁ svaṁ diśaś cānavalokayan \\
praśāntātmā vigata-bhīr brahmacāri-vrate sthitaḥ \\
manaḥ saṁyamya mac-citto yukta āsīta mat-paraḥ}
{Holding the body, head, and neck straight and steady, gazing at the tip of the nose without looking around, being motionless, tranquil in mind and fearless, firm in the practice of celibacy, and subduing the mind, centering all thoughts on Me and having Me as the supreme goal, one should sit, being completely absorbed.}

\Verse[6.15]
{yuñjann evaṁ sadātmānaṁ yogī niyata-mānasaḥ \\
śāntiṁ nirvāṇa-paramāṁ mat-saṁsthām adhigacchati}
{Applying oneself constantly in this way, the \textit{yogī} whose mind is controlled attains peace which culminates in supreme immersion, thereby abiding with Me forever.}

\Verse[6.16]
{nāty-aśnatas tu yogo ’sti na caikāntam anaśnataḥ \\
na cāti-svapna-śīlasya jāgrato naiva cārjuna}
{O Arjuna, \textit{yoga} is not for one who overeats, nor for one who fasts excessively; nor is it for one who sleeps too much, nor for one who stays awake for lengthy periods.}

\Verse[6.17]
{yuktāhāra-vihārasya yukta-ceṣṭasya karmasu \\
yukta-svapnāvabodhasya yogo bhavati duḥkha-hā}
{For one who is moderate in eating, recreation, sleep, and wakefulness and who is regulated in the performance of actions, \textit{yoga} becomes the dispeller of all suffering.}


\enquote{Kṛṣṇa is saying you should practice your \textit{sādhana} for self-purification. Without self-purification, you can’t advance. This purification is very important. He is emphasizing purification through meditation. Purify yourself by chanting the Name of God, meditate on the Lord, meditate on His glory, meditate on His form, meditate on His Name.}

\enquote{For a \textit{yogī} to sit in Times Square is easy, because that one has transcended everything. But people who have not yet transcended, still have to find this quiet place in the outside where they will not be disturbed, where the mind will not run left and right.}

\enquote{The time and the place of meditation are very important, because when you set a certain time to meditate, you are inviting God to be there with you. It’s like if you invite a friend to come to your place and then you are not there to welcome your friend. How would it be? Or you are invited somewhere, you are ringing the bell and you see all the lights off. Then someone turns the light on and you see the person in pyjamas saying, \enquote{Ai! I thought it was tomorrow!} How would it be? You would be shocked, no? So, here it is the same thing. When you set a certain time to sit for meditation, you are inviting God to manifest Himself at that moment. And you can have this expectation that, as you sent the invitation, God will come and be eagerly waiting for you inside yourself at that time.}

\section{Verses 18-32: The Goal of Meditation}\label{sec-verses-18-32-the-goal-of-meditation}

\Verse[6.18]
{yadā viniyataṁ cittam ātmany evāvatiṣṭhate \\
niḥspṛhaḥ sarva-kāmebhyo yukta ity ucyate tadā}
{When the subdued mind rests in the Self alone and one is free from desire for any kind of pleasure, one is said to be well established in \textit{yoga}.}

\Verse[6.19]
{yathā dīpo nivāta-stho neṅgate sopamā smṛtā \\
yogino yata-cittasya yuñjato yogam ātmanaḥ}
{Just as the light of a lamp in a windless place does not flicker, so it is with the \textit{yogi} who has a controlled mind and relentlessly practices union with the \textit{ātmā}. This is the analogy that is cited in this regard.}

\Verse[6.20]
{yatroparamate cittaṁ niruddhaṁ yoga-sevayā \\
yatra caivātmanātmānaṁ paśyann ātmani tuṣyati}
{When the mind, restrained by the continuous practice of \textit{yoga}, comes to a standstill and one perceives the \textit{ātmā} by the mind in that state of absorption, one rejoices in the \textit{ātmā} alone.}

\Verse[6.21]
{sukham ātyantikaṁ yat tad buddhi-grāhyam atīndriyam \\
vetti yatra na caivāyaṁ sthitaś calati tattvataḥ}
{In that state one experiences a boundless joy which is known to the intellect, but unknowable to the senses. Established therein, such a \textit{yogī} does not waver from that reality.}

\Verse[6.22]
{yaṁ labdhvā cāparaṁ lābhaṁ manyate nādhikaṁ tataḥ \\
yasmin sthito na duḥkhena guruṇāpi vicālyate}
{Having attained that limitless happiness, one considers no other gain as superior and established therein, one is not shaken even by terrible suffering.}

\Verse[6.23]
{taṁ vidyād duḥkha-saṁyoga-viyogaṁ yoga-saṁjñitam \\
sa niścayena yoktavyo yogo ’nirviṇṇa-cetasā}
{That state of dissociation from the contact with suffering should be known as \textit{yoga}. This \textit{yoga} must be practiced with firm resolve and a mind free from despondency.}

\Verse[6.24]
{saṅkalpa-prabhavān kāmāṁs tyaktvā sarvān aśeṣataḥ \\
manasaivendriya-grāmaṁ viniyamya samantataḥ}
{Completely giving up all desires arising from the mind, 
one should restrain the senses from all sides through the mind alone.}

\Verse[6.25]
{śanaiḥ śanair uparamed buddhyā dhṛti-gṛhītayā \\
ātma-saṁsthaṁ manaḥ kṛtvā na kiñcid api cintayet}
{Having fixed the mind on the \textit{ātmā}, one should gradually stop the mind's fluctuations with a determined intellect and not think of anything else.}

\Verse[6.26]
{yato yato niścarati manaś cañcalam asthiram \\
tatas tato niyamyaitad ātmany eva vaśaṁ nayet}
{From wherever the fickle and unsteady mind wanders, one should withdraw it and bring it back under control, fixing it on the \textit{ātmā} alone.}

\Verse[6.27]
{praśānta-manasaṁ hy enaṁ yoginaṁ sukham uttamam \\
upaiti śānta-rajasaṁ brahma-bhūtam akalmaṣam}
{The supreme happiness comes to the \textit{yogī} whose mind is fully pacified, whose passions are subdued, who has realized Brahman, and is spotlessly pure.}

\Verse[6.28]
{yuñjann evaṁ sadātmānaṁ yogī vigata-kalmaṣaḥ \\
sukhena brahma-saṁsparśam atyantaṁ sukham aśnute}
{Thus freed from mental impurities and constantly centering the mind, the \textit{yogī} easily attains the endless bliss of union with Brahman.}

\Verse[6.29]
{sarva-bhūta-stham ātmānaṁ sarva-bhūtāni cātmani \\
īkṣate yoga-yuktātmā sarvatra sama-darśanaḥ}
{The one whose mind is absorbed in the Supreme through \textit{yoga} has equal vision at all times and thus sees the Paramātmā in all beings and all beings within the Paramātmā.}

\Verse[6.30]
{yo māṁ paśyati sarvatra sarvaṁ ca mayi paśyati \\
tasyāhaṁ na praṇaśyāmi sa ca me na praṇaśyati}
{For one who sees Me everywhere and everything in Me, I am never separated from them, and they are never separated from Me.}

\Verse[6.31]
{sarva-bhūta-sthitaṁ yo māṁ bhajaty ekatvam āsthitaḥ \\
sarvathā vartamāno ’pi sa yogī mayi vartate}
{One who is fixed in this vision of oneness and worships Me dwelling in all beings, rests in Me as a true \textit{yogī}, in whatever circumstances of life.}

\Verse[6.32]
{ātmaupamyena sarvatra samaṁ paśyati yo ’rjuna \\
sukhaṁ vā yadi vā duḥkhaṁ sa yogī paramo mataḥ}
{O Arjuna, one who sees everything equally, whether happiness or suffering, by comparison to his own \textit{ātmā}, is considered to be the highest \textit{yogī}.}

\enquote{Here Kṛṣṇa makes an analogy between a still mind and an unmoving flame: the mind is \enquote{motionless like the light of a lamp in a windless place.} A flame is always moving, but when it is sheltered, blocked from all sides, it doesn’t move, it doesn’t flicker; it is stable, it shines, it radiates light! Like that, when one is fully absorbed in the divine Self, God within oneself, nothing can deviate one from the path and automatically one starts to shine all the divine qualities.}

\enquote{When through deep meditation, one has perceived the true reality, \enquote{which is perceived by the intelligence and is beyond the senses}; when one has perceived and attained this bliss through meditation on God; when God has revealed this supreme aspect of Himself within oneself, \enquote{established therein, the soul can no longer fall away from the spiritual truth of its being.} In this state, one perceives this oneness and the everlasting, imperishable joy and bliss emanating from God within oneself. One is God-realized, one is a true \textit{yogī} and one is never disunited from God. Whatever one does in life, whether it is having children, eating, drinking, or sleeping, if one is centered in that state, nothing can touch that person. One stays ever free. That’s why you see that in ancient times, the sages were all \textit{karma-yogīs}, they were all married, even the \textit{gurus}, and their wives were supporting their work. All those greater mothers also attained the same state of God-realization as their husbands, due to their surrender, their service, their \textit{bhakti}, and the dedication that they had to their husbands.}

\enquote{Lord Kṛṣṇa says that the ones who have attained this supreme state, whatever they do in the outside world, they are completely free, even if it may appear that they are limited. One might ask, \enquote{If you are a \textit{yogī}, why do you need to sleep? If you are a \textit{yogī}, why do you need to eat?} But as they are established in the Self, they are beyond duality, they are beyond the judgment of right and wrong, they are beyond the judgment of being husband and wife, they are beyond the duality of man and woman, they are beyond heat and cold, they are beyond \enquote{good} and \enquote{bad} and other opposites.}

\enquote{Everything that you do, once you have attained bliss, is infused with that bliss\ldots When bliss awakens inside of you, it starts to consume you and it is forever. It will keep growing. There is not an end to that bliss\ldots All the saints of the Lord are absorbed in that bliss.}

\section{Verses 33-36: Arjuna's Doubt}\label{sec-verses-33-36-arjuna-s-doubt}

\Verse[6.33]
{\hspace*{1em}arjuna uvāca \\
yo ’yaṁ yogas tvayā proktaḥ sāmyena madhusūdana \\
etasyāhaṁ na paśyāmi cañcalatvāt sthitiṁ sthirām}
{\hspace*{1em}Arjuna said: \\
O Madhusūdana, I don't see this \textit{yoga} of equanimity taught by You as stable and unwavering, due to the mind being restless.}

\Verse[6.34]
{cañcalaṁ hi manaḥ kṛṣṇa pramāthi balavad dṛḍham \\
tasyāhaṁ nigrahaṁ manye vāyor iva suduṣkaram}
{The mind is exceedingly restless, turbulent, powerful, and stubborn. Controlling it I consider to be as difficult as controlling the wind.}

\Verse[6.35]
{\hspace*{1em}śrī-bhagavān uvāca \\
asaṁśayaṁ mahā-bāho mano durnigrahaṁ calam \\
abhyāsena tu kaunteya vairāgyeṇa ca gṛhyate}
{\hspace*{1em}Bhagavān Kṛṣṇa said: \\
O mighty-armed Arjuna, without doubt, the mind is unsteady and hard to subdue. But by repeated practice and detachment it can be brought under control, O Kaunteya.}

\Verse[6.36]
{asaṁyatātmanā yogo duṣprāpa iti me matiḥ \\
vaśyātmanā tu yatatā śakyo ’vāptum upāyataḥ}
{In My opinion, it is hard for a person with an unrestrained mind to attain this \textit{yoga}. However, it can be attained by one who strives for it with a subdued mind through proper means.}

\enquote{Arjuna’s question was how to gain control of this unsteady mind. Bhagavān has replied to him saying that it is possible to do so through constant practice of mental discipline and observance, to observe the mind, to observe how the mind is thinking, to keep the mind on the path of reflection\ldots that will free oneself from attachment. But He is saying that one must practice. Wherever you focus your mind, that mind takes a shape. It’s like liquid. It doesn’t have a form itself, but it takes the form of a receptacle. You pour water in a bottle, it takes the form of a bottle.}

\enquote{Such is this mind also. If you feed that mind with negativity, it will become negative. \enquote{If you are sitting here, O Arjuna, and telling Me, ‘No, no, I can’t do it, I can’t do it, I can’t do it,’ you will not be able to.} If you sit and say, \enquote{I can’t, I can’t, I can’t,} you are building a brick wall where you are closing yourself inside. You are blinding yourself, you are finding thousands of excuses for not doing your \textit{sādhana}, for not practicing your \textit{kriyā}. Here Bhagavān says, \enquote{Practice is very important. Don’t find excuses.} Stop saying, \enquote{I can’t, I can’t, I can’t}; sing, \enquote{I can, I can, I can} and, \enquote{I will, I will, I will.} And when you sit down in your meditation, have this pure conviction that the Almighty Lord Himself is with you.}

\enquote{Kṛṣṇa says that when your mind is jumping from one thing to the other, don’t run into your weakness, don’t run into your fear! Don’t think that your desire for God is too much or too far away. One should always be alert and not allow the mind to think about anything but God. He has given so many divine forms, He has given so many Divine Names to enable mankind to completely focus on Him. When you sit for meditation, if you perceive your mind going somewhere else, let it go, but then bring it back. Show your power over it through the intellect. Show that the Great Observer is more powerful than the mind. That’s why you can bring back your thought and control it. Remind yourself continuously, even if you don’t notice when the mind ran away and slipped out. When it slips out, don’t dwell on the object that has attracted the mind: instead bring the mind back and fix it \enquote{in the Self.} Fix it on God! God dwells in the identity of the soul as the Supersoul.}

\section{Verses 37-46: The Fate of the Failed Yogī}\label{sec-verses-37-46-the-fate-of-the-failed-yogi}

\Verse[6.37]
{\hspace*{1em}arjuna uvāca \\
ayatiḥ śraddhayopeto yogāc calita-mānasaḥ \\
aprāpya yoga-saṁsiddhiṁ kāṁ gatiṁ kṛṣṇa gacchati}
{\hspace*{1em}Arjuna said: \\
O Kṛṣṇa, what destination does one attain who has faith, but lacks diligent effort and whose mind deviates from the path, thus failing to attain the perfection of \textit{yoga}?}

\Verse[6.38]
{kaccin nobhaya-vibhraṣṭaś chinnābhram iva naśyati \\
apratiṣṭho mahā-bāho vimūḍho brahmaṇaḥ pathi}
{O mighty-armed  Lord, having fallen from both material success and spiritual realization, does one not perish like a broken cloud, deluded and without any firm standing on the path toward the Supreme?}

\Verse[6.39]
{etan me saṁśayaṁ kṛṣṇa chettum arhasy aśeṣataḥ \\
tvad-anyaḥ saṁśayasyāsya chettā na hy upapadyate}
{O Kṛṣṇa, please completely remove this doubt of mine since there is no one else other than You who can dispel it.}

\Verse[6.40]
{\hspace*{1em}śrī-bhagavān uvāca \\
pārtha naiveha nāmutra vināśas tasya vidyate \\
na hi kalyāṇa-kṛt kaścid durgatiṁ tāta gacchati}
{\hspace*{1em}Bhagavān Kṛṣṇa said: \\
My dear Pārtha, neither in this world nor in the next, is there destruction for such a person, because no one who does good ever attains misfortune.}

\Verse[6.41]
{prāpya puṇya-kṛtāṁ lokān uṣitvā śāśvatīḥ samāḥ \\
śucīnāṁ śrīmatāṁ gehe yoga-bhraṣṭo ’bhijāyate}
{Having attained the realms of the righteous and dwelt there for many long years, one who has fallen from \textit{yoga} is born in the house of the pure and prosperous.}

\Verse[6.42]
{atha vā yoginām eva kule bhavati dhīmatām \\
etad dhi durlabha-taraṁ loke janma yad īdṛśam}
{Otherwise, one takes birth in a family of truly wise \textit{yogīs}; certainly such a birth is exceedingly rare in this world.}

\Verse[6.43]
{tatra taṁ buddhi-saṁyogaṁ labhate paurva-dehikam \\
yatate ca tato bhūyaḥ saṁsiddhau kuru-nandana}
{In such a birth, one gains contact with the spiritual wisdom acquired in the previous body, and from there one endeavors again for perfection, O joy of the Kurus.}

\Verse[6.44]
{pūrvābhyāsena tenaiva hriyate hy avaśo ’pi saḥ \\
jijñāsur api yogasya śabda-brahmātivartate}
{Due to previous practice, one is drawn to that path, even unintentionally. Even one who merely inquires about \textit{yoga}, transcends the ritualistic principles of the \textit{Vedas}.}

\Verse[6.45]
{prayatnād yatamānas tu yogī saṁśuddha-kilbiṣaḥ \\
aneka-janma-saṁsiddhas tato yāti parāṁ gatim}
{But the \textit{yogi} who strives diligently and is cleansed from impurities, attains perfection after many lifetimes and thereafter reaches the supreme goal.}

\Verse[6.46]
{tapasvibhyo ’dhiko yogī jñānibhyo ’pi mato ’dhikaḥ \\
karmibhyaś cādhiko yogī tasmād yogī bhavārjuna}
{The \textit{yogī} is considered to be superior to those who perform austerity, superior to the \textit{jñānīs}, as well as superior to the ritualists; therefore, become a \textit{yogī}, O Arjuna.}

\enquote{If you don’t reach it in one life, it doesn’t get wasted, because spirituality is not tangible. Spirituality is not something which is made up of matter, but it is made up of faith inside of you. What is happening right now is a continuation from what you have started before. And if in this life you don’t attain it, it will still carry on in the next life.}

\enquote{Those who have faith in God, they are not lost. This is His assurance.}

\enquote{The surroundings create an effect on people. Even if one has grace, it can stay dormant inside one’s heart. Deep inside one has the feeling, \enquote{I am incomplete.} However, due to the environment, due to the surroundings, due to the mind’s activities, they only get drawn to the outside reality. They get tempted and controlled by the outside reality, and then they don’t have the grace to find their path in this lifetime. However, through the grace that they received, through the \textit{puṇya} that they accumulated, they are born again on this Earth in spiritual families which will contribute to their spiritual growth\ldots This means that the ones who try their best, never lose. They enjoy celestial regions for long years and are born again on Earth to continue on their spiritual advancement.}

\enquote{In this life, through \textit{sādhana}, through following the words of the master, the one who has faith in one’s practice, who has faith in God, transcends the enjoyment of this world, as well as the next one. And due to the fruits of past good \textit{puṇya}, one continues practicing and advancing on the path to Enlightenment.}

\section{Verse 47: The Highest of All Yogīs}\label{sec-verse-47-the-highest-of-all-yogis}
\Verse[6.47]
{yoginām api sarveṣāṁ mad-gatenāntar-ātmanā \\
śraddhāvān bhajate yo māṁ sa me yuktatamo mataḥ}
{Among all the \textit{yogīs}, one who has faith and worships Me with a mind fully immersed in Me, is considered to be the most perfectly united with Me.}

\enquote{Sometimes people take a spiritual path, they get initiated, they follow the master, and then they feel that that path is not their path. Then, they go away not knowing that God had placed them on that path for a reason\ldots Due to their difficulty in following what the master had asked from them, that grace that they received even just to be in the presence of the master will become dormant inside of them until they are truly ready. Then, they will be called back again. And if they die, they will be called again in their next life.}

\enquote{The soul sees the path that will lead one to the spiritual family---\textit{vasudhaiva-kuṭumbakaṁ}---this unification of one family\ldots A real family is one that shares with you, who stands with you, however you are, who supports you on your way. That’s real family.}
