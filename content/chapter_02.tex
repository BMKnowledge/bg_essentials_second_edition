\chapter{Sāṅkhya-yoga}\label{chap-sankhya-yoga}
\chaptersubtitle{The Eternal Nature of the Self}
%%%%%%%%%%%%%%%%%%%%%%%%%%%%%%%%%%%%%%%%%%%%%%%%%%%%%%%%%

\noindent Despite Kṛṣṇa urging Arjuna to shake off his weakness, he continues to express his unwillingness to fight. Eventually, in a helpless state, Arjuna surrenders to Kṛṣṇa and begs for guidance. At this point, Kṛṣṇa assumes the role of the \textit{satguru}\footnote{A \textit{guru} who can grant God-realization by mere will.} and teaches him about the nature of the Self---the \textit{ātmā}\footnote{The individual, eternal, conscious, and blissful Self, distinct from the body-mind complex and yet infusing life into it.} present in every individual. He describes how the \textit{ātmā} is indestructible, eternal, and beyond this material world. Just as we cast off old clothes and adorn ourselves with new ones, the soul sheds old bodies and takes on new ones. As a \textit{kṣatriya} (warrior), therefore, Arjuna should have no reservation in carrying out his duty since ultimately nobody can be killed.

Having established Arjuna in this philosophical understanding, Kṛṣṇa proceeds to teach one of the fundamental concepts of the \textit{Gītā}: \textit{karma-yoga}. He explains that those of small intelligence perform various duties with the hope of gaining some limited reward; this attachment to results is what produces \textit{karma}.\footnote{Any action done with the body or mind that produces further consequences.} But Kṛṣṇa urges Arjuna to carry out his duty as a warrior without any attachment to the results. By engaging in righteous action without any selfish gain, one can be freed from all \textit{karmic} consequences. With the senses restrained and the mind controlled, one is not distracted by the ways of the world. In this desire-less state, a \textit{yogī} achieves perfect wisdom and peace.

This overview of \textit{karma-yoga}\footnote{The path by which one continues to act in the world with detachment. Such acts, therefore, do not create further consequences.
} provides the centerpiece for Kṛṣṇa's teachings over the next four chapters. We learn about the real meaning of action, what is true renunciation, as well as the ultimate state that is reached when practicing this \textit{yoga}.

\section{Verses 1-9: Arjuna Seeks Refuge in Kṛṣṇa as the Guru}\label{sec-verses-1-9-arjuna-seeks-refuge-in-krsna-as-the-guru}

\Verse[2.1]
{\hspace*{1em}sañjaya uvāca \\
taṁ tathā kṛpayāviṣṭam aśru-pūrṇākulekṣaṇam \\
viṣīdantam idaṁ vākyam uvāca madhusūdanaḥ}
{\hspace*{1em}Sañjaya said: \\
To Arjuna, who was thus despondent and overwhelmed with pity, his eyes filled with confusion and tears, Madhusūdana spoke these words:}

\Verse[2.2]
{\hspace*{1em}śrī-bhagavān uvāca \\
kutas tvā kaśmalam idaṁ viṣame samupasthitam \\
anārya-juṣṭam asvargyam akīrti-karam arjuna}
{\hspace*{1em}Bhagavān Kṛṣṇa said: \\
O Arjuna, from where has this weakness come in this hour of danger? It is unworthy of noble persons and will not lead to the attainment of Heaven. Instead, it causes dishonor.}

\Verse[2.3]
{klaibyaṁ mā sma gamaḥ pārtha naitat tvayy upapadyate \\
kṣudraṁ hṛdaya-daurbalyaṁ tyaktvottiṣṭha parantapa}
{Do not give in to this cowardice, O Arjuna, it does not befit you. Throw away this lowly weakness of heart and arise, O vanquisher of enemies!}

\Verse[2.4]
{\hspace*{1em}arjuna uvāca \\
kathaṁ bhīṣmam ahaṁ saṅkhye droṇaṁ ca madhusūdana \\
iṣubhiḥ pratiyotsyāmi pūjārhāv ari-sūdana}
{\hspace*{1em}Arjuna said: \\
O Madhusūdana, destroyer of enemies, how can I fire arrows against Bhīṣma and Droṇa in battle, who are worthy of worship?}

\Verse[2.5]
{gurūn ahatvā hi mahānubhāvān \\
śreyo bhoktuṁ bhaikṣyam apīha loke \\
hatvārtha-kāmāṁs tu gurūn ihaiva \\
bhuñjīya bhogān rudhira-pradigdhān}
{It would be better to live in this world, even by begging, than to kill these noble, venerable elders. Our enjoyment of avaricious pleasures, after killing them here, would certainly be stained with blood.}

\Verse[2.6]
{na caitad vidmaḥ kataran no garīyo \\
yad vā jayema yadi vā no jayeyuḥ \\
yān eva hatvā na jijīviṣāmas \\
te ’vasthitāḥ pramukhe dhārtarāṣṭrāḥ}
{We do not know which of the two is better for us---defeating them or being conquered by them. Those sons of Dhṛtarāṣṭra after killing whom we would no longer wish to live, are verily standing here before us.}

\Verse[2.7]
{kārpaṇya-doṣopahata-svabhāvaḥ \\
pṛcchāmi tvāṁ dharma-sammūḍha-cetāḥ \\
yac chreyaḥ syān niścitaṁ brūhi tan me\\
śiṣyas te ’haṁ śādhi māṁ tvāṁ prapannam}
{With my nature overcome by the stain of weakness and my mind confused about my duty, I urge You to tell me clearly what would be better for me. I am Your disciple who has taken refuge in You. Please instruct me.}

\Verse[2.8]
{na hi prapaśyāmi mamāpanudyād \\
yac chokam ucchoṣaṇam indriyāṇām \\
avāpya bhūmāv asapatnam ṛddhaṁ \\
rājyaṁ surāṇām api cādhipatyam}
{Even the attainment of a prosperous, unchallenged kingdom on this Earth---or lordship over the gods---I cannot see what could remove this grief that dries up my senses.}

\Verse[2.9]
{\hspace*{1em}sañjaya uvāca \\
evam uktvā hṛṣīkeśaṁ guḍākeśaḥ parantapaḥ \\
na yotsya iti govindam uktvā tūṣṇīṁ babhūva ha}
{\hspace*{1em}Sañjaya said: \\
Having thus spoken to Hṛṣīkeśa, Arjuna, the conqueror of sleep and vanquisher of enemies, said to Govinda, \enquote{I will not fight} and became silent.}

\enquote{In the last chapter, we saw how Arjuna feels confused and how depressed he has become, because when the mind is focused and concentrated on the negative, everything becomes negative. We will see how Lord Kṛṣṇa gives Arjuna the knowledge of the \textit{Gītā} and how He starts to talk to Arjuna about the knowledge of the Self.}

\enquote{Kṛṣṇa is a great psychologist; He doesn’t directly go and give Arjuna everything. He goes slowly with him, step by step. He doesn’t change him, but He gradually allows Arjuna to transform himself.}

\enquote{In this chapter, Kṛṣṇa becomes the \textit{guru}. All this time, Arjuna has been looking at Kṛṣṇa as his cousin and his friend. But now, seeing this terrible, pitiful state of Arjuna, Kṛṣṇa starts to teach. At this moment, Kṛṣṇa becomes the \textit{guru} and Arjuna, the disciple. But Arjuna’s willingness to listen to Kṛṣṇa is very important!}

\enquote{How can Kṛṣṇa tell something to Arjuna, if Arjuna doesn’t want to listen? In the first chapter, you saw that Kṛṣṇa doesn’t say much: He keeps quiet and listens to what Arjuna has to say. But when Arjuna is tired of talking, then Kṛṣṇa reveals to him the \textit{Gītā}. Only when you can listen, can you change! If you can’t listen, how will you change? You will never change. Life is like this: if you don’t learn to listen and observe, you will not transform.}

\enquote{The instruction of the \textit{guru} is like a seed which is planted in fertile land. If the disciple’s heart is like a stone, nothing will grow from it.}

\enquote{When a disciple takes refuge in the \textit{guru}, the disciple must completely accept the superiority of the master. This is what Arjuna is accepting.}

\enquote{When you come to the master, you have to come as an empty vessel so that you can be filled. That mind must be emptied to receive. You have to have this willingness to change to have the full benefit.}

\enquote{If you start to think about what you are losing, you will never get anything, because on the spiritual path you may lose everything on the outside, but in reality, you are gaining everything.}

\enquote{Arjuna’s eyes are brightened and his mind is completely transformed. His whole attitude has changed. With folded hands, he knows that by taking this role of the disciple, that the Lord, the God Almighty, the omniscient, and the knower of all hearts, has taken the form of the supreme master. And that He, who is full of Love, greatness, virtue, and knowledge, He who is non-attached, who can’t be touched by \textit{karma} or by anything, is his dear friend. And from being his best friend, He has become the supreme guide and the supreme divinity.}

\enquote{This is the greatness of the \textit{guru}. The \textit{guru} does not have just one role. He is not just a teacher. He takes different forms, a multitude of aspects. Because the life of the \textit{guru} is not for Himself, but for others. The help and support, the knowledge, the power, and the affection that the \textit{guru} has for the devotee, is amazing and exquisite.}

\newpage
\section{Verses 10-30: The Eternal Nature of the Soul}\label{sec-verses-10-30-the-eternal-nature-of-the-soul}

\Verse[2.10]
{tam uvāca hṛṣīkeśaḥ prahasann iva bhārata \\
senayor ubhayor madhye viṣīdantam idaṁ vacaḥ}
{O descendant of Bharata, as he was grieving between the two armies, Hṛṣīkeśa, slightly smiling, spoke the following words to him.}

\Verse[2.11]
{\hspace*{1em}śrī-bhagavān uvāca \\
aśocyān anvaśocas tvaṁ prajñā-vādāṁś ca bhāṣase \\
gatāsūn agatāsūṁś ca nānuśocanti paṇḍitāḥ}
{\hspace*{1em}Bhagavān Kṛṣṇa said: \\
You are grieving for those who should not be grieved for, yet you speak words of wisdom. The truly wise ones lament neither for the dead nor for the living.}

\Verse[2.12]
{na tv evāhaṁ jātu nāsaṁ na tvaṁ neme janādhipāḥ \\
na caiva na bhaviṣyāmaḥ sarve vayam ataḥ para}
{Surely there has never been a time when I didn't exist, nor you, nor any of these kings, nor will there ever be a time when we all don't exist.}

\Verse[2.13]
{dehino ’smin yathā dehe kaumāraṁ yauvanaṁ jarā \\
tathā dehāntara-prāptir dhīras tatra na muhyati}
{Just as childhood, youth, and old age take place in the body, the embodied \textit{ātmā} attains another body at death. A wise person is not bewildered about this.}

\Verse[2.14]
{mātrā-sparśās tu kaunteya śītoṣṇa-sukha-duḥkha-dāḥ \\
āgamāpāyino ’nityās tāṁs titikṣasva bhārata}
{The contact of the senses with their objects, O Kaunteya, gives rise to feelings of cold and heat, pleasure and pain. They come and go and are impermanent, so endure them without being disturbed, O descendant of Bharata.}

\Verse[2.15]
{yaṁ hi na vyathayanty ete puruṣaṁ puruṣarṣabha \\
sama-duḥkha-sukhaṁ dhīraṁ so ’mṛtatvāya kalpate}
{O best of men, the wise person who is unaffected by these, and to whom pain and pleasure are the same, is indeed worthy of immortality.}

\Verse[2.16]
{nāsato vidyate bhāvo nābhāvo vidyate sataḥ \\
ubhayor api dṛṣṭo ’ntas tv anayos tattva-darśibhiḥ}
{There is no existence of the unreal, and there is no non-existence of the real. This is the conclusion regarding these two that is perceived by those who have realized the Truth.}

\Verse[2.17]
{avināśi tu tad viddhi yena sarvam idaṁ tatam \\
vināśam avyayasyāsya na kaścit kartum arhati}
{Know that which pervades this entire body to be indestructible. None can cause the destruction of the imperishable \textit{ātmā}.}

\Verse[2.18]
{antavanta ime dehā nityasyoktāḥ śarīriṇaḥ \\
anāśino ’prameyasya tasmād yudhyasva bhārata}
{The bodies of the eternal, imperishable, and incomprehensible \textit{ātmā} dwelling in the body are said to be perishable. Therefore fight, O descendant of Bharata.}

\Verse[2.19]
{ya enaṁ vetti hantāraṁ yaś cainaṁ manyate hatam \\
ubhau tau na vijānīto nāyaṁ hanti na hanyate}
{One who believes the \textit{ātmā} to be the killer, and one who thinks it can be killed---both do not know; for the \textit{ātmā} neither slays nor is it slain.}

\Verse[2.20]
{na jāyate mriyate vā kadācin \\
nāyaṁ bhūtvā bhavitā vā na bhūyaḥ \\
ajo nityaḥ śāśvato ’yaṁ purāṇo \\
na hanyate hanyamāne śarīre}
{The \textit{ātmā} is never born, nor does it ever die, nor is it the case that once having come to existence it will cease to exist in the future. It is unborn, eternal, everlasting and most ancient. It is not killed when the body is slain.}

\Verse[2.21]
{vedāvināśinaṁ nityaṁ ya enam ajam avyayam \\
kathaṁ sa puruṣaḥ pārtha kaṁ ghātayati hanti kam}
{O son of Pṛthā, if one knows this \textit{ātmā} to be indestructible, unborn, unchanging, and eternal, how and whom does that person kill or cause to be killed?}

\Verse[2.22]
{vāsāṁsi jīrṇāni yathā vihāya \\
navāni gṛhṇāti naro ’parāṇi \\
tathā śarīrāṇi vihāya jīrṇāny \\
anyāni saṁyāti navāni dehī}
{Just as a person who has discarded worn-out clothes puts on new ones, so the embodied \textit{ātmā}, having discarded its worn-out bodies, enters into new ones.}

\Verse[2.23]
{nainaṁ chindanti śastrāṇi nainaṁ dahati pāvakaḥ \\
na cainaṁ kledayanty āpo na śoṣayati mārutaḥ}
{Weapons do not cut the \textit{ātmā}; fire does not burn it, water does not wet it, and wind does not dry it.}

\Verse[2.24]
{acchedyo ’yam adāhyo ’yam akledyo ’śoṣya eva ca \\
nityaḥ sarva-gataḥ sthāṇur acalo ’yaṁ sanātanaḥ}
{It cannot be cut; it cannot be burnt; it cannot be wetted, and it cannot be dried. It is eternal, all-pervading, changeless, immovable, and everlasting.}

\Verse[2.25]
{avyakto ’yam acintyo ’yam avikāryo ’yam ucyate \\
tasmād evaṁ viditvainaṁ nānuśocitum arhasi}
{This \textit{ātmā} is said to be unmanifest, inconceivable, and unchanging. Knowing the \textit{ātmā} in this way, you should therefore no longer grieve.}

\Verse[2.26]
{atha cainaṁ nitya-jātaṁ nityaṁ vā manyase mṛtam \\
tathāpi tvaṁ mahā-bāho nainaṁ śocitum arhasi}
{Even if you consider this \textit{ātmā} to repeatedly go through birth and death, O Arjuna, even then you should not lament.}

\Verse[2.27]
{jātasya hi dhruvo mṛtyur dhruvaṁ janma mṛtasya ca \\
tasmād aparihārye ’rthe na tvaṁ śocitum arhasi}
{Indeed, death is certain for one who is born, and rebirth is certain for one who has died; therefore do not grieve over what is inevitable.}

\Verse[2.28]
{avyaktādīni bhūtāni vyakta-madhyāni bhārata \\
avyakta-nidhanāny eva tatra kā paridevanā}
{O descendant of Bharata! All beings are unmanifest in the beginning, manifest in the middle and unmanifest at death. So why lament over this?}

\Verse[2.29]
{āścarya-vat paśyati kaścid enam \\
āścarya-vad vadati tathaiva cānyaḥ \\
āścarya-vac cainam anyaḥ śṛṇoti \\
śrutvāpy enaṁ veda na caiva kaścit}
{One person sees the \textit{ātmā} as a wonder, likewise another speaks of it as a wonder; yet another hears of it as a wonder. But even after hearing about it, no one really knows it.}

\Verse[2.30]
{dehī nityam avadhyo ’yaṁ dehe sarvasya bhārata \\
tasmāt sarvāṇi bhūtāni na tvaṁ śocitum arhasi}
{The embodied \textit{ātmā} is in the bodies of all, O descendant of Bharata. It is eternal and indestructible, therefore you should not grieve for any living being.}

\enquote{Now is the beginning of the \textit{Gītā}. The real \textit{Gītā} starts with the smile of the Lord. All that came before was just a preparation for this.}

\enquote{When the \textit{guru} sees that the disciple is really ready, He has the same smile! The disciple can ask ten thousand times and the \textit{guru} will give whatever He has to give. But when the disciple is truly ready, even without asking, all will be given. This is the smile that Lord Kṛṣṇa gives to Arjuna. He says, \enquote{Ah! I see that now you are ready.}}

\enquote{The \textit{guru} will not tell you things which will elevate your ego or your pride. He will tell you things which you don’t want to hear. This is how He crushes the pride out of you; He removes that ignorance out of you\ldots He will show you the way. He will polish you, He will help you to control the mind, He will help you to conquer the senses and conquer the mind. He will help you to banish this unhappiness from you, so that automatically through discrimination, through reflection, through analysis, and the true knowledge of the Self, that true happiness, the permanent happiness, awakens inside of you.}

\enquote{If you want true wisdom, know that there is no birth and no death. You are the \textit{ātmā}, you are eternal. For the \textit{ātmā}, there is no beginning, there is no end. So why do you mourn? You have come here to attain a higher reality; you have come here to do a higher work than what you realize. Kṛṣṇa tells Arjuna that the wise never grieve in this way. The one who has the knowledge of the Self, the realized one, knows that the \textit{ātmā} is eternal and that there is no point in mourning or grieving. The wise person knows that God is the embodiment of true knowledge\ldots The wise person knows that it is only Him who is abiding in everything. He is the Self of all. He is the indestructible Absolute.}

\enquote{[Kṛṣṇa says,] \enquote{There was never a time when I did not exist, nor you! As I existed before everything, you were also with Me.} This is the relationship that we have with God. We are always with Him and He is always with us, whether you want it or not. You can be the greatest atheist and you can say God doesn’t exist; it doesn’t bother Him.}

\enquote{You have an identity. You will always have it. You have ever existed with Him. You are never separate from Him. We all existed before our birth, before we manifested here in these bodies, and we will carry on existing long after these bodies have disappeared. That \textit{ātmā} will always live even if everything gets destroyed.}

\enquote{The body goes through recycling, because it’s made of the five elements, and the five elements go back into nature. But the Self remains untouched, unchanged, immutable, indestructible and it is eternal. It is not created. If something is not created, that means it can’t die.}

\enquote{How you see the body is important. It doesn’t mean that now that you have heard that the body is decaying and deteriorating, one shouldn’t care about it. Then realization will not come to you, because this body is the vehicle to attain that realization.}

\enquote{Even the Lord Himself manifests on Earth in a body. But you should not forget about the \textit{ātmā} also. Nowadays people concentrate so much only on the external reality that they forget that they have a soul.}


\enquote{So, awake, my dear soul. Realize your Self. You are the \textit{ātmā}. Stop being in the sleeping state, because in a sleeping state you will not realize anything. You have to be awake to realize your Self.}


\section{Verses 31-38: The Duty of a Warrior}\label{sec-verses-31-38-the-duty-of-a-warrior}

\Verse[2.31]
{sva-dharmam api cāvekṣya na vikampitum arhasi \\
dharmyād dhi yuddhāc chreyo ’nyat kṣatriyasya na vidyate}
{Additionally, considering your personal duty, you should not hesitate, since for a \textit{kṣatriya}, there is nothing more auspicious than a righteous war.}

\Verse[2.32]
{yadṛcchayā copapannaṁ svarga-dvāram apāvṛtam \\
sukhinaḥ kṣatriyāḥ pārtha labhante yuddham īdṛśam}
{O Pārtha, the \textit{kṣatriyas} rejoice when they get such an opportunity for battle. A war that comes of its own accord opens the gates to Heaven.}

\Verse[2.33]
{atha cet tvam imaṁ dharmyaṁ saṅgrāmaṁ na kariṣyasi \\
tataḥ sva-dharmaṁ kīrtiṁ ca hitvā pāpam avāpsyasi}
{But if you do not fight this righteous war, thus turning away from personal duty and honor, you will incur sin.}

\Verse[2.34]
{akīrtiṁ cāpi bhūtāni kathayiṣyanti te ’vyayām \\
sambhāvitasya cākīrtir maraṇād atiricyate}
{Then people will forever speak of your disgrace, and for an honorable man, dishonor is worse than death.}

\Verse[2.35]
{bhayād raṇād uparataṁ maṁsyante tvāṁ mahā-rathāḥ \\
yeṣāṁ ca tvaṁ bahu-mato bhūtvā yāsyasi lāghavam}
{The great warriors will think that you have fled from the battlefield in fear and you will earn the disrespect of those who held you in high esteem.}

\Verse[2.36]
{avācya-vādāṁś ca bahūn vadiṣyanti tavāhitāḥ \\
nindantas tava sāmarthyaṁ tato duḥkha-taraṁ nu kim}
{Moreover, your enemies, ridiculing your ability, will use words which should never be uttered. Indeed what could be more painful than that?}

\Verse[2.37]
{hato vā prāpsyasi svargaṁ jitvā vā bhokṣyase mahīm \\
tasmād uttiṣṭha kaunteya yuddhāya kṛta-niścayaḥ}
{If slain, you shall gain Heaven; if victorious, you shall enjoy the Earth. Therefore, arise, O son of Kuntī, with firm resolve to fight.}

\Verse[2.38]
{sukha-duḥkhe same kṛtvā lābhālābhau jayājayau \\
tato yuddhāya yujyasva naivaṁ pāpam avāpsyasi}
{Considering pleasure and pain, gain and loss, victory and defeat to be the same, prepare yourself for battle. In this way, you will not incur sin.}

\enquote{Kṛṣṇa is saying everybody knows that they are not the body, everybody knows that they are not the mind, they are not the intellect, but yet, they still identify themselves with them. This is due to the lack of true knowledge, lack of true understanding. You have received this knowledge through books; reading wonderful books has given you this insight. You have heard many talks about it, but it has just remained as words which are pleasurable to you. It doesn’t have any effect upon you. You are still running after the external pleasures; you are still running after things that are perishable.}

\enquote{If you are looking at the soul only from the point of view of the mind, do you think that the mind can understand the soul? A mind which is always moving around is \enquote{\textit{cañcala},} \enquote{restless,} dancing and jumping like a monkey, from one thought to the other. How can the mind, which is in constant movement, be still and perceive the unmovable? The mind will always move from one thing to another until one has gone deeply into the \textit{sādhana} (spiritual practice), deeply into meditation and calmed the mind. Only then, will the soul reveal itself. But before that, if the mind is still jumping around, the soul is unknowable.}

\enquote{\enquote{Considering pleasure and pain, gain and loss, victory and defeat to be the same}: Remind yourself that this is all just a game. Take this pair of opposites as being equal. Even if they appear different, they are the same. A wise man should not be completely taken over by joy when something is happening, but just stay calm. And those who are wise don’t complain when there is some trouble. They analyse the situation. They are in a balanced state in both situations: in a happy moment or in an unhappy moment.}

\enquote{When you do your duty in life, accepting what God is giving you to do, it draws you closer and closer to God-realization. By accepting what God is giving you in your daily life, you are opening the gates to receive more. By showing that you are doing your duty by accepting what God gives you in life, you are showing God that you are ready for a greater purpose in this life, for greater work. Kṛṣṇa says, \enquote{If you run away from this, it will not bring you any good. It will hurt you. It will be unrighteous. Especially on such an occasion, where doing this duty will help you open all the doors to Heaven.} He says, \enquote{Accept life. Accept whatever God gives you.}}

\section{Verses 39-53: The Introduction of Karma-yoga}\label{sec-verses-39-53-the-introduction-of-karma-yoga}

\Verse[2.39]
{eṣā te ’bhihitā sāṅkhye buddhir yoge tv imāṁ śṛṇu \\
buddhyā yukto yayā pārtha karma-bandhaṁ prahāsyasi}
{The knowledge which I have taught to you so far is based on Sāṅkhya.\footnote{A philosophy which states that there are two realities: the Puruṣa (Self) and \textit{prakṛti} (matter). Through an analysis of \textit{prakṛti}, one can realize oneself as the Puruṣa, which is utterly distinct from material nature.} Now listen to the teaching concerning \textit{karma-yoga}. O Pārtha, by being endowed with this wisdom, you will escape the bondage of \textit{karma}.}

\Verse[2.40]
{nehābhikrama-nāśo ’sti pratyavāyo na vidyate \\
svalpam apy asya dharmasya trāyate mahato bhayāt}
{On this path, there is no loss of invested efforts nor is there any failure. Even a little practice of this \textit{yoga} saves one from great fear.}

\Verse[2.41]
{vyavasāyātmikā buddhir ekeha kuru-nandana \\
bahu-śākhā hy anantāś ca buddhayo ’vyavasāyinām}
{On this path the resolute intellect is one-pointed, O descendant of Kuru; but the thoughts of the irresolute are many-branched and endless.}

\Verse[2.42–43]
{yām imāṁ puṣpitāṁ vācaṁ pravadanty avipaścitaḥ \\
veda-vāda-ratāḥ pārtha nānyad astīti vādinaḥ \\
kāmātmānaḥ svarga-parā janma-karma-phala-pradām \\
kriyā-viśeṣa-bahulāṁ bhogaiśvarya-gatiṁ prati}
{O Pārtha, those who are unwise delight in the words of the \textit{Vedas}, declaring \enquote{there is nothing superior to this!} Full of desire and with heaven as their goal, such people utter flowery hymns about special rites to gain power and pleasure, which lead to further birth and \textit{karma}.}

\Verse[2.44]
{bhogaiśvarya-prasaktānāṁ tayāpahṛta-cetasām \\
vyavasāyātmikā buddhiḥ samādhau na vidhīyate}
{The intellect of those who cling to pleasure and power and whose minds have been carried away by such flowery speech, cannot be firmly established in single-pointed focus.}

\Verse[2.45]
{trai-guṇya-viṣayā vedā nistrai-guṇyo bhavārjuna \\
nirdvandvo nitya-sattva-stho niryoga-kṣema ātmavān}
{The \textit{Vedas}\footnote{The earliest body of Indian scripture, consisting of the \textit{Ṛg Veda}, \textit{Sāma Veda}, \textit{Yajur Veda}, and \textit{Atharva Veda}, which lay down the fundamental principles of Hinduism.} deal with the three \textit{guṇas},\footnote{The three qualities within \textit{prakṛti} (matter): \textit{sattva} (purity), \textit{rajas} (activity), and \textit{tamas} (darkness). The different \textit{guṇas} work together to shape and influence the experience we have in the material world.} O Arjuna. You must become free from these \textit{guṇas} and the pairs of opposites. Be ever established in \textit{sattva-guṇa},\footnote{Quality of gentility, good conduct, enlightened understanding, purity, and detachment.} free from concerns of gain and protection---instead, be established in the \textit{ātmā}.}

\Verse[2.46]
{yāvān artha udapāne sarvataḥ samplutodake \\
tāvān sarveṣu vedeṣu brāhmaṇasya vijānataḥ}
{As much use there is for a well when there is water everywhere, that much use there is in all the \textit{Vedas} for a realized \textit{brāhmaṇa}.}

\Verse[2.47]
{karmaṇy evādhikāras te mā phaleṣu kadācana \\
mā karma-phala-hetur bhūr mā te saṅgo ’stv akarmaṇi}
{You have only the right to act, but never to the fruits of action. Do not consider yourself the cause of actions' outcomes, nor let there be any attachment to avoiding action.}

\Verse[2.48]
{yoga-sthaḥ kuru karmāṇi saṅgaṁ tyaktvā dhanañjaya \\
siddhy-asiddhyoḥ samo bhūtvā samatvaṁ yoga ucyate}
{O Dhanañjaya, established in \textit{yoga} and abandoning attachment, perform action, remaining equal in success and failure. This equanimity is called \textit{yoga}.}

\Verse[2.49]
{dūreṇa hy avaraṁ karma buddhi-yogād dhanañjaya \\
buddhau śaraṇam anviccha kṛpaṇāḥ phala-hetavaḥ}
{O Dhanañjaya, action done with attachment is greatly inferior to this \textit{yoga} of wisdom, so seek refuge in that wisdom. The petty-minded are motivated by the fruits of action.}

\Verse[2.50]
{buddhi-yukto jahātīha ubhe sukṛta-duṣkṛte \\
tasmād yogāya yujyasva yogaḥ karmasu kauśalam}
{United with such wisdom, one rises above both righteous and unrighteous deeds while living. Therefore, engage yourself in this \textit{yoga}. This \textit{yoga} is expertise in action.}

\Verse[2.51]
{karma-jaṁ buddhi-yuktā hi phalaṁ tyaktvā manīṣiṇaḥ \\
janma-bandha-vinirmuktāḥ padaṁ gacchanty anāmayam}
{The sages who are established in this wisdom renounce the fruits of action and attain the state beyond all suffering, and are freed from the bondage of rebirth.}

\Verse[2.52]
{yadā te moha-kalilaṁ buddhir vyatitariṣyati \\
tadā gantāsi nirvedaṁ śrotavyasya śrutasya ca}
{When your intellect crosses beyond the forest of delusion, you will be indifferent to what you have heard and what you will hear in the future.}

\Verse[2.53]
{śruti-vipratipannā te yadā sthāsyati niścalā \\
samādhāv acalā buddhis tadā yogam avāpsyasi}
{When your intellect, confused by the words of the \textit{Vedas} remains unshaken in single-mindedness, then you will have attained the state of \textit{yoga}.}

\enquote{Here Kṛṣṇa is saying that you should do your work, but you should not attach to the fruit of the work. You have your duty. You have to work. But your work will bear fruits when you do it out of love, when you love what God has given you and happily do it knowing that you are serving Him. This is where work becomes worship. When you think of Him throughout the time you are working, then everything will be perfect, because it doesn’t come from you. It comes from Him. It is Him who is working through you.}

\enquote{Realized souls are not attached to anything. Even if they do an action which is inappropriate in the minds of society, they are free from every \textit{karma} related to it. Their ordinary speech, even their modes of sitting and walking, have a special characteristic: it is all through the vibration, and it is all for the sake of others, not for their own sake.}

\enquote{When you are attached, you become ineffective, because when you are only focused on trying to do things properly, you are not free! You don’t have any freedom inside of you, because you are tense. Kṛṣṇa says, \enquote{If you do your action with the aim of serving the Lord, if you renounce this attachment to the result of the action, you will attain God-realization. You can’t renounce your actions, but do your actions with a selfless motive. Do them without egoism.} If you try to renounce your actions by force, you will not succeed.}

\enquote{Take success and failure in the same way. Purify the mind. Only when the mind is purified, only when the mind is calm, only when the mind is focused and centered on the Supreme Lord, then you are free. If the mind is wandering in all directions, nothing good can come out of it. You are never sure, \enquote{Am I right or not?}\ldots You have hundreds of questions that arise inside of your mind. This is an uncontrolled mind.}

\enquote{When you do your \textit{sādhana}, first, try to calm your mind. What awakens is the first instinct, the first feeling, the first wave which awakens inside of you. That's what you have to learn to listen to, because it comes from your consciousness, it comes from your higher Self. If you can distinguish between the mind and your consciousness, you will always take the right decision. The mind always runs towards the pairs of opposites. It always wants to justify itself. So, control that mind. A controlled mind, which is not running left-right in all directions, will give you balance and equilibrium in life.}

\section{Verses 54-72: The Qualities of a Realized Person}\label{sec-verses-54-72-the-qualities-of-a-realized-person}

\Verse[2.54]
{\hspace*{1em}arjuna uvāca \\
sthita-prajñasya kā bhāṣā samādhi-sthasya keśava \\
sthita-dhīḥ kiṁ prabhāṣeta kim āsīta vrajeta kim}
{\hspace*{1em}Arjuna said: \\
O Keśava, what is the description of one whose intellect is fixed and who is seated in transcendence? How does such a person speak? How does he sit and how does he move?}

\Verse[2.55]
{\hspace*{1em}śrī-bhagavān uvāca \\
prajahāti yadā kāmān sarvān pārtha mano-gatān \\
ātmany evātmanā tuṣṭaḥ sthita-prajñas tadocyate}
{\hspace*{1em}The Lord said: \\
When one gives up all desires arising in the mind, O Pārtha, and is satisfied within oneself by the Self alone, then one is said to be of steady wisdom.}

\Verse[2.56]
{duḥkheṣv anudvigna-manāḥ sukheṣu vigata-spṛhaḥ \\
vīta-rāga-bhaya-krodhaḥ sthita-dhīr munir ucyate}
{One whose mind is not perturbed by suffering, who does not crave after happiness, who is free from desire, fear and anger---such a person is called a sage of steady intellect.}

\Verse[2.57]
{yaḥ sarvatrānabhisnehas tat tat prāpya śubhāśubham \\
nābhinandati na dveṣṭi tasya prajñā pratiṣṭhitā}
{He who has no attachment anywhere, who feels neither attraction nor aversion when encountering the agreeable or the disagreeable---his wisdom is steady.}

\Verse[2.58]
{yadā saṁharate cāyaṁ kūrmo ’ṅgānīva sarvaśaḥ \\
indriyāṇīndriyārthebhyas tasya prajñā pratiṣṭhitā}
{When one completely withdraws the senses from the sense objects, just as a tortoise retracts its limbs, then one's wisdom is firmly established.}

\Verse[2.59]
{viṣayā vinivartante nirāhārasya dehinaḥ \\
rasa-varjaṁ raso ’py asya paraṁ dṛṣṭvā nivartate}
{The sense objects slacken their pull for someone who abstains from them, but their taste remains. Once having seen the Supreme, even that taste vanishes.}

\Verse[2.60]
{yatato hy api kaunteya puruṣasya vipaścitaḥ \\
indriyāṇi pramāthīni haranti prasabhaṁ manaḥ}
{O Kaunteya! The turbulent senses forcefully carry away the mind of even a wise person who is striving to control them.}

\Verse[2.61]
{tāni sarvāṇi saṁyamya yukta āsīta mat-paraḥ \\
vaśe hi yasyendriyāṇi tasya prajñā pratiṣṭhitā}
{Having controlled all the senses, one should abide in the state of meditation, focusing on Me; for one who has controlled his senses, wisdom is firmly established.}

\Verse[2.62]
{dhyāyato viṣayān puṁsaḥ saṅgas teṣūpajāyate \\
saṅgāt sañjāyate kāmaḥ kāmāt krodho ’bhijāyate}
{When deliberating upon sense-objects, a person develops an attachment to them; from attachment ensues desire, from desire arises anger.}

\Verse[2.63]
{krodhād bhavati sammohaḥ sammohāt smṛti-vibhramaḥ \\
smṛti-bhraṁśād buddhi-nāśo buddhi-nāśāt praṇaśyati}
{From anger arises delusion; from delusion, there is loss of memory; from loss of memory the destruction of discrimination occurs; and with the destruction of discrimination, one is lost.}

\Verse[2.64]
{rāga-dveṣa-vimuktais tu viṣayān indriyaiś caran \\
ātma-vaśyair vidheyātmā prasādam adhigacchati}
{But one who is self-controlled, who interacts with sense objects with senses that are free from attraction and aversion and under the control of the Self, attains tranquility.}

\Verse[2.65]
{prasāde sarva-duḥkhānāṁ hānir asyopajāyate \\
prasanna-cetaso hy āśu buddhiḥ paryavatiṣṭhate}
{In this state of tranquility, all his sorrows are overcome. Certainly, the intellect of a person with a serene mind soon becomes firmly established in Me.}

\Verse[2.66]
{nāsti buddhir ayuktasya na cāyuktasya bhāvanā \\
na cābhāvayataḥ śāntir aśāntasya kutaḥ sukham}
{For one who is not self-controlled, there is no discernment nor proper awareness. For one devoid of such practice, there can be no peace, and without peace, how can there be happiness?}

\Verse[2.67]
{indriyāṇāṁ hi caratāṁ yan mano ’nuvidhīyate \\
tad asya harati prajñāṁ vāyur nāvam ivāmbhasi}
{Indeed, when the mind follows the wandering senses, it carries away one’s intelligence, just as the wind drives a ship across the water.}

\Verse[2.68]
{tasmād yasya mahā-bāho nigṛhītāni sarvaśaḥ \\
indriyāṇīndriyārthebhyas tasya prajñā pratiṣṭhitā}
{Therefore, O mighty-armed Arjuna, one whose senses are entirely restrained from their objects, his wisdom is firmly established.}

\Verse[2.69]
{yā niśā sarva-bhūtānāṁ tasyāṁ jāgarti saṁyamī \\
yasyāṁ jāgrati bhūtāni sā niśā paśyato muneḥ}
{What is night to all beings, to that the self-controlled is awake, and that to which all beings are awake, is night for the enlightened one.}

\Verse[2.70]
{āpūryamāṇam acala-pratiṣṭhaṁ samudram āpaḥ praviśanti yadvat \\
tadvat kāmā yaṁ praviśanti sarve sa śāntim āpnoti na kāma-kāmī}
{Just as rivers enter into the sea, which remains full, steady and immovable, so the person into whom all desires flow attains peace, not the one who craves for pleasures.}

\Verse[2.71]
{vihāya kāmān yaḥ sarvān pumāṁś carati niḥspṛhaḥ \\
nirmamo nirahaṅkāraḥ sa śāntim adhigacchati}
{Having abandoned all desires, the person who lives without any craving, and does not have any notion of \enquote{I} and \enquote{mine,} attains real peace.}

\Verse[2.72]
{eṣā brāhmī sthitiḥ pārtha naināṁ prāpya vimuhyati \\
sthitvāsyām anta-kāle ’pi brahma-nirvāṇam ṛcchati}
{This is the realized state, O Pārtha, having attained which one is no longer deluded. By abiding in this state even at the hour of death, one attains the bliss of Brahman.}

\enquote{Don’t run behind the sense objects. Long for something permanent. Long for God’s grace. Don’t long for wealth, external comforts, the desire for name, fame, glory, pride, ego, and vanity. All this will not make you free. It will make you more miserable, because it will keep you away from realizing your Self. All this will awaken the egoistic self-importance, this big \enquote{I-ness} inside of you.}

\enquote{This big \enquote{I} will bring lots of expectations into your life. It will awaken lots of desires for many things.}

\enquote{If you are doing \textit{yoga}, if you are doing spiritual things, humility has to be there, humbleness must be there. If there is no humility, you will never have peace.}

\enquote{The whole problem starts when you start thinking. When you start thinking, you start comparing. When you start comparing, you lose everything. There is no peace of mind. And when there is no peace of mind, everything is destroyed.}

\enquote{Whatever you have, be happy. God always gives you what you need. He also knows best when is the right time to give you what you need. If you learn to accept this, no matter what God will give you, it will be for your spiritual growth. Even all these desires will not have any effect on the person who is fully centered in the Divine, whose aim is clear. Nothing will affect one’s mental state. If one is peaceful, there will be nothing to worry about. One knows that all is the will of God. One knows that all comes from God. It is He who gives, and it is He who takes.}

\enquote{When you are rebelling against somebody else, you are not rebelling against anybody except yourself. It’s just because you can’t see and accept your own fault that you are projecting it on people around you. That’s what Kṛṣṇa is saying: from desire, anger comes forth and explodes! But when your mind is controlled, then you don’t disturb anybody. You bring peace for the people around you also.}

\enquote{Those who have realized God, they reflect a different light. Whatever these great souls do is not for themselves\ldots Whatever they do, they do in accordance with the will of God. They don’t do it for their own pleasures, they don’t do it for their own attachment or their own wants. Even if it appears that they are not doing anything, even if they are piling up a mountain of sand, they are always doing something for the benefit of the world\ldots The same thing that they have inside of them, they give to others.}

\enquote{The \textit{Bhagavad Gītā} is not just literature. These are not just nice words that Kṛṣṇa wanted to say to Arjuna to prove that he has to fight. In reality, the \textit{Bhagavad Gītā} is life itself. It is your life from the moment you start to search.}

\enquote{When Kṛṣṇa is talking to Arjuna, He is not only talking to Arjuna, He is talking to everyone about how to transcend from just being a human being, and how to have this connection with Him, to build this relationship with Him. That’s what \textit{yoga} stands for. In the \textit{Bhagavad Gītā} we talk about \textit{yoga}, but \textit{yoga} is this eternal relationship which we have with God. The world, everything that you see around, is trying to make you forget about this relationship, saying, \enquote{No, you are not this! This is all just a fantasy.} Does the world teach you about your soul? No, it doesn’t. Does the world teach you how to live your life? No, it doesn’t. But the world wants to control you. Everywhere you see people wanting to control you. Society is like that.}



\enquote{So, long for the grace of the Lord, long for Him; don’t let yourself be put under the spell of \textit{māyā}.\footnote{The influence of the material world that draws us away from our true relationship with God.} Because \textit{māyā} doesn’t have any grip upon those who are centered in the Lord\ldots they are free. They go everywhere. They radiate their peace, freedom, and true happiness freely to all, without expectation. They become a lantern to those seated in darkness. They become a guide to those who sit in darkness so that they can also start shining their light. This is the duty of a devotee, the duty of a \textit{bhakta}:\footnote{One who is devoted to God.} once they have tasted that sweetness of the Love of God, and once they are strong enough, they should help others find that light.}
