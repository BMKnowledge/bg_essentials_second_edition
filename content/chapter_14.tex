\chapter{Guṇa-traya-vibhāga-yoga}\label{chap-guna-traya-vibhaga-yoga}
\chaptersubtitle{The Three Modes of Nature}
%%%%%%%%%%%%%%%%%%%%%%%%%%%%%%%%%%%%%%%%%%%%%%%%%%%%%%%%%

\noindent In the opening verses, Kṛṣṇa states that He is the one who gives life to the unconscious \textit{prakṛti} (material world) which then gives birth to all things. Out of this \textit{prakṛti} comes the different \textit{guṇas} (qualities) known as \textit{sattva}, \textit{rajas} and \textit{tamas}. These \textit{guṇas} bind the Self to this material body.

\textit{Sattva} is goodness which brings wisdom, purity, detachment, and kindness. \textit{Rajas} is passion that encourages activity, restlessness, and attachment. \textit{Tamas} is ignorance that brings negligence, lethargy, and sleep. At various times and situations, one \textit{guṇa} will predominate over the others, influencing our state and experience of life.

Kṛṣṇa tells Arjuna that those who exist in \textit{sattva} at death will be in true knowledge and attain higher realms. Those in \textit{rajas} will come back to a life of action in this world, and those who dwell in \textit{tamas} will be dragged down to be born among the deluded. But real eternal freedom is gained by one who rises beyond all three of the \textit{guṇas}. Such a person is not swayed by this material world and is unaffected by pain, pleasure, joy, and sorrow. In conclusion, Kṛṣṇa reminds Arjuna that it is through \textit{bhakti} that one can cross over the bondage of the \textit{guṇas} to realize the supreme state.

\section{Verses 1-4: Kṛṣṇa Gives Life to This Material World}\label{sec-verses-1-4-krsna-gives-life-to-this-material-world}


\Verse[14.1]
{\hspace*{1em}śrī-bhagavān uvāca \\
paraṁ bhūyaḥ pravakṣyāmi jñānānāṁ jñānam uttamam \\
yaj jñātvā munayaḥ sarve parāṁ siddhim ito gatāḥ}
{\hspace*{1em}Bhagavān Kṛṣṇa said: \\
I shall once again speak about the best and highest wisdom among all forms of knowledge, having realized which, all sages have attained the supreme perfection beyond this world.}

\Verse[14.2]
{idaṁ jñānam upāśritya mama sādharmyam āgatāḥ \\
sarge ’pi nopajāyante pralaye na vyathanti ca}
{Taking refuge in this wisdom, they attain the same nature as Me. As a result, they are not born at the time of creation and remain undisturbed even at the time of dissolution.}

\Verse[14.3]
{mama yonir mahad brahma tasmin garbhaṁ dadhāmy aham \\
sambhavaḥ sarva-bhūtānāṁ tato bhavati bhārata}
{My womb is the totality of material existence in which I place the seed from which arises the origin of all beings, O descendant of Bharata.}

\Verse[14.4]
{sarva-yoniṣu kaunteya mūrtayaḥ sambhavanti yāḥ \\
tāsāṁ brahma mahad yonir ahaṁ bīja-pradaḥ pitā}
{O Kaunteya, whatever forms are created in various wombs, material nature is their womb and I am the seed-giving father.}

\enquote{Bhagavān says that all creation is His womb, and is where He places the seed. He places each spirit according to their qualities, their \textit{dharma}, and certain things that they have to do; like seeds that sprout into various forms.}

\enquote{He also says that your spirit is free, it is never touched by these qualities. But \textit{karma}---the qualities---that roams around the spirit determines the form of creation. If they are good qualities, one is born as a human being, with good nature and manners; you are brought on your spiritual path, finding your spiritual way.}

\enquote{When the \textit{ātmā} starts to manifest into this world, it utilizes matter---\textit{prakṛti}---to manifest. From the subtle state it manifests into the gross state, which implies all forms, including not only all the species in this world, but also of the different spheres in the celestial worlds. He is also referring to the \textit{devas}, like Indra, and to the Rākṣasas, the demons, the lower beings, spirits, and ghosts, which symbolize the \textit{tamasic} lower qualities.}

\enquote{It’s through \textit{prakṛti} that His will is manifested, but He’s ever-free from it. Everything and everybody is under His influence and supervision. Wherever He has placed each one and however He is guiding each one, is perfect. And one has to learn to accept this.}

\section{Verses 5-18: How the Guṇas Influence Our Experience}\label{sec-verses-5-18-how-the-gunas-influence-our-experience}

\Verse[14.5]
{sattvaṁ rajas tama iti guṇāḥ prakṛti-sambhavāḥ \\
nibadhnanti mahā-bāho dehe dehinam avyayam}
{\textit{Sattva}, \textit{rajas}, and \textit{tamas}---these \textit{guṇas} arise from \textit{prakṛti} and bind the immutable embodied Self dwelling in the body, O mighty-armed Arjuna.}

\Verse[14.6]
{tatra sattvaṁ nirmalatvāt prakāśakam anāmayam \\
sukha-saṅgena badhnāti jñāna-saṅgena cānagha}
{Of these, \textit{sattva} is illuminating since it is pure and free from contamination. O sinless one, it binds through attachment to happiness and knowledge.}

\Verse[14.7]
{rajo rāgātmakaṁ viddhi tṛṣṇā-saṅga-samudbhavam \\
tan nibadhnāti kaunteya karma-saṅgena dehinam}
{Know that \textit{rajas} has the nature of passion and is the origin of intense craving and attachment. It binds the embodied Self through attachment to action, O Kaunteya.}

\Verse[14.8]
{tamas tv ajñāna-jaṁ viddhi mohanaṁ sarva-dehinām \\
pramādālasya-nidrābhis tan nibadhnāti bhārata}
{Know that \textit{tamas} gives rise to ignorance and deludes all embodied beings. It binds through negligence, laziness, and sleep, O descendant of Bharata.}

\Verse[14.9]
{sattvaṁ sukhe sañjayati rajaḥ karmaṇi bhārata \\
jñānam āvṛtya tu tamaḥ pramāde sañjayaty uta}
{\textit{Sattva} binds one to happiness, and \textit{rajas} to action, O descendant of Bharata. But \textit{tamas}, obscuring knowledge, binds to negligence.}

\Verse[14.10]
{rajas tamaś cābhibhūya sattvaṁ bhavati bhārata \\
rajaḥ sattvaṁ tamaś caiva tamaḥ sattvaṁ rajas tathā}
{O descendant of Bharata! Overpowering \textit{rajas} and \textit{tamas}, \textit{sattva} prevails, subduing \textit{sattva} and \textit{tamas}, \textit{rajas} prevails, and likewise, subduing \textit{sattva} and \textit{rajas}, \textit{tamas} prevails}

\Verse[14.11]
{sarva-dvāreṣu dehe ’smin prakāśa upajāyate \\
jñānaṁ yadā tadā vidyād vivṛddhaṁ sattvam ity uta}
{When illuminating wisdom arises in all the sense organs of this body, then one should know that \textit{sattva} is predominant.}

\Verse[14.12]
{lobhaḥ pravṛttir ārambhaḥ karmaṇām aśamaḥ spṛhā \\
rajasy etāni jāyante vivṛddhe bharatarṣabha}
{Greed, intense activity, initiation of various different tasks, restlessness, and craving---these arise, when \textit{rajas} is dominant, O best of the Bharatas.}

\Verse[14.13]
{aprakāśo ’pravṛttiś ca pramādo moha eva ca \\
tamasy etāni jāyante vivṛddhe kuru-nandana}
{Darkness, inertia, negligence, and delusion---these arise when \textit{tamas} dominates, O descendant of Kuru.}

\Verse[14.14]
{yadā sattve pravṛddhe tu pralayaṁ yāti deha-bhṛt \\
tadottama-vidāṁ lokān amalān pratipadyate}
{If the embodied Self meets death while \textit{sattva} predominates, then it attains the pure realms of those who know the Supreme.}

\Verse[14.15]
{rajasi pralayaṁ gatvā karma-saṅgiṣu jāyate \\
tathā pralīnas tamasi mūḍha-yoniṣu jāyate}
{Meeting death when \textit{rajas} predominates, one is born among those attached to action. Similarly, one who dies when \textit{tamas} prevails is born among beings in delusion.}

\Verse[14.16]
{karmaṇaḥ sukṛtasyāhuḥ sāttvikaṁ nirmalaṁ phalam \\
rajasas tu phalaṁ duḥkham ajñānaṁ tamasaḥ phalam}
{The result of virtuous action is said to be \textit{sattvic} and pure, but the result of \textit{rajasic} action is suffering, and the result of \textit{tamasic} action is ignorance.}

\Verse[14.17]
{sattvāt sañjāyate jñānaṁ rajaso lobha eva ca \\
pramāda-mohau tamaso bhavato ’jñānam eva ca}
{From \textit{sattva} arises knowledge, from \textit{rajas} comes greed, and from \textit{tamas} arise negligence, delusion, and ignorance.}

\Verse[14.18]
{ūrdhvaṁ gacchanti sattva-sthā madhye tiṣṭhanti rājasāḥ \\
jaghanya-guṇa-vṛtti-sthā adho gacchanti tāmasāḥ}
{Those established in \textit{sattva} rise upwards; those in \textit{rajas} remain in the middle; and those in \textit{tamas}, whose character is established in the vilest qualities, go downwards.}

\enquote{Bhagavān is describing here the qualities of the three \textit{guṇas}: how they manifest and what qualities one perceives. This is not about judging others. It’s not about seeing these qualities in other people. Be careful! These qualities are not in other people; they are in you, in each individual. You have this human body due to these three \textit{guṇas}. Bhagavān has previously said that this world is created due to these three \textit{guṇas}. And these three \textit{guṇas} include also your physical body and the mind. It’s not about pointing a finger at others. It’s about looking at yourself---which \textit{guṇa} predominates inside of you. This is also not about judging yourself. Of course, if you are fully \textit{tamasic} you will start judging. But be aware what you have to change and what you have to transcend to purify yourself.}

\textbf{Qualities of\textit{ sattva-guṇa}:}\enquote{The \textit{sattvic guṇa} is a touch of the divine positive nature. This \textit{guṇa} guides one in doing all good actions\ldots one is not attached to power, name, fame, glamour, or wealth. Everything becomes balanced. The mind, the mental state, is always at peace, even if everything else is busy, one’s mind is calm and serene.}

\enquote{Someone who has \textit{sattvic} qualities focuses their enjoyment on the Lord. Whatever one does is to achieve God-realization. Everything they do is with purity. Everything is bound by a great knowledge of the Lord, a great knowledge of the Self, and someone who has these \textit{sattvic} qualities is happy; they are in bliss. They are not worried about how and what. They are not worried about what they have. They know that the Lord will take care of them and they long for their spiritual advancement.}

\textbf{Qualities of \textit{rajo-guṇa}:} \enquote{For one who has \textit{rajasic} qualities, the mind always dwells on external enjoyment or desires, greed, envy, lust and many activities where the mind is in a permanent state of restlessness, constantly moving left, right, left, right, left, right; so, one becomes miserable. Due to this state of mind, one is dragged into the cycle, the whirlpool, of birth and death. Constantly coming out and in, in and out, one remains a slave. One would say, \enquote{When will I be free?} But to be free one has to transcend these qualities.}

\enquote{Bhagavān says here that one with \textit{rajasic} qualities does not learn to attain the grace of God, to attain the Lord through devotion and Love. How many people really understand Love? Not many. How many people really live this Love? Not many. People whose minds always dwell on outside things don’t know about this Love; they have a certain image of love. They create a certain reality and they call it love, when in reality it’s not love at all. It is attachment and this type of love falls into this category. Whereas true Love frees oneself, and true Love is the core of what one has to attain.}

\textbf{Qualities of \textit{tamo-guṇa}:} \enquote{Bhagavān says the roots of \textit{tamasic} qualities arise due to ignorance. You can see this in the world nowadays, when people are not aware that they have a life. They walk like zombies in the street, not present. They are completely deluded by these qualities.}

\enquote{The different activities that reflect these qualities include sleeping, lazing around and so on, always playing computer games non-stop. Even when they go away from the computer, the mind is drawn back to the computer game.}

\enquote{With these \textit{tamasic} qualities, this attachment to these lower frequencies, to lower manifestations, one becomes a slave to them. They can’t be free. And such is the deluded mind: the mind is so active that it doesn’t want them to realize themselves, and so the soul goes dormant. The mind that is focused on \textit{tamasic} qualities and has these tendencies is blind; one with such a mind doesn’t want to hear about God at all! They don’t want to hear about anything that is positive, just like the Rākṣasas, demons, and \textit{asuras}. Their minds are constantly drawn to harming and to hurting. This \textit{tamasic} mind doesn’t know anything else. Such people are kept in the cycle of birth and death. They can’t be free from this cycle.}

\section{Verses 19-20: Transcending the Guṇas}\label{sec-verses-19-20-transcending-the-gunas}

\Verse[14.19]
{nānyaṁ guṇebhyaḥ kartāraṁ yadā draṣṭānupaśyati \\
guṇebhyaś ca paraṁ vetti mad-bhāvaṁ so ’dhigacchati}
{When the seer perceives that there is no doer other than the \textit{guṇas}, and knows itself to be superior to the \textit{guṇas}, it attains My divine nature.}

\Verse[14.20]
{guṇān etān atītya trīn dehī deha-samudbhavān \\
janma-mṛtyu-jarā-duḥkhair vimukto ’mṛtam aśnute}
{Transcending these three \textit{guṇas} that give rise to the body, the embodied Self is liberated from birth, death, old age, and suffering, and attains immortality.}

\enquote{When the mind, senses and the vital air are diverted from the normal activities or functions like hearing, seeing, eating, drinking, reflecting, sleeping, sitting and so on---when one attains true knowledge, which one gets on the spiritual path---and one is absorbed through meditation on the Self\ldots the mind runs inwardly.}

\enquote{One doesn’t go on an outside journey, but an inner journey. When one follows this inner journey, automatically one rises---due to the \textit{sattvic} \textit{bhāva}---and the state of mind at that moment changes. The state of mind is not to focus on the outside reality. Due to these \textit{sattvic} qualities inside of you, one makes one’s journey inside. When one follows the inside journey, one perceives that there’s another reality---the Great Observer---and one perceives that it is the Self that observes the reality of life, and that all this drama of the \textit{sattvic}, \textit{rajasic}, and \textit{tamasic} qualities is just outside. These qualities are not related to the \textit{ātmā}. They are related only to the body. At this moment, true consciousness awakens. In this state, one rises above the \textit{guṇas} and one perceives God within oneself, the Supersoul, the Lord of all.}

\enquote{One who recognizes God within oneself is not connected to the three \textit{guṇas}. One who perceives God in everything is not bound by the three \textit{guṇas} and automatically, one who is absorbed in such a state of divine consciousness, doesn’t identify with the outside reality. One perceives that beyond these \textit{guṇas} is only Bhagavān Himself who is the absolute form and that it is Him only in His formless aspect (\textit{nirguṇa} state), that manifests everything.}

\section{Verses 21-25: Qualities of Those Above the Guṇas}\label{sec-verses-21-25-qualities-of-those-above-the-gunas}

\Verse[14.21]
{\hspace*{1em}arjuna uvāca \\
kair liṅgais trīn guṇān etān atīto bhavati prabho \\
kim-ācāraḥ kathaṁ caitāṁs trīn guṇān ativartate}
{\hspace*{1em}Arjuna said: \\
By what signs is he characterized as one who has gone beyond the three \textit{guṇas}? How does such a person behave? And how does one overcome the three \textit{guṇas}, O Lord?}

\Verse[14.22–25]
{\hspace*{1em}śrī-bhagavān uvāca \\
prakāśaṁ ca pravṛttiṁ ca moham eva ca pāṇḍava \\
na dveṣṭi sampravṛttāni na nivṛttāni kāṅkṣati \\
udāsīna-vad āsīno guṇair yo na vicālyate \\
guṇā vartanta ity evaṁ yo ’vatiṣṭhati neṅgate \\
sama-duḥkha-sukhaḥ sva-sthaḥ sama-loṣṭāśma-kāñcanaḥ \\
tulya-priyāpriyo dhīras tulya-nindātma-saṁstutiḥ \\
mānāpamānayos tulyas tulyo mitrāri-pakṣayoḥ \\
sarvārambha-parityāgī guṇātītaḥ sa ucyate}
{\hspace*{1em}Bhagavān Kṛṣṇa said: \\
O Pāṇḍava, that person is said to have transcended the \textit{guṇas} who does not resent illumination, activity, or delusion when they appear, nor longs for them when they are absent; who remains indifferent and undisturbed by the \textit{guṇas}, who remains firm and does not waver, realizing that the \textit{guṇas} alone are acting; who is equal in pleasure and pain, established within; who sees a lump of earth, a stone, and a piece of gold as the same; who is firmly resolved; who is equal to those who are dear and those who are not; who is steady-minded and equal to praise and blame; who is indifferent to honor and dishonor, equal to friends and enemies, and who has given up all worldly undertakings.}


\enquote{Bhagavān is revealing the qualities of those who are above the \textit{guṇas}. They are not touched by blame or by honor. A mind which is completely under control exhibits the highest degree of calmness, serenity, tranquility. That mind is not touched by any turmoil. It remains equal under all circumstances.}

\enquote{One who has become completely absorbed in the cosmic body of the Lord, whose mind has transformed and become light, is above the game of the three \textit{guṇas}. They observe only\ldots They are not bound by judgment. Their eyes still see, but there’s no judgment of what they see. Their nose still smells, but there’s no good or bad. Even if they do something that for the mind of men appears crazy, they are not bound by prejudice. They stay always free. Honor and dishonor is the same to them.}

\enquote{As long as you entertain these three \textit{guṇas}, you are in the grip of \textit{māyā} and you are dancing to Her tune. As I always say, \textit{māyā} puts on the music, and you dance like a puppet. Like a tamed animal in a circus, kept in a cage. When the circus man throws his whip, the animal starts dancing, like the tamed monkey that turns and jumps around. This is similar to what Bhagavān Kṛṣṇa is saying: the circus of life. As long as one doesn’t break through and rise above this, one remains in this circus. And as long as you stay in the game of this circus, there is judgment. But with the disappearance of all these \textit{guṇas}, when you are above, looking from the \textit{ātmā’s} point of view, when you sit in meditation, you perceive there is no judgment!}

\enquote{Bhagavān Kṛṣṇa Himself is saying, \enquote{Be in this state! Don’t be in a state of judgment. And don’t entertain the game of the \textit{guṇas}; that is, don’t say, \enquote{I am \textit{rajasic}, \textit{sattvic}, or \textit{tamasic}.}} Also, don’t look for this in others, because it’s easy to do. Now that you have heard about these three \textit{guṇas}, you could start saying, \enquote{Ah, that person is \textit{rajasic}, that person is \textit{tamasic}, and that person is \textit{sattvic}.} This is a “three-\textit{guṇas} fever.” Step away from it, and you will be healthy!}

\enquote{When you are fully absorbed in the Divine within you, you become a \textit{guṇātītaḥ}, you go above the three \textit{guṇas}. In such a state you are in complete bliss. That’s why bliss can’t be explained. You can’t say bliss is happiness or bliss is sadness. You can’t say that this person who is jumping around and happy is in a very blissful state. No! Because bliss is a state when you are fully absorbed in your Self.}

\section{Verses 26-27: By Devotion One Rises Above the Guṇas}\label{sec-verses-26-27-by-devotion-one-rises-above-the-gunas}

\Verse[14.26]
{māṁ ca yo ’vyabhicāreṇa bhakti-yogena sevate \\
sa guṇān samatītyaitān brahma-bhūyāya kalpate}
{But one who serves Me with unwavering devotion transcends the \textit{guṇas} and becomes fit to realize Brahman.\footnote{Used here to refer to the impersonal conception of the supreme reality. Its abstract features (immortal, unchanging, eternal, blissful, etc.) have a concrete existence in the Supreme Lord.}}

\Verse[14.27]
{brahmaṇo hi pratiṣṭhāham amṛtasyāvyayasya ca \\
śāśvatasya ca dharmasya sukhasyaikāntikasya ca}
{I am indeed the foundation of the immortal and unchanging Brahman, as well as the basis of the eternal \textit{dharma} and absolute happiness.}


\enquote{Bhagavān Kṛṣṇa says, \enquote{I am this eternal God Himself who is formless and pervades over this whole universe. I am the One who sustains; I am the One who creates; I am the One who destroys. I am the Supreme, unmanifest and formless God. I am Śiva. I am Viṣṇu. I am Rāma. I am Kṛṣṇa. I am Jesus. I am Allah. I am Yahweh. I am all these Names. Here I am in front of you as Vāsudeva Kṛṣṇa.} All these forms and aspects of the Divine have been manifested throughout time to remind people of their real \textit{dharma}; of why they have come here; why they are here; of how to free themselves; of how to attain their cosmic body and how to rise above this duality; how to rise above the game of \textit{māyā} and how to rise above these three \textit{guṇas}. And only through this will one attain supreme bliss, will one attain this blissful state, will one become blissful. Kṛṣṇa Himself says here, \enquote{It’s easy to attain this state. Just surrender to Me! If you love Me, if you serve Me, if you hear My glory, I shall reveal this supreme bliss to you. Not outside, but within yourself and from within yourself you will beam this out to all.} This is God-realization. This is surrender and this is true happiness.}
