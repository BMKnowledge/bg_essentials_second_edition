\chapter{Karma-sannyāsa-yoga}\label{chap-karma-sannyasa-yoga}
\noindent\textbf{CHAPTER 5}\par
\paragraph*{\MakeUppercase{True Renunciation}}
%%%%%%%%%%%%%%%%%%%%%%%%%%%%%%%%%%%%%%%%%%%%%%%%%%%%%%%%%


\noindent Kṛṣṇa acknowledges that renunciation of external action (\textit{jñāna-yoga}) and acting without attachment (\textit{karma-yoga}), are two spiritual paths, but nonetheless they both deliver the same goal: they both lead to knowledge of the Self. For this reason, they are ultimately non-different. However, due to the ease of practice, \textit{karma-yoga} is judged to be the better of the two.

This knowledge of the Self makes us realize that despite performing all kinds of activity, in truth we do nothing. All action is simply the senses engaging with the world and does not contaminate our true nature. Established in this state, we become untouched by the movements of life. We develop equanimity, beyond pain and pleasure. Eventually we rise beyond every form of judgment and see the same divinity in all living beings. Such \textit{yogīs}, who are detached and have conquered desire, master the mind, which is the ultimate cause of pain and suffering in the material world.

\Verse[5.1]
{\hspace*{1em}arjuna uvāca \\
sannyāsaṁ karmaṇāṁ kṛṣṇa punar yogaṁ ca śaṁsasi \\
yac chreya etayor ekaṁ tan me brūhi suniścitam}
{\hspace*{1em}Arjuna said: \\
Kṛṣṇa, You recommend the renunciation of action and then again praise \textit{karma-yoga}. Tell me conclusively which of the two is better.}

\Verse[5.2]
{\hspace*{1em}śrī-bhagavān uvāca \\
sannyāsaḥ karma-yogaś ca niḥśreyasa-karāv ubhau \\
tayos tu karma-sannyāsāt karma-yogo viśiṣyate}
{\hspace*{1em}Bhagavān Kṛṣṇa said: \\
Renunciation of action and \textit{karma-yoga} both lead to the highest good. But of the two, \textit{karma-yoga} is superior to the renunciation of action.}

\Verse[5.3]
{jñeyaḥ sa nitya-sannyāsī yo na dveṣṭi na kāṅkṣati \\
nirdvandvo hi mahā-bāho sukhaṁ bandhāt pramucyate}
{One who neither resents nor desires anything should be known as a permanent renunciate. One who is free from the pairs of opposites is easily liberated from bondage, O mighty-armed Arjuna.}

\Verse[5.4]
{sāṅkhya-yogau pṛthag bālāḥ pravadanti na paṇḍitāḥ \\
ekam apy āsthitaḥ samyag ubhayor vindate phalam}
{The inexperienced speak of \textit{jñāna-yoga} and \textit{karma-yoga} as separate, but not the wise. One who is firmly established in even just one path attains the fruit of both.}

\Verse[5.5]
{yat sāṅkhyaiḥ prāpyate sthānaṁ tad yogair api gamyate \\
ekam sāṅkhyaṁ ca yogam ca yaḥ paśyati sa paśyati}
{The state which is attained by the \textit{jñāna-yogīs} is also attained by the \textit{karma-yogīs}. The one who sees both \textit{jñāna-yoga} and \textit{karma-yoga} as one sees truly.}

\Verse[5.6]
{sannyāsas tu mahā-bāho duḥkham āptum ayogataḥ \\
yoga-yukto munir brahma na cireṇādhigacchati}
{But renunciation is hard to achieve without \textit{karma-yoga}, O mighty-armed Arjuna. The sage absorbed in this \textit{yoga} attains realization without delay.}

\Verse[5.7]
{yoga-yukto viśuddhātmā vijitātmā jitendriyaḥ \\
sarva-bhūtātma-bhūtātmā kurvann api na lipyate}
{One who is engaged in \textit{karma-yoga}, has a pure heart, has subdued the mind and conquered the senses, perceives the \textit{ātmā} as the essence of all beings and remains untainted even while acting.}

\enquote{Arjuna is still examining what Kṛṣṇa said in the previous chapters about \textit{jñāna-yoga}, the path of knowledge, and \textit{karma-yoga}, the path of action. The Lord says, \enquote{Both paths are okay.} But the mind of Arjuna is still troubled, because if Kṛṣṇa considers knowledge superior to action, why does He also urge him to fight? If knowledge is superior to action, then Kṛṣṇa should just say, \enquote{Okay, renounce everything, go and sit in a cave!} This doubt is still playing in Arjuna’s mind, so he is unable to really understand what Kṛṣṇa is trying to tell him. He continues to look at this from the point of view of the mind, asking Lord Kṛṣṇa, \enquote{Which one will be better for me? Which path should I adopt to attain this true wisdom, to attain the wisdom of the Self?} And in which way should he be disciplined---by meditating on this knowledge or by doing his duty without any personal expectation, without any attachment, dedicating all his action to God? Arjuna is sitting there in his chariot, on the battlefield, sincerely asking, \enquote{Which path, Lord?}}

\enquote{Here the Lord says that both \textit{yogas} will bring you to supreme bliss. When you are disciplined in your \textit{sādhana} and doing it for the sake of attaining God, you will attain salvation. But Kṛṣṇa gives priority to \textit{karma-yoga} saying to Arjuna, \enquote{You are a \textit{karma-yogī}, you are married, you have a family, you have children, you are a warrior! It’s not meant for you to go sit and meditate in a cave.} It is interesting to note here that Kṛṣṇa is talking to everybody in this world, saying, \enquote{Find Me in whatever you do! I am in your house! The moment you sit to do your \textit{sādhana}, I am there. I am also in the work that you are doing.}}

\enquote{Most people usually work in order to get something, so there is always an expectation. But when work is only done to serve God, to please Him, one beholds the Absolute Lord in whatever one does. Then, there is a mixture of both \textit{yogas}, \textit{karma-yoga} and \textit{jñāna-yoga}, because the \textit{karma-yogī} is acting as a true renunciate. On the other hand, there are many people who pretend to renounce, who appear to renounce, yet they haven’t renounced anything. The renunciation that Lord Kṛṣṇa is referring to is not a superficial renunciation. It means to renounce very deeply and sincerely. It’s of no use to say, \enquote{I have renounced everything! I have renounced the outside world,} and then sit there and think, \enquote{Oh, I wish I had this} or \enquote{I wish I had that.} When you renounce something, you have to let go of it in the mind itself.}

\enquote{There are many people who are still living in the world outside, yet they are true renunciates; they are free! They are not attached to the fruits of their actions. Kṛṣṇa says, \enquote{That’s the best path!} It’s not because one wears a robe that one becomes a renunciate. One becomes a renunciate when one renounces the expectations, this aggressiveness of the mind; when one has control over the senses; when one perceives that the aim of life is only to serve God in everything. This is true renunciation! The true renunciate can be a \textit{karma-yogī}, someone who is doing one’s duty in the outside world, yet who is completely surrendered to the Divine.}

\enquote{Renouncing everything for something will not bring you peace, because there will be always an expectation. And when this expectation is not fulfilled, what will arise? Frustration. And frustration will lead you to anger. Anger will lead you to mental stress.}

\enquote{You have incarnated in this world to do certain things, and once you are here, you can’t run away. Even a saint who has renounced everything, sitting in the cave, in a hut, is performing action too.}

\newpage
\section{Verses 8-17: The Self is Untouched by Action}\label{sec-verses-8-17-the-self-is-untouched-by-action}

\Verse[5.8–9]
{naiva kiñcit karomīti yukto manyeta tattva-vit \\
paśyañ śṛṇvan spṛśañ jighrann aśnan gacchan svapañ śvasan \\
pralapan visṛjan gṛhṇann unmiṣan nimiṣann api \\
indriyāṇīndriyārtheṣu vartanta iti dhārayan}
{The one who is engaged in \textit{yoga} is aware that even while seeing, hearing, touching, smelling, eating, moving, sleeping, breathing, speaking, releasing, grasping, as well as opening and closing the eyes, it is just the senses operating among sense-objects. Such a knower of truth thus considers, \enquote{I do nothing.}}

\Verse[5.10]
{brahmaṇy ādhāya karmāṇi saṅgaṁ tyaktvā karoti yaḥ \\
lipyate na sa pāpena padma-patram ivāmbhasā}
{One who acts, surrendering one’s actions to the Lord and having given up attachment, is not tainted by sin, just as a lotus leaf is unaffected by water.}

\Verse[5.11]
{kāyena manasā buddhyā kevalair indriyair api \\
yoginaḥ karma kurvanti saṅgaṁ tyaktvātma-śuddhaye}
{Having given up attachment, \textit{yogīs} perform action by the body, mind, intellect, and even with the senses alone, for the purification of the mind.}

\Verse[5.12]
{yuktaḥ karma-phalaṁ tyaktvā śāntim āpnoti naiṣṭhikīm \\
ayuktaḥ kāma-kāreṇa phale sakto nibadhyate}
{Giving up the fruits of action, the selfless \textit{yogī} attains everlasting peace. But the one who works selfishly is attached to the results of action, impelled by desire and thus remains bound.}

\Verse[5.13]
{sarva-karmāṇi manasā sannyasyāste sukhaṁ vaśī \\
nava-dvāre pure dehī naiva kurvan na kārayan}
{Mentally renouncing all actions, the embodied Self who has controlled the senses lives happily in the city of nine gates, neither acting nor causing the body to act.}

\Verse[5.14]
{na kartṛtvaṁ na karmāṇi lokasya sṛjati prabhuḥ \\
na karma-phala-saṁyogaṁ svabhāvas tu pravartate}
{The Lord does not create the sense of agency, neither does He initiate people's actions, nor does He create the connection between actions and their result---the \textit{jīva's} external nature accomplishes all this.}

\Verse[5.15]
{nādatte kasyacit pāpaṁ na caiva sukṛtaṁ vibhuḥ \\
ajñānenāvṛtaṁ jñānaṁ tena muhyanti jantavaḥ}
{The all-pervading Lord does not take on the sin or merit of any person.
Mortal beings, however, are deluded by ignorance which covers their knowledge.}

\Verse[5.16]
{jñānena tu tad ajñānaṁ yeṣāṁ nāśitam ātmanaḥ \\
teṣām āditya-vaj jñānaṁ prakāśayati tat param}
{But for some, that ignorance covering the Self is destroyed by realization; their knowledge, like the sun, reveals the Supreme.}

\Verse[5.17]
{tad-buddhayas tad-ātmānas tan-niṣṭhās tat-parāyaṇāḥ \\
gacchanty apunar-āvṛttiṁ jñāna-nirdhūta-kalmaṣāḥ}
{Those whose intellects are absorbed in Him, whose minds are focused on Him, who are fixed in Him, and who have Him as their supreme goal, attain the state from which there is no return, cleansed from all impurities by knowledge.}



\enquote{Those who are always aware of the ultimate reality remain constantly absorbed in the true Identity of God. As their minds are always focused on the Divine, they are not attached to pride and say, \enquote{I am doing this,} or \enquote{I am doing that.} No! They are always aware that it is the Divine who is doing everything through them. And if one is fully focused on the Self, focused on God and serving Him, there will be no duality. When duality disappears from the mind, there will be no trace of evil qualities like lust, greed, anger, infatuation, or egoism. All these negative qualities, these \enquote{one hundred Kauravas,} will be killed, will disappear, and vanish! The heart will be purified and the mind will be constantly under control. Even when they walk in the outside world, whatever they do doesn’t touch them at all. There is no anger in their hearts and no judgment in their minds. They are free! That’s a true \textit{yogī}! That’s a true devotee of the Lord.}

\enquote{Here Lord Kṛṣṇa uses the example of the lotus leaf. The lotus is born from muddy water, which is very dirty, but when it rises to the surface of the water, the water doesn’t touch it. Have you ever seen lotus leaves? Even if you put water on them, it quickly falls off. He meant here that you have to be like the lotus leaf, so that nothing can cling to you. When you do your \textit{sādhana}, only God should be there. If God is your aim, if God becomes everything for you, no negativity will be able to cling to you. Like water can’t stay on a lotus leaf and runs off it, if one is completely surrendered to the Divine, all the negative qualities, all the kinds of attachments which one has in this world, will disappear. Even the attachment to gaining Heaven will disappear.}

\enquote{On the other hand, every action which is done with expectation, with the aim of self-gratification or with ego, will bring pain to oneself. And due to the \textit{karma} created from this, one binds oneself to the cycle of birth and death and this will destroy life itself. And the more one is attached to this \textit{karma}, the more difficult it is to remove it. It’s like creating and building up layers of \textit{karma}.}

\enquote{A true \textit{karma-yogī} considers everything as coming from God, belonging to God and only for God’s service. Everything: the body, the senses, the mind, and the intellect, belong to God. So the \textit{karma-yogī} performs all his duties absolutely unselfishly, with a disinterested spirit. In this way, one is completely purified. In such a state, the Divine takes full control of oneself and one is totally under the inspiration and guidance of God, living and working as His instrument. Even the small \enquote{I} which is aware of God, is only Him.}

\enquote{Kṛṣṇa says, \enquote{By mere renunciation of the mind, one gets attached to the ‘nine-gated city,’ the body.} But when the mind is fully absorbed in the Divine through service, when the mind is tamed and controlled through discrimination and reasoning, it turns automatically within. Then, when the mind turns within, the intellect awakens and through this intellect, through this wisdom, the heart opens up to Love. Through that Love, one is fully absorbed in the Supersoul, in God consciousness within one’s soul. In this state, it is the soul itself that is perceiving: at first the mind was looking at the soul, and now the soul is looking at the Supersoul.}

\enquote{Those who are under the control of the \textit{indriyas}, the sense organs, and the sense enjoyment claim doership. They are not aware of the divine Self within themselves due to this ignorance. Here the Lord says, \enquote{I am beyond this.} All this game of the outside reality is done by \textit{prakṛti}. Everything which is manifested, which is tangible, is \textit{prakṛti}. But this doesn’t affect Him or touch Him. He is the Great Observer. He is the soul of souls who is not touched by anything.}

\enquote{Lord Kṛṣṇa emphasizes that as you are the \textit{ātmā}, you also are out of this game: you are not the doer. It is just the human mind that gives you the limited perception that you are the doer.}

\enquote{The natural function of the heart is to beat and to pump the blood throughout the body. The natural function of the eyes is to see. The natural role of fire is to burn. The natural role of water is to wet. All these have been created through \textit{prakṛti}, only through His will! So Mother Nature carries on Her own work. But when one gets realized, one rises above material nature. However, if the \textit{jīva} has not risen above material nature, if one doesn’t realize that one is hanging onto the \textit{guṇas}, how can one be free? If people don’t realize their true essence as being part of the Supreme, if they don’t see the divinity within themselves, if they get stuck in the duality of like and dislike, good and evil, then they unite themselves with material nature and limit themselves to only that outward reality.}

\enquote{Very often people think that God is only \enquote{good} and doesn’t have any imperfections inside of Him. But how can He be God, if He doesn’t have any imperfections? He is the source of everything! He has all the \enquote{good} inside Him and He has all the imperfections inside Him. And He is also above all these qualities! That’s what makes Him God, because He is beyond this duality, beyond His own will, beyond His own creation.}

\enquote{Once the eyes are open, one will perceive that light, not only in oneself, but everywhere, because that light which the divinity carries is not limited; it is not only here in yourself, but it is in everything\ldots When you are fully absorbed into the Divine, you don’t need anybody’s confirmation. That automatically will shine through you. You will glow. Here the glow is not an external glow, but a glow from the inside out, and that glow reflects upon you. That glow will consume every trace of ignorance in the body, in the mind, everywhere. And that’s what will set you free.}

\enquote{The first thing that one has to do is perceive Kṛṣṇa within oneself. When you sit down in your meditation you are absorbed and seeing the \textit{guru} and God within you. So, this is where the inner \textit{guru} reveals Himself. So, this wisdom, this awakening, will make you luminous and will transform your gloomy self into a luminous Self. You will start to shine. You will transform everything around. The misdeeds, misapprehension, the impure thoughts---everything will be purified. So, when everything is glowing with this luminous light, this uncreated light of Kṛṣṇa within you, then how can there be sin? Once you have attained that level, nothing can bring you back. You can’t fall back. You will stay immersed in that divine bliss even being in the world itself.}

\section{Verses 18-26: The State of Equanimity}\label{sec-verses-18-26-the-state-of-equanimity}

\Verse[5.18]
{vidyā-vinaya-sampanne brāhmaṇe gavi hastini \\
śuni caiva śva-pāke ca paṇḍitāḥ sama-darśinaḥ}
{The sages perceive with equal vision, a learned and humble \textit{brāhmaṇa}, a cow, an elephant, a dog, and an outcaste.}

\Verse[5.19]
{ihaiva tair jitaḥ sargo yeṣāṁ sāmye sthitaṁ manaḥ \\
nirdoṣaṁ hi samaṁ brahma tasmād brahmaṇi te sthitāḥ}
{Those whose minds are lodged in equanimity have risen beyond the conditions of the material world, even while still here in this world. Since Brahman is without faults and equanimous, they are considered to be fixed in Brahman.}

\Verse[5.20]
{na prahṛṣyet priyaṁ prāpya nodvijet prāpya cāpriyam \\
sthirabuddhir asammūḍho brahma-vid brahmaṇi sthitaḥ}
{One who knows Brahman and is established in Brahman possesses a stable intellect, is free from delusion and neither rejoices when gaining what is pleasant nor grieves when receiving what is unpleasant.}

\Verse[5.21]
{bāhya-sparśeṣv asaktātmā vindaty ātmani yat sukham \\
sa brahma-yoga-yuktātmā sukham akṣayam aśnute}
{One whose mind is detached from external sensations finds the happiness that is natural to the \textit{ātmā} and one whose entire self is absorbed in the Lord through \textit{yoga} enjoys endless bliss.}

\Verse[5.22]
{ye hi saṁsparśa-jā bhogā duḥkha-yonaya eva te \\
ādy-antavantaḥ kaunteya na teṣu ramate budhaḥ}
{Pleasures that arise from contact with the external world are indeed sources of suffering alone. Having a beginning and an end, O son of Kuntī, a wise person does not rejoice in them.}

\Verse[5.23]
{śaknotīhaiva yaḥ soḍhuṁ prāk śarīra-vimokṣaṇāt \\
kāma-krodhodbhavaṁ vegaṁ sa yuktaḥ sa sukhī naraḥ}
{In this very human life, someone who can tolerate the impulses arising from desire and anger before shedding the body is a true \textit{yogī} and a joyful person.}

\Verse[5.24]
{yo ’ntaḥ-sukho ’ntar-ārāmas tathāntar-jyotir eva yaḥ \\
sa yogī brahma-nirvāṇaṁ brahma-bhūto ’dhigacchati}
{One who is joyful within, whose delight is within, and who is illumined within, is a true \textit{yogī}. He has realized Brahman and attains the bliss of Brahman.}

\Verse[5.25]
{labhante brahma-nirvāṇam ṛṣayaḥ kṣīṇa-kalmaṣāḥ \\
chinna-dvaidhā yatātmānaḥ sarva-bhūta-hite ratāḥ}
{The sages whose doubts have been destroyed, who are self-controlled and are devoted to the welfare of all beings, become cleansed of all impurities and attain the bliss of Brahman.}

\Verse[5.26]
{kāma-krodha-vimuktānāṁ yatīnāṁ yata-cetasām \\
abhito brahma-nirvāṇaṁ vartate viditātmanām}
{For the ascetics who are free from desire and anger, whose minds are controlled, and who have realized the \textit{ātmā}, the bliss of Brahman is near.}

\enquote{Here \enquote{\textit{brāhmaṇa}} doesn’t mean a priest or somebody who is born into a certain caste. No, Kṛṣṇa says here, \enquote{You are \textit{brāhmaṇa} when you have realized God, when you have realized your relationship with Him, when you have attained Him.}}

\enquote{Those who have this true knowledge\ldots are fully aware that every duty is only service to the Lord.}

\enquote{The sages, the \textit{sādhus}, the saints, perceive only the oneness: they see God equally in each person. They see beyond outward appearances and respect each person’s role as being in accordance with the will of God. They have this deep knowledge which allows them to see that there is perfection everywhere! There is nothing that is imperfect! Imperfection is due to the ignorance in the minds of men. But if one has transcended one’s human nature and realized the divine Self, and the divine qualities manifest through one, then one sees that the whole creation which has come out of Him, is a projection of His own will.}

\enquote{The one who is ignorant finds fault everywhere and proclaims, \enquote{I am better than you. What I believe is best! My God is better than your God. Why do you have so many gods?} Then due to ignorance people fight over all these things. But the truly wise ones, who have realized themselves, whose knowledge is not based on mere study of scriptures or on books, who have perceived the Truth within themselves, who have perceived the Divine, God within their true Self, don’t find fault anywhere.}

\enquote{People can give them either honor or dishonor. People can criticize them because of lack of understanding. But to the saint, it doesn’t make any difference. For the one who is seated in God consciousness, it doesn’t matter if life brings joy or sorrow. His faith is never shaken!}

\enquote{One who takes delight in the sense enjoyment of the outside world, whose mind is always running towards judgment and criticizing everything, will never experience this supreme, blissful state. The joy born of dispassion, the bliss which the realized soul attains, is far greater than any so-called joy and happiness of the outside reality. People’s happiness level always varies, doesn’t it? Today they \enquote{wake up with the right foot,} and they are happy; tomorrow they \enquote{wake up with the left foot} and they are not happy. The ones who are constantly awake in the divine Self, who are constantly in the blissful state, always \enquote{wake up with the right foot,} even if the left foot touches the ground first. They are constantly meditating on God, so they perceive only this One reality, Nārāyaṇa. They eat: Nārāyaṇa. They sleep: Nārāyaṇa. They talk: Nārāyaṇa. They dance: Nārāyaṇa. They laugh: Nārāyaṇa. There is not a single moment when Nārāyaṇa is not present, for each breath is Nārāyaṇa.}

\enquote{No one is responsible for your happiness or your unhappiness except yourself. You are the cause of your own happiness. When you are surrendered wholeheartedly to God, when your mind, body and spirit are fully in accordance with the will of God, serving God, meditating on Him, this becomes the source of your true happiness and bliss! But when the mind runs towards the external reality thinking, \enquote{Yes, this is it!,} when one is finding great joy in the outside world, this will bring deep sorrow to oneself. The one who enjoys the sense objects, automatically becomes evil-minded, greedy, lustful and angry. One will have to go through enormous suffering in this life and even later on, in future lifetimes. And then one will ask, \enquote{What have I done to God to suffer so much? Why me, and not my neighbor?}}

\enquote{God is the Supreme Light of all lights. The whole universe is illumined by His effulgence. He is the Supreme soul in all beings. A true \textit{yogī} is ever conscious of such a God. He sees the Ultimate as His own Self. He remains constantly absorbed in His Love, His Peace and His Light. Even though the world appears real, the true \textit{yogī} sees beyond the external reality. He sees beyond what normal people see. He sees the source. He sees within his own Self the Supreme Light, the Light of God, \enquote{\textit{antarātmā},} \enquote{\textit{antara-jyoti}.} \enquote{\textit{Antara-jyoti}} is the inner Light; \enquote{\textit{antarātmā}} is the Supreme Light within the core of his soul itself. Then, one starts to emanate the inner Light and becomes a source of salvation, not only for oneself, but for others.}

\enquote{Wherever one goes, one brings this Light to the people. These ones don’t think only for themselves. They say, \enquote{I have reached that realization. I want every soul to realize themselves. I want everybody to understand how the mind is and I want everybody to be free.} That’s what the \textit{guru} does. The \textit{gurus} have already realized and attained the Supreme Lord, but yet, they come down of their own accord to help, and to uplift humanity\ldots You have to change. You have to transform. You have a life. You are a living identity. You have to move out of this ignorance and be aware of the knowledge of the Self.}

\enquote{When you take shelter at the feet of the master, you have the reflection of the master upon you. As you come into the Light of the master, the Light of the master shines upon you\ldots Reflect your master. Let the Light you have received from your spiritual path enlighten others also.}

\section{Verses 27-29: Introducing Yoga Practice}\label{sec-verses-27-29-introducing-yoga-practice}

\Verse[5.27–28]
{sparśān kṛtvā bahir bāhyāṁś cakṣuś caivāntare bhruvoḥ \\
prāṇāpānau samau kṛtvā nāsābhyantara-cāriṇau \\
yatendriya-mano-buddhir munir mokṣa-parāyaṇaḥ \\
vigatecchā-bhaya-krodho yaḥ sadā mukta eva saḥ}
{Keeping out all thoughts of external objects, focusing the inner gaze between the eyebrows alone, regulating the inward and outward breaths within the nostrils and having controlled the senses, mind, and intellect, the sage aspiring for liberation who is free from desire, fear, and anger is verily always free.}

\Verse[5.29]
{bhoktāraṁ yajña-tapasāṁ sarva-loka-maheśvaram \\
suhṛdaṁ sarva-bhūtānāṁ jñātvā māṁ śāntim ṛcchati}
{Thus knowing Me to be the supreme enjoyer of all sacrifices and austerities, the Lord of all worlds, and the friend of every being, one attains peace.}

\enquote{Usually the human mind hangs onto the outside reality, but when the mind is tamed, it starts to look within. Then, the focus of the mind is towards the true knowledge of the soul. When you sit in meditation, the mind starts to perceive everything in a different way; the mind looks up to the soul, to the Great Observer. And when the soul is turned into itself, it rests in the greater Supersoul, who is Nārāyaṇa. When the soul rests in the Supersoul, it still continues its activities in the world, but it is the Supersoul who is doing this activity. At this point, one is fully absorbed into the Supersoul. One is fully absorbed into this awareness, that God alone is the reality, nothing else.}

\enquote{He says, \enquote{I am your best Friend, I am the Friend of every being. Those who attain that peace, those who attain that level of consciousness, they realize that whatever I am doing is not for My sake. Whatever I am doing is for your sake and for the sake of uplifting humanity.} That’s the humbleness of God. He lowers Himself so that you can be dear to Him.}
